\subsection{Representation dimension}\label{sec:repdimension}

In this section we look at the representation dimension of an algebra. This is another useful homological invariant of the representation theory for a finite dimensional algebra. The representation dimension is less than or equal to 2 if and only if $\Lambda$ is representation finite, so it is natural to think that the representation dimension in some sense measures how far $\Lambda$ is from being representation finite. In \cref{cor:repdim_less_than_3_implies FDC} we show that $\findim(\Lambda)$ is finite when $\repdim(\Lambda) \leq 3$, and in \cref{sec:stable_hereditary_algebras} and \cref{sec:special_biserial_algebras} we give a two examples of families of algebras that satisfy this.

\begin{defn}[Representation dimension]
	Let $\Lambda$ be a finite dimensional algebra. The \emph{representation dimension} of $\Lambda$, denoted $\repdim(\Lambda)$, is the minimal global dimension of $\End(M)^{\operatorname{op}}$ for $M$ a generator-cogenerator in $\mod\Lambda$. We call a generator-cogenerator that achieves this minimum an \emph{Auslander-generator}.
\end{defn}

The representation dimension can also be defined using $\mathcal M$-resolutions, which we define here.

\begin{defn}[$\mathcal M$-resolutions]
	Let $X$ be an object in $\mod\Lambda$ and $\mathcal M$ a contravariantly finite subcategory. We consider a diagram as the one below.
	\begin{center}
	\begin{tikzcd}
		\cdots \ar[rd, two heads] \ar[r] & M_2 \ar[rd, two heads] \ar[r] & M_1 \ar[rd, two heads] \ar[r] & M_0 \ar[rd, two heads]\\
		&\Omega_M^3 X \ar[u, hook] & \Omega_M^2  \ar[u, hook] X & \Omega_M X  \ar[u, hook] & X
	\end{tikzcd}
	\end{center}
	If the maps $M_n \twoheadrightarrow \Omega_M^nX$ are minimal right $\mathcal M$-approximations for $n\geq 0$ (they need not be surjective), and $\Omega_M^{n+1} \hookrightarrow M_n$ are their kernels, then this is a minimal \emph{$\mathcal M$-resolution} of $X$. The \emph{$\mathcal M$-res-dimension} of $X$ is the length of this sequence of (nonzero) $M_i$'s, and the $\mathcal M$-res-dimension of $\Lambda$ is the supremum of the dimension of its objects.
\end{defn}

An $\mathcal M$-resolution of $X$ should be thought of as a projective resolution of $\Hom(-, X)|_{\mathcal M}$ in the category of coherent functors on $\mathcal M$. When $\mathcal M = \add M$ the category of coherent functors is isomorphic to $\mod \End(M)^{\op}$, where $\Hom(-, X)|_{\mathcal M}$ corresponds to $\Hom(M, X)$. In the proof of the next proposition we use this correspondence, and we write $M$-$\operatorname{res-dim}$ instead of $(\add M)$-$\operatorname{res-dim}$. 

\begin{prop}\label{prop:repdim_resdim+2}\cite[Chapter~III.5]{Aus71}
	If the representation dimension of $\Lambda$ is at least $2$, then $\repdim(\Lambda) - 2$ equals the minimum of $M$-$\operatorname{res-dim}(\mod \Lambda)$ for $M$ a generator-cogenerator. In fact, for any generator-cogenerator, $M$-$\operatorname{res-dim}(\mod \Lambda)$ is two less than the global dimension of $\End(M)^{\op}$.
	\begin{proof}
		Let $M$ be a generator-cogenerator. We first show that the global dimension of $\End(M)^{\op}$ is less than or equal to $M$-$\operatorname{res-dim}(\mod \Lambda) + 2$. 
		
		The functor $\Hom(M, -)$ is an equivalence from $\add M$ to $\proj\End(M)^{\op}$, which maps minimal $M$-approximations to projective covers. Let $X$ be any module in $\mod\End(M)^{\op}$ with projective dimension at least 2. Then it has a projective presentation 
		\begin{center}
			\begin{tikzcd}
				\Omega^2X \ar[r] & (M,M_1) \ar[r] & (M,M_0) \ar[r] & X.
			\end{tikzcd}
		\end{center}
		Because of the equivalence this is induced by a map $f\colon M_1\to M_0$. Since $\Hom(M,-)$ is left exact we have that $\Omega^2X \cong \Hom(M, \ker f)$, and so the projective dimension of $X$ is $2$ more than the resolution dimension of $\ker f$ with respect to $M$. Hence  we have that $$\operatorname{gl.dim}\End(M)^{\op} \leq M\operatorname{-res-dim}(\mod \Lambda) + 2.$$
		
		Next we prove the other inequality.
		
		Since $M$ is a cogenerator, any module $Y$ in $\mod\Lambda$ has a copresentation 
		\begin{center}
		\begin{tikzcd}
			0 \ar[r] & Y \ar[r] & M_0 \ar[r, "f"] & M_1.
		\end{tikzcd}
		\end{center}
		Applying $(M,-) := \Hom(M,-)$ we get
		\begin{center}
		\begin{tikzcd}[column sep=24pt]
		0 \ar[r] & (M,Y) \ar[r] & (M,M_0) \ar[r]{}{f \circ -} & (M,M_1) \ar[r] & \cok(M,f) \ar[r] & 0.
		\end{tikzcd}
		\end{center}
		If the projective dimension of $\cok(M,f)$ is less than 2, then $(M, Y)$ is a direct summand of $(M, M_0)$. This means that $(M,Y) \cong (M, M')$, so the minimal $M$-approximation of $Y$ is $M'$, and $(M, \Omega_M Y) = 0$. Since $M$ is a generator this means $\Omega_M Y = 0$ and thus $M\operatorname{-res-dim}(Y)=0$.
		
		So provided the projective dimension of $\cok(M,f)$ is larger than or equal to 2, it equals $M\operatorname{-res-dim}(Y)+2$. In particular the global dimension of $\End(M)^{\op}$ is larger than or equal to $M$-$\operatorname{res-dim}(\mod \Lambda) + 2$. Hence they are equal.
	\end{proof}
\end{prop}

The next two results paint an important picture of the representation dimension as an invariant, but are not relevant for the other results in this thesis.

\begin{theorem}\cite[Corollary~1.2]{Iya02}
	The representation dimension of an artin algebra is always finite.
	\begin{proof}
		This was proven by Iyama in 2002. The proof is omitted here, but can be found in their paper \cite{Iya02}.
	\end{proof}
\end{theorem}

\begin{prop}\cite[Chapter~III.4]{Aus71}
	The representation dimension of $\Lambda$ is less than or equal to $2$ if and only if $\Lambda$ is representation finite.
	\begin{proof}
		Assume $\Lambda$ is representation finite and let $M$ be the direct sum of all indecomposable modules up to isomorphism. Then $M$ is a generator-cogenerator. Let $X$ be an $\End(M)^{\operatorname{op}}$-module with projective presentation
		\begin{center}
			\begin{tikzcd}
				(M,M_1) \ar[r] & (M, M_0) \ar[r] & X \ar[r] & 0.
			\end{tikzcd}
		\end{center}
		Let $M_2$ be the kernel of $M_1 \to M_0$. Since $M$ is the sum of all indecomposables $M_2$ is in $\add M$, so 
		\begin{center}
			\begin{tikzcd}
				0 \ar[r] & (M, M_2) \ar[r] & (M,M_1) \ar[r] & (M, M_0) \ar[r] & X \ar[r] & 0
			\end{tikzcd}
		\end{center}
		is a projective resolution of $X$. So $\Lambda$ has representation dimension at most 2.
		
		Assume $\Lambda$ has representation dimension at most 2, and let $M$ be an Auslander-generator. We want to show that $\add M = \mod\Lambda$. Let $X$ be any $\Lambda$-module, and let 
		\begin{center}
			\begin{tikzcd}
				0 \ar[r] & X \ar[r] & I_0 \ar[r] & I_1
			\end{tikzcd}
		\end{center}
		be a minimal injective presentation. If $I_0 \to I_1$ is split then $X$ is injective and thus in $\add M$. Let $M_X$ be a minimal $M$-approximation of $X$, let $\Omega_M X$ be the kernel of the approximation, and let $Y$ be the cokernel of $(M, I_0) \to (M, I_1)$. Then 
		\begin{center}
			\begin{tikzcd}
				(M,\Omega_M X) \ar[r] & (M,M_X) \ar[r] & (M, I_0) \ar[r] & (M, I_1) \ar[r] & Y \ar[r] & 0
			\end{tikzcd}
		\end{center}
		is a minimal exact sequence. Since the global dimension of $\End(M)^{\operatorname{op}}$ is at most 2 this means that $(M, \Omega_M X)=0$. Consequently we have that $\Omega_M X = 0$ and that $X=M_X$, so $X$ is in $\add M$. Thus $\Lambda$ is representation finite.
	\end{proof}
\end{prop}

We conclude this subsection by proving that $\findim(\Lambda)$ is finite when $\Lambda$ has representation dimension at most 3. To do this we first prove a slight generalization of this.

\begin{theorem}\cite[Corollary~8]{IgTo05}
	If $\Lambda = \End_\Gamma(P)^{\op}$ for an algebra $\Gamma$ with global dimension at most 3, and $P$ projective, then $\findim(\Lambda) < \infty$.
	\begin{proof}
		Let $X$ be any $\Lambda$-module with finite projective dimension. Then it has a projective presentation $(P, P_1) \to (P,P_0) \to X \to 0$ where $(P,P_i)=\Hom_\Gamma(P,P_i)$ with $P_i \in \add P$. Since $(P,-)$ is an equivalence from $\add P$ to $\proj\Lambda$, this corresponds to a map $P_1 \to P_0$ which we can extend to a projective resolution in $\mod\Gamma$:
		\begin{center}
			\begin{tikzcd}
			0 \ar[r] & P_3 \ar[r] & P_2 \ar[r] & P_1 \ar[r] & P_0.
			\end{tikzcd}
		\end{center}
		Applying the exact functor $(P, -)$, we get an exact sequence
		\begin{center}
			\begin{tikzcd}
			0 \ar[r] & (P,P_3) \ar[r] & (P,P_2) \ar[r] & (P,P_1) \ar[r] & (P,P_0)\ar[r] & X \ar[r] & 0.
			\end{tikzcd}
		\end{center}
		Truncating this we get a short exact sequence
		\begin{center}
			\begin{tikzcd}
			0 \ar[r] & (P, P_3) \ar[r] & (P, P_2) \ar[r] & \Omega^2 X \ar[r] & 0.
			\end{tikzcd}
		\end{center}
		Then by \cref{thm:projdim_bounded_by_psi} the projective dimension of $\Omega^2 X$ has an upper bound given by $\psi((P, P_3)\oplus (P, P_2))+1$. Which means that
		$$\pd X \leq \psi((P, P_3)\oplus (P, P_2))+3 \leq \psi((P,\Gamma))+3.$$
		Since this bound doesn't depend on $X$, we have $\findim(\Lambda) < \infty$.
	\end{proof} 
\end{theorem}

\begin{cor}\label{cor:repdim_less_than_3_implies FDC}
	If $\repdim(\Lambda) \leq 3$, then $\findim(\Lambda) < \infty$.
	\begin{proof}
		If $\Lambda$ has rep-dimension less than or equal to 3, then there is a generator-cogenerator $M$ in $\mod\Lambda$ such that $\Gamma := \End_\Lambda(M)^{\op}$ has global dimension 3 or less. Then, since $M$ is a generator, $\Lambda$ is in $\add M$, and so $\Hom_\Lambda(M, \Lambda)$ is a projective $\Gamma$-module with 
		$$\End_\Gamma(\Hom_\Lambda(M, \Lambda))^{\op} = \End_\Lambda(\Lambda)^{\op} = \Lambda.$$
	\end{proof}
\end{cor}
