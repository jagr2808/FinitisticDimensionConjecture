\subsection{Special biserial algebras}\label{sec:special_biserial_algebras}

In this section we consider two finite dimensional algebras, with a homomorphism between them. We denote these by $\Lambda$ and $\Gamma$, and we denote their radicals by $J_\Lambda$ and $J_\Gamma$ respectively.

The goal of the section is to show that special biserial algebras have representation dimension less than or equal to 3, and consequently that they have finite finitistic dimension. We do this in several parts. In \cref{thm:radical_embedding_repdim_3} we show that an algebra that has a radical embedding into a representation finite algebra has representation dimension at most 3. In \cref{thm:mod_out_socle} and \cref{prop:special_biserail_algerbas_are_binomial} we show that for every special biserial algebra there is a string algebra with larger representation dimension. Lastly in \cref{thm:special_biserial_algebra_repdim_3} we construct a radical embedding of any string algebra into a representation finite algebra.

First we discuss some general properties of homomorphisms of algebras.

\begin{defn}[Coinduced module]
	Given a homomorphism of algebras $\psi\colon\Lambda \to \Gamma$ we can consider every $\Gamma$-module as a $\Lambda$-module, where multiplication by $\lambda$ is given by multiplication with $\psi(\lambda)$. This defines a functor $\mod\Gamma \to \mod\Lambda$ known as \emph{restriction of scalars}. The right adjoint to this functor is called the \emph{coinduction functor}. For a $\Lambda$-module $M$ the coinduced module is the $\Gamma$-module defined as
	$$M' := \Hom_\Lambda(\Gamma, M)$$
	where we consider $\Gamma$ as a $\Lambda$-$\Gamma$-bimodule through restriction of scalars. If we identify $M$ with $\Hom_\Lambda(\Lambda, M)$ then the counit of the adjunction is given by precomposing with $\psi$. Specifically we get the map
	\begin{center}
		\begin{tikzcd}
			M'\ar[r, "\varepsilon_M"] & M\\
			f \ar[r, mapsto] & f(\psi(1)) = f(1).
		\end{tikzcd}
	\end{center}
\end{defn}

\begin{prop}\cite[Lemma~2.2]{EHIS04}\label{prop:coinduction_right_adjoint}
	The coinduced functor as defined above is the right adjoint to restriction of scalars, and $\varepsilon$ is the counit.
	\begin{proof}
		Let $M$ be a $\Lambda$-module and let $N$ be a $\Gamma$-module. Then we get an isomoprhism from the Hom-Tensor adjunction 
		$$\Hom_\Gamma(N, \Hom_\Lambda(\Gamma, M)) \cong \Hom_\Lambda(\Gamma\otimes_\Gamma N, M).$$
		Notice that $_\Lambda\Gamma\otimes_\Gamma N \cong {{}_\Lambda N}$ is exactly restriction of scalars. Further the counit $\Gamma\otimes_\Gamma M' = M' \to M$ is given by $f\mapsto f(1)$, which is exactly how we defined $\varepsilon$ above.
	\end{proof}
\end{prop}

Next, in preperation for \cref{thm:radical_embedding_repdim_3}, we restrict to the case where $\psi$ is the inclusion of a radical embbeding.

\begin{defn}[Radical embedding]
	A subalgebra $\Lambda \subseteq \Gamma$ is called a \emph{radical embedding} if the two radicals coincide, $J_\Lambda = J_\Gamma$.
\end{defn}

\begin{lemma}\cite[Lemma~2.3]{EHIS04}\label{lem:epsilon_semisimple_(co)kernel}
	If $\Lambda \subseteq \Gamma$ is a radical embedding, then $\ker \varepsilon_M$ and $\cok\varepsilon_M$ are both semisimple for any $\Lambda$-module $M$.
	\begin{proof}
		If we apply $\Hom_\Lambda(-, M)$ to the short exact sequence of $\Lambda$-modules \begin{tikzcd}
			0 \ar[r] & \Lambda \ar[r, "\psi"] & \Gamma \ar[r] & \Gamma/\Lambda \ar[r] & 0,
		\end{tikzcd} we get 
		\begin{center}
			\begin{tikzcd}
				0 \ar[r] & \Hom(\Gamma/\Lambda, M) \ar[r] & M' \ar[r, "\varepsilon_M"] & M \ar[dr, two heads] \ar[r] & \Ext^1(\Gamma/\Lambda, M)\\
				  &&    && \cok\varepsilon_M \ar[u, hookrightarrow]
			\end{tikzcd}
		\end{center}
		Thus $\Hom(\Gamma/\Lambda, M)$ is the kernel of $\varepsilon_M$ and the cokernel is a submodule of $\Ext^1(\Gamma/\Lambda, M)$. Since $J_\Gamma = J_\Lambda \subseteq \Lambda$ we have that $(\Gamma/\Lambda)J_\Lambda = 0$. Thus $J_\Lambda\Hom(\Gamma/\Lambda, M)$ and $J_\Lambda\Ext^1(\Gamma/\Lambda, M)$ are both 0, which means they are both semisimple. Since $\cok\varepsilon_M$ is a submodule of $\Ext^1(\Gamma/\Lambda, M)$, it is also semisimple.
	\end{proof}
\end{lemma}

We now use the radical embedding to say something about the representation dimension of $\Lambda$.

\begin{theorem}\cite[Theorem~1.1]{EHIS04}\label{thm:radical_embedding_repdim_3}
	If $\Gamma$ is representation finite and $\Lambda \subseteq \Gamma$ is a radical embedding, then the representation dimension of $\Lambda$ is at most 3.
	\begin{proof}
		Since $\Gamma$ is representation finite there is a finite set of indecomposable $\Gamma$-modules up to isomorphism. Let $X$ be the direct sum of all of these. Since $\Lambda$ is a subalgebra of $\Gamma$ we can consider $X$ as a $\Lambda$-module. Now define $V$ to be $\Lambda \oplus D\Lambda \oplus X$, i.e. $V$ is the sum of all projective $\Lambda$-modules, all injective $\Lambda$-modules, and all $\Gamma$-modules. We claim that $V$-$\operatorname{res-dim}(\Lambda) \leq 1$, which by \cref{prop:repdim_resdim+2} would imply that $\repdim(\Lambda) \leq 3$.
		
		As in \cref{thm:stably_hereditary_repdim_3} we do this by showing that for any  $\Lambda$-module $M$ there is a short exact sequence
		\begin{center}
			\begin{tikzcd}
			0 \ar[r] & V_1 \ar[r] & V_0 \ar[r] & M \ar[r] & 0
			\end{tikzcd}
		\end{center}
		with $V_i$ in $\add V$, such that 
		\begin{center}
			\begin{tikzcd}
			0 \ar[r] & (V,V_1) \ar[r] & (V,V_0) \ar[r] & (V,M) \ar[r] & 0
			\end{tikzcd}
		\end{center}
		is exact. 
		
		Now let $M$ be any $\Lambda$-module. If $M$ is injective, then $M$ is in $\add V$, and so we may simply choose $V_0 = M$ and $V_1=0$. From here on out assume that $M$ has no injective summands. 
		
		Let $M'$ be the coinduced module of $M$, and $\varepsilon_M\colon M' \to M$ be the counit map. Now if we let $P$ be the projective cover of $\cok \varepsilon_M$, then by lifting the map $P \to \cok\varepsilon_M$ we get a surjective map $M' \oplus P \to M$. Since $M'$ is a $\Gamma$-module and $P$ is projective $M'\oplus P$ is in $\add V$. We let this be our $V_0$.
		
		Next, we let $V_1$ be the kernel of the map $V_0 \to M$. Then we wish to show that this is in $\add V$. Since $M \to \cok\varepsilon_M$ is an epimorphism and $P \to \cok\varepsilon_M$ is a projective cover, we can lift this to a morphism $P \to M$. Taking the pullback along $\Image \varepsilon_M \to M$ we get a commutative diagram:
		\begin{center}
		\begin{tikzcd}
			0 \ar[r] & K \ar[r]\ar[d]\arrow[dr, phantom, "\usebox\pullback" , very near start, color=black] & P \ar[r]\ar[d] & \cok\varepsilon_M \ar[r] \ar[d, equal] & 0\\
			0 \ar[r] & \Image \varepsilon_M \ar[r] & M \ar[r] & \cok\varepsilon_M \ar[r] & 0
		\end{tikzcd}
		\end{center} 
		By \cref{lem:epsilon_semisimple_(co)kernel} we have that $ \cok\varepsilon_M $ is semisimple, and thus $K = J_\Lambda P$. Since $J_\Lambda=J_\Gamma$ this means that $J_\Lambda P$ is a $\Gamma$-module, and thus is in $\add V$. Next we take the pullback again, this time along $M' \to \Image \varepsilon_M$. 
		\begin{center}
			\begin{tikzcd}
			0\ar[r]& \ker\varepsilon_M \ar[d, equal]\ar[r] & M'\prod\limits_M J_\Lambda P\arrow[dr, phantom, "\usebox\pullback" , very near start, color=black]\ar[r]\ar[d] & J_\Lambda P\ar[r]\ar[d] & 0\\
			0\ar[r]& \ker\varepsilon_M \ar[r]& M'\ar[r] & \Image \varepsilon_M\ar[r] &  0
			\end{tikzcd}
		\end{center} 
		Notice that $M'\prod\limits_M J_\Lambda P = M'\prod\limits_M P$, which is the kernel of $V_0 \to M$. In other words it is equal to $V_1$.
		
		Since $J_\Lambda P$ is a $\Gamma$-module we get a map of abelian groups by postcomposing with $\varepsilon_M$:
		\begin{center}
			\begin{tikzcd}
			\Hom_\Gamma(J_\Lambda P, M') \ar[r, "\varepsilon_M \circ -"] & \Hom_\Lambda(J_\Lambda P, M)\\
			f \ar[r, mapsto]& (p \mapsto f(p)(1))
			\end{tikzcd}
		\end{center} 
		This is excatly the isomoprhism of the adjuntion between restriction of scalars and the coinduction functor in \cref{prop:coinduction_right_adjoint}.

		In other words the map $J_\Lambda P \to \Image \varepsilon_M$ factorizes through $M'$. Then using the pullback property, we get that the map $V_1 \to J_\Lambda P$ splits, and so $V_1 = \ker\varepsilon_M \oplus J_\Lambda P$. 
		
		We have already established that $J_\Lambda P$ is a $\Gamma$-module. By \cref{lem:epsilon_semisimple_(co)kernel} we have that $\ker\varepsilon_M$ is semisimple. We now show that $\ker\varepsilon_M$ is in $\add V$, by showing that all simple modules are.
		
		Let $S$ be a simple $\Lambda$-module, and let $e$ be an idempotent such that $S \cong \Lambda e / J_\Lambda e$. We have a semisimple $\Gamma$-module $\hat{S}:=\Gamma e / J_\Gamma e$ that contains $S$. Since $J_\Gamma = J_\Lambda$ we have that $\hat{S}$ is also semisimple as a $\Lambda$-module. Thus $S$ is a direct summand of $\hat{S}$. Since $\hat{S}$ is in $\add V$, we get that $S$ is as well. Thus $V_1$ is in $\add V$.
		
		Lastly we show that we get an exact sequence
		\begin{center}
			\begin{tikzcd}
			0\ar[r] & (V, V_1) \ar[r] & (V, V_2) \ar[r] & (V, M) \ar[r] & 0.
			\end{tikzcd}
		\end{center} 
		The only thing we need to show is that the right map is surjective. We do this by verifying the three cases for an indecomposable summand of $V$. Firstly let $W$ be a $\Gamma$-module. Then $\Hom_\Lambda(W, V_2)$ breaks up as a direct sum into $\Hom_\Lambda(W, M') \oplus \Hom_\Lambda(W, P)$. We saw in \cref{prop:coinduction_right_adjoint} that the composition
		\begin{tikzcd}[column sep=10pt]
			\Hom_\Gamma(W, M') \ar[r, "\subseteq"] & \Hom_\Lambda(W, M') \ar[r] & \Hom_\Lambda(W, M)
		\end{tikzcd}
		is an isomorphism. Thus the map \begin{tikzcd}[column sep=10pt]
			\Hom_\Gamma(W, M') \ar[r] & \Hom_\Lambda(W, M)
		\end{tikzcd} is surjective.
		
		If $W$ is projective, then $\Hom_\Lambda(W, -)$ is exact, and there is nothing we need to show.
		
		Since we assumed $M$ had no injective summands, if $W$ is an indecomposable injective, then a map $W \to M$ cannot be injective. This means that it factors through $W/\socle(W)$. Since $D(W/\socle(W)) = (DW)J_\Lambda = (DW)J_\Gamma$ this means that $W/\socle(W)$ is a $\Gamma$-module. Then from the argument above it follows that the map is surjective.
		
		This shows that $V$-$\operatorname{res-dim}(\Lambda) \leq 1$, and thus the representation dimension of $\Lambda$ is at most 3.
	\end{proof}
\end{theorem}

Now we move away from the case where $\psi$ is a radical embedding, and instead look at a specific quotient map.

\begin{theorem}\cite[Proposition~1.2]{EHIS04}\label{thm:mod_out_socle}
	Let $\Lambda$ be a basic finite dimensional algebra and let $P$ be a basic projective-injective $\Lambda$-module. Then the socle of $P$ is a two-sided ideal, which allows us to define the ring $\Gamma := \Lambda / \socle P$. Then we have that $\repdim(\Lambda) \leq \max\{2, \repdim(\Gamma)\}$. 
	\begin{proof}
		First we show that the socle of $P$ is a two-sided ideal. Multiplication on the right defines a homomorphism $-\cdot \lambda\colon \Lambda \to \Lambda$. Any homomorphism maps the socle to the socle, so $(\socle P) \cdot \lambda \subseteq \socle \Lambda$. Now let $s \in \socle P$ be some element such that $s\lambda$ is non-zero. Then the injective envelope $I(s)$ is a direct summand of $P$ and thus projective-injective. Further since $-\cdot \lambda\colon (s) \to (s\lambda)$ is an injective map, $I(s)$ is mapped injectively into $\Lambda$ by $-\cdot \lambda$, which means $-\cdot\lambda\colon I(s) \to \Lambda$ splits. Since $\Lambda$ is basic this means that $I(s)\lambda \subseteq P$, and thus $s\lambda \in \socle P$, so the socle of $P$ is a two-sided ideal.
		
		Next we note that any indecomposable $\Lambda$-module is either a $\Gamma$-module, or a direct summand of $P$. To see this, let $M$ be any indecomposable $\Lambda$-module and consider $(\socle P)M$. If this is zero, then $M$ is a $\Gamma$-module. If on the other hand there is some $s \in \socle P$ and $m \in M$ such that $sm \neq 0$, then let $I(s)$ be the injective envelope of $s$ and let $e$ be the idempotent such that $I(s) = \Lambda e$. Then we get a map $I(s) \to M$ which maps $\lambda e$ to $\lambda e m$. Since $sm \neq 0$ this maps the socle of $I(s)$ injectively. Now, since $I(s)$ is injective this mean that $I(s)$ is a direct summand of $M$. Since $M$ is indecomposable we have that $M \cong I(s)$, and thus $M$ is a direct summand of $P$.
		
		Now we show that $\repdim(\Lambda) \leq \max\{2, \repdim(\Gamma)\}$. By \cref{prop:repdim_resdim+2} it suffices to find a generator-cogenerator $V$ that satisfies the inequality $V$-$\operatorname{res-dim}(\mod \Lambda) \leq \max\{0, \repdim(\Gamma)-2\}$. Let $N$ be the generator-cogenerator in $\mod\Gamma$ that achieves the minimal resolution dimension. Then we claim $V = N \oplus P$ is our desired generator-cogenerator. This is a generator-cogenerator because any indecomposable projective or injective module that is not a summand of $P$ will be  a summand of $N$, since all $\Lambda$-modules that are not summands of $P$ are $\Gamma$-modules.
		
		To show that $V$-$\operatorname{res-dim}(\mod \Lambda) \leq \max\{0, \repdim(\Gamma)-2\}$ we explicitly construct the resolutions. Let $M$ be an indecomposable $\Lambda$-module. Then we wish to construct an exact sequence
		\begin{center}
		\begin{tikzcd}
			0 \ar[r] & V_n \ar[r] & \cdots \ar[r] & V_1 \ar[r] & V_0 \ar[r] & M \ar[r] & 0
		\end{tikzcd}
		\end{center} 
		such that $V_i$ is in $\add V$, $n\leq \max\{0, \repdim(\Gamma)-2\}$, and $\Hom(V, -)$ is exact on the sequence. If $M$ is a summand of $P$ we may choose $V_0=M$ and $V_i=0$ for $i>0$.
		
		If $M$ is not a summand of $P$ then $M$ is a $\Gamma$-module. Then we already have an exact sequence
		\begin{center}
			\begin{tikzcd}
				0 \ar[r] & N_n \ar[r] & \cdots \ar[r] & N_1 \ar[r] & N_0 \ar[r] & M \ar[r] & 0
			\end{tikzcd}
		\end{center} 
		with $N_i \in \add N$. Since $\Lambda \to \Gamma$ is surjective we get that $\Hom_\Lambda(N, -)=\Hom_\Gamma(N,-)$ on $\Gamma$-modules. So if we apply $\Hom_\Lambda(N, -)$ to the sequence it remains exact. Lastly since $\Hom(V,-) = \Hom(N, -) \oplus \Hom(P, -)$ and $\Hom(P, -)$ is an exact functor, if we apply $\Hom(V, -)$ to the sequences it still remains exact. Thus $V$-$\operatorname{res-dim}(\mod \Lambda) \leq \max\{0, \repdim(\Gamma)-2\}$ and $\repdim(\Lambda) \leq \max\{2, \repdim(\Gamma)\}$. 
	\end{proof}
\end{theorem}

We now give the definition of special biserial algebras, and string algebras.

\begin{defn}[Special biserial algebra]
	A finite dimensional algebra $\Lambda$ is called \emph{special biserial} if it is isomorphic to a path algebra $kQ/I$ such that
	\begin{enumerate}[i)]
		\item Each vertex in $Q$ is the initial vertex for at most two arrows, and the terminal vertex for at most two arrows.
		\item For any arrow $\beta$ in $Q$ there is at most one arrow $\alpha$ such that $\alpha\beta \not\in I$ and at most one arrow $\gamma$ such that $\beta\gamma \not\in I$.
	\end{enumerate}
	A special biserial algebra is called a \emph{string algebra} if it is also monomial, i.e. if $I$ is generated by paths. 
\end{defn}

We now show that given a special biserial algebra we can always construct a string algebra, by moding out socles of projective injective modules like in \cref{thm:mod_out_socle}.

\begin{prop}\label{prop:special_biserail_algerbas_are_binomial}
	If $\Lambda = kQ/I$ is special biserial, then $I$ is generated by monomial and binomial relations. Further if $\gamma + t\gamma'$ is a binomial relation such that $\gamma \not\in I$, then $(\gamma)$ is the socle of a projective-injective module.
	\begin{proof}
		Let $\rho$ be a relation. Then we may assume $\rho$ is some linear combinations of paths which start in the same vertex and end in the same vertex. Assume by induction that $\rho$ is a combination of $n$ distinct paths for some $n\geq 3$, and let $\gamma^1$, $\gamma^2$, and $\gamma^3$ be three of those paths. Write each path as a composition of arrows $\gamma^1 = \alpha^1_{t_1} \cdots \alpha^1_1\alpha^1_0$,  $\gamma^2 = \alpha^2_{t_2} \cdots \alpha^2_1\alpha^2_0$, and  $\gamma^3 = \alpha^3_{t_3} \cdots \alpha^3_1\alpha^3_0$.
		
		Since there can be at most two arrows out of any vertex, it cannot be the case that $\alpha^1_0$, $\alpha^2_0$, and $\alpha^3_0$ are all distinct. Let us assume $\alpha^1_0 = \alpha^2_0$. Since we assume $\gamma^1$ and $\gamma^2$ are distinct there must be a smallest $k$ such that $\alpha^1_k \neq \alpha^2_k$. But then it must be the case that either $\alpha^1_k\alpha^1_{k-1}$ or $\alpha^2_k\alpha^1_{k-1}$ is a relation. That means that either $\gamma^1$ or $\gamma^2$ is a relation. Thus $\rho$ is the sum of a monomial relation and a relation that is the linear combination of $(n-1)$ paths. Then by induction each relation in $I$ is the sum of binomial relations.
		
		Now let $\gamma + t\gamma'$ be a binomial relation such that $\gamma \not\in I$. Let $i$ be the origin vertex of $\gamma$, let $j$ be the terminal vertex, and let $e_i$ and $e_j$ be the corresponding idempotents. Then we claim that $\Lambda e_i$ is projective-injective, and that $(\gamma)$ is its socle.
		
		As above decompose the two paths into a product of arrows $\gamma = \alpha_{t}\cdots \alpha_1\alpha_0$ and $\gamma' = \alpha'_{t'}\cdots \alpha_1\alpha_0$, and let $k$ be the smallest integer such that $\alpha_k \neq \alpha'_k$. If $k$ is bigger than 0, then as before we get that either $\alpha_k\alpha_{k-1}$ or $\alpha'_k\alpha_{k-1}$ is a relation. Consequently both $\gamma$ and $\gamma'$ would be relations contradicting our assumption. Similarly if we let $k$ be the smallest integer such that $\alpha_{t-k} \neq \alpha'_{t'-k}$ we get that $k$ cannot be bigger than 0, by exactly the same argument. This means that $\alpha_0 \neq \alpha'_0$ and that $\alpha_t \neq \alpha'_{t'}$, which will be important later.

		We show that $(\gamma)$ is simple, by showing that $\alpha\gamma$ is a relation for every arrow $\alpha$. We have that $\alpha(\gamma + t\gamma')$ is a relation. Since $\alpha_t \neq \alpha'_{t'}$ we have that either $\alpha\alpha_t = 0$ or $\alpha\alpha'_{t'} = 0$. If $\alpha\alpha_t=0$, then $\alpha\gamma=0$ and we are done. If $\alpha\alpha'_{t'} = 0$, then $\alpha\gamma'=0$ which means that $\alpha\gamma = \alpha(\gamma + t\gamma') - t\alpha\gamma'$ is as well. So $(\gamma)$ is simple and hence in the socle of $\Lambda e_i$.
		
		By exactly the same argument as above, any path in $\Lambda e_i$ is an initial subpath of either $\gamma$ or $\gamma'$. This gives us that $\socle \Lambda e_i = (\gamma)$.

		Lastly we need to show that $\Lambda e_i$ is injective. We can do this by constructing an isomorphism $\varphi \colon \Lambda e_i \to D(e_j \Lambda)$. We define the map by $\varphi(e_i) = \gamma^*$. By the same argument as before $(\gamma)$ is the socle of $e_j\Lambda$ as right modules. Thus $\gamma^*$ generates the top of $D(e_j \Lambda)$, and $\varphi$ is surjective. Since $\varphi(\gamma) = e_j^*$ and $(\gamma)$ is the socle of $\Lambda e_i$ we have that $\varphi$ is injective, and so it is an isomorphism.

		Hence $\Lambda e_i$ is projective-injective, and so $(\gamma)$ is the socle of a projective-injective module. 
	\end{proof} 
\end{prop}

This explains where the name \emph{special biserial} comes from; the radical of each indecomposable projective of a special biserial algebra is biserial. I.e. it is the sum of two uniserial modules. In fact for an indecomposable projective $P$, either $P$ is uniserial or $JP/\socle P$ is the direct sum of two uniserial modules, as visualized in \cref{fig:biserial_projectives}.

\begin{figure}
	\centering
\setlength{\tabcolsep}{30pt}
\begin{tabular}{ccc}
	\begin{tikzcd}
	\bullet\ar[d]\\
	\bullet\ar[d]\\
	\vdots\ar[d]\\
	\bullet\ar[d]\\
	\bullet
	\end{tikzcd}
	&
	\begin{tikzcd}[ampersand replacement=\&, column sep = 10pt]
	\&\bullet\ar[dl]\ar[dr]\\
	\bullet\ar[d] \&\& \bullet\ar[d]\\
	\vdots\ar[d] \&\& \vdots\ar[d]\\
	\bullet \&\& \bullet\ar[d]\\
	\&\&\bullet
	\end{tikzcd}
	&
	\begin{tikzcd}[ampersand replacement=\&, column sep = 10pt]
	\&\bullet\ar[dl]\ar[dr]\\
	\bullet\ar[d] \&\& \bullet\ar[d]\\
	\vdots\ar[d] \&\& \vdots\ar[d]\\
	\bullet\ar[dr] \&\& \bullet\ar[dl]\\
	\&\bullet
	\end{tikzcd}
\end{tabular}

\caption{The possible shapes for an indecomposable projective module.}\label{fig:biserial_projectives}
\end{figure}

Combining \cref{thm:mod_out_socle} and \cref{prop:special_biserail_algerbas_are_binomial} we can reduce the problem of computing the representation dimension of a special biserial algebra to string algebras, by modding out all binomial relations. We now do this to show that special biserial algebras have representation dimension at most 3.

\begin{theorem}\cite[Corollary~1.3]{EHIS04}\label{thm:special_biserial_algebra_repdim_3}
	If $\Lambda = kQ/I$ is a special biserial algebra, then $\repdim(\Lambda) \leq 3$, and thus $\findim(\Lambda) < \infty$.
	\begin{proof}
		By \cref{thm:mod_out_socle} we may assume $\Lambda$ is a string algebra. If we can construct a radical embedding of $\Lambda$ into a representation finite algebra, then by \cref{thm:radical_embedding_repdim_3} our result would follow.

		For any vertex $l \in Q$ define $E(l)$ to be the number of arrows ending in $l$ and $S(l)$ the number of arrows starting in $l$. Define $c(\Lambda)$ to be the sum of the number of vertices $l$ with $E(l) \geq 2$ and the number of vertices $k$ with $S(k) \geq 2$. The proof goes by induction on $c(\Lambda)$.

		If $c(\Lambda)=0$, then $Q$ is the disjoint union of linearly oriented quivers of type $\mathbb{A}$ and cyclically oriented quivers of type $\tilde{\mathbb{A}}$. Finite dimensional algebras arising from such quivers are well known to be representation finite (c.f. \cite[Chapter~VI.2]{ARS97} or \cite[Chapter~V.3]{ASS06}), and so the identity map on $\Lambda$ is a radical embedding into an algebra of finite representation type.
		\begin{center}
			\setlength{\tabcolsep}{30pt}
			\begin{tabular}{cc}
				\begin{tikzcd}[ampersand replacement=\&]
					1\ar[r] \& {} \ar[r, dashed, no head] \& {} \ar[r] \& n
				\end{tikzcd}
				&
				\begin{tikzcd}[ampersand replacement=\&, row sep= 15pt, column sep = 15pt]
					\&0 \ar[r] \& 1 \ar[dr]\\
					n \ar[ur] \&\&\& 2 \ar[dl]\\
					\& \phantom{3} \ar[ul] \&  \phantom{3} \ar[l, dashed, no head]
				\end{tikzcd}\\
				&\\
				Linearly oriented 
				&
				Cyclically oriented\\
				quiver of type $\mathbb{A}_n$
				&
				quiver of type $\tilde{\mathbb{A}}_n$
			\end{tabular}
		\end{center}
		If $c(\Lambda) = n \geq 1$, then there is a vertex $l$ with either $E(l)=2$ or $S(l)=2$. We now construct a new string algebra $\Gamma$ and a radical embedding $\Lambda \to \Gamma$ such that $c(\Gamma) \leq n-1$.

		The two cases are completely symmetric, so we only show the case $E(l)=2$ here. Let $\alpha_1$ and $\alpha_2$ be the two arrows ending in $l$. Define the quiver $Q'$ to have the same vertices as $Q$, except we replace $l$ by two vertices $l_1$ and $l_2$. The arrows of $Q'$ are exactly the same, except now $\alpha_1$ ends in $l_1$ and $\alpha_2$ ends in $l_2$. For any arrow $\beta \in Q$ that starts in $l$, the corresponding arrow in $Q'$ starts in $l_1$ if and only if $\beta\alpha_1$ is not a relation.

		We may consider $I$ as an ideal in $kQ'$ simply by setting paths to 0 if they are no longer defined in $Q'$. Then $\Gamma := kQ'/I$ is a string algebra, and the map $\Lambda \to \Gamma$ that sends $e_l$ to $e_{l_1}+e_{l_2}$ and all other paths to themselves is a radical embedding.

		For each vertex $k \neq l$, we have $E_\Lambda(k) = E_\Gamma(k)$ and $S_\Lambda(k) = S_\Gamma(k)$. We also have $E_\Lambda(l) = 2$,  $E_\Gamma(l_1) = E_\Gamma(l_2) = 1$, and $S_\Lambda(l) = S_\Gamma(l_1)+S_\Gamma(l_2)$. Since $S_\Lambda(l) \leq 2$ it follows that $c(\Gamma) \leq n-1$.

		By induction there is a radical embedding of $\Lambda$ into an algebra $\Gamma$ with $c(\Gamma)=0$, which is representation finite. Then by \cref{thm:radical_embedding_repdim_3} we get that $\repdim(\Lambda) \leq 3$, and by \cref{cor:repdim_less_than_3_implies FDC} we have $\findim(\Lambda) < \infty$.
	\end{proof}
\end{theorem}