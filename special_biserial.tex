\cite{EHIS04}

In this section we shall consider two finite dimensional algebras, where one is the subalgebra of another. We will denote these by $\Lambda$ and $\Gamma$, and we will denote their radicals by $J_\Lambda$ and $J_\Gamma$ respectively.

\begin{defn}[Radical embedding]
	A subalgebra $\Lambda \subseteq \Gamma$ is called a \emph{radical embedding} if the two radicals coincide, $J_\Lambda = J_\Gamma$.
\end{defn}

\begin{lemma}\cite[Lemma~2.3]{EHIS04}\label{lem:epsilon_semisimple_(co)kernel}
	A homomorphism of algebras $\Lambda \to \Gamma$ induces a homomorphism of $\Lambda$-modules $\varepsilon_M\colon \Hom_\Lambda(\Gamma, M) \to \Hom_\Lambda(\Lambda, M)=M$, by considering $\Gamma$ as a $\Lambda$ module. If $\Lambda \subseteq \Gamma$ is a radical embedding, then $\ker \varepsilon_M$ and $\Image \varepsilon_M$ are both semisimple for any $\Lambda$-module $M$.
	\begin{proof}
		content...
	\end{proof}
\end{lemma}

\begin{theorem}
	If $\Gamma$ is representation finite and $\Lambda \subseteq \Gamma$ is a radical embedding, then the representation dimension of $\Lambda$ is at most 3.
	\begin{proof}
		Since $\Gamma$ is representation finite there is a finite set of indecomposable $\Gamma$-modules up to isomorphism. Let $X$ be the direct sum of all of these. Since $\Lambda$ is a subalgebra of $\Gamma$ we can consider $X$ as a $\Lambda$-module. Now define $V$ to be $\Lambda \oplus D\Lambda \oplus X$, i.e. $V$ is the sum of all projective $\Lambda$, all injective $\Lambda$ modules, and all $\Gamma$-modules. We claim $\End_\Lambda(V)^{\op}$ has global dimension at most 3.
		
		As in \cref{thm:stably_hereditary_repdim_3} we do this by showing that for any  $\Lambda$-module $M$ there is a short exact sequence
		\begin{center}
			\begin{tikzcd}
			0 \ar[r] & V_3 \ar[r] & V_2 \ar[r] & M \ar[r] & 0
			\end{tikzcd}
		\end{center}
		with $V_i$ in $\add V$, such that 
		\begin{center}
			\begin{tikzcd}
			0 \ar[r] & (V,V_3) \ar[r] & (V,V_2) \ar[r] & (V,M) \ar[r] & 0
			\end{tikzcd}
		\end{center}
		is exact. 
		
		Now let $M$ be any $\Lambda$-module. If $M$ is injective, then $M$ is in $\add V$, and so we may simply choose $V_2 = M$ and $V_3=0$. From here on out assume that $M$ has no injective summands. 
		
		By considering $\Gamma$ as a $\Lambda$-module, we define the $\Gamma$-module $M' := \Hom_\Lambda(\Gamma, M)$. Evaluation at 1 gives us a $\Lambda$-linear map $\varepsilon_M \colon M' \to M$, introduced in \cref{lem:epsilon_semisimple_(co)kernel}. Now if we let $P$ be the projective cover of $\cok \varepsilon_M$ then we get a surjective map $M' \oplus P \to M$. Since $M'$ is a $\Gamma$-module and $P$ is projective $M'\oplus P$ is in $\add V$. We let this be our $V_2$.
		
		Next, we let $V_3$ be the kernel of the map $V_2 \to M$, and we wish to show that this is in $\add V$. Since $M \to \cok\varepsilon_M$ is an epimorphism and $P \to \cok\varepsilon_M$ is a projective cover, we can lift this to a morphism $P \to M$. Taking the pullback along $\Image \varepsilon_M \to M$ we get a commutative diagram:
		\begin{center}
		\begin{tikzcd}
			0 & K & P & \cok\varepsilon_M & 0\\
			0 & \Image \varepsilon_M & M & \cok\varepsilon_M & 0
		\end{tikzcd}
		\end{center} 
		By \cref{lem:epsilon_semisimple_(co)kernel} we have that $ \cok\varepsilon_M $ is semisimple, and thus $K = J_\Lambda P$. Since $J_\Lambda=J_\Gamma$ this means that $J_\Lambda P$ is a $\Gamma$-module, and thus is in $\add V$. Next we take the pullback again, this time along $M^- \to \Image \varepsilon_M$. 
		\begin{center}
			\begin{tikzcd}
			0& \ker\varepsilon_M \ar[d, equal] & M^-\prod\limits_M P & J_\Lambda P & 0\\
			0& \ker\varepsilon_M & M^- & \Image \varepsilon_M &  0
			\end{tikzcd}
		\end{center} 
		Notice that $M^-\prod\limits_M P$ is exactly what we defined to be $V_3$.
		
		Since $J_\Lambda P$ is a $\Gamma$-module we get a map of abelian groups by postcomposing with $\varepsilon_M$:
		\begin{center}
			\begin{tikzcd}
			\Hom_\Gamma(J_\Lambda P, M^-) \ar[r, "\varepsilon_M \circ -"] & \Hom_\Lambda(J_\Lambda P, M)\\
			f \ar[r, mapsto]& (p \mapsto f(p)(1))
			\end{tikzcd}
		\end{center} 
		Remember that $M^-$ is defined as $\Hom_\Lambda(B, M)$. This map is exactly the same as the isomorphism coming from the Hom-Tensor adjunction, thus it is an isomorphism.
		\begin{center}
			\begin{tikzcd}
			\Hom_\Gamma(J_\Lambda P, M^-) \ar[r, "\cong"] & \Hom_\Lambda(\Gamma \otimes_\Gamma J_\Lambda P, M)\\
			f \ar[r, mapsto]& (b\otimes p \mapsto f(p)(b))
			\end{tikzcd}
		\end{center} 
		In other words the map $P \to \Image \varepsilon_M$ factorizes through $M^-$. Then using the pullback property, we get that the map $V_3 \to J_\Lambda P$ splits, and so $V_3 = \ker\varepsilon_M \oplus J_\Lambda P$. We have already established that $J_\Lambda P$ is a $\Gamma$-module, the same is true of $\ker\varepsilon_M$. To see this simply note that $\ker\varepsilon_M = \Hom_\Lambda(\Gamma/\Lambda, M)$ and that $\Gamma/\Lambda$ is a $\Lambda$-$\Gamma$-bimodule. Thus $V_3$ is a $\Gamma$-module, and so it is in $\add V$.
		
		Lastly we show that we get an exact sequence
		\begin{center}
			\begin{tikzcd}
			0 & (V, V_3) & (V, V_2) & (V, M) & 0.
			\end{tikzcd}
		\end{center} 
		The only thing we need to show is that the last map is an epimorphism. We do this by verifying the three cases for an indecomposable summand of $V$. Firstly let $W$ be a $\Gamma$-module. Then $\Hom_\Lambda(W, V_2)$ breaks up as a direct sum into $\Hom_\Lambda(W, M^-) \oplus \Hom_\Lambda(W, P)$ \todo{blabla}
		
		If $W$ is surjective, then $\Hom_\Lambda(W, -)$ is exact, and there is nothing we need to show.
		
		If $W$ is an indecomposable injective, since we assumed $M$ had no injective summands, a map $W \to M$ cannot be injective. This means that it factors through $W/\socle(W)$. Since $D(W/\socle(W)) = (DW)J_\Lambda = (DW)J_\Gamma$ this means that $W/\socle(W)$ is a $\Gamma$-module. Then from the argument above it follows that the map is surjective.
		
		Now just as in \cref{thm:stably_hereditary_repdim_3} if we let $N$ be any module over $\End(V)^{\op}$ then it has a projective presentation
		\begin{center}
			\begin{tikzcd}
			(V,V_1) \ar[r, "f\circ-"] & (V,V_0) \ar[r] & N \ar[r] & 0
			\end{tikzcd}
		\end{center}
		If we let $M$ denote the kernel of $f$ and we choose $V_3$ and $V_2$ as above, then we get a projective resolution of $N$ by
		\begin{center}
			\begin{tikzcd}[column sep=20pt]
			0\ar[r] & (V,V_3) \ar[r] & (V,V_2) \ar[r] & (V,V_1) \ar[r] & (V,V_0) \ar[r] & N \ar[r] & 0.
			\end{tikzcd}
		\end{center}
		This shows that the global dimension of $\End(V)^{\op}$ is at most 3, and likewise the representation dimension of $\Lambda$ is at most 3.
	\end{proof}
\end{theorem}

\begin{theorem}
	Let $\Lambda$ be a basic finite dimensional algebra and let $P$ be a basic projective-injective $\Lambda$-module. Then the socle of $P$ is a two-sided ideal, which allows us to define the ring $\Gamma := \Lambda / \socle P$. Then we have that $\repdim(\Lambda) \leq \max\{2, \repdim(\Gamma)\}$. 
	\begin{proof}
		First we show that the socle of $P$ is a two-sided ideal. Multiplication on the right defines a homomorphism $-\cdot \lambda\colon \Lambda \to \Lambda$. Any homomorphism maps the socle to the socle, so $(\socle P) \cdot \lambda \subseteq \socle \Lambda$. Now let $s \in \socle P$ be some element such that $s\lambda$ is non-zero. Then the injective envelope $I(s)$ is a direct summand of $P$ and thus projective-injective. Further since $-\cdot \lambda\colon (s) \to (s\lambda)$ is an injective map, $I(s)$ is mapped injectively into $\Lambda$ by $-\cdot \lambda$, which means $-\cdot\lambda\colon I(s) \to \Lambda$ splits. Since $\Lambda$ is basic this means that $I(s)\lambda \subseteq P$, and thus $s\lambda \in \socle P$, so the socle of $P$ is a two-sided ideal.
		
		Next we note that any indecomposable $\Lambda$-module is either a $\Gamma$-module, or a direct summand of $P$. To see this, let $M$ be any indecomposable $\Lambda$-module and consider $(\socle P)M$. If this is zero, then $M$ is a $\Gamma$-module. If on the other hand there is some $s \in \socle P$ and $m \in M$ such that $sm \neq 0$, then let $I(s)$ be the injective envelope of $s$ and let $e$ be the idempotent such that $I(s) = \Lambda e$. Then we get a map $I(s) \to M$ which maps $\lambda e$ to $\lambda e m$. Since $sm \neq 0$ this maps the socle of $I(s)$ inejctively. Now, since $I(s)$ is inejctive this mean that $I(s)$ is a direct summand of $M$. Since $M$ is indecomposable we have that $M \cong I(s)$, and thus $M$ is a direct summand of $P$.
		
		Now we show that $\repdim(\Lambda) \leq \max\{2, \repdim(\Gamma)\}$. By \cref{prop:repdim_resdim+2} it suffices to find a generator-cogenerator $V$ such that $V$-$\operatorname{res-dim}(\mod \Lambda) \leq \max\{0, \repdim(\Gamma)-2\}$. Let $N$ be the generator-cogenerator in $\mod\Gamma$ that achieves the minimal resolution dimension. Then we claim $V = N \oplus P$ is our desired generator-cogenerator. This is a generator-cogenerator because any indecomposable projective or injective module that is not a summand of $P$ will be  a summand of $N$, since all $\Lambda$-modules that are not summands of $P$ are $\Gamma$-modules.
		
		To show that $V$-$\operatorname{res-dim}(\mod \Lambda) \leq \max\{0, \repdim(\Gamma)-2\}$ we explicitly construct the resolutions. Let $M$ be an indecomposable $\Lambda$-module. Then we wish to construct an exact sequence
		\begin{center}
		\begin{tikzcd}
			0 & V_n & \cdots & V_1 & V_0 & M & 0
		\end{tikzcd}
		\end{center} 
		such that $V_i$ is in $\add V$, $n\leq \max\{0, \repdim(\Gamma)-2\}$, and $\Hom(V, -)$ is exact on the sequence. If $M$ is a summand of $P$ we may choose $V_0=M$ and $V_i=0$ for $i>0$.
		
		If $M$ is not a summand of $P$ then $M$ is a $\Gamma$-module. Then we already have an exact sequence
		\begin{center}
			\begin{tikzcd}
			0 & N_n & \cdots & N_1 & N_0 & M & 0
			\end{tikzcd}
		\end{center} 
		with $N_i \in \add N$. Since all the modules in the sequence are $\Gamma$-modules and $\Lambda \to \Gamma$ is surjective we get the same whether we apply $\Hom_\Lambda(N, -)$ or $\Hom_\Gamma(N,-)$. Lastly since $\Hom(V,-) = \Hom(N, -) \oplus \Hom(P, -)$ and $\Hom(P, -)$ is an exact functor, we get that applying $\Hom(V, -)$ to the sequences gives us something exact. Thus $V$-$\operatorname{res-dim}(\mod \Lambda) \leq \max\{0, \repdim(\Gamma)-2\}$ and $\repdim(\Lambda) \leq \max\{2, \repdim(\Gamma)\}$. 
	\end{proof}
\end{theorem}

\begin{defn}[Special biserial algebra]
	A finite dimensional algebra $\Lambda$ is called \emph{special biserial} if it is isomorphic to a path algebra $kQ/I$ such that
	\begin{itemize}
		\item Each vertex in $Q$ is the initial vertex for at most two arrows, and the terminal vertex for at most two arrows.
		\item For any arrow $\beta$ in $Q$ there is at most on arrow $\alpha$ such that $\alpha\beta \not\in I$ and at most one arrow $\gamma$ such that $\beta\gamma \not\in I$.
	\end{itemize}
	A special biserial algebra is called a \emph{string algebra} if it is also monomial. I.e. $I$ is generated by paths. 
\end{defn}

\begin{prop}
	If $\Lambda = kQ/I$ is special biserial, then $I$ is generated by monomial and binomial relations. Further if $\gamma + t\gamma'$ is a binomial relation such that $\gamma \not\in I$, then $(\gamma)$ is the socle of a projective-injective module.
	\begin{proof}
		Let $\rho$ be a relation. Then we may assume $\rho$ is some linear combinations of paths which start in the same vertex and end in the same vertex. Assume by induction that $\rho$ is a combination of $n$ distinct paths for some $n\geq 3$, and let $\gamma^1$, $\gamma^2$, and $\gamma^3$ be three of those paths. Write each path as a composition of arrows $\gamma^1 = \alpha^1_{t_1} \cdots \alpha^1_1\alpha^1_0$,  $\gamma^2 = \alpha^2_{t_2} \cdots \alpha^2_1\alpha^2_0$, and  $\gamma^3 = \alpha^3_{t_3} \cdots \alpha^3_1\alpha^3_0$.
		
		Since there can be at most two arrows out of any vertex, it cannot be the case that $\alpha^1_0$, $\alpha^2_0$, and $\alpha^3_0$ are all distinct. Let us assume $\alpha^1_0 = \alpha^2_0$. Since we assume $\gamma^1$ and $\gamma^2$ are distinct there must be a smallest $k$ such that $\alpha^1_k \neq \alpha^2_k$. But then it must be the case that either $\alpha^1_k\alpha^1_{k-1}$ or $\alpha^2_k\alpha^1_{k-1}$ is a relation. That means that either $\gamma^1$ or $\gamma^2$ is a relation. Thus $\rho$ is the sum of a monomial relation and a relation that is the linear combination of $(n-1)$ paths. Then by induction each relation in $I$ is the sum of binomial relations.
		
		Now let $\gamma + t\gamma'$ be a binomial relation such that $\gamma \not\in I$. Let $i$ be the origin vertex of $\gamma$, let $j$ be the terminal vertex, and let $e_i$ and $e_j$ be the corresponding idempotents. Then we claim that $\Lambda e_i$ is projective-injective, and that $(\gamma)$ is its socle.
		
		As above decompose the two paths into a product of arrows $\gamma = \alpha_{t}\cdots \alpha_1\alpha_0$ and $\gamma' = \alpha'_{t'}\cdots \alpha_1\alpha_0$, and let $k$ be the smallest integer such that $\alpha_k \neq \alpha'_k$. If $k$ is bigger than 0, then as before we get that either $\alpha_k\alpha_{k-1}$ or $\alpha'_k\alpha_{k-1}$ is a relation. Consequently both $\gamma$ and $\gamma'$ would be relations contradicting our assumption. Similarly if we let $k$ be the smallest integer such that $\alpha_{t-k} \neq \alpha'_{t'-k}$ we get that $k$ cannot be bigger than 0. \todo{blabla}
		
		$(\gamma)$ simple because $\alpha(\gamma + t\gamma')$ relation + $\alpha\gamma'$ relation gives $\alpha\gamma$ relation
		
		 
	\end{proof} 
\end{prop}

This explains where the name \emph{special biserial} comes from; the radical of each indecomposable projective of a special biserial algebra is biserial. That is, it is the sum of two uniserial modules. In fact for an indecomposable projective $P$, either $P$ is uniserial or $JP/\socle P$ is the direct sum of two uniserial modules.

\begin{center}
\setlength{\tabcolsep}{30pt}
\begin{tabular}{ccc}
	\begin{tikzcd}
	\bullet\ar[d]\\
	\bullet\ar[d]\\
	\vdots\ar[d]\\
	\bullet\ar[d]\\
	\bullet
	\end{tikzcd}
	&
	\begin{tikzcd}[ampersand replacement=\&, column sep = 10pt]
	\&\bullet\ar[dl]\ar[dr]\\
	\bullet\ar[d] \&\& \bullet\ar[d]\\
	\vdots\ar[d] \&\& \vdots\ar[d]\\
	\bullet \&\& \bullet\ar[d]\\
	\&\&\bullet
	\end{tikzcd}
	&
	\begin{tikzcd}[ampersand replacement=\&, column sep = 10pt]
	\&\bullet\ar[dl]\ar[dr]\\
	\bullet\ar[d] \&\& \bullet\ar[d]\\
	\vdots\ar[d] \&\& \vdots\ar[d]\\
	\bullet\ar[dr] \&\& \bullet\ar[dl]\\
	\&\bullet
	\end{tikzcd}
\end{tabular}

Possible shapes for indecomposable projectives
\end{center}