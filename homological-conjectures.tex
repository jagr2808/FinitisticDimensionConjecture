
The finitistic dimension conjecture is part of a larger family of homological conjectures about finite dimensional algebras. In this section we outline some of these conjectures, and show which conjectures imply each other.

All of the conjectures are formulated as a specific property conjectured to hold for all finite dimensional algebras. In \cref{prop:conj_on_individual_algebras} we summarize how these implications work on the level of individual algebras.

\subsection*{Finitistic Dimension Conjecture (FDC)}
\begin{defn}[Finitistic dimension]
	For a finite dimensional algebra $\Lambda$ the \emph{finitistic dimension} of $\Lambda$, denoted $\findim(\Lambda)$ is defined by
	$$\findim(\Lambda) = \{\pd M \mid M \in \mod\Lambda, \pd M < \infty\}.$$
\end{defn}

There is also the analogous definition for $\Mod\Lambda$, which is sometimes called the \emph{big finitistic dimension}, and is denoted $\Findim(\Lambda)$. A natural question to ask, which is sometimes also called the finitistic dimension conjecture is whether $\findim(\Lambda)$ always equals $\Findim(\Lambda)$. This was shown to be false by Zimmermann-Huisgen in 1992 \cite{ZH92}. The conjecture we consider is due to Rosenberg and Zelinsky \cite{Bass60}, and asks about when the finitistic dimension is finite.

\begin{conj}[Finitistic dimension conjecture]
	For a finite dimensional algebra the finitistic dimension is always finite.
	$$\findim(\Lambda) < \infty$$
\end{conj}

\subsection*{Wakamatsu Tilting Conjecture (WTC)}
In 1988 Wakamatsu introduced a generalization of tilting modules, now known as Wakamatsu tilting modules \cite{Wak88}.

\begin{defn}[Wakamatsu tilting]
	Let $T$ be a module in $\mod\Lambda$ for a finite dimensional algebra $\Lambda$. Then $T$ is \emph{Wakamatsu tilting} if
	\begin{enumerate}[i)]
		\item We have that $\Ext^n(T,T)=0$ for all $n >0$.
		\item There is an exact sequence 
		\begin{center}
			\begin{tikzcd}
				\eta\colon 0 \ar[r] & \Lambda \ar[r, "d_{-1}"] & T_0 \ar[r, "d_0"] & T_1 \ar[r, "d_1"] & \cdots
			\end{tikzcd}
		\end{center}
		where $T_i$ is in $\add T$.
		\item The sequence $\Hom(\eta, T)$ is exact. Which is equivalent to the condition that $\Ext^1(\Ker d_i, T)=0$ for every differential $d_i$ in $\eta$.
	\end{enumerate}
\end{defn}

The definition is distinct from the definition of a tilting module in two key ways: the projective dimension of $T$ is not assumed to be finite, and $\eta$ is not assumed to be bounded. The Wakamatsu tilting conjecture states that this last condition is unnecessary.

\begin{conj}[Wakamatsu tilting conjecture] 
	If $T$ is Wakamatsu tilting and has finite projective dimension, then $T$ is a tilting module. In other words we can choose $\eta$ to be bounded.
\end{conj}

\subsection*{Gorenstein Symmetry Conjecture (GSC)}

\begin{defn}[Gorenstein algebra]
	A finite dimensional algebra is said to be Gorenstein if all projective modules have finite injective dimension and all injective modules have finite projective dimension.
\end{defn}

The Gorenstein symmetry conjecture says that we only need one of the two conditions for our algebra to be Gorenstein.

\begin{conj}[Gorenstein symmetry conjecture] 
	If $\Lambda$ is a finite dimensional algebra the injective dimension of $\Lambda$ as a left module is finite if and only if the projective dimension of $D\Lambda_\Lambda$ is finite.
\end{conj}

The conjecture describes a sort of symmetry between $\Lambda$ and $\Lambda^{\op}$. An equivalent formulation would be that $\Lambda$ has finite injective dimension as a left module if and only if it has finite injective dimension as a right module.

%$X \to I_0 \to \cdots I_n$ implies $\pd X \leq \max \pd I_i$

\subsection*{Vanishing Conjecture (VC)}
%If $\Lambda$ is a finite dimensional algebra we denote by $K^b(\inj \Lambda)$ the homotopy category of bounded complexes of injective modules. The category $K^{+,b}(\inj\Lambda)$ is the homotopy category of complexes of injectives that are bounded below, and bounded in homology. There is an equivalence of categories between $K^{+,b}(\inj\Lambda)$ and the bounded derived category $\D^b(\Lambda)$. This allows us to consider $K^b(\inj\Lambda)$ as a subcategory of $\D^b(\Lambda)$. Using this we define the perpendicular subcategory

We remind the reader that when $\Lambda$ is a finite dimensional algebra, we have an equivalence of categories between $K^{+,b}(\inj\Lambda)$ and the bounded derived category $\D^b(\Lambda)$ given by injective resolutions. This allows us to consider $K^b(\inj\Lambda)$ as a subcategory of $\D^b(\Lambda)$. Using this we define the perpendicular subcategory

$$K^b(\inj\Lambda)^\perp = \{X \in \D^b(\Lambda) \mid \Hom(I, X)=0 \text{ for all } I \in K^b(\inj\Lambda)\}.$$

The vanishing conjecture then states that this subcategory is 0.
\begin{conj}[Vanishing conjecture] 
	If $\Lambda$ is a finite dimensional algebra, then $K^b(\inj\Lambda)^\perp = 0$.
\end{conj}

In \cref{sec:Unbounded_derived_category} we investigate an analog of this conjecture for the unbounded derived category.

\subsection*{Nunke Condition (NuC)}
The Nunke condition is similar to the vanishing conjecture in that it considers modules which are ``perpendicular'' to the injective modules. Such a module is called a \emph{Nunke module}, and an algebra is said to satisfy the Nunke condition if the only Nunke module is the zero module.

\begin{conj}[Nunke condition] 
	If $X \neq 0$ is a non-zero module over a finite dimensional algebra $\Lambda$, then there is an $n \geq 0$ such that $\Ext^n(D\Lambda, X) \neq 0$. 
\end{conj}

\subsection*{Strong Nakayama Conjecture (SNC)}
The strong Nakayama conjecture is simply the dual of the Nunke condition. We include both in this summary for completeness sake.

\begin{conj}[strong Nakayama Conjecture] 
	If $X \neq 0$ is a non-zero module over a finite dimensional algebra $\Lambda$, then there is an $n \geq 0$ such that $\Ext^n(X,\Lambda) \neq 0$. 
\end{conj}

\subsection*{Generalized Nakayama Conjecture (GNC)}
The generalized Nakayama conjecture is a slight weakening of the Strong Nakayama conjecture.

\begin{conj}[generalized Nakayama conjecture] 
	If $S$ is a simple module over a finite dimensional algebra $\Lambda$, then there is an $n \geq 0$ such that $\Ext^n(S, \Lambda) \neq 0$. 
\end{conj}

We can also formulate the conjecture as all indecomposable injectives appearing in the minimal injective resolution of $\Lambda$. We give a short proof that this is an equivalent formulation here.

\begin{prop}\label{prop:GNC_reformulated}
	A finite dimensional algebra $\Lambda$ satisfies GNC if and only if every indecomposable injective appears in the minimal injective resolution of $\Lambda$.
	\begin{proof}
		Let the minimal injective resolution of $\Lambda$ is given by 
		\begin{center}
			\begin{tikzcd}
				0 \ar[r] & \Lambda \ar[r] & I_0 \ar[r] & I_1 \ar[r] & \cdots
			\end{tikzcd}
		\end{center}
		Since the resolution is minimal, we have that $\Ext^n(S, \Lambda) = \Hom(S, I_n)$. This is non-zero if and only if the socle of $S$ is a summand of $I_n$. Thus $\Ext^n(S, \Lambda)$ is non-zero for some $n$ if the injective envelope of $S$ appears in the minimal resolution, and $I$ is a summand of $I_n$ if $\Ext^n(\operatorname{soc}I, \Lambda) \neq 0$.
	\end{proof}
\end{prop}

\subsection*{Auslander--Reiten Conjecture (ARC)}

\begin{conj}[Auslander--Reiten conjecture] 
	Let $\Lambda$ be  finite dimensional algebra. If $M$ is a generator in $\mod\Lambda$ such that  $\Ext^n(M, M) = 0$ for all $n > 0$, then $M$ is projective. 
\end{conj}

\subsection*{Nakayama Conjecture (NC)}

\begin{defn}[Dominant dimension]
	Let $\Lambda$ be a finite dimensional algebra, and let
	\begin{center}
		\begin{tikzcd}
		0 \ar[r] & \Lambda \ar[r] & I^0 \ar[r] & I^1 \ar[r] & \cdots
		\end{tikzcd}
	\end{center}
	be the minimal injective resolution of $\Lambda$. Then the \emph{dominated dimension} of $\Lambda$ is $$\domdim(\Lambda) = \inf\{n \; | \; I^n\text{ is not projective} \}.$$
\end{defn}

\begin{conj}[Nakayama conjecture] 
	If $\Lambda$ has infinite dominant dimension, then $\Lambda$ is selfinjective.
\end{conj}

\subsection{Implications}
The homological conjectures are related in the way presented in the diagram below.

\begin{tikzcd}
\text{FDC} \ar[r, Rightarrow]\ar[d, Rightarrow] & \text{WTC} \ar[r, Rightarrow] & \text{GSC}\\
\text{VC}\ar[r, Rightarrow] & \text{NuC}\ar[r, Leftrightarrow] & \text{SNC}\ar[r, Rightarrow] & \text{GNC}\ar[r, Leftrightarrow] & \text{ARC}\ar[r, Rightarrow] & \text{NC}
\end{tikzcd}

The remainder of this section is used to prove these implications.

\begin{theorem} \cite[Proposition~4.4]{MR04}
	The finitistic dimension conjecture implies the Wakamatsu tilting conjecture.
	\begin{proof}
		Assume $\Lambda$ satisfies FDC, and let $T$ be a Wakamatsu tilting module with $\pd T < \infty$. By definition we have an exact sequence
		\begin{center}
			\begin{tikzcd}
			\eta\colon 0 \ar[r] & \Lambda \ar[r, "d_{-1}"] & T_0 \ar[r, "d_0"] & T_1 \ar[r, "d_1"] & \cdots
			\end{tikzcd}
		\end{center}
		We want to show that $\eta$ can be replaced by a bounded sequence of the same form. 
		
		Let $K_i$ denote the kernel of $d_i$. First we prove by induction on $i$ that $\Ext^{>0}(K_i, T)=0$. For $i=0$ we have $K_0=\Lambda$, so we have $\Ext^{>0}(K_0, T)=0$. Now assume that $\Ext^{>0}(K_i, T)=0$ for some $i \geq 0$. We have a short exact sequence
		\begin{center}
			\begin{tikzcd}
			0  \ar[r] & K_{i}  \ar[r] & T_{i}  \ar[r] & K_{i+1}  \ar[r] & 0.
			\end{tikzcd}
		\end{center}
		Applying the long exact sequence in $\Ext(-, T)$ we get
		\begin{center}
			\begin{tikzcd}[column sep=18pt]
			\Ext^{n}(T_{i}, T) \ar[r] & \Ext^{n}(K_{i}, T) \ar[r] & \Ext^{n+1}(K_{i+1}, T) \ar[r] & \Ext^{n+1}(T_{i}, T)
			\end{tikzcd}
		\end{center}
		Since $T_{i}$ is in $\add T$ we have that $\Ext^n(T_{i}, T)=0$ for all $n>0$. Then by exactness we have that $\Ext^{n+1}(K_{i+1}, T) \cong \Ext^{n}(K_i, T) = 0$ for all $n \geq 1$. Since $T$ is Wakamatsu tilting we have that $\Ext^1(K_{i+1}, T) = 0$, so by induction $\Ext^{>0}(K_i, T)=0$ for all $i\geq 0$.
		
		By a similar argument we now wish to show that $$\Ext^1(K_m, K_{m-1}) \cong \Ext^{i}(K_m, K_{m-i})$$ for all $i \leq m$. We proceed by induction on $i$. When $i=1$ the statement is evident. Now assume that $$\Ext^1(K_m, K_{m-1}) \cong \Ext^{i}(K_m, K_{m-i})$$ for some $i \geq 1$. Then it is sufficient to show that $$ \Ext^{i}(K_m, K_{m-i}) \cong  \Ext^{i+1}(K_m, K_{m-i-1}).$$
		We have a short exact sequence
		\begin{center}
			\begin{tikzcd}
			0  \ar[r] & K_{m-i-1}  \ar[r] & T_{m-i-1}  \ar[r] & K_{m-i}  \ar[r] & 0.
			\end{tikzcd}
		\end{center}
		Taking the long exact sequence in $\Ext(K_m, -)$ we get the exact sequence
		\begin{center}
			\begin{tikzcd}
			\Ext^{i}(K_m, T_{m-i-1}) \ar[r] & \Ext^{i}(K_m, K_{m-i}) \ar[dl, out=-10, in=170, overlay]\\ 
			\Ext^{i+1}(K_m, K_{m-i-1}) \ar[r] & \Ext^{i+1}(K_m, T_{m-i-1}).
			\end{tikzcd}
		\end{center}
		Since we showed above that $\Ext^{>0}(K_m, T) = 0$ and $T_{m-i-1}$ is in $\add T$ we get that $\Ext^{>0}(K_m, T_{m-i-1}) = 0$. Thus $\Ext^{i}(K_m, K_{m-i}) \cong \Ext^{i+1}(K_m, K_{m-i-1})$, and by induction we have that 
		$$\Ext^1(K_m, K_{m-1}) \cong \Ext^{i}(K_m, K_{m-i})$$ for all $i \leq m$.
		
		Next we show that $\pd K_i < \infty$ for all $i \geq 0$. We proceed by induction on $i$. The projective dimension of $K_0=\Lambda$ is 0, which is finite. For $i>0$ we have a short exact sequence
		\begin{center}
			\begin{tikzcd}
			0  \ar[r] & K_{i-1}  \ar[r] & T_{i-1}  \ar[r] & K_{i}  \ar[r] & 0.
			\end{tikzcd}
		\end{center}
		Therefore $\pd K_i \leq \sup\{\pd T_{i-1}, \pd K_{i-1} - 1\} < \infty$.
		
		Lastly, let $n=\findim(\Lambda) < \infty$. Then we have that 
		$$\Ext^1(K_{n+1}, K_{n}) \cong \Ext^{n+1}(K_{n+1}, K_{0}) = 0$$
		where the last equality comes from $\pd K_{n+1} \leq n$. Now if we apply $\Hom(K_{n+1}, -)$ to the short exact sequence
		\begin{center}
			\begin{tikzcd}
			0  \ar[r] & K_{n}  \ar[r] & T_{n}  \ar[r] & K_{n+1}  \ar[r] & 0,
			\end{tikzcd}
		\end{center}
		we get an exact sequence 
		\begin{center}
			\begin{tikzcd}
			\Hom(K_{n+1}, T_n) \ar[r] & \Hom(K_{n+1}, K_{n+1}) \ar[r] & \Ext^1(K_{n+1}, K_{n})=0.
			\end{tikzcd}
		\end{center}
		This means that $K_{n+1}$ is a direct summand of $T_n$, and thus is in $\add T$. Then we get abounded version of $\eta$ by
		\begin{center}
			\begin{tikzcd}[column sep=23pt]
			\eta'\colon 0 \ar[r] & \Lambda \ar[r, "d_{-1}"] & T_0 \ar[r, "d_0"] & T_1 \ar[r, "d_1"] & \cdots \ar[r, "d_{n-1}"] & T_n \ar[r, "d_n"] & K_{n+1} \ar[r] & 0.
			\end{tikzcd}
		\end{center}
		Hence $T$ is a tilting module, and thus $\Lambda$ satisfies WTC.
	\end{proof}
\end{theorem}

\begin{theorem}
	The Wakamatsu tilting conjecture implies the Gorenstein symmetry conjecture.
	\begin{proof}
		The left module $D(\Lambda_\Lambda)$ is Wakamatsu tilting. WTC then gives us that if $D(\Lambda_\Lambda)$ has finite projective dimension, then $_\Lambda\Lambda$ has a finite coresolution by modules in $\add D(\Lambda_\Lambda)$. In other words $_\Lambda\Lambda$ has finite injective dimension.
		
		For the other direction assume $_\Lambda\Lambda$ has finite injective dimension. Then the right module $D(_\Lambda\Lambda)$ has finite projective dimension, so WTC gives us that $\Lambda_\Lambda$ has finite injective dimension. Which means $D(\Lambda_\Lambda)$ has finite projective dimension.
	\end{proof}
\end{theorem}

\begin{theorem} \cite[1.2]{Hap93} \label{thm:FDC_implies_VC}
	The finitistic dimension conjecture implies the vanishing conjecture.
	\begin{proof}
		Assume $\Lambda$ doesn't satisfy VC, and let $I^\bullet \in K^b(\inj\Lambda)^\perp$ be non-zero complex. Since $\D^b(\Lambda) \cong K^{+,b}(\inj\Lambda)$ we may assume $I^\bullet$ is a complex of injectives, and without loss of generality we may assume it is concentrated in degrees $i \geq 0$, and that $d^0\colon I^0 \to I^1$ is not split mono. Since if it's concentrated in degrees $i \geq k$ we can just shift it, and if $d^0$ is split mono, then replacing $I^0$ by $0$ and $I^1$ by $I^1/I^0$ gives a homotopic complex.
			
		The module $\Hom(D\Lambda, I^i)$ is in $\add\Hom(D\Lambda, D\Lambda) = \add\Lambda$ so $\Hom(D\Lambda, I^\bullet)$ is a complex of projectives. We show that this complex is acyclic by considering the following diagram.
		
		\begin{center}
			\begin{tikzcd}
			0 \ar[r] \ar[d] & D\Lambda \ar[r] \ar[d, "f"] \ar[dl, dashed]& 0 \ar[d]\\
			I^{i-1} \ar[r, "d^{i-1}"] & I^i \ar[r, "d^i"] & I^{i+1}
			\end{tikzcd}
		\end{center}
		
		Since $I^\bullet$ is in $K^b(\inj\Lambda)^\perp$ and $D\Lambda$ is in $K^b(\inj\Lambda)$, we have that whenever $d^if=0$, $f^\bullet$ is homotopic to 0. Meaning $f$ factors through $d^{i-1}$. This means that $\Hom(D\Lambda, I^\bullet)$ is an acyclic complex. Further since $\Hom(D\Lambda, -)$ is an equivalence between $\inj\Lambda$ and $\proj\Lambda$ and $d^0$ is not split mono, we have that $\Hom(D\Lambda, d^0)$ is not split mono.
		
		The cokernel of $\Hom(D\Lambda, d^i)$ has a projective resolution of length $i$. This resolution is the direct sum of its minimal resolution and an acyclic bounded complex of projectives. Since bounded acyclic complexes of projectives are split and $\Hom(D\Lambda, d^0)$ is not, we must have that the minimal resolution has length $i$, and so $\findim(\Lambda) = \infty$.
	\end{proof}
\end{theorem}

\begin{theorem} \cite[1.2]{Hap93} \label{thm:VC_implies_Nuc}
	The vanishing conjecture implies the Nunke condition.
	\begin{proof}
		Assume $\Lambda$ doesn't satisfy NuC. That is, there is an $X \neq 0$ with $\Ext^i(D\Lambda, X) = 0$ for all $i \geq 0$. Then we claim that $X$ considered as a stalk complex is in $K^b(\inj\Lambda)^\perp$. To show this we proceed by induction on the width of $I^\bullet \in K^b(\inj\Lambda)$. If the width is 1, then $I^\bullet = I[-i] \in K^b(\inj\Lambda)$ is a stalk complex. Then $\D^b(I[-i], X) = \Ext^i(I, X)$, which is 0 because $I$ is in $\add D\Lambda$ and $\Ext^i(D\Lambda, X)=0$.
		
		Let $I^\bullet \in K^b(\inj\Lambda)$ be a complex of width $n$. without loss of generality we may assume $I^\bullet$ is concentrated in degrees $0 \leq i < n$. Then 
		$$I^{>0} \to I \to I^{0} \to I^{>0}[1]$$ 
		is a triangle with $I^{>0}$ of width $n-1$ and $I^0$ of width 1. Taking the long exact sequence in $\D^b(-,X)$ it follows that $\D^b(I, X)=0$. So $X$ is a non-zero complex in $K^b(\inj\Lambda)^\perp$, and hence $\Lambda$ does not satisfy VC.
	\end{proof}
\end{theorem}

%Before we proceed we write out the details for the reformulation of the Auslander--Reiten conjecture, that was mentioned before. \todo{Maybe its cleaner to just use this as the conjecture directly...}
%
%\begin{prop}\label{prop:ARC_reformulated}
%	The Auslander--Reiten conjecture is equivalent to the statement that if $M$ is a generator with $\Ext^n(M, M) = 0$ for $n > 0$, then $M$ is projective.
%	\begin{proof}
%		Assume ARC and that $M$ satisfies the hypothesis. Then since $M$ is a generator $\Lambda$ is in $\add M$ and thus $\Ext^n(M, \Lambda)=0$. So $\Ext^n(M, M\oplus \Lambda)=0$ and $M$ is projective.
%		
%		For the other direction Assume $M$ satisfies $\Ext^n(M, M \oplus \Lambda)=0$. Then $\Ext^n(M \oplus \Lambda, M\oplus \Lambda) = 0$, so $M \oplus \Lambda$ is projective, which means that $M$ is projective. 
%	\end{proof}
%\end{prop}

%To prove the last two implications we need some results from the theory of Wedderburn projectives. The results we need are stated below, and are proved in \cref{sec:wedderburn_correspondence}. We remind the reader that we write $(-,-)$ to mean $\Hom(-,-)$.
%
%\begin{restatable}{restate-thm}{Wederburnequivalence} \label{thm:hom_generator_equivalence}
%	Let $\Lambda$ be an artin algebra and $M$ a generator. Let $\Gamma = \End(M)^{\operatorname{op}}$ and $P=(M, \Lambda)$. Then we have the following:
%	\begin{enumerate}[i)]
%		\item We have an isomorphism of rings $\End_\Gamma(P)^{\operatorname{op}} \cong \Lambda$ and an isomorphism of $\Lambda$-modules $(P_\Lambda, \Gamma) \cong M$.
%		\item The composition $(P,-)\circ (M,-)$ is the identity on $\mod\Lambda$.
%		\item The functor $(M,-)$ maps injective $\Lambda$-modules to injective $\Gamma$-modules. 
%	\end{enumerate}
%\end{restatable}
%
%\begin{defn}[Wedderburn projective]
%	Let $\Gamma$ be an artin algebra and let $P$ be a finitely generated projective $\Gamma$-module. Let $\Lambda = \End(P)^{\operatorname{op}}$ and $M=(P, \Gamma)$. The module $P$ is said to be \emph{Wedderburn projective} if $\End(M)^{\operatorname{op}}=\Gamma$.
%\end{defn}
%
%\begin{restatable}{restate-thm}{Wederburncriterion}\label{thm:wedderburn_criterion}
%	Let $\Gamma$ be an artin algebra and $P$ a projective $\Gamma$-module. If $P$ contains the projective cover of all simple modules that appear in the socle of an injective copresentation of $\Gamma$, then $P$ is Wedderburn projective.
%\end{restatable}

Before we can prove the equivalence between the generalized Nakayama conjecture and the Auslander--Reiten Conjecture we will need the following proposition.

\begin{prop}\label{prop:hom_generator_preserves_injectives}
	Let $M$ be a module and $I$ an injective module. If the projective cover of the socle of $I$ is in $\add M$, then $(M,I)$ is an injective $\Gamma:=\End(M)^{\operatorname{op}}$-module. In particular if $M$ is a generator then $(M,-)$ preserves injectives.
	\begin{proof}
		Let $J \leq \Gamma$ be a left ideal and let $\psi\colon J \to (M,I)$ be any $\Gamma$-linear map. By \cref{lem:injectives_for_noetherian_ring} in in the appendix it is enough to show that $\psi$ factors through $\Gamma$ to conclude that $(M, I)$ is injective. Assume $J$ is generated by $\{f_i\}$. If we can find $\gamma\colon M \to I$ such that $\gamma \circ f_i = \psi(f_i)$ then we would get our factorization of $\psi$ by 
		\begin{tikzcd}
		J \ar[r, hookrightarrow] & \Gamma \ar[r, "\gamma \circ -"] & (M, I).
		\end{tikzcd} To construct such a $\gamma$ we consider the following diagram.
		\begin{center}
			\begin{tikzcd}
			\bigoplus M \ar[dr, "\sum \psi(f_i)"] \ar[d, swap, "\sum f_i"]\\
			M \ar[r, swap, dashed, "\gamma"] & I
			\end{tikzcd}
		\end{center}
		We want to show that the kernel of $\sum \psi(f_i)$ contains the kernel of $\sum f_i$, so that we can use the injective property of $I$. To see this let $K$ be the kernel of $\sum f_i$ and let $K'$ be the kernel of $\sum \psi(f_i)$. If $K'$ does not contain $K$, then $Q:= K/K'\cap K$ is a nonzero module that is mapped injectively into $I$. So the socle of $Q$ is a summand of the socle of $I$. Then by assumption the projective cover of the socle of $Q$ is in $\add M$, so there is a non-zero projective map $M \to Q$. By the lifting property of projectives we get a map $M \to K$ such that the composition with $\sum \psi(f_i)$ is non-zero.
		
		Let $a_i$ be the composition 
		\begin{tikzcd}[column sep=15pt]
		M \ar[r] & K \ar[r, hookrightarrow] & \bigoplus M \ar[r, "\pi_i"] & M.
		\end{tikzcd}
		Then we get $\sum f_i \circ a_i = 0$. Applying $\psi$ we get $\sum \psi(f_i)\circ a_i = 0$, which gives a contradiction since $a_i$ was explicitly constructed such that $\sum \psi(f_i)\circ a_i$ is non-zero. Thus $K'$ contains $K$.
		
		Using this we get the following commutative diagram:
		\begin{center}
			\begin{tikzcd}
			\bigoplus M \ar[d] \ar[dd, bend right=60, swap, "\sum f_i"] \ar[dr, "\sum \psi(f_i)"] \\
			\left(\bigoplus M\right)/ K \ar[r] \ar[d, hookrightarrow] & I\\
			M \ar[ur, dashed, swap, "\exists\gamma"]
			\end{tikzcd}
		\end{center}
		Since $I$ is injective it lifts monomorphisms so we know that $\gamma$ exists. Thus $(M, I)$ is an injective $\Gamma$-module.
	\end{proof}
\end{prop}

\begin{theorem}\label{thm:GNC_implies_ARC}
	The generalized Nakayama conjecture implies the Auslander--Reiten conjecture.
	\begin{proof}
		The proof goes by contraposition. Assume $\Lambda$ does not satisfy ARC. Then we have a nonprojective generator $M$ such that $\Ext^n(M, M)=0$ for all $n>0$. We wish to show that $\Gamma := \End(M)^{\operatorname{op}}$ does not satisfy GNC. Let
		\begin{center}
		\begin{tikzcd}
			0 \ar[r] & M \ar[r] & I_0 \ar[r] & I_1 \ar[r] & \cdots
		\end{tikzcd}
		\end{center}
		be an injective resolution of $M$. Since $\Ext^n(M,M)=0$, when we apply $(M,-):=\Hom(M,-)$ we get an exact sequence.
		\begin{center}
		\begin{tikzcd}
			0 \ar[r] & \Gamma \ar[r] & (M,I_0) \ar[r] & (M,I_1) \ar[r] & \cdots
		\end{tikzcd}
		\end{center}
		By \cref{prop:hom_generator_preserves_injectives} this is an injective resolution of $\Gamma$.
		
		Since $M$ is a non-projective generator it has every indecomposable projective as a summand and a nonprojective summand. So $M$ has more indecomposable summands than $\Lambda$ which means that $\Gamma$ has more indecomposable projectives than $\Lambda$. It follows that $\Gamma$ also has more injectives and thus has an injective not on the form $(M, I)$. Since all modules that appear in the injective resolution of $\Gamma$ are on the form $(M, I)$, not all indecomposable injectives appear in the resolution. Therefore by \cref{prop:GNC_reformulated} we have that $\Gamma$ does not satisfy GNC.
	\end{proof}
\end{theorem}

\begin{theorem} \cite[Theorem 3.4.3]{Yam96}
	The Auslander--Reiten conjecture implies the generalized Nakayama conjecture.
	\begin{proof}
		Assume that ARC holds, and let $\Gamma$ be a finite dimensional algebra. We wish to show that $\Gamma$ satisfies GNC. Let the minimal injective resolution of $\Gamma$ be given by
		\begin{center}
			\begin{tikzcd}
			0 \ar[r] & \Gamma \ar[r] & I_0 \ar[r] & I_1 \ar[r] & \cdots
			\end{tikzcd}
		\end{center}
		Let $I$ be the minimal injective module such that each $I_i$ is in $\add I$. If we can show that $I$ is a cogenerator, then it will follow that $\Gamma$ satisfies GNC. Let $P=DI$ be the projective right $\Gamma$-module dual to $I$, and let $\Lambda = \End_\Gamma(P)$ be its endomorphism ring. 
		
		Using the Hom-Tensor adjunction we see that 
		\begin{align*}
		D(P \otimes_\Gamma X) &= \Hom_k(P \otimes_\Gamma X, k)\\
		&= \Hom_\Gamma(P, \Hom_k(X, k))\\
		&= \Hom_\Gamma(P, D X)
		\end{align*}
		In particular we have that $D(P\otimes_\Gamma I) = \End_\Gamma(P) = \Lambda$ as right $\Lambda$-modules, and so $P\otimes_\Gamma I = D\Lambda$. 
		
		Now let $\mathcal S$ be the subcategory of $\mod \Gamma$ of modules that have a copresentation of modules in $\add I$. Then we claim there is an equivalence of categories
		
		\begin{center}
			\begin{tikzcd}[column sep = 50pt]
			\mathcal S \ar[r, bend left=10]{}{P\otimes_\Gamma-} & \mod\Lambda \ar[l, bend left=10]{}{\Hom_\Lambda(P,-)}
			\end{tikzcd}
		\end{center}
		
		To see this we first note the following identities
		\begin{align*}
		\Hom_\Lambda(P, P\otimes_\Gamma I) &= \Hom_\Lambda(P, D\Lambda)\\
		&= \Hom_k(\Lambda \otimes_\Lambda P, k) \\
		&= DP = I\\
		\\
		P \otimes_\Gamma \Hom_\Lambda(P, D\Lambda) &= P \otimes_\Gamma DP\\
		&= D\Lambda
		\end{align*}
		
		Since $P_\Gamma$ is projective $P\otimes_\Gamma -$ is exact, so both functors are left exact. This means they induce equivalences between the subcategories with copresentations in $\add I$ and $\add D\Lambda$ respectively. Thus we get our wanted equivalence.
		
		Now if we apply $P\otimes_\Gamma -$ to the injective resolution $I_\bullet$, we get an injective resolution of $P\otimes_\Gamma\Gamma = P$ as a $\Lambda$-module. Applying $\Hom_\Lambda(P, -)$ gives us back the complex $I_\bullet$ and thus we have that $\Ext_\Lambda^n(P, P)=0$ for all $n>0$. 
		
		Since $\Hom_\Lambda(P, -)$ is non-vanishing, $P$ is a generator in $\mod\Lambda$. Since by assumption ARC holds, we get that $P$ is projective as a $\Lambda$-module. Thus $\Hom_\Lambda(P, -)$ is right exact. Since $\Hom_\Lambda(P, P) = \Hom_\Lambda(P, P\otimes \Gamma) = \Gamma$ we get that $\Hom_\Lambda(P,-)$ induces an equivalence between modules with a presentation in $\add P$ and modules with a presentation in $\add \Gamma$. We conclude that $\mathcal S = \mod\Gamma$, and thus that $I$ is a cogenerator.
		
		Since $I$ is a cogenerator all indecomposable injective modules appear in the resolution of $\Gamma$, an thus $\Gamma$ satisfies GNC.
	\end{proof}
\end{theorem}
%
%\begin{theorem}
%	The Auslander--Reiten conjecture implies the Nakayama conjecture.
%	\begin{proof}
%		Assume $\Gamma$ does not satisfy NC. In other words $\Gamma$ has dominant dimension $\infty$, but is not self injective. We then want to show that there exists a ring that does not satisfy ARC. Let
%		\begin{center}
%		\begin{tikzcd}
%			0 \ar[r] & \Gamma \ar[r] & I_0 \ar[r] & I_1
%		\end{tikzcd}
%		\end{center}
%		be an injective copresentation of $\Gamma$. Let $P$ be the sum of the projective covers of all nonisomorphic simple modules in the socle of $I_0$. Then by \cref{thm:wedderburn_criterion} we have that $P$ is Wedderburn projective.
%		
%		Let $\Lambda = \End(P)^{\operatorname{op}}$ and let $M = \Hom(P, \Gamma)$. Then $M$ is a nonprojective generator. If we can show that $\Ext^{>0}(M,M)=0$, then we will have shown that $\Lambda$ does not satisfy ARC.
%		
%		We have functors $(M,-)\colon\mod\Lambda \to \mod\Gamma$ and $(P,-)\colon\mod\Gamma \to \mod\Lambda$. By \cref{thm:hom_generator_equivalence} $(M, -)$ is fully faithful and $(P,-)\circ (M,-) = \operatorname{id}_{\Lambda}$.
%		
%		Let $0\to M \to Q_0 \to Q_1$ be an injective copresentation of $M$. Applying $(M,-)$ we get an injective copresentation of $\Gamma$. We conclude that all the projective-inejctive modules are in the essential image of $(M,-)$.
%		
%		In other words if $I^\bullet$ is the minimal injective resolution of $\Gamma$ then $Q^\bullet := (P, I^\bullet)$ is the minimal injective resolution of $M$, and $(M, Q^\bullet)=I^\bullet$. This means that $(M, Q^\bullet)$ is exact away from 0, so $\Ext^{>0}(M,M)=0$. 
%		
%		But then $M$ is a nonprojectvie generator with $\Ext^{>0}(M,M)=0$, so $\Lambda$ does not satisfy ARC.
%	\end{proof}
%\end{theorem}
%
%Combining the implications above we see that the generalized Nakayama conjecture implies the Nakayama conjecture. There is however a much simpler proof of this fact which we include below.

\begin{prop}\cite{AR75} 
	The generalized Nakayama conjecture implies the Nakayama conjecture
	\begin{proof}
		Assume $\Lambda$ satisfies GNC and that the dominant dimension of $\Lambda$ is $\infty$. As shown in \cref{prop:GNC_reformulated} if $\Ext^\bullet(S, \Lambda)$ is nonzero that means the injective envelope $I(S)$ appears in the minimal injective resolution of $\Lambda$. If all injectives appear in the resolution and the dominant dimension is infinity then all injectives are projective. Thus $\Lambda$ is self injective, and hence $\Lambda$ satisfies NC. 
	\end{proof}
\end{prop}

The proofs above do not necessarily work on the level of individual algebras. For example for the proof that WTC implies GSC we need to assume that WTC holds for both $\Lambda$ and $\Lambda^{\op}$ to prove that $\Lambda$ satisfies GSC. We list the relationships between the conjectures for individual algebras.

\begin{prop}\label{prop:conj_on_individual_algebras}
	The implications between the conjectures on the level of individual algebras can be described as follows: 
	\begin{enumerate}[i)]
		\item If $\Lambda$ satisfies FDC, then $\Lambda$ also satisfies WTC.
		\item If both $\Lambda$ and $\Lambda^{\op}$ satisfy WTC, then both $\Lambda$ and $\Lambda^{\op}$ also satisfy GSC.
		\item The implications FDC $\Rightarrow$ VC $\Rightarrow$ NuC, both hold on the level of individual algebras.
		\item An algebra $\Lambda$ satisfies Nuc if and only if $\Lambda^{\op}$ satisfies SNC.
		\item The implications SNC $\Rightarrow$ GNC $\Rightarrow$ NC, both gold on the level of individual algebras.
		\item If $\Gamma$ satisfies GNC whenever $\Gamma = \End_\Lambda(M)^{\op}$ for a generator $M$ in $\mod\Lambda$, then $\Lambda$ satisfies ARC.
		\item If $\End(I)^{\op}$ satisfies ARC, where $I$ is an injective module such that $\add I$ contains every injective in the minimal resolution of $\Lambda$, then $\Lambda$ satisfies GNC.
		\item If $\End(P)^{\op}$ satisfies ARC, where $P$ is the projective cover of $\operatorname{soc}I$ where $I$ is the sum of all indecomposable projective-injective $\Lambda$-modules, then $\Lambda$ satisfies NC. \todo{I removed the proof for this, since it factors through ARC implies GNC, but it gives a different condition on individual algebras}
		\item An algebra $\Lambda$ satisfies NC if and only if $\Lambda^{\op}$ does \cite[Theorem~4]{Mu68}. \todo{read proof more carefully}
	\end{enumerate}
\end{prop}

%To prove the last two implications we need some results from the theory of Wedderburn projectives. The results we need are stated below, and are proved in \cref{sec:wedderburn_correspondence}. We remind the reader that we write $(-,-)$ to mean $\Hom(-,-)$.

\begin{restatable}{restate-thm}{Wederburnequivalence} \label{thm:hom_generator_equivalence}
	Let $\Lambda$ be an artin algebra and $M$ a generator. Let $\Gamma = \End(M)^{\operatorname{op}}$ and $P=(M, \Lambda)$. Then we have the following:
	\begin{enumerate}[i)]
		\item We have an isomorphism of rings $\End_\Gamma(P)^{\operatorname{op}} \cong \Lambda$ and an isomorphism of $\Lambda$-modules $(P_\Lambda, \Gamma) \cong M$.
		\item The composition $(P,-)\circ (M,-)$ is the identity on $\mod\Lambda$.
		\item The functor $(M,-)$ maps injective $\Lambda$-modules to injective $\Gamma$-modules. 
	\end{enumerate}
\end{restatable}

\begin{defn}[Wedderburn projective]
	Let $\Gamma$ be an artin algebra and let $P$ be a finitely generated projective $\Gamma$-module. Let $\Lambda = \End(P)^{\operatorname{op}}$ and $M=(P, \Gamma)$. The module $P$ is said to be \emph{Wedderburn projective} if $\End(M)^{\operatorname{op}}=\Gamma$.
\end{defn}

\begin{restatable}{restate-thm}{Wederburncriterion}\label{thm:wedderburn_criterion}
	Let $\Gamma$ be an artin algebra and $P$ a projective $\Gamma$-module. If $P$ contains the projective cover of all simple modules that appear in the socle of an injective copresentation of $\Gamma$, then $P$ is Wedderburn projective.
\end{restatable}

\subsection{Wedderburn correspondence}\label{sec:wedderburn_correspondence}

In this section we give the relevant theory needed to prove the implications between homological conjectures involving the Auslander--Reiten conjecture.

The theory is about understanding the relationship between the functors $(P,-)\colon \mod \Gamma \to \mod\Lambda$ and $(M,-)\colon \mod\Lambda \to \mod\Gamma$, where $P$ is a projective $\Gamma$-module, $M=(P, \Gamma)$ and $\Lambda = \End(P)^{\op}$. First we show a general result about how the homfunctor interacts with injective modules.

\begin{prop}\label{prop:hom_generator_preserves_injectives}
	Let $M$ be a module and $I$ an injective module. If the projective cover of the socle of $I$ is in $\add M$, then $(M,I)$ is an injective $\Gamma:=\End(M)^{\operatorname{op}}$-module. In particular if $M$ is a generator then $(M,-)$ preserves injectives.
	\begin{proof}
		Let $J \leq \Gamma$ be a left ideal and let $\psi\colon J \to (M,I)$ be any $\Gamma$-linear map. By \cref{lem:injectives_for_noetherian_ring} \todo{[...] In the appendix} it is enough to show that $\psi$ factors through $\Gamma$ to conclude that $(M, I)$ is injective. Assume $J$ is generated by $\{f_i\}$. If we can find $\gamma\colon M \to I$ such that $\gamma \circ f_i = \psi(f_i)$ then we would get our factorization of $\psi$ by 
		\begin{tikzcd}
			J \ar[r, hookrightarrow] & \Gamma \ar[r, "\gamma \circ -"] & (M, I).
		\end{tikzcd} To construct such a $\gamma$ we consider the following diagram.
		\begin{center}
			\begin{tikzcd}
			\bigoplus M \ar[dr, "\sum \psi(f_i)"] \ar[d, swap, "\sum f_i"]\\
			M \ar[r, swap, dashed, "\gamma"] & I
			\end{tikzcd}
		\end{center}
		We want to show that the kernel of $\sum \psi(f_i)$ contains the kernel of $\sum f_i$, so that we can use the injective property of $I$. To see this let $K$ be the kernel of $\sum f_i$ and let $K'$ be the kernel of $\sum \psi(f_i)$. If $K'$ does not contain $K$, then $Q:= K/K'\cap K$ is a nonzero module that is mapped injectively into $I$. So the socle of $Q$ is a summand of the socle of $I$. Then by assumption the projective cover of the socle of $Q$ is in $\add M$, so there is a non-zero projective map $M \to Q$. By the lifting property of projectives we get a map $M \to K$ such that the composition with $\sum \psi(f_i)$ is non-zero.
		
		Let $a_i$ be the composition 
		\begin{tikzcd}[column sep=15pt]
		M \ar[r] & K \ar[r, hookrightarrow] & \bigoplus M \ar[r, "\pi_i"] & M.
		\end{tikzcd}
		Then we get $\sum f_i \circ a_i = 0$. Applying $\psi$ we get $\sum \psi(f_i)\circ a_i = 0$, which gives a contradiction since $a_i$ was explicitly constructed such that $\sum \psi(f_i)\circ a_i$ is non-zero. Thus $K'$ contains $K$.
		
		Using this we get the following commutative diagram:
		\begin{center}
			\begin{tikzcd}
			\bigoplus M \ar[d] \ar[dd, bend right=60, swap, "\sum f_i"] \ar[dr, "\sum \psi(f_i)"] \\
			\left(\bigoplus M\right)/ K \ar[r] \ar[d, hookrightarrow] & I\\
			M \ar[ur, dashed, swap, "\exists\gamma"]
			\end{tikzcd}
		\end{center}
		Since $I$ is injective it lifts monomorphisms so we know that $\gamma$ exists. Thus $(M, I)$ is an injective $\Gamma$-module.
	\end{proof}
\end{prop}

Combining the proposition we just proved with Yoneda's Lemma we have the necessary tools to prove \cref{thm:hom_generator_equivalence}. We restate it her for the convenience of the reader.
\Wederburnequivalence*
\begin{proof}
	\begin{enumerate}[i)]
		\item[]
		\item By Yoneda's Lemma we have an equivalence $$(M,-)\colon\add M \to \add (M,M)=\proj\Gamma.$$ Since $M$ is a generator, $\Lambda$ is in $\add M$. So 
		$$\End(P)=\End((M,\Lambda)) = \End(\Lambda)=\Lambda^{\operatorname{op}}$$ 
		\begin{center}
			and
		\end{center}
		$$(P,\Gamma)=((M,\Lambda),(M,M))=(\Lambda,M)=M.$$
		\item Let $X$ be a $\Lambda$-module. Since $\add M$ has only a finite number of indecomposable modules it is functorially finite. \todo{Have not defined this yet...} So we can take an $M$-resolution of $X$.
		$$\cdots \to M_1 \to M_0 \to X \to 0$$
		Since $\add M$ contains the projectives this is exact. Applying $(M,-)$ we get a projective resolution of $(M,X)$. Since $(M, X)$ is determined by its projective resolution and $X$ is determined by its $M$-resolution we need only show that $(P,-)\circ (M,-)$ is the identity on $\add M$. Then again by Yoneda's Lemma $(P, (M, M')) = (\Lambda, M')=M'$.
		\item Since $M$ is a generator it contains the projective cover of all simple modules. Then \cref{prop:hom_generator_preserves_injectives} gives us that $(M, -)$ maps injective modules to injective modules.
	\end{enumerate}
\end{proof}

\begin{prop}
	Let $P$ be a projective $\Gamma$-module, and let $\Lambda = \End(P)^{\operatorname{op}}$. Then $(P, -)\colon\mod \Gamma \to \mod \Lambda$ is fully faithful on $\add I(P/JP)$, where $P/JP$ denotes the top of $P$, and $I(P/JP)$ its injective envelope.
	\begin{proof}
		Let $I$ and $I'$ be in $\add I(P/JP)$. We want to show that the map $\Hom_\Gamma(I, I') \to \Hom_\Lambda((P,I), (P, I'))$ is an isomorphism. First we show injectivity. Let $f\colon I\to I'$ be a non-zero map. Then the socle of $\Image f$ is a semisimple submodule of $I'$, so it is in $\add P/JP$. Then there exists a nonzero map from $P$ to $\Image f$. Since $P$ is projective this lifts to a map $\hat{f}\colon P\to I$. Then $f \circ \hat{f}$ is non-zero, so $\Hom_\Gamma(I, I') \to \Hom_\Lambda((P,I), (P, I'))$ is injective.
		
		The argument for surjectivity is similar to that for \cref{prop:hom_generator_preserves_injectives}. Let $\psi\colon(P,I)\to (P, I')$ be a $\Lambda$-linear map. Let $f_i\colon P\to I$ generate $(P,I)$ as a $\Lambda$-module. Consider the following diagram:
		\begin{center}
			\begin{tikzcd}
			\bigoplus P \ar[dr, "\sum \psi(f_i)"] \ar[d, swap, "\sum f_i"]\\
			I \ar[r, swap, dashed, "?"] & I'
			\end{tikzcd}
		\end{center}
		We wish to show that there is a map at ? completing the diagram. We first show that $K':=\ker \sum \psi(f_i)$ contains $K:=\ker \sum f_i$. Assume for the sake of contradiction that it does not. Then $Q := K / K' \cap K$ is mapped injectively into $I'$ by $\sum \psi(f_i)$. So the socle of $Q$ is in $\add P/JP$, and we have a non-zero map $P \to Q$.
		
		Since $P$ is projective this extends to a map $P \to K$. Let $a_i$ be the compositions 
		\begin{tikzcd}[column sep = 15pt]
			P \ar[r] & K \ar[r] & \bigoplus P \ar[r, "\pi_i"] & P.
		\end{tikzcd}
		Then clearly $\sum f_i \circ a_i = 0$, but $\sum \psi(f_i) \circ a_i$ is non-zero. Since $\psi$ is $\Lambda$-linear this is a contradiction, so $K'$ contains $K$.
		
		Then we get an induced commutative diagram:
		\begin{center}
			\begin{tikzcd}
			\bigoplus P \ar[d] \ar[dd, bend right=60, swap, "\sum f_i"] \ar[dr, "\sum \psi(f)"] \\
			\left(\bigoplus P\right)/ K \ar[r] \ar[d, hookrightarrow] & I'\\
			I \ar[ur, dashed, swap, "\exists"]
			\end{tikzcd}
		\end{center}
		Now because $I'$ is injective we know that there is a lift, and so the map $\Hom_\Gamma(I, I') \to \Hom_\Lambda((P,I), (P, I'))$ is surjective, and thus an isomorphism.
	\end{proof} 
\end{prop}

We conclude this section by giving a proof of \cref{thm:wedderburn_criterion}.

\Wederburncriterion*
\begin{proof}
	Let $\Gamma \to I_0 \to I_1$ be a minimal injective presentation of $\Gamma$. Then by \cref{prop:hom_generator_preserves_injectives} we have that $(P, I_0) \to (P,I_1)$ is an injective presentation of $(P,\Gamma)$. The proposition gives us that $(P,-)$ is fully faithful on $I_0$ and $I_1$. Since the endomorphisms of $\Gamma$ are exactly endomorphisms of $I_0 \to I_1$ up to homotopy this means that $$\Gamma=\End_\Gamma(\Gamma)^{\op} = \End_\Lambda((P, \Gamma))^{\op}$$
	So $P$ is Wedderburn projective.
\end{proof}