In this section we will discuss a reduction technique known as recollement. The idea of reduction techniques is to reduce the work of proving an algebra has finite finitistic dimension to proving the same for ``simpler'' algebras. In \cref{sec:Triangular_matrix_rings} we will consider a reduction technique of triangular matrix algebras. The triangular matrix rings are closely related to recollements, and we discuss their relationship more closely in \todo{ref}

OVERGANG?

\begin{defn}[Recollement]\label{def:recollement}
	A \emph{recollement} between triangulated categories $\mathcal T'$, $\mathcal T$ and $\mathcal T''$ is a collection of six functors satisfying:
\begin{center}
\begin{tikzcd}[column sep=4cm]
\mathcal T' \ar[r, "i_*=i_!"{name=i}] & 
\ar[l, swap, "i^*"{name=il}, bend right=30] \ar[l, "i^!"{name=ir}, bend left=30]
\mathcal T \ar[r, "j^!=j^*"{name=j}] & 
\ar[l, swap, "j_!"{name=jl}, bend right=30] \ar[l, "j_*"{name=jr}, bend left=30]
\mathcal T''
\arrow[phantom, from=il, to=i, "\dashv" rotate=-90]
\arrow[phantom, from=i, to=ir, "\dashv" rotate=-90]
\arrow[phantom, from=jl, to=j, "\dashv" rotate=-90]
\arrow[phantom, from=j, to=jr, "\dashv" rotate=-90]
\end{tikzcd}	
\end{center}

\begin{enumerate}[(i)]
	\item All functors are exact, and we have adjoint pairs $(i^*, i_*)$, $(i_!, i^!)$, $(j_!, j^!)$, $(j^*, j_*)$. 
	\item \label{recollement:vanishing_composition}The composition $j^*i_*=0$ vanishes.
	\item \label{item:i_fully-faith} 
	We have natural isomorphisms $i^*i_* \cong i^!i_! \cong \operatorname{id}_{\mathcal T'}$ induced by the units and counits of the adjunctions. 
	\item \label{item:j_fully-faith}
	We have natural isomorphisms $j^!j_! \cong j^*j_* \cong \operatorname{id}_{\mathcal T''}$, also induced by the units and counits. 
	\item \label{recollement:triangles}
	For every $X \in \mathcal T$ we have the following distinguished triangles:
	\begin{center}
	\begin{tikzcd}
	j_!j^!X \ar[r, "\varepsilon"] & X \ar[r, "\eta"] & i_*i^* X \ar[r] & j_!j^!X[1]\\		
	i_!i^!X \ar[r, "\varepsilon"] & X \ar[r, "\eta"] & j_*j^* X \ar[r] & i_!i^!X[1].
	\end{tikzcd}
	\end{center}
\end{enumerate}
Note that (\ref{item:i_fully-faith}) and (\ref{item:j_fully-faith}) are equivalent to $i_*$, $j_!$, and $j_*$ being fully faithful.
\end{defn}

We are specifically interested in recollements where the triangulated categories in question are (bounded) derived categories of finite dimensional algebras.

OVERGANG?

\begin{lemma} \label{lem:adjoint_preserves_bounded_proj/inj}
	Let \begin{tikzcd}
	\D^b(\Lambda') \ar[r, "i_*"{name=i}, bend right=20] & 
	\ar[l, swap, "i^*"{name=il}, bend right=20]
	\D^b(\Lambda)
	\end{tikzcd} be exact functors with an adjoint pair $(i^*,i_*)$. Then $i^*$ preserves bounded projective complexes and $i_*$ preserves bounded injective complexes.
	\begin{proof}
		The bounded projective complexes can be characterized up to isomorphism as the complexes $P$ such that for any complex $Y$ there is an integer $t_Y$ with $\D^b(\Lambda) (P, Y[t])=0$ for $t\geq t_Y$. One can see this by using the equivalence $\D^b(\Lambda) \cong K^{-,b}(\proj\Lambda)$.
		
		Let $P$ be a bounded complex of projectives in $\D^b(\Lambda)$. Then we want to show that $i^*P$ is as well. Let $Y$ be any complex in $\D^b(\Lambda')$. Then $\D^b(\Lambda')(i^*P, Y[t]) = \D^b(\Lambda)(P, i_*Y[t])$, so since $P$ is a bounded complex of projectives there is $t_Y$ such that this vanishes for $t \geq t_Y$. 
		
		The statement for injectives is exactly dual, and so we do not write it out here, but leave it to the reader.
	\end{proof}
\end{lemma}

FILLER

\begin{lemma} \label{lem:uniform_bound_on_homology}
	Let \begin{tikzcd}
	\D^b(\Lambda') \ar[r, "i_*"{name=i}] & 
	\ar[l, swap, "i^*"{name=il}, bend right=30] \ar[l, "i^!"{name=ir}, bend left=30]
	\D^b(\Lambda)
	\end{tikzcd} be exact functors with adjoint pairs $(i^*,i_*)$ and $(i_*, i^!)$. Then the homology of $i_*X$ is uniformly bounded for $X\in\mod\Lambda'$ considered as a complex concentrated in degree 0. I.e. there is an $r$, independent of $X$, such that $H^{j}(i_*X) = 0$ for $j\not\in(-r, r)$.
	\begin{proof}
		We first prove that there is an $r'$, independent of $X$, such that $H^{j}(i_*X)=0$ for $j \geq r'$.
		Let $P$ be $i^*\Lambda \in \D^b(\Lambda')$. Then by \cref{lem:adjoint_preserves_bounded_proj/inj} $P$ is a bounded complex of projectives.
		
		Thus there is an $r'$ such that $P^{-j}=0$ for $j \geq r'$. Then $$\D^b(\Lambda')(P, X[j]) = \D^b(\Lambda)(\Lambda, i_*X[j]) = H^{j}(i_*X)=0$$ for $j\geq r'$ and any $\Lambda'$-module $X$, when considered as a complex concentrated in degree 0.
		
		Next we prove that there is an $r''$ such that $H^{-j}(i_*X)=0$ for $j \geq r''$. The argument is completely dual. Let $I$ be $i^!D\Lambda \in \D^b(\Lambda') \cong K^{+,b}(\inj\Lambda')$. Then again by \cref{lem:adjoint_preserves_bounded_proj/inj} $I$ is a bounded complex of injectives.
		
		Thus there is an $r''$ such that $I^{j}=0$ for $j \geq r''$. Then $$\D^b(\Lambda')(X, I[j]) = \D^b(\Lambda)(i_*X, D\Lambda[j]) = H^{-j}(i_*X)=0$$ for $j\geq r''$ and any $\Lambda'$-module $X$, when considered as a complex concentrated in degree 0.
		
		Letting $r$ be the maximum of $r'$ and $r''$ we get that $H^{j}(X)$ is zero outside of $(-r, r)$.
	\end{proof}
\end{lemma}

Now that we have a good understanding of how the functors in a recollement interact with homology, we can use this to say something about the projective dimension of modules, and thus about the finitistic dimension.

\begin{theorem}\cite[3.3]{Hap93}
	Given a recollement between bounded derived categories 
	\begin{center}
		\begin{tikzcd}[column sep=4cm]
		\D^b(\Lambda') \ar[r, "i_*=i_!"{name=i}] & 
		\ar[l, swap, "i^*"{name=il}, bend right=30] \ar[l, "i^!"{name=ir}, bend left=30]
		\D^b(\Lambda) \ar[r, "j^!=j^*"{name=j}] & 
		\ar[l, swap, "j_!"{name=jl}, bend right=30] \ar[l, "j_*"{name=jr}, bend left=30]
		\D^b(\Lambda''),
		\arrow[phantom, from=il, to=i, "\dashv" rotate=-90]
		\arrow[phantom, from=i, to=ir, "\dashv" rotate=-90]
		\arrow[phantom, from=jl, to=j, "\dashv" rotate=-90]
		\arrow[phantom, from=j, to=jr, "\dashv" rotate=-90]
		\end{tikzcd}	
	\end{center}
	 then we have that $\findim(\Lambda) < \infty$ if and only if we have that $\findim(\Lambda') < \infty$ and $\findim(\Lambda'') < \infty$.
	\begin{proof}
		Assume $\findim(\Lambda) < \infty$. We begin by showing that $\findim(\Lambda') < \infty$.
		
		Let $T = \Lambda' / \rad\Lambda'$ be the sum of all simple $\Lambda'$-modules. Then the projective dimension of $X$ is the largest $t$ for which $\Ext^t(X, T) \neq 0$. Let $X$ be a module in $\mod \Lambda'$ with finite projective dimension. We consider $X$ as a complex concentrated in degree 0. Then since $X$ is isomorphic to its projective resolution, by \cref{lem:adjoint_preserves_bounded_proj/inj} $i_*X$ is a bounded complex of projectives. Say:
		$$i_*X = 0 \to P^{-s} \to \cdots \to P^{s'} \to 0$$
		By \cref{lem:uniform_bound_on_homology} we know there is an $r$ independent of $X$ such that $H^{-j}(i_*X)=0$ for $j \geq r$. Truncating $i_*X$ at $-r$ gives a projective resolution of $\ker d^{-r}_{i_*X}$. So $\ker d^{-r}_{i_*X}$ has projective dimension $-r-(-s) = s-r$. Since $\findim(\Lambda)<\infty$ this means that $s \leq r + \findim(\Lambda)$.
		
		Since $i_*T$ is in $\D^b(\Lambda)$ it is a bounded complex, in particular there is a $t_0$ such that $i_*T^{t}=0$ for $t \geq t_0$. Then by the bounds above $\D^b(\Lambda)(i_*X, i_*T[t]) = 0$ for $t \geq t_0 + s \geq t_0 + r + \findim(\Lambda)$. Since $i_*$ is fully faithful this equals $\D^b(\Lambda')(X, T[t])$, and so $\findim(\Lambda') \leq t_0 + r + \findim(\Lambda)$. In particular it is finite.
		
		The proof for $\findim(\Lambda'')$ is the same, just replacing $i_*$ with $j_!$. We leave writing out the details to the reader.
		
		For the converse assume $\Lambda'$ and $\Lambda''$ both have finite finitistic dimension. Let $T = \Lambda / \rad\Lambda$, and $X$ be a $\Lambda$-module with finite projective dimension, and consider both modules as a complex concentrated in degree 0. By \cref{def:recollement}(\ref{recollement:triangles}) we have distinguished triangles:
		\begin{center}
			\begin{tikzcd}
				j_!j^!X \ar[r] & X \ar[r] & i_*i^* X \ar[r] & j_!j^!X[1]\\		
				i_!i^!T \ar[r] & T \ar[r] & j_*j^* T \ar[r] & i_!i^!T[1].
			\end{tikzcd}
		\end{center}
		We write $(-,-)_m$ instead of $\D^b(\Lambda)(-,-[m])$, and make the following abbreviation:
		\begin{align*}
			X_j &:= j_!j^!X & X_i &:= i_*i^* X \\
			T_i &:= i_!i^!T & T_j &:= j_*j^* T.
		\end{align*} 
		Taking the long exact sequence in homfuntors we get the long exact sequences:
		\begin{center}
			\begin{tikzcd}[column sep=0.5cm]
			\cdots \ar[r] & (X, T_i)_m \ar[r] & (X, T)_m \ar[r] & (X, T_j)_m \ar[r] & (X, T_i)_{m+1} \ar[r] & \cdots\\
			\cdots \ar[r] & (X_i, T_i)_m \ar[r] & (X, T_i)_m \ar[r] & (X_j, T_i)_m \ar[r] & (X_i, T_i)_{m+1} \ar[r] & \cdots\\
			\cdots \ar[r] & (X_i, T_j)_m \ar[r] & (X, T_j)_m \ar[r] & (X_j, T_j)_m \ar[r] & (X_i, T_j)_{m+1} \ar[r] & \cdots\\
			\end{tikzcd}
		\end{center}
		Using the fact that $j^*i_* = j^!i_! = 0$ from \cref{def:recollement}(\ref{recollement:vanishing_composition}) we deduce that
		\[\arraycolsep=2pt\def\arraystretch{1.33}
		\begin{array}{ccccccl}
			(X_i, T_j)_m &=& (i_*i^* X, j_*j^* T)_m &=& (j^*i_*i^*X, j^*T)_m &=& 0\\
			\span\span\span\text{and}&\\
			(X_j, T_i)_m &=& (j_!j^!X, i_!i^!T)_m &=& (j^!X, j^!i_!i^!T)_m &=& 0.
		\end{array}\]
		Combining this with the long exact sequences gives us that 
		$$(X_i, T_i)_m = (X, T_i)_m \text{ and } (X_j, T_j)_m = (X, T_j)_m.$$ 
		If we can show that $(X_i, T_i)_m$ and $(X_j, T_j)_m$ are bounded, then $(X, T_i)_m$ and $(X, T_j)_m$ would be bounded as well. Consequently we would have that $(X, T)_m$ is bounded. This would give us a bound on the projective dimension of $X$.
		
		We start by bounding $(X, T_i)_m = (X_i, T_i)_m$. First note that since  $i^*i_* \cong \operatorname{id}$ we have that
		\begin{align*}
			(X_i, T_i)_m = (i_*i^* X, i_!i^!T)_m = (i^*i_*i^* X, i^!T)_m = (i^* X, i^!T)_m
		\end{align*}
		Since $X$ has finite projective dimension we can think of it as a bounded complex of projectives. Then by \cref{lem:adjoint_preserves_bounded_proj/inj} $i^*X$ is as well. By the second half of \cref{lem:uniform_bound_on_homology} (using $(i^*, i_*)$ instead of $(i_*, i^!)$) we have that there is an $r$ such that $H^{-j}(i^*X)=0$ for all $j \geq r$. This means that thinking of $i^*X$ as a complex of projectives, it is 0 in degree $-t$ for all $t \geq r + \pd\ker d^{-r}_{i^*X}$, in particular it is 0 for all $t\geq r + \findim(\Lambda')$. Since $i^!T$ is a bounded complex, it has an upper bound, say $t_0$. Thus $(i^* X, i^!T)_m = 0$ for all $m \geq t_0 + r + \findim(\Lambda')$.
		
		The bound on $(X, T_j)_m$ is similar, using the finitistic dimension of $\Lambda''$. Taking the maximum of these two bounds we get a bound on $(X, T)_m$, which gives a bound on the projective dimension independent of $X$, hence a bound on $\findim(\Lambda)$. 
	\end{proof}
\end{theorem}