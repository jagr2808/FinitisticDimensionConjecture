In representation theory of finite dimensional algebras, there are several related conjectures known as the ``homological conejctures''. The strongest of these conjectures is the Finitistic Fimension Conjecture, which states that the finitistic dimension of a finite dimensional algebra is always finite.

The finitistic dimension was introduced by Auslander--Buchsbaum in the late 1950s to study commutative noetherian rings. They proved that for a local noetherian commutative ring the finitistic dimension equals the depth\cite{AB57}. Later it was shown by Bass and Gruson--Raynaud that for any commutative noetherian ring the (big) finitistic dimension equals the Krull dimension\cite{Bass62,RG71}.

The non-commutative case turned out to be more difficult. In 1960 Bass published to important questions about the finitistic dimension\cite{Bass60}, which he credits to Rosenberg and Zelinsky. Their first question asks whether the small finitistic dimension equals the big finitistic dimension. \todo{text}

- Zimmermann-Huisgen proved false \cite{ZH92}

