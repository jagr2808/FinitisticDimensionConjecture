\section{The Igusa--Todorov functions} \label{sec:Igusa-Todorov}

In this section we introduce the Igusa--Todorov functions, which are important tools for bounding the projective dimensions of modules in $\mod \Lambda$. The main theorem is \cref{thm:projdim_bounded_by_psi} in which we give a bound for the projective dimension of modules in a short exact sequence. In \cref{sec:repdimension} we use this to show that algebras with representation dimension at most 3, has finite finitistic dimension, and in \cref{sec:stable_hereditary_algebras} and \cref{sec:special_biserial_algebras} we give a examples of two classes of algebras which are known to have representation dimension 3.

Let $K_0$ be the abelian group generated by isomorphism classes of modules in $\mod\Lambda$, with relations given by $[A\oplus B] - [A] - [B] = 0$ for any modules $A$ and $B$, and $[P]=0$ whenever $P$ is projective. We define the linear map $L\colon K_0\to K_0$ by $L[A] = [\Omega A]$. For any module $X$, we let $[\add X]$ be the finitely generated subgroup of $K_0$ generated by modules in $\add X$. 

Fitting's lemma (\cref{thm:Fittings_lemma}) tells us that there is an integer $\eta_X$ such that the homomorphism $L\colon L^m[\add X] \to L^{m+1}[\add X]$ is an isomorphism for every $m \geq \eta_X$. We use this to define two important functions from $\mod \Lambda$ to $\mathbb N$.

\begin{defn}[The Igusa--Todorov functions]
	We define two functions $\phi$ and $\psi$ from $\mod\Lambda$ to $\mathbb N$. For a module $M \in \mod\Lambda$ we define $\phi(M)$ to be the integer $\eta_M$ coming from Fitting's lemma, as explained above. In other words, $\phi(M)$ is the smallest integer such that $$L\colon L^m[\add M] \to L^{m+1}[\add M]$$ is an isomorphism for every $m \geq \phi(M)$. We define $\psi(M)$ in a similar way, but adding on an extra term to account for the structure of $\Omega^{\phi(M)}M$. 
	$$\psi(M) = \phi(M) + \sup\left\lbrace\pd Z \; \middle| \; \pd Z < \infty, Z \in \add \Omega^{\phi(M)}M\right\rbrace$$
\end{defn}

We now list the properties needed to prove our main theorem.

\begin{lemma} \cite[Lemma~3]{IgTo05} \label{lem:properties_of_psi}
	\begin{enumerate}[i)]
		\item $\psi(M) = \pd M$, when $\pd M < \infty$.
		\item $\psi(M^k) = \psi(M)$.
		\item $\psi(M) \leq \psi(M\oplus N)$.
		\item If $Z$ is a direct summand of $\Omega^n(M)$ where $n \leq \phi(M)$ and $\pd Z < \infty$, then we have that $\pd Z + n \leq \psi(M)$.
	\end{enumerate}
	\begin{proof}
		\begin{enumerate}[i)]
			\item[] %empty line
			\item If $\pd M < \infty$, then $L^m[\add M] \neq 0$ whenever $m < \pd M$, and $L^m[\add M] =0$ whenever $m \geq \pd M$. So $\psi(M)=\phi(M)=\pd M$.
			\item The two subcategories $\add M^k$ and $\add M$ are equal. So, since $\psi$ is defined only in terms of the additive subcategory $\add M$, we have that $\psi(M^k)=\psi(M)$.
			\item  The subcategory $\add M$ is contained in $\add M\oplus N$, so if $L$ is injective when restricted to $L^m[\add M\oplus N]$ then $L$ is injective when restricted to $L^m[\add M]$. Thus we have $\phi(M) \leq \phi({M\oplus N})$. Further $$\Omega^{\phi({M\oplus N})-\phi(M)}\left(\add\Omega^{\phi(M)}M \right) \subseteq \add\Omega^{\phi({M\oplus N})} M\oplus N,$$ 
			so $\psi(M) \leq \psi(M\oplus N)$.
			\item Let $p=\pd Z$ and $k = \phi(M) - n$. Then $\Omega^k Z$ is in $\add \Omega^{\phi(M)}M$ and has finite projective dimension, so $\pd\Omega^k Z + \phi(M) \leq \psi(M)$. Thus $$\pd Z + n = p + n = (p-k) + \phi(M) \leq \pd\Omega^k Z + \phi(M) \leq \psi(M).$$
		\end{enumerate}
	\end{proof}
\end{lemma}

We will now apply these properties to get a bound on the projective dimension of modules in a short exact sequence, in terms of the $\psi$-function.

\begin{theorem}\cite[Theorem~4]{IgTo05} \label{thm:projdim_bounded_by_psi}
	Let $0 \to A \to B \to C \to 0$ be a short exact sequence of modules with $\pd C < \infty$. Then $\pd C \leq \psi(A\oplus B)+1$.
	\begin{proof}
		Let $P_A^\bullet$ and $P_C^\bullet$ be the minimal projective resolutions of $A$ and $C$. Then we get a map of short exact sequences
		\begin{center}
		\begin{tikzcd}
			0 \ar[r]  & P_A^0 \ar[r] \ar[d] & P_A^0 \oplus P_C^0 \ar[r] \ar[d] & P_C^0 \ar[r] \ar[d] & 0\\
			0 \ar[r] & A \ar[r] & B \ar[r] & C \ar[r] & 0 
		\end{tikzcd}
		\end{center}
		Applying the Snake Lemma we get $0 \to \Omega A \to \Omega B \oplus P \to \Omega C \to 0$ for some projective module $P$. Thus for some $n \leq \pd C$ we have $L^n[A] = L^n[B]$, and let $n$ be the minimal such number. Clearly $n \leq \phi(A\oplus B
			)$. Let $X = \Omega^n A$, then our sequence of $n$-syzygies looks like
		\begin{center}
			\begin{tikzcd}
			0 \ar[r] & X \ar[r] & X\oplus P \ar[r] & \Omega^nC \ar[r] & 0.
			\end{tikzcd}
		\end{center}
		Let $f$ be the composition
		\begin{tikzcd}
			X \ar[r] & X \oplus P \ar[r, "\pi_X"] & X.
		\end{tikzcd}
		Then by Fitting's Lemma (\cref{cor:fittings_lemma_artin}) $X$ decomposes as a direct sum into two summands $X = Z \oplus Y$ such that $f = f_Z \oplus f_Y$ with $f_Y$ an isomorphism and $f_Z$ nilpotent. In other words the sequence above can be written as
		\begin{center}
			\begin{tikzcd}
			0 \ar[r] & Z\oplus Y \ar[r] & Z \oplus Y\oplus P \ar[r] & \Omega^nC \ar[r] & 0.
			\end{tikzcd}
		\end{center}
		with the left map being
		$$\begin{bmatrix}
			f_Z & 0\\
			0 & f_Y\\
			* & *
		\end{bmatrix} \sim
		\begin{bmatrix}
		f_Z & 0\\
		0 & 1_Y\\
		* & 0
		\end{bmatrix} $$
		So by changing basis this restricts to another short exact sequence
		\begin{center}
			\begin{tikzcd}
			0 \ar[r] & Z \ar[r] & Z \oplus P \ar[r] & \Omega^nC \ar[r] & 0.
			\end{tikzcd}
		\end{center}
		Let $T = \Lambda/J$ and apply the long exact sequence in $\Ext(-, T)$. Then we get an exact sequence
		\begin{center}
			\begin{tikzcd}[column sep=13pt]
			\Ext^k(Z, T) \ar[r] & \Ext^k(Z \oplus P, T) \ar[r] & \Ext^{k+1}(\Omega^nC, T) \ar[r] & \Ext^{k+1}(Z, T).
			\end{tikzcd}
		\end{center}
		Because $\Ext^k(Z \oplus P, T) \cong \Ext^k(Z, T)$, the left map is induced just by $f_Z$. Now, since $f_Z$ is nilpotent, the induced map is surjective if and only if $\Ext^k(Z, T)=0$. We know that, since $\Omega^nC$ has finite projective dimension, $\Ext^{k+1}(\Omega^n C, T)$ is $0$ for $k$ large enough. Then we must have that $\Ext^k(Z, T)=0$, and thus $Z$ has finite projective dimension. Specifically we have bounds given by $\pd\Omega^n C -1 \leq \pd Z \leq \pd\Omega^n C$.
		
		Since $Z$ is a direct summand of $\Omega^n (A\oplus B)$ of finite projective dimension, \cref{lem:properties_of_psi} gives us that $\pd Z + n \leq \psi(A \oplus B)$, and thus $\pd \Omega^n C - 1 + n = \pd C - 1 \leq \psi(A \oplus B)$.
	\end{proof}
\end{theorem}

With a bit of diagram chasing we can extend this theorem to get a bound for the projective dimensions of $A$ and $B$ as well.

\begin{cor}\label{cor:projdim_bounded_by_psi}
	Let $0 \to A \to B \to C \to 0$ be a short exact sequence of modules. 
	\begin{enumerate}[i)]
		\item \label{cor:projdim_bounded_by_psi_i}
		If $\pd A < \infty$, then $\pd A \leq \psi(\Omega B \oplus \Omega C)+1$.
		\item \label{cor:projdim_bounded_by_psi_ii}
		If $\pd B < \infty$ then $\pd B \leq \psi(\Omega A \oplus \Omega^2 C) + 2$.
	\end{enumerate}
	\begin{proof}
		Let $P_B \to B$ be a projective cover of $B$. Then we have a commutative diagram:
		\begin{center}
			\begin{tikzcd}
			0 \ar[r]  & 0 \ar[r] \ar[d] & P_B \ar[r, equal] \ar[d] & P_B \ar[r] \ar[d] & 0\\
			0 \ar[r] & A \ar[r] & B \ar[r] & C \ar[r] & 0 
			\end{tikzcd}
		\end{center}
		Applying the Snake Lemma we get a short exact sequence
		\begin{center}
			\begin{tikzcd}
				0 \ar[r] & \Omega B \ar[r] & \Omega C \oplus P \ar[r] & A \ar[r] & 0
			\end{tikzcd}
		\end{center} 
		for some projective module $P$. Then using the theorem we have that if $\pd A \leq \infty$, then $\pd A \leq \psi(\Omega B \oplus \Omega C \oplus P) + 1 = \psi(\Omega B \oplus \Omega C) + 1$.
		
		Applying the same reasoning to $0 \to \Omega B \to \Omega C \oplus P \to A \to 0$ gives us that if $\pd B \leq \infty$, then $\pd\Omega B \leq \psi(\Omega A \oplus \Omega^2 C) + 1$. Hence we get that $\pd B \leq  \psi(\Omega A \oplus \Omega^2 C) + 2$.
	\end{proof}
\end{cor}

These are all the results we need about the Igusa--Todorov functions. We will now use them to find families of algebras with $\findim(\Lambda) < \infty.$

%% MOVED
%\begin{theorem}\cite[Corollary~8]{IgTo05}
%	If $\Lambda = \End_\Gamma(P)^{\op}$ for an algebra $\Gamma$ with global dimension at most 3, and $P$ projective, then $\findim(\Lambda) < \infty$.
%	\begin{proof}
%		Let $X$ be any $\Lambda$-module with finite projective dimension. Then it has a projective presentation $(P, P_1) \to (P,P_0) \to X \to 0$ where $(P,P_i)=\Hom_\Gamma(P,P_i)$ with $P_i \in \add P$. Since $(P,-)$ is an equivalence from $\add P$ to $\proj\Lambda$ this corresponds to a map $P_1 \to P_0$ which we can extend to a projective resolution in $\Gamma$:
%		\begin{center}
%			\begin{tikzcd}
%			0 \ar[r] & P_3 \ar[r] & P_2 \ar[r] & P_1 \ar[r] & P_0.
%			\end{tikzcd}
%		\end{center}
%		Applying the exact functor $(P, -)$, we get an exact sequence
%		\begin{center}
%			\begin{tikzcd}
%			0 \ar[r] & (P,P_3) \ar[r] & (P,P_2) \ar[r] & (P,P_1) \ar[r] & (P,P_0)\ar[r] & X \ar[r] & 0.
%			\end{tikzcd}
%		\end{center}
%		Truncating this we get a short exact sequence
%		\begin{center}
%			\begin{tikzcd}
%			0 \ar[r] & (P, P_3) \ar[r] & (P, P_2) \ar[r] & \Omega^2 X \ar[r] & 0.
%			\end{tikzcd}
%		\end{center}
%		Then by \cref{thm:projdim_bounded_by_psi} the projective dimension of $\Omega^2 X$ is bounded by $\psi((P, P_3)\oplus (P, P_2))+1$. Which means
%		$$\pd X \leq \psi((P, P_3)\oplus (P, P_2))+3 \leq \psi((P,\Gamma))+3$$
%		Since this bound doesn't depend on $X$, $\Lambda$ has finite finitistic dimension.
%	\end{proof} 
%\end{theorem}
%
%\begin{cor}
%	If $\repdim(\Lambda) \leq 3$ then $\findim(\Lambda) < \infty$.
%	\begin{proof}
%		If $\Lambda$ has rep-dimension less than or equal to 3 then by \cref{prop:repdim_auslander_generator} there is a generator-cogenerator $M$ in $\mod\Lambda$ such that $\Gamma := \End_\Lambda(M)$ has global dimension 3 or less. Then since $M$ is a generator $\Lambda$ is in $\add M$ and so $\Hom_\Lambda(M, \Lambda)$ is a projective $\Gamma$-module with $\End_\Gamma(\Hom_\Lambda(M, \Lambda)) = \End_\Lambda(\Lambda) = \Lambda$.
%	\end{proof}
%\end{cor}