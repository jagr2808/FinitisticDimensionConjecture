We remind the reader that throughout this section $\Lambda$ is a finite dimensional algebra, and $J$ is its radical. The Loewy length of an algebra is the smallest integer $n$ such that $J^n = 0$. In this section show that algebras with short Loewy length have finite finitistic dimension.

The may theorem of this section is \cref{thm:half_rep_finite} in which we prove that ``half representation finite'' algebras satisfies the finitistic dimension conjecture. The reader should note that \cref{thm:J2_equals_0_implies_FDC} and \cref{thm:J3_equals_0_implies_FDC} are special cases of \cref{thm:half_rep_finite}, but we include alternate proofs here.

\begin{theorem}\label{thm:J2_equals_0_implies_FDC}
	If $J^2=0$ then $\findim(\Lambda) < \infty$.
	\begin{proof}
		Let $d = \max\{\pd S_i \mid \pd S_i < \infty\}$ where $S_i$ ranges over the simple $\Lambda$-modules. Let $M$ be a module with $\pd M < \infty$. Let $P \to M$ be a projective cover. Then $\Omega M$ is contained in $JP$ and since $J^2P=0$, $\Omega M$ is annihilated by $J$ and is thus semisimple. This means $\pd \Omega M \leq d$, and thus $\pd M \leq d+1$. So $\findim(\Lambda) \leq d+1 < \infty$.
	\end{proof}
\end{theorem}

The proof for the case of $J^3=0$ uses the Igusa--Todorov function from \cref{sec:Igusa-Todorov}.

\begin{theorem}\cite[Corollary~6]{IgTo05}\label{thm:J3_equals_0_implies_FDC}
	If $J^3=0$ then $\findim(\Lambda) < \infty$.
	\begin{proof}
		Let $M$ be a module with $\pd M < \infty$, and let $P^0 \to M$ be its projective cover. Since $\Omega M \subseteq JP^0$ we have $J^2\Omega M = 0$. Let $P \to \Omega M$ be a projective cover. Since $J^2\Omega M = 0$ we can factorize this as $P \to P/J^2P \to \Omega M$, and we get a short exact sequence
		\begin{center}
		\begin{tikzcd}
			0 \ar[r] & (\Omega^2 M + J^2P) / J^2 P \ar[r] & P / J^2 P \ar[r] & \Omega M \ar[r] & 0
		\end{tikzcd}
		\end{center}
		Let $\psi$ be the Igusa--Todorov function as introduced in \cref{sec:Igusa-Todorov}. Since $\Omega^2 M \subseteq JP$ we have that $(\Omega^2 M + J^2P) / J^2 P$ is semisimple. Then by \cref{lem:properties_of_psi} $\psi((\Omega^2 M + J^2P) / J^2 P) \leq \psi(\Lambda / J)$, and $\psi(P / J^2 P) \leq \psi(\Lambda / J^2)$.
		
		Applying \cref{thm:projdim_bounded_by_psi} to the short exact sequence above we thus get $\pd \Omega M \leq \psi(\Lambda / J \oplus \Lambda / J^2) + 1$, and so $\pd M \leq \psi(\Lambda / J \oplus \Lambda / J^2) + 2$, and $\findim(\Lambda) < \infty$.
	\end{proof}
\end{theorem}

The main theorem of this section is just a very slight generalization of the proof of the $J^3=0$ case. 

\begin{theorem}\cite{Wang94}\label{thm:half_rep_finite}
	If $J^{2l+1} = 0$ and $\Lambda / J^l$ is representation finite, then $\findim(\Lambda) < \infty$.
	\begin{proof}
		Let $M$ be a module with $\pd M < \infty$. We have a short exact sequence 
		\begin{center}
			\begin{tikzcd}
			0 \ar[r] & J^l\Omega M \ar[r] & \Omega M \ar[r] & \Omega M / J^l\Omega M \ar[r] & 0.
			\end{tikzcd}
		\end{center}
		Since $\Omega M \subseteq JP^0_M$ we have $J^{2l}\Omega M = 0$. This means that $J^l\Omega M$ and $\Omega M / J^l\Omega M$ are $\Lambda / J^l$-modules. We use this, the fact that $\Lambda / J^l$ is representation finite, and the Igusa--Todorov function to create a bound for $\pd M$.
		
		Applying \cref{cor:projdim_bounded_by_psi} (\ref{cor:projdim_bounded_by_psi_ii}) we have that:
		$$ \pd \Omega M \leq \psi(\Omega (J^l\Omega M)\oplus\Omega^2(\Omega M / J^l\Omega M)) + 2.$$ 
		Since $\Lambda / J^l$ is representation finite, there are only finitely many indecomposable $\Lambda / J^l$-modules, up to isomorphism. Let $V$ be the sum of all of them. Then since $J^l\Omega M$ and $\Omega M / J^l\Omega M$ are in $\add V$, using \cref{lem:properties_of_psi} we have that 
		$$\psi(\Omega (J^l\Omega M)\oplus\Omega^2(\Omega M / J^l\Omega M)) \leq \psi(\Omega V \oplus \Omega^2 V).$$
		So $\pd M \leq \psi(\Omega V \oplus \Omega^2 V) + 3$, and thus $\findim(\Lambda) < \infty$.
	\end{proof}
\end{theorem}
