\section{Vanishing radical powers}\label{sec:vanishing_radical}

We remind the reader that throughout this section $\Lambda$ is a finite dimensional algebra, and $J$ is its radical. The Loewy length of an algebra is the smallest integer $n$ such that $J^n = 0$. In this section show that algebras with short Loewy length have finite finitistic dimension.

Historically the two important conditions for showing that $\findim(\Lambda) < \infty$ has been that $J^2=0$ and $J^3=0$. Note that both of these are special case of \cref{thm:half_rep_finite}, where we show that $\findim(\Lambda) < \infty$ for ``half representation finite'' algebras. This proof is due to Wang\cite{Wang94}.

We first give an alternate proof for the case $J^2=0$.

\begin{theorem}\label{thm:J2_equals_0_implies_FDC}
	If $\Lambda$ is a finite dimensional algebra with $J^2=0$, then $\findim(\Lambda) < \infty$.
	\begin{proof}
		Let $d = \max\{\pd S_i \mid \pd S_i < \infty\}$ where $S_i$ ranges over the simple $\Lambda$-modules. Let $M$ be a module with $\pd M < \infty$. Let $P \to M$ be a projective cover. Then $\Omega M$ is contained in $JP$ and since $J^2P=0$, $\Omega M$ is annihilated by $J$ and is thus semisimple. This means $\pd \Omega M \leq d$, and thus $\pd M \leq d+1$. So $\findim(\Lambda) \leq d+1 < \infty$.
	\end{proof}
\end{theorem}

The proof of the above theorem is relatively elementary, but it first appeared as a corollary to a more general result by Mochizuki\cite{Moc65}. We outline the proof of this as well.

\begin{theorem}\cite[Theorem~3.1]{Moc65}
	Let $\Lambda$ be a finite dimensional algebra such that $J^i/J^{i+1}$ has finite projective dimension for all $i \geq 2$. Then $\findim(\Lambda) < \infty$.
	\begin{proof}
		As before let $d = \max\{\pd S_i \mid \pd S_i < \infty\}$ where $S_i$ ranges over the simple $\Lambda$-modules. We want to show that $\findim(\Lambda) \leq d+1$.

		First note that since $J^i/J^{i+1}$ is semisimple and of finite projective dimension, we have $\pd J^i/J^{i+1} \leq d$. Now let $M$ be a $\Lambda$-module with finite projective dimension. We see that $J^iM/J^{i+1}M$ is in $\add J^i/J^{i+1}$, because it is semisimple and for each nonzero simple summand $(\lambda m)$, we have that $(\lambda m) \cong (\lambda) \subseteq J^i/J^{i+1}$.

		So $\pd J^iM/J^{i+1}M \leq d$ for all $i\geq 2$. For each $i$ we have a short exact sequence
		\begin{center}
			\begin{tikzcd}
				0 \ar[r] & J^{i+1}M \ar[r] & J^iM \ar[r] & J^iM/J^{i+1}M \ar[r] & 0,
			\end{tikzcd}
		\end{center}
		which gives us that $\pd J^i M \leq \max\{ \pd J^{i+1}M, J^iM/J^{i+1}M \}$. Since there is an $n$ such that $J^n M = 0$ it follows by induction that $\pd J^2M \leq d$.

		If we consider the short exact sequence
		\begin{center}
			\begin{tikzcd}
				0 \ar[r] & J^{2}M \ar[r] & M \ar[r] & M/J^{2}M \ar[r] & 0,
			\end{tikzcd}
		\end{center}
		we get that $\pd M/J^2M \leq \max\{\pd J^2M+1, \pd M\}$. In particular, when $M=\Lambda$, we get $\pd \Lambda/J^2 \leq d+1$. If we let $P \to M$ be the projective cover of $M$, we get a short exact sequence
		\begin{center}
			\begin{tikzcd}
				0 \ar[r] & K \ar[r] & P/J^2P \ar[r] & M/J^{2}M \ar[r] & 0
			\end{tikzcd}
		\end{center}
		for some module $K \subseteq JP/J^2P$. Since we assumed $M$ had finite projective dimension, and $\pd M/J^2M \leq \max\{\pd J^2M+1, \pd M\}$, both $M/J^2M$ and $P/J^2P$ has finite projective dimension. Thus $K$ is a semisimple module with finite projective dimension, and we have $\pd K \leq d$. Thus $\pd M/J^2M \leq \max\{\pd K + 1, \pd P/J^2P\} \leq d+1$.

		Lastly since $\pd M \leq \max\{\pd J^2M, M/J^2M\}$ we get that $\pd M \leq d+1$, and consequently $\findim(\Lambda) \leq d+1 < \infty$.
	\end{proof}
\end{theorem}

The case for $J^3=0$ was first proved by Green--Huisgen-Zimmerman\cite[Theorem~16]{GZH91}. Simplified proofs where given by Fuller--Saorin\cite{FS92}, and Igusa--Todorov\cite[Corollary~6]{IgTo05}. Igusa--Todorov's proof was then generalized by Wang to so called ``half representation finite'' algebras\cite{Wang94}. We give this proof here.

%Here we give a simplified proof, due to Igusa--Todorov, using the $\psi$-function from \cref{sec:Igusa-Todorov}.

% \begin{theorem}\cite[Corollary~6]{IgTo05}\label{thm:J3_equals_0_implies_FDC}
% 	If $J^3=0$ then $\findim(\Lambda) < \infty$.
% 	\begin{proof}
% 		Let $M$ be a module with $\pd M < \infty$, and let $P^0 \to M$ be its projective cover. Since $\Omega M \subseteq JP^0$ we have $J^2\Omega M = 0$. Let $P \to \Omega M$ be a projective cover. Since $J^2\Omega M = 0$ we can factorize this as $P \to P/J^2P \to \Omega M$, and we get a short exact sequence
% 		\begin{center}
% 		\begin{tikzcd}
% 			0 \ar[r] & (\Omega^2 M + J^2P) / J^2 P \ar[r] & P / J^2 P \ar[r] & \Omega M \ar[r] & 0
% 		\end{tikzcd}
% 		\end{center}
% 		Let $\psi$ be the Igusa--Todorov function as introduced in \cref{sec:Igusa-Todorov}. Since $\Omega^2 M \subseteq JP$ we have that $(\Omega^2 M + J^2P) / J^2 P$ is semisimple. Then by \cref{lem:properties_of_psi} $\psi((\Omega^2 M + J^2P) / J^2 P) \leq \psi(\Lambda / J)$, and $\psi(P / J^2 P) \leq \psi(\Lambda / J^2)$.
		
% 		Applying \cref{thm:projdim_bounded_by_psi} to the short exact sequence above we thus get $\pd \Omega M \leq \psi(\Lambda / J \oplus \Lambda / J^2) + 1$, and so $\pd M \leq \psi(\Lambda / J \oplus \Lambda / J^2) + 2$, and $\findim(\Lambda) < \infty$.
% 	\end{proof}
% \end{theorem}

% The main theorem of this section is just a very slight generalization of the proof of the $J^3=0$ case, and is due to Wang. 

\begin{theorem}\cite{Wang94}\label{thm:half_rep_finite}
	If $\Lambda$ is a finite dimensional algebra with $J^{2l+1} = 0$ and $\Lambda / J^l$ is representation-finite, then $\findim(\Lambda) < \infty$.
	\begin{proof}
		Let $M$ be a module with $\pd M < \infty$. We have a short exact sequence 
		\begin{center}
			\begin{tikzcd}
			0 \ar[r] & J^l\Omega M \ar[r] & \Omega M \ar[r] & \Omega M / J^l\Omega M \ar[r] & 0.
			\end{tikzcd}
		\end{center}
		Since $\Omega M \subseteq JP^0_M$ we have $J^{2l}\Omega M = 0$. This means that $J^l\Omega M$ and $\Omega M / J^l\Omega M$ are $\Lambda / J^l$-modules. We use this, the fact that $\Lambda / J^l$ is representation finite, and the Igusa--Todorov function to create a bound for $\pd M$.
		
		Applying \cref{cor:projdim_bounded_by_psi} (\ref{cor:projdim_bounded_by_psi_ii}) we have that:
		$$ \pd \Omega M \leq \psi(\Omega (J^l\Omega M)\oplus\Omega^2(\Omega M / J^l\Omega M)) + 2.$$ 
		Since $\Lambda / J^l$ is representation finite, there are only finitely many indecomposable $\Lambda / J^l$-modules, up to isomorphism. Let $V$ be the sum of all of them. Then since $J^l\Omega M$ and $\Omega M / J^l\Omega M$ are in $\add V$, using \cref{lem:properties_of_psi} we have that 
		$$\psi(\Omega (J^l\Omega M)\oplus\Omega^2(\Omega M / J^l\Omega M)) \leq \psi(\Omega V \oplus \Omega^2 V).$$
		So $\pd M \leq \psi(\Omega V \oplus \Omega^2 V) + 3$, and thus $\findim(\Lambda) < \infty$.
	\end{proof}
\end{theorem}
