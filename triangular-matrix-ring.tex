\subsection{Triangular matrix rings}\label{sec:Triangular_matrix_rings}

In this section we will relate the finitistic dimension of the triangular matrix ring $\Lambda = \begin{pmatrix}
R & 0\\
M& S
\end{pmatrix}$ to the finitistic dimension of $R$ and $S$. Specifically the finitistic dimension of $\Lambda$ will be finite if the finitistic dimensions of both $R$ and $S$ are finite. 

In \cref{sec:recollemt_of_triangular_rings} we give some further conditions on $M$ for which we get a recollement between the bounded derived categories of $S$, $R$ and $\Lambda$.

We will first define the concept of a comma category and describe some of its homological properties. In \cref{thm:findim_of_comma_cat} we give a bound on the finitistic dimension of the comma category. Then in \cref{prop:triangular_matrix_is_comma_cat} we show that for $\Lambda$ a triangular matrix ring as above, we have that $\mod \Lambda$ is isomorphic to the comma category of $M\otimes_R - \colon \mod R \to \mod S$, which means we get a bound on $\findim(\Lambda)$.

\begin{defn}[Comma category]
	Let $\mathcal A$ and $\mathcal B$ be categories and let $F\colon\mathcal A \to \mathcal B$ be a functor. Then the \emph{comma category} $(F, \mathcal  B)$ has as objects triplets $(A, B, f)$ with $A \in \mathcal  A$, $B \in \mathcal  B$, and $f\colon FA \to B$ a morphism in $\mathcal  B$. The morphisms are pairs $(\alpha, \beta)\colon(A, B, f) \to (A', B', f')$ with $\alpha\colon A \to A'$ and $\beta\colon B \to B'$ such that the following diagram commutes:
	\begin{center}
		\begin{tikzcd}
			FA \ar[r, "f"] \ar[d, swap, "F\alpha"] & B \ar[d, "\beta"]\\
			FA' \ar[r, "f'"] & B'.
		\end{tikzcd}
	\end{center}
	The composition is what one would expect. Namely, $(\alpha, \beta) \circ (\alpha', \beta') = (\alpha \circ \alpha ', \beta \circ \beta')$.
\end{defn}

\begin{prop}\label{prop:comma-cat_abelian}
	If $\mathcal A$ and $\mathcal B$ are abelian categories and $F$ is right exact, then the comma category $(F, \mathcal  B)$ is abelian. Further a sequence
	\begin{center}
	\begin{tikzcd}
		(A'', B'', f'') \ar[r]{}{(\alpha', \beta')} & (A, B, f)\ar[r]{}{(\alpha, \beta)}  & (A', B', f')
	\end{tikzcd}
	\end{center}
	is exact if and only if the two related sequences in $\mathcal A$ and $\mathcal B$ are exact.
	\begin{center}
		\begin{tikzcd}
			A'' \ar[r, "\alpha'"] & A\ar[r, "\alpha"]  & A'\\
			B'' \ar[r, "\beta'"] & B\ar[r, "\beta"]  & B'
		\end{tikzcd}
	\end{center}
	\begin{proof}
		We need to show that $(F, \mathcal B)$ has kernels and cokernels, and that for any map the image equals the coimage. First we show that it contains kernels. Let $(\alpha, \beta)\colon(A, B, f) \to (C, D, g)$ be a morphism in the comma category. Then we have a diagram:
		\begin{center}
		\begin{tikzcd}
			& F \ker \alpha \ar[r, "F\iota_\alpha"] \ar[d, dashed, "\theta"]& FA \ar[r, "F\alpha"] \ar[d, "f"] & FC \ar[d, "g"]\\
			0 \ar[r] & \ker \beta \ar[r, "\iota_\beta"] & B \ar[r, "\beta"] & D
		\end{tikzcd}
		\end{center}
		Since $\beta f F\iota_\alpha = f' F\alpha F \iota_\alpha = 0$ there is a unique $\theta$ making the diagram commute. I claim the kernel of $(\alpha, \beta)$ is $(\ker \alpha, \ker \beta, \theta)$. Indeed if $(\alpha', \beta')\colon (A', B', f') \to (A, B, f)$ is any map such that $(\alpha, \beta) \circ (\alpha', \beta') = 0$, then $\alpha\alpha'=0$ and $\beta\beta'=0$. This means both $\alpha'$ and $\beta'$ factor uniquely through $\iota_\alpha$ and $\iota_\beta$. Let $\alpha''$ and $\beta''$ be the morphisms such that $\alpha' = \iota_\alpha \circ \alpha''$ and $\beta' = \iota_\beta \circ \beta''$. Then we claim $(\alpha', \beta')$ factors through $(\iota_\alpha, \iota_\beta)$ as indicated in the diagram below.
		\begin{center}
			\begin{tikzcd}
			FA' \ar[d, "f'"] \ar[r, "F\alpha''"] & F \ker \alpha \ar[r, "F\iota_\alpha"] \ar[d, "\theta"]& FA  \ar[d, "f"] \\
			B' \ar[r, "\beta''"] & \ker \beta \ar[r, "\iota_\beta"] & B 
			\end{tikzcd}
		\end{center}
		The only thing left to verify is that the left square commutes. This follows from the outer rectangle commuting, and that $\iota_\beta$ is a monomorphism.
		
		Showing that cokernels exists is similar, but relies on $F$ being right exact. The construction is completely dual, but to verify commutativity at the end instead of using that $\iota_\beta$ is mono we must use that $F\pi_\alpha\colon FA' \to F\cok \alpha$ is an epimorphism. This follows from $F$ being right exact. We leave the details to the reader. 
		
		Since kernels and cokernels are directly induced by the kernels and cokernels in $\mathcal A$ and $\mathcal B$ it is clear that a sequence in $(F, \mathcal B)$ is exact if and only if the two related sequences are exact. Similarly that the image equals the coimage follows from this being true in $\mathcal A$ and $\mathcal B$.
	\end{proof}
\end{prop}

For the rest of this section we assume $F$ is a right exact functor between abelian catgeories so that the comma category is abelian. We also assume $\mathcal A$ and $\mathcal B$ has enough projectives. In particular we are interested in the case when $\mathcal A$ and $\mathcal B$ are module categories over finite dimensional algebras.

\begin{defn}
	For $\mathcal A$ and $\mathcal B$ abelian categories and $F$ right exact we define the following functors:
		\begin{center}
		\begin{tikzcd}[row sep=3pt]
		T\colon\mathcal A \times \mathcal{B} \ar[r]& (F, \mathcal B)\\
		(A, B)  \ar[r, mapsto]& (A, B \oplus FA, FA \hookrightarrow FA \oplus B)\\
		(\alpha, \beta)  \ar[r, mapsto]& (\alpha, F\alpha \oplus \beta)
		\end{tikzcd}
		\end{center}
		
		\begin{center}
			 \begin{tikzcd}[row sep=3pt]
			U\colon(F, \mathcal B) \ar[r]& \mathcal A \times \mathcal{B}\\
			(A, B, f) \ar[r, mapsto]& (A, B)\\
			(\alpha, \beta) \ar[r, mapsto]& (\alpha, \beta)
			\end{tikzcd}
			\hspace{1cm}
		\begin{tikzcd}[row sep=3pt]
		C\colon(F, \mathcal B)\ar[r]& \mathcal A \times \mathcal B\\
		(A, B, f)  \ar[r, mapsto]& (A, \cok f)\\
		(\alpha, \beta)  \ar[r, mapsto]& (\alpha, \hat{\beta})
		\end{tikzcd}
		\end{center}

		\begin{center}
		\begin{tikzcd}[row sep=3pt]
		Z\colon\mathcal A \times \mathcal{B} \ar[r]& (F, B)\\
		(A, B)  \ar[r, mapsto]& (A, B, 0)\\
		(\alpha, \beta)  \ar[r, mapsto]& (\alpha, \beta)
		\end{tikzcd}
		\end{center}
\end{defn}

\begin{prop}
	With the definitions above $U$ and $Z$ become exact functors.
	\begin{proof}
		Using the characterization of exact sequences shown in \cref{prop:comma-cat_abelian} a short exact sequence in $(F, \mathcal B)$ is a commutative diagram
		\begin{center}
			\begin{tikzcd}
				 & FA'' \ar[r, "F\alpha'"] \ar[d, "f''"] & FA \ar[r, "F\alpha"] \ar[d, "f"] & FA' \ar[r] \ar[d, "f'"] & 0\\
				0 \ar[r] & B'' \ar[r, "\beta'"] & B \ar[r, "\beta"] & B' \ar[r] & 0
			\end{tikzcd}
		\end{center}
		such that the sequences 
		\begin{center}
		\begin{tikzcd}
		0 \ar[r] & A'' \ar[r, "\alpha'"] & A \ar[r, "\alpha"] & A' \ar[r] & 0\\
		0 \ar[r] & B'' \ar[r, "\beta'"] & B \ar[r, "\beta"] & B' \ar[r] & 0
		\end{tikzcd} 
		\end{center}
		are short exact. Since when we apply $U$ we simply get the product of these two sequences, $U$ is exact.
		
		Similarly for $Z$ since the two sequences we start with are assumed to be exact the resulting sequence will be exact by the characterization in \cref{prop:comma-cat_abelian}.
	\end{proof}
\end{prop}

\begin{prop}\cite[Proposition~1.3]{FGR75}
	The pairs of functors $(T, U)$ and $(C, Z)$ form adjoint pairs.
	\begin{proof}
		We want to establish an isomorphism
		$$\Hom(T(A, B), (A', B', f)) \cong \Hom((A, B), (A', B')).$$ 
		A morphism $\left(\alpha, 
		\begin{bmatrix}
		\beta & \gamma
		\end{bmatrix}\right) \colon 
		T(A, B) \to (A', B', f)$  is given by a commutative diagram
		\begin{center}
		\begin{tikzcd}[ampersand replacement=\&, row sep = 25pt]
			T(A, B) \colon
			\ar[d, swap]{}{\left(\alpha, \begin{bmatrix}
					\beta & \gamma
			\end{bmatrix}\right)} 
		\& FA 
			\ar{r}{\begin{bmatrix}
			0 \\ 1
			\end{bmatrix}} 
			\ar[d, swap, "F\alpha"]
		\& B \oplus FA 
			\ar{d}{\begin{bmatrix}
			\beta & \gamma
			\end{bmatrix}} \\
			(A', B', f)\colon \& FA' \ar[r, "f"] \& B'.
		\end{tikzcd}
		\end{center}
		The isomorphism is then given by sending this to $(\alpha, \beta)$. This is clearly surjective. 
		
		For injectivity assume $(\alpha, \beta) = 0$, then $\gamma = \begin{bmatrix}
		\beta & \gamma
		\end{bmatrix}\begin{bmatrix}
		0 \\ 1
		\end{bmatrix} = fF\alpha= 0$. So the map is injective, and $(T, U)$ is an adjoint pair.
		
		Next we consider $(C, Z)$. We want an isomorphism 
		\begin{align*}
			\Hom(C(A, B, f), (A', B')) &= \Hom((A, \cok f), (A', B')) \\
			&\cong \Hom((A, B, f), (A', B', 0)).
		\end{align*} 
		A morphism in $\Hom((A, B, f), (A', B', 0))$ is a commutative diagram
		\begin{center}
			\begin{tikzcd}
			FA \ar[r, "f"] \ar[d, swap, "F\alpha"] & B \ar[d, "\beta"]\\
			FA' \ar[r, "0"] & B'
			\end{tikzcd}
		\end{center}
		Since $\beta f = 0$, we have that $\beta$ factors through the cokernel of $f$ uniquely. Let the factorization be given by the map $\beta'\colon \cok f \to B'$. Then we send this diagram to $(\alpha, \beta')$. Since the choice of $\beta'$ was unique this is an isomorphism, so $(C, Z)$ is an adjoint pair.
	\end{proof}
\end{prop}

\begin{cor}
	The functors $T$ and $C$ preserve projective objects.
	\begin{proof}
		What we need to check is that for projective objects $P$ and $Q$ in $(\mathcal A \times \mathcal B)$ and $(F, \mathcal B)$ respectively we have that $\Hom(TP, -)$ and $\Hom(CQ, -)$ are exact. By adjointness these are equal to $\Hom(P, U-)$ and $\Hom(Q, Z-)$ respectively. Since $U$ and $Z$ are exact this holds, and so $T$ and $C$ preserve projective objects.
	\end{proof}
\end{cor}

We will now use these four functors to understand the structure of projective objects in the comma category, and consequently projective resolutions.

\begin{prop}\cite[Corollary~1.6c]{FGR75}
	For a projective object $P$ in $(F, \mathcal B)$ we have that $T(C(P)) \cong P$, in particular all projectives are of the form $T(P')$ for a projective $P' \in \mathcal A \times \mathcal B$.
	\begin{proof}
		Let $P$ be given by $f\colon FA \to B$. Applying $C$ we get $(A, \cok f)$. We have morphisms $P \to ZC(P)$ and $TC(P) \to ZC(P)$ given by the following diagram
		\begin{center}
		\begin{tikzcd}
			FA \ar[d, equal] \ar[r, "f"] & B \ar[d, two heads]\\
			FA \ar[r, "0"] & \cok f\\
			FA \ar[u, equal] \ar[r, hookrightarrow]  & \cok f \oplus FA. \ar[u, swap,  two heads]
		\end{tikzcd}
		\end{center}
		By the projective property of $P$ there is some morphism $\beta$ factorizing the map $P \to ZC(P)$, which gives us the diagram:
		\begin{center}
			\begin{tikzcd}
			FA \ar[d, equal] \ar[r, "f"] & B \ar[d, "\beta"]\\
			FA \ar[d, equal] \ar[r, hookrightarrow]  & \cok f \oplus FA \ar[d,  two heads]\\
			FA \ar[r, "0"] & \cok f.
			\end{tikzcd}
		\end{center}
		Since $FA \hookrightarrow \cok f \oplus FA$ is split mono, $f$ is split mono. This means that $B$ splits as a direct sum of the image and cokernel of $f$, i.e. $B$ is isomorphic to $\cok f \oplus \Image f \cong \cok f \oplus FA$. From the diagram we see that $\beta$ induces an isomorphism on each component, and thus $\beta$ is an isomorphism. So we have $P \cong TC(P)$.
	\end{proof}
\end{prop}

\begin{prop}\cite[Lemma~4.16]{FGR75}\label{prop:pd_in_commacat}
	Let $X = (A, B, f)$ be an object in the comma category. Then $\pd X \geq \pd A$, and if $A=0$ then $\pd X = \pd B$.
	\begin{proof}
		We first show that $\pd X \geq \pd A$. Note that there is an equality $\pd C(X) = \max\{ \pd A, \pd \cok f \}$ so we always have $\pd C(X) \geq \pd A$. If $\pd X = \infty$ then the statement holds so let us assume $\pd X = n < \infty$. We proceed by induction on $n$. If $n=0$ then $C(X)$ is projective so $\pd X = \pd C(X) = \pd A = 0$. Next assume the statement holds whenever the projective dimension is less than $n$. Let $P \to A$ and $P' \to \cok f$ be epimorphisms from projectives. Then we have an epimorphism $T(P, P') \to X$. If we let $\Omega A$ be the kernel of $P \to A$ and $X' = (\Omega A, K, \theta)$ be the kernel of $T(P, P') \to X$ as shown in the following diagram
		\begin{center}
		\begin{tikzcd}
			& F\Omega A \ar[r] \ar[d, "\theta", swap] & FP \ar[r] \ar[d, hookrightarrow] & FA \ar[r] \ar[d, "f"] & 0\\
			0 \ar[r] & K \ar[r] & P' \oplus FP \ar[r] & B \ar[r] & 0,
		\end{tikzcd}
		\end{center}
		then we have $\pd A \leq \pd \Omega A + 1$ and $\pd X = \pd X' + 1$. By induction we have that $\pd X' \geq \pd \Omega A$ and so $\pd X \geq \pd \Omega A +1 \geq \pd A$. 
		
		If $A=0$ then we can associate $C(X)=(0, B)$ with $B$. Any projective resolution $P_B^\bullet$ of $B$ gives a resolution of $X$ by $T(0, P_B^\bullet)$, and any resolution $P_X^\bullet$ of $X$ gives a resolution of $(0, B)$ by $C(P_X^\bullet)$. Thus $\pd X = \pd B$.
	\end{proof}
\end{prop}

Now we are ready for the main theorem of this section, where we give an upper bound on the finitistic dimension of the comma category.

\begin{theorem}\cite[Theorem~4.20]{FGR75}\label{thm:findim_of_comma_cat}
	The finitistic dimension of the comma category $(F, \mathcal B)$ is bounded above by $\findim(\mathcal A) + \findim(\mathcal B) + 1$.
	\begin{proof}
		Let $X=(A, B, f)$ be an element of the comma category with finite projective dimension. Let $P_A^\bullet$ be a projective resolution of $A$ shorter than $\findim(\mathcal A)$. Similar to what we did in \cref{prop:pd_in_commacat} define $P_X^0$ to be $T(P_A^0, P(\cok f))$ where $P(\cok f)$ is a projective module with an epimorphism onto $\cok f$. Then let the kernel of $P_X^0 \to X$ be $(\Omega A, K^0, \theta^0)$. We continue inductively, defining $P_X^n$ to be $T(P_A^n, \cok \theta^{n-1})$. Then $\Omega^{\findim(\mathcal A)+1} X = (0, K^{\findim(\mathcal A)}, 0)$. Then by \cref{prop:pd_in_commacat} we know that $\pd\Omega^{\findim(\mathcal A)+1}X = \pd K^{\findim(\mathcal A)} \leq \findim(\mathcal B)$. So $$\pd X \leq \findim(\mathcal A) + \findim(\mathcal B) + 1.$$
	\end{proof}
\end{theorem}

Before applying this to triangular matrix rings, let us have a look at a simple example.

\begin{example}\label{ex:triangular_matrix_ring}
	If $k$ is a field, $\mathcal A = \mathcal B = \mod k$, and $F$ is the identity, then the comma category $(F, \mathcal B)$ is equivalent to the category of finite dimensional representations of $\mathbb A_2$ over $k$. 
\end{example}

In this example $\mathcal A$ and $\mathcal B$ both have finitistic dimension 0, while $(F, \mathcal B)$ has finitistic dimension 1. So the bound shown above is tight. 

\begin{defn}[Triangular matrix ring]
	Let $R$ and $S$ be rings, and let $M$ be an $S$-$R$-bimodule. Then the \emph{triangular matrix ring} $\begin{pmatrix}
	R & 0\\
	M & S
	\end{pmatrix}$ is the ring of all matricies $\begin{bmatrix}
	r & 0\\
	m & s
	\end{bmatrix}$ with $r\in R$, $s\in S$, and $m\in M$. The multplication is given by
	$$\begin{bmatrix}
	r & 0\\
	m & s
	\end{bmatrix}\begin{bmatrix}
	r' & 0\\
	m' & s'
	\end{bmatrix}=\begin{bmatrix}
	rr' & 0\\
	mr' + sm' & ss'
	\end{bmatrix}.$$
\end{defn}

We have already hinted at an example of this in \cref{ex:triangular_matrix_ring}. The algebra $k\mathbb A_2$ is isomorphic to the matrix ring $\begin{pmatrix}
k & 0\\
k & k
\end{pmatrix}$, and we saw how $\mod k\mathbb A_2$ becomes the comma category for a functor between $\mod k$ and $\mod k$. In fact whenever $\Lambda$ is a triangular matrix ring, the module category $\mod \Lambda$ will be the comma category for a specific functor.

\begin{prop}\label{prop:triangular_matrix_is_comma_cat}
	If $\Lambda = \begin{pmatrix}
	R & 0\\
	M & S
	\end{pmatrix}$ is a triangular matrix ring and $M$ is finitely generated as an $S$-module, then $\mod \Lambda$ is isomorphic to the comma category $(M \otimes_R -, \mod S)$. In particular this holds if $\Lambda$ is also a finite dimensional algebra.
	\begin{proof}
		Notice, if $N$ is a $\Lambda$-module, then as an abelian group $N$ splits as a direct sum into
		$$N= N_R \oplus N_S :=
		\begin{bmatrix}
		1 & 0\\
		0 & 0
		\end{bmatrix}N \oplus
		\begin{bmatrix}
		0 & 0\\
		0 & 1
		\end{bmatrix}N.$$
		
		By restriction of scalars we can think of $N_R$ as an $R$-module and $N_S$ as an $S$-module. Further multiplication by $\begin{bmatrix}
		0 & 0\\
		m & 0
		\end{bmatrix}$ is 0 on $N_S$ and maps $N_R$ into $N_S$. So $N$ consists of an $R$-module $N_R$, an $S$-module $N_S$ and a $S$-$R$-linear map $M \to \Hom_\mathbb{Z}(N_R, N_S)$, or equivalently an $S$-linear map $M \otimes_R N_R \to N_S$.
		
		This gives us the equivalence between $\mod \Lambda$ and $(M \otimes_R -, \mod S)$.
	\end{proof}
\end{prop} 

\begin{cor}
	When $\Lambda$ is the triangular matrix algebra above, then 
	$$\findim(\Lambda) \leq \findim(R) + \findim(S)+1.$$
\end{cor}

\subsection{Recollements for triangular matrix rings}\label{sec:recollemt_of_triangular_rings}

There is an analogues definition of recollement between abelian categories. If $\Lambda$ is a triangulated matrix algebra as above then we do get a recollement of abelian categories

\begin{center}
	\begin{tikzcd}[column sep=4cm]
		\mod S \ar[r, ""{name=i}]{}{\operatorname{inc}} & 
		\ar[l, swap, ""{name=il}, bend right=30]{}{\Lambda / \Lambda e_R \Lambda \otimes_\Lambda -} \ar[l, ""{name=ir}, bend left=30]{}{\Hom(\Lambda e_S, -)}
		\mod \Lambda \ar[r, ""{name=j}]{}{\Hom(\Lambda e_R, -)=e_R\Lambda \otimes-} & 
		\ar[l, swap, ""{name=jl}, bend right=30]{}{\Lambda e_R \otimes -} \ar[l, ""{name=jr}, bend left=30]{}{\Hom(e_R\Lambda, -)}
		\mod R
		\arrow[phantom, from=il, to=i, "\dashv" rotate=-90]
		\arrow[phantom, from=i, to=ir, "\dashv" rotate=-90]
		\arrow[phantom, from=jl, to=j, "\dashv" rotate=-90]
		\arrow[phantom, from=j, to=jr, "\dashv" rotate=-90]
	\end{tikzcd}	
\end{center}

\todo{something about recoll of abcats}

By taking derived functors we get a recollement of unbounded derived categories, which also restricts to a recollement between $D^-(S)$, $D^-(\Lambda)$ and $D^-(R)$\cite[Corollary~15]{Ko91}.

This does not in general restrict to a recollement of bounded derived categories, but if $M$ has finite projective dimension both as an $R$-module and an $S$-module then it does. 


% Does the same idea of recollement works if we consider $\D^-$ instead of $\D^b$? Clearly not since Happell restricts to $\D^b$, where is does it break? ANSWER: In $\D^-$ we cannot characterize bounded complexes of injectives in the same way, so we cannot bound $H^i(X)$ from below. 

% Note $R = e_R\Lambda e_R$ and $S=\Lambda/\Lambda e_R \Lambda$.

% This gives a recollement of abelian categories. Under what conditions does this extend to one for derived categories?? When $M$ is projective as $S$ and $R$ module then the top functors are exact. The other functors are always exact. If $M$ has finite projective dimension then the derived functors should be well defined on the bounded derived category. Can we say something meaningful in the unbounded case \todo{?}
