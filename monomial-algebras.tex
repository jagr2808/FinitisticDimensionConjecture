\cite{GKK91, IgZa90}

In this section we will show a particularly nice way to construct a minimal projective resolution of the right module $\Lambda / J$ for a monomial algebra $\Lambda$. We will use this to compute $\Tor_i(\Lambda /J, M)$ and/or $\Ext^i(M, D\Lambda/J)$ to get a bound on the projective dimension of all modules $M$.

\begin{defn}[Monomial algebra]
	A \emph{monomial algebra} is a path algebra with admissible relations that are generated by monomials. That is, we do not allow the generators for the relations to consist of nontrivial linear combinations of paths.
\end{defn}
\todo{We may assume rho contains J2}
\begin{defn}[$m$-chains]\cite{GKK91}
	Let $\Lambda = k\Gamma / (\rho)$ be a monomial algebra, with $\rho$ a minimal generating set of paths. As usual we define $\Gamma_0$ to be the vertices of $\Gamma$, and $\Gamma_1$ to be the arrows. Recursively define the set of $(m-1)$-chains, $\Gamma_m$, as the paths $\gamma$ with the following criteria:
	\begin{enumerate}[i)]
		\item $\gamma = \beta\delta\tau$ with $\beta \in \Gamma_{m-2}$, $\beta\delta \in \Gamma_{m-1}$, and $\tau$ a non-zero path of length at least 1.
		\item $\delta\tau$ is 0 in $\Lambda$, i.e. it is in the ideal of relations.
		\item $\gamma$ is left-minimal in the sense that if $\gamma = \gamma' \sigma$ such that $\gamma'$ satisfies the above conditions, then $\gamma = \gamma'$.
	\end{enumerate}
\end{defn}

The sets of $m$-chains will become the generating sets for the projectives in our projective resolution. But first we will prove some properties of them.

\begin{lemma}
	Any $\gamma\in \Gamma_m$ for $m \geq 1$ can be factored uniquely as $\gamma_1\gamma_0$ with $\gamma_1 \in \Gamma_{m-1}$, and $\gamma_0$ a non-zero path of length at least 1.
	\begin{proof}
		When $m=1$ this should be clear, since $\Gamma_1$ is the set of arrows, and $\Gamma_0$ is the set of vertices, so if $\gamma \in \Gamma_1$ is an arrow $i\to j$ then $\gamma = e_j\gamma$.
		
		When $m > 1$ we know from the definition of $\Gamma_m$ that $\gamma$ can be written as $\gamma_1\gamma_0$. Assume there is another decomposition $\gamma = \gamma'_1\gamma'_0$. Then without loss of generality we may assume that $\gamma'_1$ is shorter than $\gamma_1$. Then there is a $\sigma$ such that $\gamma'_1\sigma = \gamma_1$. By minimality this means that $\gamma'_1=\gamma_1$, and so the decomposition is unique.
	\end{proof} 
\end{lemma} 

From now on we will write $R$ for the ring $\Lambda/J$. Let $k\Gamma_m$ be the free vectorspace generated by $\Gamma_m$. Notice that $k\Gamma_m$ has a canonical structure as a $R$-$R$-bimodule. This means we can get projective right $\Lambda$-modules $P^m := k\Gamma_m\otimes_R\Lambda$.

\begin{prop}
Define the map $\delta_m \colon P^m \to P^{m-1}$ by $\delta_m(\gamma \otimes \alpha) = \gamma_1 \otimes \gamma_0\alpha$ where $\gamma_1\gamma_0$ is the unique decomposition of $\gamma$, and define $\delta_0 \colon P^0 \to \Lambda /J$ by $\delta_0(e_i\otimes \alpha) = e_i\alpha + J$. Then we have a minimal projective resolution of the right $\Lambda$-module $\Lambda/J$ by

\begin{center}
\begin{tikzcd}
	\cdots \ar[r] & P^3 \ar[r, "\delta_3"] & P^2 \ar[r, "\delta_2"] & P^1 \ar[r, "\delta_1"] & P^0 \ar[d, two heads, "\delta_0"]\ar[r] & 0\\
	&&&&\Lambda / J
\end{tikzcd}
\end{center}

\begin{proof}
	For all $i$ the module $P^i$ is projective as a right $\Lambda$-module and the image of $\delta_m$ is clearly contained in $P^{m-1}J$, so the only thing left to show is exactness. First we show that $\delta_m\delta_{m-1}=0$. Let $\gamma\otimes \alpha$ be in $P^m$ for $m \geq 2$. Then we can decompose $\gamma$ uniquely as $\gamma_2\gamma_1\gamma_0$ and $\delta_m\delta_{m-1}(\gamma\otimes \alpha) = \gamma_2\otimes\gamma_1\gamma_0\alpha$. By the way we defined $\Gamma_m$, $\gamma_1\gamma_0$ is 0 in $\Lambda$, and so $\gamma_2\otimes\gamma_1\gamma_0\alpha = 0$.
	
	Next we want to show that $\Ker\delta_{m-1} \subseteq \Image\delta_m$. Let $x = \sum \gamma^i\otimes \alpha^i$ be in $\Ker\delta_{m-1}$.  Decomposing each $\gamma^i$ into $\gamma_1^i\gamma_0^i$ we get $\sum \gamma_1^i \otimes \gamma_0^i\alpha^i=0$. Grouping like terms and relabeling we get an expression of the form 
	$$x = \sum_j \gamma_j \otimes \sum_{k=0}^{n_j} \gamma_j^{k}\alpha_j^{k}=0$$ 
	with $\gamma_i \neq \gamma_j$ and $\gamma_j^k \neq \gamma_j^l$ when $i\neq j$ and $k\neq l$. We deduce that 
	$$\sum_{k=0}^{n_j} \gamma_j^{k}\alpha_j^{k}=0.$$ 
	Since $\Lambda$ only has monomial relations this means that $\gamma_j^{k}\alpha_j^{k}=0$. Because of this we have that $\gamma_j\gamma_j^{k}\alpha_j^{k}=\zeta_j^k\sigma_j^k$ for some $m$-chain $\zeta_j^k$ and $\sigma_j^k$ some path (possibly of length 0). This gives us that $x$ is the image of $$\sum \zeta_j^k\otimes \sigma_j^k$$ 
	by $\delta_m$. Hence $\Ker\delta_{m-1} \subseteq \Image\delta_m$, and the sequence is exact. So this gives a minimal projective resolution of $\Lambda/J$ as a right $\Lambda$-module.
\end{proof}
\end{prop}

\begin{defn}
	We call a path $\tau$ in $\Gamma$ a \emph{special segment} for $\Lambda = k\Gamma/(\rho)$ if there is a path $\gamma$ such that $\gamma\tau$ is a minimal relation.
\end{defn}

\begin{prop}
	Let $d$ be the number of special segments for $\Lambda$. If $s \geq d+3$ and $\gamma$ is in $\Gamma_s$, then for any integer $N$ there is an $n \geq N$ and a $\gamma' \in \Gamma_n$ such that for any path $\tau$ we have $\gamma\tau \in \Gamma_{s+1}$ if and only if $\gamma'\tau \in \Gamma_{n+1}$.
	\todo{finish}
\end{prop}
