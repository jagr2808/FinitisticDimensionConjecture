\section{Monomial algebras}\label{sec:monomial_algebras}
%\cite{GKK91, IgZa90}

In this section we show a particularly nice way to construct a minimal projective resolution of the right module $\Lambda / J$ for a monomial algebra $\Lambda$. We use this to compute $\Tor_i(\Lambda /J, M)$ to get a bound on the projective dimension of all modules $M$.

In \cref{prop:projective_res_of_top_monomial_alg} we define the projective resolution. Then in \cref{thm:monomial_algebra_satisafies_FDC} we use this to get a bound ion the finitistic dimension, giving us that monomial algebras satisfy the finitistic dimension conjecture.

\begin{defn}[Monomial algebra]
	A \emph{monomial algebra} is a path algebra with admissible relations that are generated by monomials. That is, we do not allow the generators for the relations to consist of nontrivial linear combinations of paths.
\end{defn}

From now on we will assume that the relations of our algebra are contained in $J^2$. If our relations includes an arrow or a vertex, we may simply replace our quiver by one where said vertex or arrow is removed. Thus we do not lose any generality by assuming this.

We will now define the set of $m$-chains, which will serve as a basis for our projective resolution.

\begin{defn}[$m$-chains]\cite{GKK91}
	Let $\Lambda = kQ / (\rho)$ be a monomial algebra, with $\rho$ a minimal generating set of paths. As usual we define $Q_0$ to be the vertices of $Q$, and $Q_1$ to be the arrows. Recursively define the set of $(m-1)$-chains, $Q_m$, as the paths $\gamma$ with the following criteria:
	\begin{enumerate}[i)]
		\item $\gamma = \beta\delta\tau$ with $\beta \in Q_{m-2}$, $\beta\delta \in Q_{m-1}$, and $\tau$ a non-zero path of length at least 1.
		\item $\delta\tau$ is 0 in $\Lambda$, i.e. it is in the ideal of relations.
		\item $\gamma$ is left-minimal in the sense that if $\gamma = \gamma' \sigma$ such that $\gamma'$ satisfies the above conditions, then $\gamma = \gamma'$.
	\end{enumerate}
\end{defn}

Before we can construct our projective resolution we will need a key property of $m$-chains. 

\begin{lemma}\label{lem:unique_factorization_of_chains}
	Any $\gamma\in Q_m$ for $m \geq 1$ can be factored uniquely as $\gamma_1\gamma_0$ with $\gamma_1 \in Q_{m-1}$, and $\gamma_0$ a non-zero path of length at least 1.
	\begin{proof}
		When $m=1$ this should be clear, since $Q_1$ is the set of arrows, and $Q_0$ is the set of vertices, so if $\gamma \in Q_1$ is an arrow $i\to j$ then $\gamma = e_j\gamma$.
		
		When $m > 1$ we know from the definition of $Q_m$ that $\gamma$ can be written as $\gamma_1\gamma_0$. Assume there is another decomposition $\gamma = \gamma'_1\gamma'_0$. Then without loss of generality we may assume that $\gamma'_1$ is shorter than $\gamma_1$. Then there is a $\sigma$ such that $\gamma'_1\sigma = \gamma_1$. By minimality this means that $\gamma'_1=\gamma_1$, and so the decomposition is unique.
	\end{proof} 
\end{lemma} 

From now on we write $R$ for the ring $\Lambda/J$, which we identify with the subring of $\Lambda$ generated by the paths of length 0. Let $kQ_m$ be the free vector space generated by $Q_m$. Notice that $kQ_m$ has a canonical structure as an $R$-$R$-bimodule. This means we can construct projective right $\Lambda$-modules by $P^m := kQ_m\otimes_R\Lambda$.

\begin{prop}\label{prop:projective_res_of_top_monomial_alg}
We define a map $\delta_m \colon P^m \to P^{m-1}$ by the formula $\delta_m(\gamma \otimes \alpha) = \gamma_1 \otimes \gamma_0\alpha$, where $\gamma_1\gamma_0$ is the unique decomposition of $\gamma$, and we define $\delta_0 \colon P^0 \to \Lambda /J$ by $\delta_0(e_i\otimes \alpha) = e_i\alpha + J$. Then we get a minimal projective resolution of the right $\Lambda$-module $\Lambda/J$ by

\begin{center}
\begin{tikzcd}
	\cdots \ar[r] & P^3 \ar[r, "\delta_3"] & P^2 \ar[r, "\delta_2"] & P^1 \ar[r, "\delta_1"] & P^0 \ar[d, two heads, "\delta_0"]\ar[r] & 0\\
	&&&&\Lambda / J
\end{tikzcd}
\end{center}
\end{prop}

Before proving this proposition we require the following lemma.

\begin{lemma}\cite[Lemma~2.1]{GKK91}\label{lem:decompose_kernel_delta_m}
	Let $M$ be a $\Lambda$-module, and $x$ an element in the kernel of of $\delta_m\otimes M\colon kQ_m\otimes_R M \to kQ_{m-1} \otimes_R M$. Write $x$ on the form
	$$x = \sum_j\sum_{k=0}^{n_j} \gamma_j  \gamma_j^k \otimes m_j^k$$
	with $\gamma_i \in Q_{m-1}$ and $\gamma_i \neq \gamma_j$ when $i \neq j$ and $\gamma_j^k \neq \gamma_j^l$  when $k \neq l$. Then 
	$$\sum_{k=0}^{n_j} \gamma_j  \gamma_j^k \otimes m_j^k$$
	is also in the kernel for each $j$.
	\begin{proof}
		Let $x$ be as given above. Applying $\delta_m\otimes M$ we get that 
		$$\sum_j \gamma_j \otimes \sum_{k=0}^{n_j} \gamma_j^{k}m_j^{k}=0.$$ 
		Since the $\gamma_j$s are distinct we can deduce that 
		$$ \sum_{k=0}^{n_j} \gamma_j^{k}m_j^{k}=0.$$ 
		From this it follows that
		$$\sum_{k=0}^{n_j} \gamma_j  \gamma_j^k \otimes m_j^k$$
		is also in the kernel of $\delta_m \otimes M$.
	\end{proof}
\end{lemma}

Using this lemma we can now prove the proposition.

\begin{proof}[Proof of \cref{prop:projective_res_of_top_monomial_alg}]
	For all $i$ the module $P^i$ is projective as a right $\Lambda$-module and the image of $\delta_m$ is clearly contained in $P^{m-1}J$, so the only thing left to show is exactness. First we show that $\delta_m\delta_{m-1}=0$. Let $\gamma\otimes \alpha$ be in $P^m$ for $m \geq 2$. Then we can decompose $\gamma$ uniquely as $\gamma_2\gamma_1\gamma_0$ and $\delta_m\delta_{m-1}(\gamma\otimes \alpha) = \gamma_2\otimes\gamma_1\gamma_0\alpha$. By the way we defined $Q_m$, $\gamma_1\gamma_0$ is 0 in $\Lambda$, and so $\gamma_2\otimes\gamma_1\gamma_0\alpha = 0$.
	
	Next we want to show that $\Ker\delta_{m-1} \subseteq \Image\delta_m$. Let $x$ be in the kernel of $\delta_{m-1}$. By \cref{lem:decompose_kernel_delta_m} it is sufficient to assume $x$ is of the form
	$$\sum_k \gamma  \gamma_k \otimes \alpha_k$$
	with $\gamma \in Q_{m-2}$ and the $\gamma_k$s all distinct. Then $\sum_k \gamma_k\alpha_k = 0$. By the minimality conditions in the way we define $m$-chains we have that none of the $\gamma_k$s divide each other on the left. Since $\Lambda$ only has monomial relations, this gives us that $\gamma_{k}\alpha_{k}=0$.
	
	Because of this we have that $\gamma\gamma_k\alpha_k=\zeta_k\sigma_k$ for some $m$-chain $\zeta_k$ and some path $\sigma_k$ (possibly of length 0). This gives us that $x$ is the image of 
	$$\sum_k \zeta_k\otimes \sigma_k$$ 
	by $\delta_m$. Hence $\Ker\delta_{m-1} \subseteq \Image\delta_m$, and the sequence is exact. So this gives a minimal projective resolution of $\Lambda/J$ as a right $\Lambda$-module.
\end{proof}

The next thing we will do is find a repeating pattern in this resolution to aid us in bounding projective dimensions. To do this we introduce the concept of a special segment.

\begin{defn}[Special segments]
	We call a path $\tau$ in $Q$ a \emph{special segment} for $\Lambda = kQ/(\rho)$ if there is a path $\gamma$ such that $\gamma\tau$ is a minimal relation.
\end{defn}

Note that when we decompose an $m$-chain $\gamma$ in \cref{lem:unique_factorization_of_chains} into $\gamma_1\gamma_0$, then $\gamma_0$ is a special segment, and that the set of special segments is finite.

\begin{lemma}\cite[Theorem~2.2]{GKK91}\label{lem:monomial_relation_repetition}
	Let $d$ be the number of special segments for $\Lambda$. If $s \geq d+3$ and $\gamma$ is in $Q_s$, then for any integer $N$ there is an $n \geq N$ and a $\hat{\gamma} \in Q_n$ such that for any path $\tau$ and any integer $r \geq 1$ we have $\gamma\tau \in Q_{s+r}$ if and only if $\hat{\gamma}\tau \in Q_{n+r}$.
	\begin{proof}
		Applying \cref{lem:unique_factorization_of_chains} recursively we get that $\gamma$ can be written as $\gamma = \tau_0\tau_1\cdots \tau_{s-1}$ where $\tau_0\tau_1 \cdots \tau_{i-1} \in Q_i$. In particular each $\tau_i$ is a special segment.
		
		Since $s \geq d+3$ we must have that there exists $i$ and $j$, $1\leq i < j \leq s-1$ such that $\tau_i=\tau_j$. Let $\beta = \tau_{i+1}\tau_{i+2}\cdots\tau_j$. Then $$\gamma_k := \tau_0\tau_1\cdots\tau_{j-1}\tau_j\beta^k\tau_{j+1}\cdots\tau_{s-1} \in Q_{s + k(j-i)}$$
		where $\beta^k$ means $\beta$ repeated $k$ times. If we now choose $k$ large enough such that $s+k(j-i) \geq N$ we can choose $n=s+k(j-i)$ and $\hat{\gamma}=\gamma_k$. Then we see that for any path $\tau$, the composition $\gamma\tau$ is in $Q_{s+r}$ if and only if $\hat{\gamma}\tau$ is in $Q_{n+r}$.
	\end{proof}
\end{lemma}

This gives us a pattern in the projective resolution that we now use to bound the finitistic dimension of our algebra.

\begin{theorem}\cite[Corollary~2.4]{GKK91}\label{thm:monomial_algebra_satisafies_FDC}
	Let $\Lambda = kQ/(\rho)$ be a monomial relation algebra. Then $\findim(\Lambda) \leq d+3$ where $d$ is the number of special segments for $\Lambda$.
	\begin{proof}
		Let $M$ be a module of finite projective dimension and let $N$ be $\pd M$. The projective dimension of $M$ can be characterized as the largest integer $c$ such that $\Tor_c(\Lambda/J, M) \neq 0$. We show that this is at most $d+3$. Let $s \geq d+3$ be an integer. Then we want to show that $\Tor_{s+1}(\Lambda/J, M)=0$. We compute this by taking the projective resolution of $\Lambda/J$ found in \cref{prop:projective_res_of_top_monomial_alg} and tensoring with $M$.
		\begin{center}
			\begin{tikzcd}
				\cdots \ar[r] & kQ_{s+2} \otimes M  \ar[r, "\delta_{s+2}\otimes M"] & kQ_{s+1} \otimes M \ar[r, "\delta_{s+1}\otimes M"]  & kQ_{s} \otimes M \ar[r] & \cdots
			\end{tikzcd}
		\end{center}
		Let $x$ be in the kernel of $\delta_{s+1}\otimes M$. Then by \cref{lem:decompose_kernel_delta_m} we may assume $x$ is on the form
		$$x = \sum_j \gamma \gamma_j \otimes m_j$$
		with $\gamma$ in $Q_s$ and all the $\gamma_j$s distinct. Then \cref{lem:monomial_relation_repetition} gives us that there is an $n \geq N$ and a $\hat{\gamma} \in Q_n$ such that $\gamma\tau$ is in $Q_{s+r}$ if and only if $\hat{\gamma}\tau$ is in $Q_{n+r}$.
		
		Then $\hat{x} := \sum \hat{\gamma}\gamma_j \otimes m_j$ is in the kernel of $\delta_{n+1}\otimes M$. Since $n+1>N=\pd M$ the complex is exact at $n+1$. This means that there are elements $\gamma_j^k$ and $m_j^k$ such that
		$$\hat{x} = \delta_{n+2} \left(\sum_j \sum_{k=0}^{n_j} \hat{\gamma}\gamma_j\gamma_j^k \otimes m_j^k\right) = 
		\sum_j \sum_{k=0}^{n_j} \hat{\gamma}\gamma_j \otimes \gamma_j^k m_j^k$$
		Since $\hat{\gamma}\gamma_j\gamma_j^k$ is in $Q_{n+2}$ if and only if $\gamma\gamma_j\gamma_j^k$ is in $Q_{s+2}$ we have that
		$$x = \delta_{s+2} \left(\sum_j \sum_{k=0}^{n_j} {\gamma}\gamma_j\gamma_j^k \otimes m_j^k\right)$$
		and thus $\Tor_{s+1}(\Lambda/J, M)=0$ so $\pd M \leq d+3$. Since $M$ was arbitrary this means that $\findim(\Lambda) \leq d+3$. 
	\end{proof}
\end{theorem}



