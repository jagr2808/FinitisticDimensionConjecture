\documentclass[11pt, a4paper, english]{article}
\usepackage[utf8]{inputenc}
\usepackage{babel, amsmath, amsthm, amssymb, amsfonts, enumitem, mathtools, centernot}
\usepackage{tikz-cd}
\usepackage{tikz-3dplot}
\usepackage{caption}
\usepackage{intcalc}
\usepackage{stmaryrd}
\usepackage{multicol}
\usepackage{cite}
\usepackage{hyperref}
\usepackage{cleveref}
\usepackage[toc,page]{appendix}
\usepackage{fancyhdr}
\usepackage{todonotes}
\newcommand\tab[1][1cm]{\hspace*{1}}
\DeclarePairedDelimiter{\ceil}{\lceil}{\rceil}

\newtheorem{theorem}{Theorem}[section]
\newtheorem{cor}{Corollary}[theorem]
\newtheorem{prop}[theorem]{Proposition}
\newtheorem{lemma}[theorem]{Lemma}
\theoremstyle{definition}
\newtheorem{defn}[theorem]{Definition}
\newtheorem{example}[theorem]{Example}

\newcommand{\C}{\mathbb{C}}
\newcommand{\Z}{\mathbb{Z}}
\DeclareMathOperator{\Hom}{Hom}
\DeclareMathOperator{\Ext}{Ext}
\DeclareMathOperator{\Tor}{Tor}
\DeclareMathOperator{\End}{End}
\DeclareMathOperator{\Aut}{Aut}
\DeclareMathOperator{\Image}{Im}
\DeclareMathOperator{\Ker}{Ker}
\DeclareMathOperator{\cok}{Cok}
\DeclareMathOperator{\depth}{depth}
\DeclareMathOperator{\inj}{inj}
\DeclareMathOperator{\proj}{proj}
\DeclareMathOperator{\add}{add}
\def\mod{\operatorname{mod}}

\setlength{\parindent}{0em}
\setlength{\parskip}{1em}

\pagestyle{fancy}
\fancyhead{}
%\fancyhead[LO, LE]{\small\emph{McKay correspondence}}
\fancyhead[LO, LE]{\small\nouppercase\rightmark}
\fancyfoot[CO, CE]{\thepage}
%\fancyfoot[RO, RE]{\thepage}
\renewcommand{\sectionmark}[1]{\markboth{}{\emph{\thesection~#1}}}
%\renewcommand{\subsectionmark}[1]{}% Remove \subsection from header

\begin{document}
\title{Finitistic dimension conjecture}
\author{Jacob Fjeld Grevstad}
\date{2020}
\maketitle
\pagenumbering{roman}

\begin{abstract}
FDC yo!
\end{abstract}
\clearpage

\tableofcontents
\clearpage

\pagenumbering{arabic}
\section*{Introduction}
\addcontentsline{toc}{section}{\protect\numberline{}Introduction}%
\markboth{section}{Introduction}

This is an introduction

\section{finitistic dimension and conjectures}

\begin{itemize}
	\item FDC - finitistic dimesnion conjecture
	Finitistic dimension is always finite
	\item WTC - Watamatsu tilting conjecture
	
	\item GSC - Gorenstein symmetry conjecture
	\item NuC - Nunke condition
	\item SNC - strong Nakayama conjecture
	\item ARC - Auslander Reiten conjecture
	\item NC - Nakayama conjecture
\end{itemize}

\subsection{Implications}
\begin{tikzcd}
FDC \ar[r]\ar[d] & WTC \ar[r] & GSC\\
NuC\ar[r] & SNC\ar[r] & ARC\ar[r] & NC
\end{tikzcd}

\begin{theorem} \cite[1.2]{Hap93}
	\item[i)] If $findim(\Lambda) < \infty$ (FDC) then $K^b(\inj\Lambda)^\perp = 0$.
	\item[ii)] If $K^b(\inj\Lambda)^\perp = 0$ then for $X\neq 0$ there exists $i$ such that, $\Ext^i(D(\Lambda), X) \neq 0$ (NuC).
	\begin{proof}
		\item[i)] Let $I^\bullet \in K^b(\inj\Lambda)^\perp$ be non-zero. Since $D^b(\Lambda) \cong K^{+,b}(\inj\Lambda)$ we may assume $I^\bullet$ is a complex of injectives, and WLOG we may assume it concentrated in degrees $i \geq 0$.
		
		$\Hom(D\Lambda, I^i)$ is in $add\Hom(D\Lambda, D\Lambda) = add\Lambda$ so $\Hom(D\Lambda, I^\bullet)$ is a complex of projectives.
		
		\begin{tikzcd}
		0 \ar[r] \ar[d] & D\Lambda \ar[r] \ar[d, "f"] \ar[dl, dashed]& 0 \ar[d]\\
		I^{i-1} \ar[r, "d^{i-1}"] & I^i \ar[r, "d^i"] & I^{i+1}
		\end{tikzcd}
		
		Since $I^\bullet$ is in $K^b(\inj\Lambda)^\perp$ and $D\Lambda$ is in $K^b(\inj\Lambda)$, whenever $d^if=0$, $f^\bullet$ is homotopic to 0. Meaning $f$ factors through $d^{i-1}$. This means that $\Hom(D\Lambda, I^\bullet)$ is an exact complex.
		
		$\cok\Hom(D\Lambda, d^i)$ has a projective resolution of length $i$. This resolution is minimal??????? Hence $findim(\Lambda)$ is infinite. \todo{Probably want to use $I$ indecomposable or something like that}
		
		\item[ii)] Assume there is an $X \neq 0$ with $\Ext^i(D\Lambda, X) = 0$ for all $i \geq 0$. Then $X$ considered as a stalk complex is in $K^b(\inj\Lambda)^\perp$. Proceed by induction: If $I[-i] \in K^b(\inj\Lambda)$ is a stalk complex then $D^b(I[-i], X) = \Ext^i(I, X)$. This is 0 because $D\Lambda$ is the sum of the indecomposable injectives.
		
		Let $I \in K^b(\inj\Lambda)$ be a complex of width $n$. WLOG assume $I$ concentrated in degrees $0 \leq i \leq n-1$. Then $$I^0 \to I \to I^{<0} \to I^0[1]$$ is a triangle, and $I^{<0}$ has width $n-1$. Taking the long exact sequence in $D^b(-,X)$ it follows that $D^b(I, X)=0$.  
	\end{proof}
\end{theorem}

\subsection{Recollement}
\begin{tikzcd}[column sep=5cm]
D^b(\Lambda') \ar[r, "i_*=i_!"{name=i}] & 
\ar[l, swap, "i^*"{name=il}, bend right=30] \ar[l, "i^!"{name=ir}, bend left=30]
D^b(\Lambda) \ar[r, "j^!=j^*"{name=j}] & 
\ar[l, swap, "j_!"{name=jl}, bend right=30] \ar[l, "j_*"{name=jr}, bend left=30]
D^b(\Lambda'')
\arrow[phantom, from=il, to=i, "\dashv" rotate=-90]
\arrow[phantom, from=i, to=ir, "\dashv" rotate=-90]
\arrow[phantom, from=jl, to=j, "\dashv" rotate=-90]
\arrow[phantom, from=j, to=jr, "\dashv" rotate=-90]
\end{tikzcd}

Sort of like a split exact sequence of functors. We want
\begin{enumerate}
	\item All functors are exact/triangulated
	\item $j^*i_*=0$
	\item $i^*i_* \cong i^!i_! \cong id$ (induced by unit/counit)
	\item $j^!j_! \cong j^*j_* \cong id$
	\item 
	\begin{tikzcd}
	j_!j^!X \ar[r, "\varepsilon"] & X \ar[r, "\eta"] & i_*i^* X \ar[r] & \Sigma\\		
	i_!i^!X \ar[r, "\varepsilon"] & X \ar[r, "\eta"] & j_*j^* X \ar[r] & \Sigma\\
	\end{tikzcd}
	
	Are triangles in $D^b(\Lambda)$
\end{enumerate}

\begin{theorem}
	Given a recollement FDC holds for midlle if and only if it holds for the two others.
	\begin{proof}
		write later..... Happel reduct technich \cite[3.3]{Hap93}
	\end{proof}
\end{theorem}

\subsection{Contravariant finiteness}

\begin{defn}[Resolving]
	A full subcategory of an abelian category is called resolving if 
	\begin{itemize}
		\item It is closed under extensions
		\item It contains the projectives
		\item It is contains the kernels of its epimorphisms
	\end{itemize}
\end{defn}

Note that the subcategory of modules with finite projective dimension is resolving.

\begin{theorem} \cite[3.8]{AR91}
	Let $\mathcal X$ be a contravariantly finite, resolving subcategory of $\mod \Lambda$. Let $X_i$ be the minimal approximation of $S_i$. Then any $X \in \mathcal X$ is a direct summand of an $X_i$-filtered module.
	\begin{proof}
		Step 1: We want to show by induction on length that any module $C$ is in an exact sequence $0 \to Y \to X \to C \to 0$ with $X$ $X_i$-filtered and $\Ext^1(\mathcal X, Y)=0$.
		
		Step 2: Whenever $C$ is in $\mathcal X$ we get that
		
		\begin{tikzcd}
		\Hom(C, X) \ar[r] & \Hom(C, C) \ar[r] & \Ext^1(C,Y) = 0
		\end{tikzcd}
		is exact, and thus $C$ is a direct summand of $X$.
	\end{proof}
\end{theorem}

\begin{cor}
	If the subcategory of modules with finite projective dimension is contravariantly finite, then the finitistic dimension is the supremum of the projective dimension of $X_i$. In particular it is finite.
\end{cor}

\section{repdimension}
\begin{defn}[dominated dimension]
	Let \begin{tikzcd}
		\Lambda \ar[r] & I_0 \ar[r] & I_1 \ar[r] & \cdots
	\end{tikzcd}
	be a minimal injective resolution of $\Lambda$. Then the dominated dimension of $\Lambda$ is $\inf\{n | I_n$ is not projective \}.
\end{defn}

\begin{defn}[rep-dimesnion]
	Let $A = \{\Gamma | domdim\Gamma \geq 2, \Lambda \text{ morita eqquivalent to } \End_\Gamma{I_0(\Gamma)}\}$ where $I_0(\Gamma)$ is the injective envelope of $\Gamma$. Then the repdimesnion of $\Lambda$ is the minimal global dimension of $\Gamma \in A$.
\end{defn}

\begin{prop}
	(all modules ar right modules)
	Repdim is the same as minimal global dimension of $\End(M)$ for $M$ being both a generator and cogenerator. 
	\begin{proof}
		Consider $\Gamma \in A$. Since $domdim\Gamma \geq 1$, $I_0(\Gamma)$ is the sum of all projective-injective modules (some probably several times). 
		
		Let $\mathcal S$ be the set of all $\Gamma$-modules with a copresentation
		\begin{center}
		\begin{tikzcd}
			0 \ar[r] & X \ar[r] & I_0 \ar[r] & I_1
		\end{tikzcd}
		\end{center}
		with $I_i$ in $\add I_0(\Gamma)$. In particular $\Gamma$ is in $\mathcal S$, because $domdim\Gamma \geq 2$.
		
		The Yoneda embedding gives an equivalence $$\Hom_\Gamma(-,I_0(\Gamma)):\add I_0(\Gamma) \to \proj\End_\Gamma(I_0(\Gamma))^{op}$$, and thus we get an equivalence $$D\Hom_\Gamma(-,I_0(\Gamma)):\add I_0(\Gamma) \to \inj\End_\Gamma(I_0(\Gamma))$$
		Since $I_0(\Gamma)$ is injective $D\Hom(-,I_0(\Gamma))$ is exact and preserves kernels, so extends to an equivalence
		$$\Hom_\Gamma(-,I_0(\Gamma)):\mathcal S \to \mod \End_\Gamma(I_0(\Gamma))$$
		
		Since $\End_\Gamma(I_0(\Gamma))$ is morita equivalent to $\Lambda$, $\mathcal S$ is equivalent to $\mod \Lambda$. $\Gamma \in \mathcal S$ is clearly a generator. To see that it is a cogenerator note that $\Gamma$ contains all the projective-injective indecomposable objects as direct summands, so there is an injection $I_0(\Gamma) \to \Gamma^n$, and since $I_0(\Gamma)$ is a cogenerator in $\mathcal S$, $\Gamma$ is aswell.
		
		Thus by the equivalence $\mathcal S \to \mod\Lambda$ there is a cogenerator-generator object $M$ such that $\End_\Lambda(M) = \End_\Gamma(\Gamma)=\Gamma$.
		
		The last step of the proof is showing that $\End(M)$ is in $A$ whenever $M$ is a generator-cogenerator.
		
		Let $0 \to M \to I_0(M) \to I_1(M)$ be an injective copresentation of $M$. Since $M$ is a cogenerator $I_i(M)$ is in $\add M$, thus we get an exact sequence of projective $\End(M)$-modules
		\begin{align} \label{eq:injective_copres_of_endM}
			0 \to \End(M) \to \Hom(M, I_0(M)) \to \Hom(M, I_1(M)).
		\end{align}
		Now we have the following isomorphisms of $\Lambda$-$\End(M)$-bimodules
		\begin{align*}
			\Hom_\Lambda(M, D\Lambda) =\\
			\Hom_k(M\otimes\Lambda, k) =\\
			\Hom_k(M, k) =\\
			DM =\\
			D\Hom_\Lambda(\Lambda, M)
		\end{align*}
		Since $\Lambda$ is in $\add M$, $\Hom(\Lambda, M)$ is projective, and thus $D\Hom(\Lambda, M) = \Hom(M, D\Lambda)$ is injective. This means that (\ref{eq:injective_copres_of_endM}) is an injective copresentation, and thus $domdim\End(M) \geq 2$.
		
		Let $I=I_0(M)$, then $\Hom(I, \Lambda)$ and $I = \Hom(\Lambda, I)$ are bimodules. Need some kind of morita theorem here?????????????????? \todo{I think this is specific to artin algebras}
	\end{proof}
\end{prop}

\begin{defn}
	Let $X$ be an object of $\mod\Lambda$ and $M$ a contravariantaly finite subcategory.
	\begin{center}
	\begin{tikzcd}
		\cdots \ar[rd, two heads] \ar[r] & M_2 \ar[rd, two heads] \ar[r] & M_1 \ar[rd, two heads] \ar[r] & M_0 \ar[rd, two heads]\\
		&\Omega_M^3 X \ar[u, hook] & \Omega_M^2  \ar[u, hook] X & \Omega_M X  \ar[u, hook] & X
	\end{tikzcd}
	\end{center}
	If $\twoheadrightarrow$ are minimal $M$-approximations (they need not be surjective), and $\hookrightarrow$ are their kernels, then this is an $M$-resolution of $X$. The $M$-res-dimesnion of $X$ is the length of the sequence of (nonzero) $M_i$s, and the $M$-res-dimesnion of $\Lambda$ is the supremum of the dimension on its objects.

\end{defn}

\begin{prop}
	$repdim-2$ is the minimum of $M$-res-dim$(\mod \Lambda)$ for $M$ both generator and cogenrator.
\end{prop}

\begin{theorem}
	FDC holds for repdim < 3, \cite[cor 8,9]{IgTo05}
\end{theorem}

\clearpage

\bibliography{mybib}
%\bibliography{mybib}
%\bibliography{intro_bib}
\bibliographystyle{alpha}
\end{document}