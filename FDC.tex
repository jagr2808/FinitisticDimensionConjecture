\documentclass[11pt, a4paper, english]{article}
\usepackage[utf8]{inputenc}
\usepackage{babel, amsmath, amsthm, amssymb, amsfonts, enumitem, mathtools, centernot}
\usepackage{tikz-cd}
\usepackage{tikz-3dplot}
\usepackage{caption}
\usepackage{intcalc}
\usepackage{stmaryrd}
\usepackage{multicol}
\usepackage{cite}
\usepackage{hyperref}
\usepackage{cleveref}
\usepackage[toc,page]{appendix}
\usepackage{fancyhdr}
\usepackage{todonotes}
\newcommand\tab[1][1cm]{\hspace*{1}}
\DeclarePairedDelimiter{\ceil}{\lceil}{\rceil}

\newtheorem{theorem}{Theorem}[section]
\newtheorem{cor}{Corollary}[theorem]
\newtheorem{prop}[theorem]{Proposition}
\newtheorem{lemma}[theorem]{Lemma}
\theoremstyle{definition}
\newtheorem{defn}[theorem]{Definition}
\newtheorem{example}[theorem]{Example}

\newcommand{\C}{\mathbb{C}}
\newcommand{\Z}{\mathbb{Z}}
\DeclareMathOperator{\Hom}{Hom}
\DeclareMathOperator{\Ext}{Ext}
\DeclareMathOperator{\Tor}{Tor}
\DeclareMathOperator{\End}{End}
\DeclareMathOperator{\Aut}{Aut}
\DeclareMathOperator{\Image}{Im}
\DeclareMathOperator{\Ker}{Ker}
\DeclareMathOperator{\cok}{Cok}
\DeclareMathOperator{\depth}{depth}
\DeclareMathOperator{\findim}{findim}
\DeclareMathOperator{\Findim}{Findim}
\DeclareMathOperator{\inj}{inj}
\DeclareMathOperator{\proj}{proj}
\DeclareMathOperator{\pd}{pd}
\DeclareMathOperator{\add}{add}
\DeclareMathOperator{\Mod}{Mod}
\def\mod{\operatorname{mod}}

\newcommand{\mymatrix}[1]{\begin{matrix}#1\end{matrix}}

\DeclareMathOperator{\D}{\mathcal{D}}

\setlength{\parindent}{0em}
\setlength{\parskip}{1em}

\pagestyle{fancy}
\fancyhead{}
%\fancyhead[LO, LE]{\small\emph{McKay correspondence}}
\fancyhead[LO, LE]{\small\nouppercase\rightmark}
\fancyfoot[CO, CE]{\thepage}
%\fancyfoot[RO, RE]{\thepage}
\renewcommand{\sectionmark}[1]{\markboth{}{\emph{\thesection~#1}}}
%\renewcommand{\subsectionmark}[1]{}% Remove \subsection from header

\begin{document}
\title{Finitistic dimension conjecture}
\author{Jacob Fjeld Grevstad}
\date{2020}
\maketitle
\pagenumbering{roman}

\begin{abstract}
FDC yo!
\end{abstract}
\clearpage

\tableofcontents
\clearpage

\pagenumbering{arabic}
\section*{Introduction}
\addcontentsline{toc}{section}{\protect\numberline{}Introduction}%
\markboth{section}{Introduction}

This is an introduction

\section{finitistic dimension and conjectures}

\begin{itemize}
	\item FDC - finitistic dimesnion conjecture
	Finitistic dimension is always finite
	\item WTC - Watamatsu tilting conjecture
	
	\item GSC - Gorenstein symmetry conjecture
	\item NuC - Nunke condition
	\item SNC - strong Nakayama conjecture
	\item ARC - Auslander Reiten conjecture
	\item NC - Nakayama conjecture
\end{itemize}

\subsection{Implications}
\begin{tikzcd}
FDC \ar[r]\ar[d] & WTC \ar[r] & GSC\\
NuC\ar[r] & SNC\ar[r] & ARC\ar[r] & NC
\end{tikzcd}

\begin{theorem} \cite[1.2]{Hap93} \label{thm:findim_implies_inj_generate}
	\item[i)] If $\findim(\Lambda) < \infty$ (FDC) then $K^b(\inj\Lambda)^\perp = 0$.
	\item[ii)] If $K^b(\inj\Lambda)^\perp = 0$ then for $X\neq 0$ there exists $i$ such that, $\Ext^i(D(\Lambda), X) \neq 0$ (NuC).
	\begin{proof}
		\item[i)] Let $I^\bullet \in K^b(\inj\Lambda)^\perp$ be non-zero. Since $\D^b(\Lambda) \cong K^{+,b}(\inj\Lambda)$ we may assume $I^\bullet$ is a complex of injectives, and WLOG we may assume it concentrated in degrees $i \geq 0$, and that $d^0:I^0 \to I^1$ is not split mono. Since if its concentrated in degrees $i \geq k$ we can just shift it, and if $d^0$ is split mono then replacing $I^0$ by $0$, and $I^1$ be $I^1/I^0$ gives a homotopic complex.
		
		$\Hom(D\Lambda, I^i)$ is in $\add\Hom(D\Lambda, D\Lambda) = \add\Lambda$ so $\Hom(D\Lambda, I^\bullet)$ is a complex of projectives.
		
		\begin{tikzcd}
		0 \ar[r] \ar[d] & D\Lambda \ar[r] \ar[d, "f"] \ar[dl, dashed]& 0 \ar[d]\\
		I^{i-1} \ar[r, "d^{i-1}"] & I^i \ar[r, "d^i"] & I^{i+1}
		\end{tikzcd}
		
		Since $I^\bullet$ is in $K^b(\inj\Lambda)^\perp$ and $D\Lambda$ is in $K^b(\inj\Lambda)$, whenever $d^if=0$, $f^\bullet$ is homotopic to 0. Meaning $f$ factors through $d^{i-1}$. This means that $\Hom(D\Lambda, I^\bullet)$ is an exact complex. Further since $\Hom(D\Lambda, -)$ is an equivalence between $\inj\Lambda$ and $\proj\Lambda$ we have that $\Hom(D\Lambda, d^0)$ is not split mono.
		
		$\cok\Hom(D\Lambda, d^i)$ has a projective resolution of length $i$. This resolution is the direct sum of the minimal resolution and an acyclic bounded complex of projectives. Since bounded acyclic complexes of projectives are split and $\Hom(D\Lambda, d^0)$ is not, we must have that the minimal resolution has length $i$, and so $\findim(\Lambda) = \infty$.
		
		\item[ii)] Assume there is an $X \neq 0$ with $\Ext^i(D\Lambda, X) = 0$ for all $i \geq 0$. Then $X$ considered as a stalk complex is in $K^b(\inj\Lambda)^\perp$. Proceed by induction: If $I[-i] \in K^b(\inj\Lambda)$ is a stalk complex then $\D^b(I[-i], X) = \Ext^i(I, X)$. This is 0 because $D\Lambda$ is the sum of the indecomposable injectives.
		
		Let $I \in K^b(\inj\Lambda)$ be a complex of width $n$. WLOG assume $I$ concentrated in degrees $0 \leq i \leq n-1$. Then $$I^0 \to I \to I^{<0} \to I^0[1]$$ is a triangle, and $I^{<0}$ has width $n-1$. Taking the long exact sequence in $\D^b(-,X)$ it follows that $\D^b(I, X)=0$.  
	\end{proof}
\end{theorem}

\section{Recollement}
\begin{defn}[Recollement]\label{def:recollement}
	A recollement is a collection of six functors satisfying:
\begin{center}
\begin{tikzcd}[column sep=4cm]
\D^b(\Lambda') \ar[r, "i_*=i_!"{name=i}] & 
\ar[l, swap, "i^*"{name=il}, bend right=30] \ar[l, "i^!"{name=ir}, bend left=30]
\D^b(\Lambda) \ar[r, "j^!=j^*"{name=j}] & 
\ar[l, swap, "j_!"{name=jl}, bend right=30] \ar[l, "j_*"{name=jr}, bend left=30]
\D^b(\Lambda'')
\arrow[phantom, from=il, to=i, "\dashv" rotate=-90]
\arrow[phantom, from=i, to=ir, "\dashv" rotate=-90]
\arrow[phantom, from=jl, to=j, "\dashv" rotate=-90]
\arrow[phantom, from=j, to=jr, "\dashv" rotate=-90]
\end{tikzcd}	
\end{center}

\begin{enumerate}
	\item All functors are exact/triangulated
	\item $j^*i_*=0$
	\item $i^*i_* \cong i^!i_! \cong id$ (induced by unit/counit) \label{item:i_fully-faith}
	\item $j^!j_! \cong j^*j_* \cong id$ \label{item:j_fully-faith}
	\item \label{recollement:triangles}
	\begin{tikzcd}
	j_!j^!X \ar[r, "\varepsilon"] & X \ar[r, "\eta"] & i_*i^* X \ar[r] & j_!j^!X[1]\\		
	i_!i^!X \ar[r, "\varepsilon"] & X \ar[r, "\eta"] & j_*j^* X \ar[r] & i_!i^!X[1]\\
	\end{tikzcd}
	
	Are triangles in $\D^b(\Lambda)$
\end{enumerate}
Note that (\ref{item:i_fully-faith}) and (\ref{item:j_fully-faith}) are equivalent to $i_*$, $j_!$, and $j_*$ are fully faithful.
\end{defn}

\begin{lemma} \label{lem:adjoint_preserves_bounded_proj/inj}
	Let \begin{tikzcd}
	\D^b(\Lambda') \ar[r, "i_*"{name=i}, bend right=20] & 
	\ar[l, swap, "i^*"{name=il}, bend right=20]
	\D^b(\Lambda)
	\end{tikzcd} be exact functors with an adjoint pair $(i^*,i_*)$. Then $i^*$ preserves bounded projective complexes and $i_*$ preserves bounded injective complexes.
	\begin{proof}
		The bounded projective complexes can be categories as the complexes $P$ such that for any complex $Y$ there is an integer $t_Y$ such that $\Hom(P, Y[t])=0$ for $t\geq t_Y$.
		
		Let $P$ be a bounded complex of projectives in $\D^b(\Lambda)$. Then we want to show that $i^*P$ is as well. Let $Y$ be any complex in $\D^b(\Lambda')$. Then $\D^b(\Lambda')(i^*P, Y[t]) = \D^b(\Lambda)(P, i_*Y[t])$, so since $P$ is a bounded complex of projectives there is $t_Y$ such that this vanishes for $t \geq t_Y$.
		
		The statement for injectives is exactly dual. 
	\end{proof}
\end{lemma}

\begin{lemma} \label{lem:uniform_bound_on_homology}
	Let \begin{tikzcd}
	\D^b(\Lambda') \ar[r, "i_*"{name=i}] & 
	\ar[l, swap, "i^*"{name=il}, bend right=30] \ar[l, "i^!"{name=ir}, bend left=30]
	\D^b(\Lambda)
	\end{tikzcd} be exact functors with adjoint pairs $(i^*,i_*)$ and $(i_*, i^!)$. Then the homology of $i_*X$ is uniformly bounded for $X\in\mod\Lambda'$. I.e. there is an $r$ such that $H^{j}(i_*X) = 0$ is 0 outside of $j\in(-r, r)$.
	\begin{proof}
		We first prove that there is an $r'$ such that $H^{j}(i_*X)=0$ for $j \geq r'$.
		Let $P$ be $i^*\Lambda \in \D^b(\Lambda')=K^{-,b}(\proj\Lambda')$. Then by \cref{lem:adjoint_preserves_bounded_proj/inj} $P$ is abounded complex of projectives.
		
		Thus there is an $r'$ such that $P^{-j}=0$ for $j \geq r'$. Then $\D^b(\Lambda')(P, X[j]) = \D^b(\Lambda)(\Lambda, i_*X[j]) = H^{j}(i_*X)=0$ for $j\geq r'$ and any $\Lambda'$-module $X$.
		
		Next we prove that there is an $r''$ such that $H^{-j}(i_*X)=0$ for $j \geq r''$. The argument is completely dual. Let $I$ be $i^!D\Lambda \in \D^b(\Lambda')=K^{+,b}(\inj\Lambda')$. Then again by \cref{lem:adjoint_preserves_bounded_proj/inj} $I$ is abounded complex of injectives.
		
		Thus there is an $r''$ such that $I^{j}=0$ for $j \geq r''$. Then $\D^b(\Lambda')(X, I[j]) = \D^b(\Lambda)(i_*X, D\Lambda[j]) = H^{-j}(i_*X)=0$ for $j\geq r''$ and any $\Lambda'$-module $X$.
		
		Letting $r$ be the maximum of $r'$ and $r''$ we get that $H^{j}(X)$ is zero outside of $(-r, r)$.
	\end{proof}
\end{lemma}

\begin{theorem}\cite[3.3]{Hap93}
	Given a recollement FDC holds for middle if and only if it holds for the two others.
	\begin{proof}
		Assume FDC holds for $\Lambda$, we begin by showing it holds for $\Lambda'$.
		
		Let $T = \Lambda' / rad\Lambda'$. Then the projective dimension of $X$ is the largest $t$ for which $\Ext^t(X, T) \neq 0$. Let $X$ be a module in $\mod \Lambda'$ with finite projective dimension. Then since $X$ is isomorphic to its projective resolution, by \cref{lem:adjoint_preserves_bounded_proj/inj} $i_*X$ is a bounded complex of projectives. Say:
		$$i_*X = 0 \to P^{-s} \to \cdots \to P^{s'} \to 0$$
		By \cref{lem:uniform_bound_on_homology} we know there is an $r$ independent of $X$ such that $H^{-j}(X)=0$ for $j \geq r$. Truncating $i_*X$ at $-r$ gives a projective resolution of $\ker d^{-r}_{i_*X}$. Since $\Lambda$ satisfies FDC this means that $s \leq r + \findim(\Lambda)$.
		
		Since $i_*T$ is in $\D^b(\Lambda)$ it is a bounded complex, in particular there is a $t_0$ such that $i_*T^{t}=0$ for $t \geq t_0$. Then by the bounds above $\D^b(\Lambda)(i_*X, i_*T[t]) = 0$ for $t \geq t_0 + r + \findim(\Lambda)$. Since $i_*$ is fully faithful this equals $\D^b(\Lambda')(X, T[t])$, and so $\findim(\Lambda') \leq t_0 + r + \findim(\Lambda)$. That is, $\Lambda'$ satisfies FDC.
		
		The proof for $\Lambda''$ is the same, just replacing $i_*$ with $j_!$.
		
		For the converse assume $\Lambda'$ and $\Lambda''$ satisfy FDC. Let $T = \Lambda / rad\Lambda$, and $X$ be a $\Lambda$-module with finite projective dimension. By \cref{def:recollement} (\ref{recollement:triangles}) we have distinguished triangles:
		\begin{center}
			\begin{tikzcd}
				j_!j^!X \ar[r] & X \ar[r] & i_*i^* X \ar[r] & j_!j^!X[1]\\		
				i_!i^!T \ar[r] & T \ar[r] & j_*j^* T \ar[r] & i_!i^!T[1]\\
			\end{tikzcd}
		\end{center}
		Let $(-,-)_m := \D^b(\Lambda)(-,-[m])$, and $X_j := j_!j^!X$, $X_i := i_*i^* X$, $T_i := i_!i^!T$, $T_j = j_*j^* T$. Then we have long exact sequences:
		\begin{center}
			\begin{tikzcd}[column sep=0.5cm]
			\cdots \ar[r] & (X, T_i)_m \ar[r] & (X, T)_m \ar[r] & (X, T_j)_m \ar[r] & (X, T_i)_{m+1} \ar[r] & \cdots\\
			\cdots \ar[r] & (X_i, T_i)_m \ar[r] & (X, T_i)_m \ar[r] & (X_j, T_i)_m \ar[r] & (X_i, T_i)_{m+1} \ar[r] & \cdots\\
			\cdots \ar[r] & (X_i, T_j)_m \ar[r] & (X, T_j)_m \ar[r] & (X_j, T_j)_m \ar[r] & (X_i, T_j)_{m+1} \ar[r] & \cdots\\
			\end{tikzcd}
		\end{center}
		We have \begin{align*}
			(X_i, T_j)_m = (i_*i^* X, j_*j^* T)_m &= (j^*i_*i^*X, j^*T)_m = 0\\
			&\text{and}\\
			(X_j, T_i)_m = (j_!j^!X, i_!i^!T)_m &= (j^!X, j^!i_!i^!T)_m = 0
		\end{align*}
		which combined with long exact sequences gives us that $(X_i, T_i)_m = (X, T_i)_m$ and $(X_j, T_j)_m = (X, T_j)_m$. If we can show that $(X_i, T_i)_m$ and $(X_j, T_j)_m$ are bounded, then $(X, T_i)_m$ and $(X, T_j)_m$, and consequently $(X, T)_m$ would be bounded. Which would give a bound on the projective dimension of $X$.
		
		We start by bounding $(X, T_i)_m = (X_i, T_i)_m$. First note that 
		\begin{align*}
			(X_i, T_i)_m = (i_*i^* X, i_!i^!T)_m = (i^*i_*i^* X, i^!T)_m = (i^* X, i^!T)_m
		\end{align*}
		Since $X$ has finite projective dimension we can think of it as a bounded complex of projectives. Then by \cref{lem:adjoint_preserves_bounded_proj/inj} $i^*X$ is as well. By the second half of \cref{lem:uniform_bound_on_homology} (using $(i^*, i_*)$ instead of $(i_*, i^!)$) we have that there is an $r$ such that $H^{-j}(i^*X)=0$ for all $j \geq r$. This means that thinking of $i^*X$ as a complex of projectives it is 0 in degree $t$ for all $t \leq -(r + \pd\ker d^{-r}_{i^*X})$, in particualr it is 0 for all $t\leq -(r + \findim(\Lambda'))$. Since $i^!T$ is a bounded complex, it has an upper bound, say $t_0$. Thus $(i^* X, i^!T)_m = 0$ for all $m \geq t_0 + r + \findim(\Lambda')$.
		
		The bound on $(X, T_j)_m$ is similar, using the finitistic dimension of $\Lambda''$. Taking the maximum of these two bounds we get a bound on $(X, T)_m$, which gives a bound on the projective dimension independent of $X$, hence a bound on $\findim(\Lambda)$. 
	\end{proof}
\end{theorem}

\section{Contravariant finiteness}

\begin{defn}[Resolving]
	A full subcategory of an abelian category is called resolving if 
	\begin{itemize}
		\item It is closed under extensions
		\item It contains the projectives
		\item It is contains the kernels of its epimorphisms
	\end{itemize}
\end{defn}

Note that the subcategory of modules with finite projective dimension is resolving.


\begin{lemma}\label{lem:coresolving_ext_vanish}
	Let $\mathcal X$ be coresolving. Then $\Ext^1(\mathcal X, Y) = 0$ implies that $\Ext^i(\mathcal X, Y)=0$ for all $i \geq 1$.
	\begin{proof}
		Since $\mathcal X$ contains the projectives $\Omega X$ is the kernel of an epimorphism in $\mathcal X$. Thus $\mathcal X$ contains all syzygies. $\Ext^i(X, Y) = \Ext^1(\Omega^{i-1}X, Y) = 0$.
	\end{proof}
\end{lemma}

\begin{prop}\label{prop:complement_closed_under_extension}
	If $\mathcal X$ is resolving then $\mathcal Y := \ker\Ext^{\geq 1}(\mathcal X, -) = \ker\Ext^{1}(\mathcal X, -)$ is closed under extensions.
	\begin{proof}
		Let $0 \to Y \to E \to Y' \to 0$ be an extension of objects in $\mathcal Y$, and let $X$ be an object of $\mathcal X$. Then we get an exact sequence  
		\begin{center}
			\begin{tikzcd}
			0=\Ext(X, Y) \ar[r] & \Ext(X, E) \ar[r] & \Ext(X, Y')=0
			\end{tikzcd}
		\end{center}
		Thus $\Ext(X, E)=0$ and $E$ is in $\mathcal Y$.
	\end{proof}
\end{prop}

\begin{lemma} \label{lem:exact_sequence_from_approximation}
	Let $\mathcal X$ be a contravariantly finite, resolving subcategory of $\mod \Lambda$. Then for every object $C \in \mod\Lambda$ there is a short exact sequence $0 \to Y \to X \to C \to 0$ with $X\to C$ minimal $\mathcal X$-approximation and $\Ext^i(\mathcal X, Y)=0$ for all $i \geq 1$.
	\begin{proof}
		Since $\mathcal X$ is contravariantly finite $C$ has a minimal approximation $X \to C$. Since $\mathcal X$ contains the projective cover of $C$ this approximation must be an epimorphism. So it is part of a short exact sequence $0 \to Y \to X \to C \to 0$. Let $X'$ be an arbitrary object in $\mathcal X$. Taking the long exact sequence in $\Ext(X', -)$ gives us
		\begin{center}
		\begin{tikzcd}
			\Hom(X', Y) \ar[r]&\Hom(X', X) \ar[r] & \Hom(X', C) \ar[dll, overlay, out=-15, in=165] \\ \Ext(X', Y) \ar[r] & \Ext(X', X) \ar[r] & \Ext(X', C)
		\end{tikzcd}
		\end{center}
		Since $X \to C$ is an approximation we know that $\Hom(X', X) \to \Hom(X', C)$ is epi. Thus if we can prove that $\Ext(X', X) \to \Ext(X', C)$ is mono we would have that $\Ext(X', Y)=0$. Assume we have an element of $\Ext(X', X)$ that is mapped to 0, i.e. we have a commutative diagram
		\begin{center}
			\begin{tikzcd}
			0 \ar[r] & X \ar[r] \ar[d] & E \ar[r] \ar[d] & X' \ar[d, equal] \ar[r] & 0\\
			0 \ar[r] & C \ar[r] & C \oplus X' \ar[r] & X' \ar[r] & 0
			\end{tikzcd}
		\end{center}
		Since $\mathcal X$ is closed under extensions $E$ is in $\mathcal X$. By composing with projection $C\oplus X' \to C$ we get a commutative triangle
		\begin{center}
			\begin{tikzcd}
			 X \ar[r] \ar[d] & E \ar[dl]\\ 
			 C 
			\end{tikzcd}
		\end{center}
		since $X \to C$ is an approximation we get that $E \to C$ factors through $X$. The endomorphism $X \to E \to X$ leaves the approximation unchanged, so by minimality it must be an isomorphism. Hence $0 \to X \to E \to X' \to 0$ is split and $\Ext(X', X) \to \Ext(X', C)$ is injective. Thus $\Ext(X', Y)=0$.
	\end{proof}
\end{lemma}


\begin{theorem} \cite[3.8]{AR91}
	Let $\mathcal X$ be a contravariantly finite, resolving subcategory of $\mod \Lambda$. Let $X_i$ be the minimal approximation of $S_i$. Then any $X \in \mathcal X$ is a direct summand of an $X_i$-filtered module.
	\begin{proof}
		The first part of the proof is to show by induction on length that any module $C$ is in an exact sequence $0 \to Y \to X \to C \to 0$ with $X$ $X_i$-filtered and $\Ext^1(\mathcal X, Y)=0$.
		
		For the base case if $C=S_i$ is simple then by \cref{lem:exact_sequence_from_approximation} we have an exact sequence $0 \to Y \to X_i \to C \to 0$. 
		
		For the induction step, assume it holds for all modules of length less than $n$, and let $C$ be a module of length $n$. Then by Jordan-Hölder $C$ is the extension of two modules of length less than $n$. Say
		\begin{center}
			\begin{tikzcd}
			0 \ar[r] & C' \ar[r] & C \ar[r] & C'' \ar[r] & 0
			\end{tikzcd}
		\end{center}
		Applying the induction hypothesis we get a diagram on the form
		\begin{center}
			\begin{tikzcd}
			& 0 \ar[d] &&0 \ar[d] &\\
			& Y' \ar[d] && Y'' \ar[d] & \\
			 & X' \ar[d] && X'' \ar[d] & \\
			0 \ar[r] & C' \ar[d] \ar[r] & C \ar[r] & C'' \ar[r] \ar[d] & 0\\
			& 0 &&0 &
			\end{tikzcd}
		\end{center}
		Taking the pullback of $X'' \to C''$ we get a diagram
		\begin{center}
			\begin{tikzcd}
			0 \ar[r] & C' \ar[r] \ar[d, equal] & E \ar[r] \ar[d] & X'' \ar[r] \ar[d] & 0\\
			0 \ar[r] & C' \ar[r] \ar[d] & C \ar[r] \ar[d] & C'' \ar[r] \ar[d] & 0\\
			&0&0&0&
			\end{tikzcd}
		\end{center}
		Since $Y'$ satisfies $\Ext^1(\mathcal X, Y') = 0$ by \cref{lem:coresolving_ext_vanish} it also satisfies $\Ext^2(\mathcal X, Y')=0$. In particular from the long exact sequence we get that $X' \to C'$ induces an isomorphism $\Ext(X'', X') \to \Ext(X'', C)$. Thus $0 \to C' \to E \to X'' \to 0$ comes from a sequence $0 \to X' \to X \to X'' \to 0$. In other words we have a diagram
		\begin{center}
			\begin{tikzcd}
			& 0 \ar[d] &&0 \ar[d] &\\
			 & Y' \ar[d] && Y'' \ar[d] & \\
			0 \ar[r] & X' \ar[d] \ar[r] & X \ar[r] \ar[d] & X'' \ar[d] \ar[r] & 0\\
			0 \ar[r] & C' \ar[d] \ar[r] & C \ar[r] & C'' \ar[r] \ar[d] & 0\\
			& 0 &&0 &
			\end{tikzcd}
		\end{center}
		Applying the snake lemma we can fill out the diagram:
		\begin{center}
			\begin{tikzcd}
			& 0 \ar[d] & 0\ar[d] &0 \ar[d] &\\
			0 \ar[r] & Y' \ar[d] \ar[r] & Y \ar[r] \ar[d] & Y'' \ar[d] \ar[r] & 0\\
			0 \ar[r] & X' \ar[d] \ar[r] & X \ar[r] \ar[d] & X'' \ar[d] \ar[r] & 0\\
			0 \ar[r] & C' \ar[d] \ar[r] & C \ar[r] \ar[d] & C'' \ar[r] \ar[d] & 0\\
			& 0 &0&0 &
			\end{tikzcd}
		\end{center}
		Since $X$ is an extension of $X_i$-filtered modules, it is also $X_i$-filtered. Since $Y$ is the extension of $Y''$ and $Y'$ it follows from \cref{prop:complement_closed_under_extension} that $\Ext(\mathcal X, Y)=0$.
		
		Hence any $C$ fits into a sequence $0 \to Y \to X \to C \to 0$ with $X$ being $X_i$-filtered and $\Ext(\mathcal X, Y)=0$.
		
		Now suppose that $C$ is in $\mathcal X$, and let $0 \to Y \to X \to C \to 0$ be as before. Then we get that
		\begin{center}
		\begin{tikzcd}
		\Hom(C, X) \ar[r] & \Hom(C, C) \ar[r] & \Ext^1(C,Y) = 0
		\end{tikzcd}
		\end{center}
		is exact, and thus $C$ is a direct summand of $X$. So every object in $\mathcal X$ is a direct summand of an $X_i$-filtered module.
	\end{proof}
\end{theorem}

\begin{cor}
	If the subcategory of modules with finite projective dimension is contravariantly finite, then the finitistic dimension is the supremum of the projective dimension of $X_i$. In particular it is finite.
\end{cor}

\section{repdimension}

Many results based on the survey \cite{Opp09}.

\begin{defn}[dominated dimension]
	Let \begin{tikzcd}
		\Lambda \ar[r] & I_0 \ar[r] & I_1 \ar[r] & \cdots
	\end{tikzcd}
	be a minimal injective resolution of $\Lambda$. Then the dominated dimension of $\Lambda$ is $\inf\{n | I_n$ is not projective \}.
\end{defn}

\begin{defn}[rep-dimesnion] \todo{can probably reformulate this in terms of projectives and left modules... is there any significance to the distinction?}
	Let $A$ be defined by $$A = \{\Gamma | domdim\Gamma \geq 2, \Lambda \text{ morita eqquivalent to } \End_\Gamma(I_0(\Gamma))\}$$ where $I_0(\Gamma)$ is the injective envelope of $\Gamma$. Then the repdimesnion of $\Lambda$ is the minimal global dimension of $\Gamma \in A$.
\end{defn}

\begin{prop}\label{prop:repdim_auslander_generator} \todo{I guess this is Auslanders original definition}
	(all modules ar right modules)
	Repdim is the same as minimal global dimension of $\End(M)$ for $M$ being both a generator and cogenerator. 
	\begin{proof}
		Consider $\Gamma \in A$. Since $domdim\Gamma \geq 1$, $I_0(\Gamma)$ is the sum of all projective-injective modules (some probably several times). 
		
		Let $\mathcal S$ be the set of all $\Gamma$-modules with a copresentation
		\begin{center}
		\begin{tikzcd}
			0 \ar[r] & X \ar[r] & I_0 \ar[r] & I_1
		\end{tikzcd}
		\end{center}
		with $I_i$ in $\add I_0(\Gamma)$. In particular $\Gamma$ is in $\mathcal S$, because $domdim\Gamma \geq 2$.
		
		The Yoneda embedding gives an equivalence $$\Hom_\Gamma(-,I_0(\Gamma)):\add I_0(\Gamma) \to \proj\End_\Gamma(I_0(\Gamma))^{op}$$, and thus we get an equivalence $$D\Hom_\Gamma(-,I_0(\Gamma)):\add I_0(\Gamma) \to \inj\End_\Gamma(I_0(\Gamma))$$
		Since $I_0(\Gamma)$ is injective $D\Hom(-,I_0(\Gamma))$ is exact and preserves kernels, so extends to an equivalence
		$$\Hom_\Gamma(-,I_0(\Gamma)):\mathcal S \to \mod \End_\Gamma(I_0(\Gamma))$$
		
		Since $\End_\Gamma(I_0(\Gamma))$ is morita equivalent to $\Lambda$, $\mathcal S$ is equivalent to $\mod \Lambda$. $\Gamma \in \mathcal S$ is clearly a generator. To see that it is a cogenerator note that $\Gamma$ contains all the projective-injective indecomposable objects as direct summands, so there is an injection $I_0(\Gamma) \to \Gamma^n$, and since $I_0(\Gamma)$ is a cogenerator in $\mathcal S$, $\Gamma$ is aswell.
		
		Thus by the equivalence $\mathcal S \to \mod\Lambda$ there is a cogenerator-generator object $M$ such that $\End_\Lambda(M) = \End_\Gamma(\Gamma)=\Gamma$.
		
		The last step of the proof is showing that $\End(M)$ is in $A$ whenever $M$ is a generator-cogenerator.
		
		Let $0 \to M \to I_0(M) \to I_1(M)$ be a minimal injective copresentation of $M$. Since $M$ is a cogenerator $I_i(M)$ is in $\add M$, thus we get an exact sequence of projective $\End(M)$-modules
		\begin{align} \label{eq:injective_copres_of_endM}
		0 \to \End(M) \to \Hom(M, I_0(M)) \to \Hom(M, I_1(M)).
		\end{align}
		Now we have the following isomorphisms of $\Lambda$-$\End(M)$-bimodules
		\begin{align*}
		\Hom_\Lambda(M, D\Lambda) =\\
		\Hom_k(M\otimes\Lambda, k) =\\
		\Hom_k(M, k) =\\
		DM =\\
		D\Hom_\Lambda(\Lambda, M)
		\end{align*}
		Since $\Lambda$ is in $\add M$, $\Hom(\Lambda, M)$ is projective, and thus $D\Hom(\Lambda, M) = \Hom(M, D\Lambda)$ is injective. This means that (\ref{eq:injective_copres_of_endM}) is an injective copresentation, and thus $domdim\End(M) \geq 2$.
		
		Since $\Hom(M, I_0(M))$ is the beginning of an injective resolution of $\End(M)$, $I_0(\End(M))$, must be a direct summand. Then $\Hom(M, I_0(M)) / I_0(\End(M))$ would map injectively into $\Hom(M, I_1(M))$, but that would mean theres a direct summand of $I_0(M)$ mapping injectively into $I_1(M)$, contradicting minimality. Thus $\Hom(M, I_0(M)) = I_0(\End(M))$.
		
		Let $I=I_0(M)$ and $\Gamma = \End_\Lambda(I)$, then $D\Hom(-,I)$ is an exact equivalence from $\add I$ to $\inj\Gamma$. Since $I$ is an injective cogenerator $\add I = \inj\Lambda$. Then because of exactness $D\Hom(-,I)$ becomes an equivalence between $K^{+,b}(\inj\Lambda)$ and $K^{+,b}(\inj\Gamma)$. Considering only those complexes with homology in degree 0, we see that $\mod\Lambda$ is equivalent to $\mod\Gamma$. So $\Lambda$ is morita equivalent to $\Gamma = \End(I_0(M)) = \End(I_0(\End(M)))$.
	\end{proof}
\end{prop}

\begin{defn}
	Let $X$ be an object of $\mod\Lambda$ and $M$ a contravariantaly finite subcategory.
	\begin{center}
	\begin{tikzcd}
		\cdots \ar[rd, two heads] \ar[r] & M_2 \ar[rd, two heads] \ar[r] & M_1 \ar[rd, two heads] \ar[r] & M_0 \ar[rd, two heads]\\
		&\Omega_M^3 X \ar[u, hook] & \Omega_M^2  \ar[u, hook] X & \Omega_M X  \ar[u, hook] & X
	\end{tikzcd}
	\end{center}
	If $\twoheadrightarrow$ are minimal $M$-approximations (they need not be surjective), and $\hookrightarrow$ are their kernels, then this is an $M$-resolution of $X$. The $M$-res-dimesnion of $X$ is the length of the sequence of (nonzero) $M_i$'s, and the $M$-res-dimesnion of $\Lambda$ is the supremum of the dimension on its objects.

\end{defn}

\begin{prop}
	Repdim-2 is the minimum of $M$-res-dim$(\mod \Lambda)$ for $M$ both generator and cogenrator (assuming repdim is at least 2).
	
	\begin{proof}
		The functor $\Hom(M, -)$ is an equivalence from $\add M$ to $\proj\End(M)$, which maps minimal $M$-approximations to projective covers. Let $X$ be any module in $\mod\End(M)$ with projective dimension at least 2. Then it has a projective presentation $$\Omega^2X \to (M,M_1) \to (M,M_0) \to X.$$
		Because of the equivalence this is induced by a map $f:M_1\to M_0$. Since $\Hom$ is left exact we have that $\Omega^2X \cong \Hom(M, \ker f)$, and so the projective dimension of $X$ is $2$ plus the $M$-res-dimension of $\ker f$.
		
		Since $M$ is a cogenerator any module $Y$ in $\mod\Lambda$ has a copresentation 
		\begin{center}
		\begin{tikzcd}
			0 \ar[r] & Y \ar[r] & M_0 \ar[r, "f"] & M_1.
		\end{tikzcd}
		\end{center}
		Applying $\Hom(M,-) =: (M,-)$ we get
		\begin{center}
		\begin{tikzcd}
		0 \ar[r] & (M,Y) \ar[r] & (M,M_0) \ar[r]{}{(M,f)} & (M,M_1) \ar[r] & \cok(M,f) \ar[r] & 0.
		\end{tikzcd}
		\end{center}
		If the projective dimension of $\cok(M,f)$ is less than 2, then $(M, Y)$ is a direct summand of $(M, M_0)$. This means that $(M,Y) \cong (M, M')$, so the minimal $M$-approximation of $Y$ is $M'$, and $(M, \Omega_M Y) = 0$. Since $M$ is a generator this means $\Omega_M Y = 0$ and thus the $M$-res-dimension of $Y$ is 0.
		
		So provided the projective dimension of $\cok(M,f)$ is larger than or equal to 2, it equals the $M$-res-dimension of $Y$ plus 2. In particular the global dimension of $\End(M)$ is 2 plus the $M$-res-dimension of $\mod\Lambda$, provided it is at least 2.
	\end{proof}
\end{prop}

\subsection{The Igusa-Todorov function}
Let $K$ be the free abelian group generated by isomorphism classes of modules, modulo the relations $[A\oplus B] = [A] + [B]$ and $[P]=0$ when $P$ is projective. Define the linear map $L:K\to K$ by $L[A] = [\Omega A]$. For any module $X$, $[\add X]$ is a finitely generated subgroup of $K$. Fitting's lemma tells us that there is an integer $\eta_X$ such that $L:L^m[\add X] \to L^{m+1}[\add X]$ is an isomorphism for every $m \geq \eta_X$. We define $\psi(X)$ to be $\eta_X + \sup\{\pd Y | Y \in \add \Omega^{\eta_X}X, \pd Y<\infty\}$.

\begin{lemma} \cite[Lemma~3]{IgTo05} \label{lem:properties_of_psi}
	\begin{enumerate}
		\item $\psi(M) = \pd M$ when $\pd M < \infty$.
		\item $\psi(M^k) = \psi(M)$
		\item $\psi(M) \leq \psi(M\oplus N)$
		\item If $Z$ is a direct summand of $\Omega^n(M)$ where $n \leq \eta_M$ and $\pd Z < \infty$, then $\pd Z + n \leq \psi(M)$.
	\end{enumerate}
	\begin{proof}
		\begin{enumerate}
			\item[] %empty line
			\item If $\pd M < \infty$ then $L^m \neq 0$ for $m < \pd M$, and $L^m =0$ for $m \geq \pd M$.
			\item $\add M^k = \add M$, and $\psi$ is only defined in terms of additive categories.
			\item  $\add M \subseteq \add M\oplus N$, so if $L$ is injective when restricted to $L^m(\add M\oplus N)$ then $L$ is injective when restricted to $L^m(\add M)$, so $\eta_M \leq \eta_{M\oplus N}$. Further $\Omega^{\eta_{M\oplus N}-\eta_M}\add\Omega^{\eta_M}M \subset \add\Omega^{\eta_{M\oplus N}} M\oplus N$, so $\psi(M) \leq \psi(M\oplus N)$.
			\item Let $p=\pd Z$ and $k = \eta_M - n$. Then $\Omega^k Z$ is in $\add \Omega^{\eta_M}M$, so $\pd\Omega^k Z + \eta_M \leq \psi(M)$. Thus $$\pd Z + n = p + n = (p-k) + \eta_M \leq \pd\Omega^k Z + \eta_M \leq \psi(M).$$
		\end{enumerate}
	\end{proof}
\end{lemma}

\begin{theorem}\cite[Theorem~4]{IgTo05} \label{thm:projdim_bounded_by_psi}
	Let $0 \to A \to B \to C \to 0$ be a short exact sequence of modules with $\pd C < \infty$. Then $\pd C \leq \psi(A\oplus B)+1$.
	\begin{proof}
		Let $P_A^\bullet$ and $P_C^\bullet$ be the minimal projective resolutions of $A$ and $C$. Then we get a map of short exact sequences
		\begin{center}
		\begin{tikzcd}
			0 \ar[r]  & P_A^0 \ar[r] \ar[d] & P_A^0 \oplus P_C^0 \ar[r] \ar[d] & P_C^0 \ar[r] \ar[d] & 0\\
			0 \ar[r] & A \ar[r] & B \ar[r] & C \ar[r] & 0 
		\end{tikzcd}
		\end{center}
		Applying the snake lemma we get $0 \to \Omega A \to \Omega B \oplus P \to \Omega C \to 0$ for some projective $P$. Thus for some $n \leq \pd C$ we have $L^n[A] = L^n[B]$, and let $n$ be the minimal such number. Clearly $n \leq \eta_{A\oplus B}$. Let $X = \Omega^n A = \Omega^n B$, then our sequence of $n$-syzygies looks like
		\begin{center}
			\begin{tikzcd}
			0 \ar[r] & X \ar[r] & X\oplus P \ar[r] & \Omega^nC \ar[r] & 0.
			\end{tikzcd}
		\end{center}
		Let $f$ be the composition
		\begin{tikzcd}
			X \ar[r] & X \oplus P \ar[r, "\pi_X"] & X.
		\end{tikzcd}
		Then by fittings lemma $X$ breaks as a direct sum into two components $X = Z \oplus Y$ such that $f = f_Z \oplus f_Y$ with $f_Y$ an isomorphism and $f_Z$ nilpotent. In other words the sequence above can be written as
		\begin{center}
			\begin{tikzcd}
			0 \ar[r] & Z\oplus Y \ar[r] & Z \oplus Y\oplus P \ar[r] & \Omega^nC \ar[r] & 0.
			\end{tikzcd}
		\end{center}
		with the left map being
		$$\begin{bmatrix}
			f_Z & 0\\
			0 & f_Y\\
			* & *
		\end{bmatrix} \sim
		\begin{bmatrix}
		f_Z & 0\\
		0 & f_Y\\
		* & 0
		\end{bmatrix} $$
		So we get another short exact sequence
		\begin{center}
			\begin{tikzcd}
			0 \ar[r] & Z \ar[r] & Z \oplus P \ar[r] & \Omega^nC \ar[r] & 0.
			\end{tikzcd}
		\end{center}
		Let $T = \Lambda/rad(\Lambda)$and apply the long exact sequence in $\Ext(-, T)$. Then we get an exact sequence
		\begin{center}
			\begin{tikzcd}
			\Ext^k(Z, T) \ar[r] & \Ext^k(Z \oplus P, T) \ar[r] & \Ext^{k+1}(\Omega^nC, T) 
			\end{tikzcd}
		\end{center}
		where the left map is induced by $f_Z$ since $\Ext^k(Z \oplus P, T) \cong \Ext^k(Z, T)$. Since $f_Z$ is nilpotent this map is surjective if and only if $\Ext^k(Z, T)=0$, and $\Omega^nC$ has finite projective dimension we have that $Z$ has finite projective dimension. In particular $\pd\Omega^n C -1 \leq \pd Z \leq \pd\Omega^n C$.
		
		Since $Z$ is a direct summand of $\Omega^n A\oplus B$ by \cref{lem:properties_of_psi} we have that $\pd Z + n \leq \psi(A \oplus B)$, and thus $\pd \Omega^n C - 1 + n = \pd C - 1 \leq \psi(A \oplus B)$.
	\end{proof}
\end{theorem}

\begin{theorem}\cite[Corollary~8]{IgTo05}
	If $\Lambda = \End_\Gamma(P)$ for an algebra $\Gamma$ with global dimension at most 3, and $P$ projective then $\findim(\Lambda) < \infty$.
	\begin{proof}
		Let $X$ be any $\Lambda$-module with finite projective dimension. Then it has a projective presentation $(P, P_1) \to (P,P_0) \to X \to 0$ where $(P,P_i)=\Hom_\Gamma(P,P_i)$ with $P_i \in \add P$. Since $(P,-)$ is an equivalence from $\add P$ to $\proj\Lambda$ this corresponds to a map $P_1 \to P_0$ which we can extend to a projective resolution in $\Gamma$:
		\begin{center}
			\begin{tikzcd}
			0 \ar[r] & P_3 \ar[r] & P_2 \ar[r] & P_1 \ar[r] & P_0.
			\end{tikzcd}
		\end{center}
		Applying the exact functor $(P, -)$, we get an exact sequence
		\begin{center}
			\begin{tikzcd}
			0 \ar[r] & (P,P_3) \ar[r] & (P,P_2) \ar[r] & (P,P_1) \ar[r] & (P,P_0)\ar[r] & X \ar[r] & 0.
			\end{tikzcd}
		\end{center}
		Truncating this we get a short exact sequence
		\begin{center}
			\begin{tikzcd}
			0 \ar[r] & (P, P_3) \ar[r] & (P, P_2) \ar[r] & \Omega^2 X \ar[r] & 0.
			\end{tikzcd}
		\end{center}
		Then by \cref{thm:projdim_bounded_by_psi} the projective dimension of $\Omega^2 X$ is bounded by $\psi((P, P_3)\oplus (P, P_2))+1$. Which means
		$$\pd X \leq \psi((P, P_3)\oplus (P, P_2))+3 \leq \psi((P,\Gamma))+3$$
		Since this bound doesn't depend on $X$, $\Lambda$ has finite finitistic dimension.
	\end{proof} 
\end{theorem}

\begin{cor}
	If $repdim(\Lambda) \leq 3$ then $\findim(\Lambda) < \infty$.
	\begin{proof}
		If $\Lambda$ has rep-dimension less than or equal to 3 then by \cref{prop:repdim_auslander_generator} there is a generator-cogenerator $M$ in $\mod\Lambda$ such that $\Gamma := \End_\Lambda(M)$ has global dimension 3 or less. Then since $M$ is a generator $\Lambda$ is in $\add M$ and so $\Hom_\Lambda(M, \Lambda)$ is a projective $\Gamma$-module with $\End_\Gamma(\Hom_\Lambda(M, \Lambda)) = \End_\Lambda(\Lambda) = \Lambda$.
	\end{proof}
\end{cor}

\section{Unbounded derived category}

If we go to the unbounded derived category we can get a sort of converse to \cref{thm:findim_implies_inj_generate}.

\begin{theorem}\cite[Theorem~4.3]{Rick19}
	If the localizing category of $D\Lambda$ is the entire unbounded derived category then $\Findim(\Lambda) < \infty$. (Note the capital F meaning the finitistic dimesnion of $\Mod\Lambda$, which is bigger than or equal to that of $\mod\Lambda$).
	
	\begin{proof}
		Assume $\Findim(\Lambda) = \infty$. Then there are modules $M_i$ with projective dimension $i$ for every $i \geq 0$. Let $P_i$ be the minimal projective resolution of $M_i$, and consider $\bigoplus P_i[-i]$ and $\prod P_i[-i]$. Both of these have homology $M_i$ in degree $i$, and are concentrated in non-negative degrees.
		
		The inclusion from the sum to the product is clearly a quasi-isomorphism. We want to show that it is not a homotopy equivalence. Assume for the sake of contradiction that it was. Then tensoring with $\Lambda/rad(\Lambda)$ would give us another homotopy equivalence. Since $\Lambda/rad(\Lambda)$ is finitely presented tensoring preserves both products and coproducts. Because all the resolutions were minimal tesnoring with $\Lambda/rad(\Lambda)$ gives us 0 differentials. In degree 0 we get $$\bigoplus \Tor_i(M_i, \Lambda/rad(\Lambda)) \to \prod \Tor_i(M_i, \Lambda/rad(\Lambda)) .$$
		Since $\Tor_i(M_i, \Lambda/rad(\Lambda))$ is nonzero for every $M_i$ this map is not an isomorphism, and so we don't have a homotopy equivalence.
		
		So the cone of the inclusion $\bigoplus P_i[-i] \to \prod P_i[-i]$, $C$, is 0 in the derived category, but non-zero in the homotopy category. Since $\Lambda$ is artinian the product of projectives is projective\cite[Theorem~3.3]{Chase60}, so $\prod P_i[-i]$ is a complex of projectives, which means that $C$ is a complex of projectives. 
		
		In other words $C$ is an acyclic lower bounded complex of projectives that is not contractible. Tensoring with $D\Lambda$ is an equivalence from projectives to injectives, so $C\otimes D\Lambda$ is an lower bounded complex of injectives that is not contractible. Such a complex cannot be acyclic so $C\otimes D\Lambda$ has homology.
		
		The homology of $C$ is 0, so $K(\Lambda)(\Lambda, C[i]) = 0$. Applying the equivalence $-\otimes D\Lambda$ we get $$\D(\Lambda)(D\Lambda, C\otimes D\Lambda [i])=K(\Lambda)(D\Lambda, C\otimes D\Lambda [i])=0.$$ This means that $C\otimes D\Lambda$ is not in the localizing category generated by $D\Lambda$, and so that is not the entire derived category.
	\end{proof}
\end{theorem}

\begin{theorem}\cite[Theorem~4.4]{Rick19}
	$\Findim(\Lambda) < \infty$ if and only if $D\Lambda^\perp \cap \D^+(\Lambda) = 0$.
	\begin{proof}
		In the theorem above we proved that when the finitistic dimension is infinite then there is a non-zero complex in $\D^+(\Lambda)$ perpendicular to $D\Lambda$. 
		
		The proof of the converse is the same as for \cref{thm:findim_implies_inj_generate}. If we have a non-zero object $X \in D\Lambda^\perp \cap \D^+(\Lambda) = 0$, then $\D(\Lambda)(D\Lambda, X)$ is a non-split complex of projectives that continue arbitrarily to the right. So the cokernels have arbitrarily big projective dimension.
	\end{proof}
\end{theorem}

\section{Personal appendix}
\begin{theorem}
	The global dimension of an artin algebra is the supremum of $k$ with $\Ext^k(T,T)\neq 0$ ($T$ sum of simples). This is also the supremum of projective dimension and supremum of injective dimension.
	\begin{proof}
		For a minimal projective resolution $\Hom(-,T)$ makes the differentials 0, and similarly with $\Hom(T,-)$ and injective resolutions. So $\Ext^k(M, T)$ is only 0 exactly when $k>\pd M$, similarly $\Ext^k(T,M)$ is only 0 when $k$ is bigger than the injective dimension. Since any module is built by extensions of simples you can prove by induction, and the long exact sequence in $\Ext(-,T)$ you get that any module has projective dimension less than or equal to that of $T$. Similarly for injective dimension.
	\end{proof}
\end{theorem}

$\findim(\Lambda)$ need not equal $\findim(\Lambda^{op}) = \sup\{ $injective dimension of $M | M$ has finite inejctive dimension$ \}$.

\begin{example} \cite{Gre20}
	Let $\Lambda=k \left.
	\left[\begin{tikzcd}
	\ar[out=150,in=210, loop, swap, looseness=3, "a"] 1 \ar[r, bend left=15, "b"] & 2 \ar[l, bend left=15, "c"]
	\end{tikzcd}\right] \middle/ (a^2, ac, ba, cbc) \right.$. Then $\findim(\Lambda) \geq 1$, but $\findim(\Lambda^{op})=0$.
	\begin{proof}
		The module $\mymatrix{1\\1} = P_1/P_2 $ ($k^2$ where $a$ acts by $\begin{bsmallmatrix}
			0 & 1\\0&0
		\end{bsmallmatrix}$, and $b$ and $c$ act trivially)
		has projective dimension 1, so $\findim(\Lambda) \geq 1$. The projective/injective modules of $\Lambda$ are:
		$$ P_1 = \mymatrix{
			&1&\\
			1 && 2\\
			&&1\\
			&&2
		},\quad P_2 = \mymatrix{
			2\\1\\2
		},\quad I_1 = \mymatrix{
			&&1\\
			1&&2\\
			&1&
		},\quad I_2 = \mymatrix{
			1\\2\\1\\2
		} $$
		If $\findim(\Lambda^{op})>0$ there would be a module with finite non-zero injective resolution. In particular it would end with a non-split epimorphism between injectives. I claim this would mean there is a non-split epimorphism $I \to I_i$ from an injective to an indecomposable injective. Obviously we get epimorphisms by composing with the projections onto summands, so we want to show that they are not split. Assume that they are, that is the map looks like
		
		\begin{center}
		\begin{tikzcd}[ampersand replacement=\&]
			I_i \oplus I \ar{r}{
				\begin{bmatrix}
				1 & 0\\ f & g
				\end{bmatrix}
			} \ar[swap]{rd}{
				\begin{bmatrix}
				1 & 0
				\end{bmatrix}
			} \& I_i \oplus I' \ar[]{d}{
				\begin{bmatrix}
				1 & 0
				\end{bmatrix}
			}\\
			\& I_i
		\end{tikzcd}.
		\end{center}
		We see that by changing basis in the domain we get the matrix $\begin{bmatrix}
		1&0\\0&g
		\end{bmatrix}$. Thus $I_i$ is mapped isomorphically to itself, which doesnt happen in a minimal resolution.
		
		The only thing left to show is that there are no non-split epimorphisms from injective modules to $I_1$ and $I_2$.
	\end{proof}
\end{example}

\clearpage

\bibliography{mybib}
%\bibliography{mybib}
%\bibliography{intro_bib}
\bibliographystyle{alpha}
\end{document}