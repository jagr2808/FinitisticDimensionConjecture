\documentclass[11pt, a4paper, english]{article}
\usepackage[utf8]{inputenc}
\usepackage{babel, amsmath, amsthm, amssymb, amsfonts, mathtools, centernot, enumerate, mathrsfs}
\usepackage{thmtools, thm-restate}
\usepackage{tikz-cd}
\usepackage{tikz-3dplot}
\usepackage{caption}
\usepackage{intcalc}
\usepackage{stmaryrd}
\usepackage{multicol}
\usepackage{cite}
\usepackage{hyperref}
\usepackage[capitalise]{cleveref}
\usepackage[toc,page]{appendix}

\usepackage[toc,page]{appendix}
\usepackage{fancyhdr}
\usepackage{todonotes}
\newcommand\tab[1][1cm]{\hspace*{1}}
\DeclarePairedDelimiter{\ceil}{\lceil}{\rceil}

\newtheorem{theorem}{Theorem}[section]
\declaretheorem[name=Theorem,sibling=theorem]{restate-thm}
\newtheorem{conj}{Conjecrture}
\newtheorem{cor}{Corollary}[theorem]
\newtheorem{prop}[theorem]{Proposition}
\newtheorem{lemma}[theorem]{Lemma}
\theoremstyle{definition}
\newtheorem{defn}[theorem]{Definition}
\newtheorem{example}[theorem]{Example}

\newcommand{\C}{\mathbb{C}}
\newcommand{\Z}{\mathbb{Z}}
\DeclareMathOperator{\Hom}{Hom}
\DeclareMathOperator{\Ext}{Ext}
\DeclareMathOperator{\Tor}{Tor}
\DeclareMathOperator{\End}{End}
\DeclareMathOperator{\Aut}{Aut}
\DeclareMathOperator{\Image}{Im}
\DeclareMathOperator{\Ker}{ker}
\DeclareMathOperator{\cok}{cok}
\DeclareMathOperator{\depth}{depth}
\DeclareMathOperator{\findim}{findim}
\DeclareMathOperator{\Findim}{Findim}
\DeclareMathOperator{\domdim}{domdim}
\DeclareMathOperator{\repdim}{repdim}
\DeclareMathOperator{\inj}{inj}
\DeclareMathOperator{\proj}{proj}
\DeclareMathOperator{\op}{op}
\DeclareMathOperator{\pd}{pd}
\DeclareMathOperator{\add}{add}
\DeclareMathOperator{\Mod}{Mod}
\def\mod{\operatorname{mod}}
\DeclareMathOperator{\rad}{rad}
\DeclareMathOperator{\socle}{soc}
\DeclareMathOperator{\D}{\mathscr{D}}

\newsavebox{\pullback}
\sbox\pullback{%
	\begin{tikzpicture}%
	\draw (0,0) -- (1ex,0ex);%
	\draw (1ex,0ex) -- (1ex,1ex);%
	\end{tikzpicture}
}

\newcommand{\mymatrix}[1]{\begin{matrix}#1\end{matrix}}

\setlength{\parindent}{0em}
\setlength{\parskip}{1em}

\pagestyle{fancy}
\fancyhead{}
%\fancyhead[LO, LE]{\small\emph{McKay correspondence}}
\fancyhead[LO, LE]{\small\nouppercase\rightmark}
\fancyfoot[CO, CE]{\thepage}
%\fancyfoot[RO, RE]{\thepage}
\renewcommand{\sectionmark}[1]{\markboth{}{\emph{\thesection~#1}}}
%\renewcommand{\subsectionmark}[1]{}% Remove \subsection from header

\begin{document}
\title{Finitistic dimension conjecture}
\author{Jacob Fjeld Grevstad}
\date{2021}
%\maketitle
\pagenumbering{roman}

\begin{abstract}
FDC yo! This is abstract!
\end{abstract}

\begin{abstract}
	Dette er en abtrakt på norsk!
\end{abstract}
\clearpage

\section*{Preface}
\addcontentsline{toc}{section}{\protect\numberline{}Preface}%
Add text here when thesis mainmatter is done. 
\begin{flushright}
	Jacob Fjeld Grevstad\\ 
	Trondheim, 2021
\end{flushright}
\clearpage

\tableofcontents
\addcontentsline{toc}{section}{\protect\numberline{}Contents}%
\clearpage

\section*{Notation}
\addcontentsline{toc}{section}{\protect\numberline{}Notation}%
\markboth{section}{Notation}
%$k$ a field $\Lambda$ findim alg, $J$ radical

Throughout this thesis $k$ will be a field, and $\Lambda$ will be a finite dimensional algebra over $k$. We will use $J$ to refer to the Jacobson radical of $\Lambda$.

%$\mod\Lambda$ finite dimensional (left-)modules, $\Mod\Lambda$ all (left-)modules.

We will use $\mod\Lambda$ to refer to the category of finite dimensional left $\Lambda$-modules, and $\Mod\Lambda$ to the category of all left $\Lambda$-modules. Any modules considered will be left modules if not specified otherwise. When there is ambiguity we may write $_\Lambda M$ to specify that we are considering $M$ as a left $\Lambda$-module, and $M_\Lambda$ to specify that we are considering $M$ as a right $\Lambda$-module. Similarly $_\Gamma M_\Lambda$ means we are considering $M$ as a $\Gamma$-$\Lambda$-bimodule.

%$_\Gamma M_\Lambda$ is a $\Gamma-\Lambda$-bimodule. Left $\Gamma$-module, right $\Lambda$-module

Since right $\Lambda$-modules are the same as left $\Lambda^{\op}$-modules we use these interchangeably. We use the symbol $D$ to denote the duality functor $D\colon \mod \Lambda \leftrightarrow \mod\Lambda^{\operatorname{op}}$ where $DM = \Hom_k(M, k)$. Typically $D\Lambda$ will refer to the left module $D\Lambda_\Lambda$.

%$D: \mod \Lambda \to \mod\Lambda^{\operatorname{op}}$ is the duality $DM = \Hom(M, k)$

A quiver is a direct graph with a finite number of vertices. We write composition of paths right to left. That is, for paths $\alpha\colon i \to j$ and $\beta\colon k\to l$ the composition $\alpha\beta$ is defined if and only if $l=i$. For a quiver $Q$, the path algebra $kQ$ is the free vector space of all paths, including a trivial path for each vertex. Multiplication of paths is defined to be composition when it is defined and 0 otherwise. The multiplication extends linearly to make $kQ$ and algebra.

%If $Q$ is a quiver we denote by $Q_0$ the set of vertices and $Q_1$ the set of arrows. We have two maps $s,t:Q_1 \to Q_0$ which assign to an arrow $\alpha: i\to j$ its vertex of origin $s(\alpha) = i$, and its vertex of termination $t(\alpha) = j$.

%Quiver/path algebra. Multiplication is written right to left

When working over a category $\mathcal C$ we will denote the set of morphisms either as $\Hom_{\mathcal C}(M, N)$ or as $\mathcal C(M,N)$. When the ambient category is clear we may also simply write $\Hom(M, N)$ or $(M, N)$.

The categories we are considering are all $k$-linear and all functors are assumed to be $k$-linear as well.

%$\Hom_{\mathcal C}(M, N)$ can be written as $\mathcal C(M,N)$ or sometimes simply $(M,N)$

For an exact category $\mathcal A$ we write:
\begin{itemize}
	\item $\D(\mathcal A)$ to refer to the derived category, 
	\item $\D^b(\mathcal A)$ to refer to the bounded derived category, 
	\item $K^b(\mathcal A)$ to refer to the bounded homotopy category, 
	\item $K^{+,b}(\mathcal A)$ (respectively $K^{-,b}(\mathcal A)$) to refer to the homotopy category of complexes bounded below (respectively above) that are bounded in homology.
\end{itemize}
We also write $\D^b(\Lambda)$ instead of $\D^b(\mod\Lambda)$ and $\D(\Lambda)$ instead of $\D(\Mod\Lambda)$. 

In all of these triangulated categories $X[i]$ will denote the complex $X$ shifted $i$ degrees down. That is, $(X[i])^n = X^{n+i}$. The hard truncation is the complex defined by $(X^{\geq n})^m$ equals $X^m$ when $m \geq n$ and 0 otherwise. We denote the hard truncation of $X$ by $X^{\geq n}$. The other hard truncation, $X^{\leq n}$, is defined similarly.

%$\D^b(\Lambda)$ bounded derived category, $K^b$, $K^{+,b}$, $K^{-, b}$, $\D$, etc. $X^{\geq n}$ hard truncation, $[i]$ shift

For a module $M$ we will write $I(M)$ for its injective envelope, and $P(M)$ for its projective cover. We may also write 
\begin{center}
	\begin{tikzcd}
		\cdots \ar[r] & P_M^2 \ar[r, "d_M^2"] & P_M^1 \ar[r, "d_M^1"] & P_M^0 \ar[d, "d_M^0"] \ar[r] & 0\\
		          &            &           &  M      
	\end{tikzcd}
\end{center} 
for its minimal projective resolution. We let the $n$th syzygies of $M$ be the kernel of $d_M^{n-1}$, denoted by $\Omega^n M$. We also define $\Omega^0 M$ to be $M$.

The projective dimension of $M$ is $i$ if $P_M^i$ is the last non-zero module in the minimal projective resolution, and $\infty$ if there is no such module. We denote the projective dimension by $\pd M$.

%$I(M)$ injective envelope, $P_M^0$ projective cover, $\cdots \to P_M^1 \to P_M^0$ projective resolution. $\Omega^n M$ syzygy.

\newpage

\pagenumbering{arabic}
\section*{Introduction}
\addcontentsline{toc}{section}{\protect\numberline{}Introduction}%
\markboth{section}{Introduction}

\section*{Introduction}
\addcontentsline{toc}{section}{\protect\numberline{}Introduction}%
\markboth{section}{Introduction}

In representation theory of finite dimensional algebras, there are several related conjectures known as the ``homological conjectures''. The strongest of these conjectures is the Finitistic Dimension Conjecture. It concerns the homological invariant called the finitistic dimension. For a noetherian ring we define
\begin{align*}
  \findim(R) &:= \sup\{\pd M \mid M \in \mod R, \pd M < \infty\},\\
  \Findim(R) &:= \sup\{\pd M \mid M \in \Mod R, \pd M < \infty\}.
\end{align*}
The finitistic dimension conjecture states that $\findim(\Lambda) < \infty$, whenever $\Lambda$ is a finite dimensional algebra. Note that $\findim(R) \leq \Findim(R)$, and so a stronger conjecture is whether $\Findim(\Lambda) < \infty$, but in this thesis we are mainly interested in the small finitistic dimension.

\subsection*{History}

The finitistic dimension was introduced by Auslander--Buchsbaum in the late 1950s to study commutative noetherian rings. They proved that for a local noetherian commutative ring the finitistic dimension equals the depth\cite{AB57}. Later it was shown by Bass and Gruson--Raynaud that for any commutative noetherian ring the (big) finitistic dimension equals the Krull dimension\cite{Bass62,RG71}.

The non-commutative case turned out to be more difficult. In 1960 Bass published two important questions about the finitistic dimension\cite{Bass60}, which they credit to Rosenberg and Zelinsky. Their first question asks whether the small finitistic dimension equals the big finitistic dimension. This was shown to be false even for monomial algebras by Huisgen-Zimmerman in 1992\cite{ZH92}. Their second question is what we here call the finitistic dimension conjecture. 

Much progress have been done on the problem over the last 60 years. Huisgen-Zimmerman has a great paper summarizing most of the results\cite{ZH95}. Here we try to do something similar to said paper, with the focus on establishing which classes of algebras the conjecture is known to hold for. We try to keep the thesis self contained by writing out all the proofs, and in addition we include some results not covered in Huisgen-Zimmermann's paper.

\subsection*{Overview}
The sections of this thesis are self-contained, and can be read independently of one another, except for \cref{sec:vanishing_radical} which relies on results from \cref{sec:Igusa-Todorov}. In \cref{sec:summary} we summarize for which algebras the conjecture is known to hold. This relies only on \cref{sec:contravariantly_finite,sec:Igusa-Todorov,sec:vanishing_radical,sec:monomial_algebras,sec:Unbounded_derived_category}, and not on \cref{sec:homological_conjectures,sec:recollement}.

In addition to the main sections of this thesis, there is an appendix, \cref{sec:appendix}, where we cover general theorems from homological algebra that would break the flow of the main text. These results are referenced when used.

In \cref{sec:homological_conjectures} we discuss the homological conjectures, and show the implications between them. All the conjectures concerns a specific property of an algebra that is conjectured to hold for all algebras. In \cref{prop:conj_on_individual_algebras} we give an overview of how the conjectures are related on the level of individual algebras.

In \cref{sec:recollement} we introduce a sort of ``short exact sequence'' of triangulated categories, known as a recollement. We show that if the derived category of $\Lambda$ is a recollement of the derived categories of $\Lambda'$ and $\Lambda''$, then finitistic dimension of $\Lambda$ is finite if and only if the finitistic dimension of both $\Lambda'$ and $\Lambda''$ are. The idea of using recollements to study the finitistic dimension is due to Happel, and most of the section is based on their paper\cite{Hap93}. We also consider a related technique concerning triangular matrix rings, due to Fossum-Griffith-Reiten\cite{FGR75}, and discuss the similarities.

In \cref{sec:contravariantly_finite} we show that if the subcategory of modules with finite projective dimension is contravariantly finite, then the algebra has finite finitistic dimension. This is a result due to Auslander--Reiten\cite{AR91}. In \cref{exam:not_contravariantly_finite}, due to Igusa--Smalø--Todorov\cite{IST90}, we show that this subcategory can fail to be contravariantly finite even for monomial algebras with radical cubed equal to 0. In \cref{exam:contravariantly_finite_dual} we show that the dual of the algebra in the previous example has contravariantly finite subcategory of modules with projective dimension. This shows that there is no immediate link between contravariant finiteness and for an algebra and its dual.

In \cref{sec:Igusa-Todorov} we introduce the Igusa--Todorov function, and use it to show that algebras with representation dimension less than or equal to 3 satisfies the finitistic dimension conjecture. We also give examples of two classes of algebras that are known to have representation dimension at most 3, due to Xi and Erdmann--Holm--Iyama--Schröer respectively\cite{Xi02,EHIS04}. Preprints of Igusa--Todorov's paper\cite{IgTo05} was circulated in the mid 90s, but it was not published until later, when several corollaries could be included.

In \cref{sec:vanishing_radical} we discuss restriction one can impose on the radical for the algebra to satisfy the finitistic dimension conjecture. Specifically we look at algebras for which $J^{2l+1}=0$ and $\Lambda/J^l$ is representation finite, and algebras where the composition factors of $J^2$ have finite projective dimension.

In \cref{sec:monomial_algebras} we show that the finitistic dimension of a monomial algebra is always finite. This proof is due to Green--Kirkman--Kuzmanovich\cite{GKK91}. An alternate proof was given by Igusa--Zacharia\cite{IgZa90}, but we don't discuss that here.

In \cref{sec:Unbounded_derived_category} we discuss a more recent result, due to Rickard\cite{Rick19}. In contrast to the rest of this thesis, instead of cinsidering the small finitistic dimension, we give a condition for when the big finitistic dimension is finite. Specifically we show that if the inejctives generate the unbounded derived category, then $\Findim(\Lambda) < \infty$. Many of the algebras considered in previous sections also satisfies this more general condition. We state this more precisely in \cref{thm:Findim_summary}(\ref{item:Findim_derived_equiv}).

\subsection*{The intended reader}
This thesis is written to be understandable to someone who has taken a course on representation theory of finite dimensional algebras and homological algebra. The reader should be familiar with: 
\renewcommand\labelitemi{---}
\begin{itemize}
  \item representation theory of quivers and path algebras,
  \item projective dimension and the $\Ext$-functor,
  \item the long exact sequence in $\Ext$ and $\Tor$,
  \item the basic definitions of category theory, including (co)limits and adjoint functors,
  \item the derived category and triangulated categories.
\end{itemize}
These subject are covered in the courses \textit{MA3203 -- Ring Theory} and \textit{MA3204 -- Homological Algebra} offered at NTNU, or in classical textbooks such as \cite{ARS97} and \cite{Wei94}.

\section{The homological conjectures}

\subsection*{Finitistic Dimension Conjecture (FDC)}
\begin{defn}[Finitistic dimension]
	For a finite dimensional algebra $\Lambda$ the \emph{finitistic dimension} of $\Lambda$, denoted $\findim(\Lambda)$ is defined by
	$$\findim(\Lambda) = \{\pd M \mid M \in \mod\Lambda, \pd M < \infty\}.$$
\end{defn}

\begin{conj}[Finitistic dimension conjecture]
	For a finite dimensional algebra the finitistic dimension is always finite.
	$$\findim(\Lambda) < \infty$$
\end{conj}

\subsection*{Wakamatsu Tilting Conjecture (WTC)}
\begin{defn}[Wakamatsu tilting]
	Let $T$ be a module in $\mod\Lambda$ for a finite dimensional algebra $\Lambda$. Then $T$ is \emph{Wakamatsu tilting} if
	\begin{enumerate}[i)]
		\item We have that $\Ext^n(T,T)=0$ for all $n >0$.
		\item There is an exact sequence 
		\begin{center}
			\begin{tikzcd}
				\eta\colon 0 \ar[r] & \Lambda \ar[r, "d_{-1}"] & T_0 \ar[r, "d_0"] & T_1 \ar[r, "d_1"] & \cdots
			\end{tikzcd}
		\end{center}
		where $T_i$ is in $\add T$.
		\item The sequence $\Hom(\eta, T)$ is exact. Which is equivalent to the condition that $\Ext^1(\Ker d_i, T)=0$ for every differential $d_i$ in $\eta$.
	\end{enumerate}
\end{defn}

Wakamatsu tilting generalizes the definition of a tilting module. A Wakamatsu tilting module is a tilting module if it has finite projective dimension and $\eta$ is bounded. The Wakamatsu tilting conjecture states that this last restriction is superfluous.

\begin{conj}[Wakamatsu tilting conjecture] 
	If $T$ is Wakamatsu tilting and has finite projective dimension, then $T$ is a tilting module. In other words we can choose $\eta$ to be bounded.
\end{conj}

\subsection*{Gorenstein Symmetry Conjecture (GSC)}
\begin{conj}[Gorenstein symmetry conjecture] 
	If $\Lambda$ is a finite dimensional algebra the injective dimension of $\Lambda$ as a left module is finite if and only if the projective dimension of $D\Lambda_\Lambda$ is finite.
\end{conj}

The conjecture describes a sort of symmetry between the projective and injective modules. Equivalently we could formulate the conjecture as $\Lambda$ having finite injective dimension as a left module if and only if it has finite injective dimension as a right module.

\subsection*{Vanishing Conjecture (VC)}
If $\Lambda$ is a finite dimensional algebra we denote by $K^b(\inj \Lambda)$ the homotopy category of bounded complexes of injectives. The category $K^{+,b}(\inj\Lambda)$ is the homotopy category of complexes of injectives that are bounded below, and bounded in homology. There is an equivalence of categories between $K^{+,b}(\inj\Lambda)$ and the bounded derived category $\D^b(\Lambda)$. This allows us to consider $K^b(\inj\Lambda)$ as a subcategory of $\D^b(\Lambda)$. Using this we define the perpendicular subcategory
$$K^b(\inj\Lambda)^\perp = \{X \in \D^b(\Lambda) \mid \Hom(I, X)=0 \text{ for all } I \in K^b(\inj\Lambda)\}.$$
The vanishing conjecture then states that this subcategory is 0.
\begin{conj}[Vanishing conjecture] 
	If $\Lambda$ is a finite dimensional algebra, then $K^b(\inj\Lambda)^\perp = 0$.
\end{conj}

\subsection*{Nunke Condition (NuC)}
\begin{conj}[Nunke condition] 
	If $X \neq 0$ is a non-zero module over a finite dimensional algebra $\Lambda$, then there is an $n \geq 0$ such that $\Ext^n(D\Lambda, X) \neq 0$. 
\end{conj}

\subsection*{Strong Nakayama Conjecture (SNC)}
The strong Nakayama conjecture is a slight weakening of the Nunke condition.

\begin{conj}[Strong Nakayama conjecture] 
	If $S$ is a simple module over a finite dimensional algebra $\Lambda$, then there is an $n \geq 0$ such that $\Ext^n(D\Lambda, S) \neq 0$. 
\end{conj}

\subsection*{Auslander--Reiten Conjecture (ARC)}

\begin{conj}[Auslander--Reiten conjecture] 
	Let $\Lambda$ be  finite dimensional algebra. If $M$ is a module over $\Lambda$ such that  $\Ext^n(M, M \oplus \Lambda) = 0$ for all $n > 0$, then $M$ is projective. 
\end{conj}

Note that if we replace $M$ by $M\oplus \Lambda$ then we get the equivalent formulation: If $M$ is a generator with $\Ext^n(M,M)=0$ for all $n>0$, then $M$ is projective.

\subsection*{Nakayama Conjecture (NC)}

\begin{defn}[Dominant dimension]
	Let $\Lambda$ be a finite dimensional algebra, and let
	\begin{center}
		\begin{tikzcd}
		0 \ar[r] & \Lambda \ar[r] & I^0 \ar[r] & I^1 \ar[r] & \cdots
		\end{tikzcd}
	\end{center}
	be the minimal injective resolution of $\Lambda$. Then the \emph{dominated dimension} of $\Lambda$ is $$\domdim(\Lambda) = \inf\{n \; | \; I^n\text{ is not projective} \}.$$
\end{defn}

\begin{conj}[Nakayama conjecture] 
	If $\Lambda$ has infinite dominant dimension, then $\Lambda$ is selfinjective.
\end{conj}

\subsection{Implications}
The homological conjectures are related in the way presented in the diagram below.

\begin{tikzcd}
\text{FDC} \ar[r, Rightarrow]\ar[d, Rightarrow] & \text{WTC} \ar[r, Rightarrow] & \text{GSC}\\
\text{VC}\ar[r, Rightarrow] & \text{NuC}\ar[r, Rightarrow] & \text{SNC}\ar[r, Rightarrow] & \text{ARC}\ar[r, Rightarrow] & \text{NC}
\end{tikzcd}

The remainder of this section is used to prove these implications.

\begin{theorem} \cite[1.2]{Hap93} \label{thm:FDC_implies_VC}
	The finitistic dimension conjecture implies the vanishing conjecture.
	\begin{proof}
		Let $I^\bullet \in K^b(\inj\Lambda)^\perp$ be non-zero. Since $\D^b(\Lambda) \cong K^{+,b}(\inj\Lambda)$ we may assume $I^\bullet$ is a complex of injectives, and without loss of generality we may assume it concentrated in degrees $i \geq 0$, and that $d^0\colon I^0 \to I^1$ is not split mono. Since if its concentrated in degrees $i \geq k$ we can just shift it, and if $d^0$ is split mono then replacing $I^0$ by $0$, and $I^1$ be $I^1/I^0$ gives a homotopic complex.
			
		The module $\Hom(D\Lambda, I^i)$ is in $\add\Hom(D\Lambda, D\Lambda) = \add\Lambda$ so $\Hom(D\Lambda, I^\bullet)$ is a complex of projectives. We show that this complex is acyclic by considering the following diagram.
		
		\begin{center}
			\begin{tikzcd}
			0 \ar[r] \ar[d] & D\Lambda \ar[r] \ar[d, "f"] \ar[dl, dashed]& 0 \ar[d]\\
			I^{i-1} \ar[r, "d^{i-1}"] & I^i \ar[r, "d^i"] & I^{i+1}
			\end{tikzcd}
		\end{center}
		
		Since $I^\bullet$ is in $K^b(\inj\Lambda)^\perp$ and $D\Lambda$ is in $K^b(\inj\Lambda)$, whenever $d^if=0$, $f^\bullet$ is homotopic to 0. Meaning $f$ factors through $d^{i-1}$. This means that $\Hom(D\Lambda, I^\bullet)$ is an acyclic complex. Further since $\Hom(D\Lambda, -)$ is an equivalence between $\inj\Lambda$ and $\proj\Lambda$ we have that $\Hom(D\Lambda, d^0)$ is not split mono.
		
		The cokernel of $\Hom(D\Lambda, d^i)$ has a projective resolution of length $i$. This resolution is the direct sum of its minimal resolution and an acyclic bounded complex of projectives. Since bounded acyclic complexes of projectives are split and $\Hom(D\Lambda, d^0)$ is not, we must have that the minimal resolution has length $i$, and so $\findim(\Lambda) = \infty$.
	\end{proof}
\end{theorem}

\begin{theorem} \cite[1.2]{Hap93} \label{thm:VC_implies_Nuc}
	The vanishing conjecture implies the Nunke condition.
	\begin{proof}
		Assume there is an $X \neq 0$ with $\Ext^i(D\Lambda, X) = 0$ for all $i \geq 0$. Then $X$ considered as a stalk complex is in $K^b(\inj\Lambda)^\perp$. Proceed by induction on the width of $I^\bullet \in K^b(\inj\Lambda)$: If the width is 1, then $I^\bullet = I[-i] \in K^b(\inj\Lambda)$ is a stalk complex. Then $\D^b(I[-i], X) = \Ext^i(I, X)$. This is 0 because $D\Lambda$ is the sum of the indecomposable injectives.
		
		Let $I^\bullet \in K^b(\inj\Lambda)$ be a complex of width $n$. without loss of generality we may assume $I^\bullet$ is concentrated in degrees $0 \leq i \leq n-1$. Then $$I^{>0} \to I \to I^{0} \to I^{>0}[1]$$ is a triangle with $I^{>0}$ of width $n-1$ and $I^0$ of width 1. Taking the long exact sequence in $\D^b(-,X)$ it follows that $\D^b(I, X)=0$. 
	\end{proof}
\end{theorem}

\begin{theorem}
	The Wakamatsu tilting conjecture implies the Gorenstein symmetry conjecture.
	\begin{proof}
	The left module $D(\Lambda_\Lambda)$ is Wakamatsu tilting. WTC then gives us that if $D(\Lambda_\Lambda)$ has finite projective dimension, then $_\Lambda\Lambda$ has a finite coresolution by modules in $\add D(\Lambda_\Lambda)$. In other words $_\Lambda\Lambda$ has finite injective dimension.
	
	For the other direction assume $_\Lambda\Lambda$ has finite injective dimension. Then the right module $D(_\Lambda\Lambda)$ has finite projective dimension, so WTC gives us that $\Lambda_\Lambda$ has finite injective dimension. Which means $D(\Lambda_\Lambda)$ has finite projective dimension.
	\end{proof}
\end{theorem}

\begin{prop}
	The Auslander--Reiten conjecture is equivalent to the statement that if $M$ is a generator with $\Ext^n(M, M) = 0$ for $n > 0$, then $M$ is projective.
	\begin{proof}
		Assume ARC and that $M$ satisfies the hypothesis. Then since $M$ is a generator $\Lambda$ is in $\add M$ and thus $\Ext^n(M, \Lambda)=0$. So $\Ext^n(M, M\oplus \Lambda)=0$ and $M$ is projective.
		
		For the other direction Assume $M$ satisfies $\Ext^n(M, M \oplus \Lambda)=0$. Then $\Ext^n(M \oplus \Lambda, M\oplus \Lambda) = 0$, so $M \oplus \Lambda$ is projective, which means that $M$ is projective. 
	\end{proof}
\end{prop}

To prove the last two implications we need some results from the theory of Wedderburn projectives. The results we need are stated below, and are proved in \cref{sec:wedderburn_correspondence}.

\begin{restatable}{restate-thm}{Wederburnequivalence} \label{thm:hom_generator_equivalence}
	Let $\Lambda$ be an artin algebra and $M$ a generator. Let $\Gamma = \End(M)^{\operatorname{op}}$ and $P=(M, \Lambda)$. Then we have the following:
	\begin{enumerate}[i)]
		\item We have an isomorphism of ring $\End_\Gamma(P)^{\operatorname{op}} \cong \Lambda$ and an isomorphism of $\Lambda$-modules $(P_\Lambda, \Gamma) \cong M$.
		\item The composition $(P,-)\circ (M,-)$ is the identity on $\mod\Lambda$.
		\item The functor $(M,-)$ maps injectives $\Lambda$-modules to injective $\Gamma$-modules. 
	\end{enumerate}
\end{restatable}

\begin{defn}[Wedderburn projective]
	Let $\Gamma$ be an artin algebra and let $P$ be a finitely generated projective $\Gamma$-module. Let $\Lambda = \End(P)^{\operatorname{op}}$ and $M=(P, \Gamma)$. The module $P$ is said to be \emph{Wedderburn projective} if $\End(M)^{\operatorname{op}}=\Gamma$.
\end{defn}

\begin{restatable}{restate-thm}{Wederburncriterion}\label{thm:wedderburn_criterion}
	Let $\Gamma$ be an artin algebra and $P$ a projective $\Gamma$-module. If $P$ contains the projective cover of all simple modules that appear in the socle of an injective copresentation of $\Gamma$, then $P$ is Wedderburn projective.
\end{restatable}

We now have the relevant tools to prove the remaining implications.

\begin{theorem}\label{thm:SNC_implies_ARC}
	The strong Nakayama conjecture implies the Auslander--Reiten conjecture.
	\begin{proof}
		We have the equality $\Ext^i_{\Lambda^{\op}}(D{\Lambda^{\op}}, M) = \Ext^i_\Lambda(DM, \Lambda)$ for any ${\Lambda^{\op}}$-module $M$. So ${\Lambda^{\op}}$ satisfies SNC if and only if for every simple module $S$ there is an $i$ such that $\Ext^i(S, \Lambda) \neq 0$.
		
		The proof goes by contraposition. Assume $\Lambda$ does not satisfy ARC. Then we have a nonprojective generator $M$ such that $\Ext^n(M, M)=0$ for all $n>0$. We wish to show that $\Gamma := \End(M)^{\operatorname{op}}$ does not satisfy SNC. Let
		\begin{center}
		\begin{tikzcd}
			0 \ar[r] & M \ar[r] & I_0 \ar[r] & I_1 \ar[r] & \cdots
		\end{tikzcd}
		\end{center}
		be an injective resolution of $M$. Since $\Ext^n(M,M)=0$, when we apply $(M,-):=\Hom(M,-)$ we get an exact sequence.
		\begin{center}
		\begin{tikzcd}
			0 \ar[r] & \Gamma \ar[r] & (M,I_0) \ar[r] & (M,I_1) \ar[r] & \cdots
		\end{tikzcd}
		\end{center}
		By \cref{thm:hom_generator_equivalence} this is an injective resolution of $\Gamma$.
		
		Since $M$ is a non-projective generator it has every indecomposable projective as a summand and a nonprojective summand. So $M$ has more indecomposable summands than $\Lambda$ which means that $\Gamma$ has more indecomposable projectives than $\Lambda$. It follows that $\Gamma$ also has more injectives and thus has an injective not on the form $(M, I)$. Let $Q$ be such an injective and let $S$ be its socle. Then $\Hom_\Gamma(S, (M, I_i)) = 0$ for all $i$, so $\Ext^i(S, \Gamma) = 0$ for all $i$. Thus $\Gamma$ does not satisfy SNC.
	\end{proof}
\end{theorem}

\begin{theorem}
	The Auslander--Reiten conjecture implies the Nakayama conjecture.
	\begin{proof}
		Assume $\Gamma$ does not satisfy NC. In other words $\Gamma$ has dominant dimension $\infty$, but is not self injective. We then want to show that there exists a ring that does not satisfy ARC. Let
		\begin{center}
		\begin{tikzcd}
			0 \ar[r] & \Gamma \ar[r] & I_0 \ar[r] & I_1
		\end{tikzcd}
		\end{center}
		be an injective copresentation of $\Gamma$. Let $P$ be the sum of the projective covers of all nonisomorphic simple modules in the socle of $I_0$. Then by \cref{thm:wedderburn_criterion} we have that $P$ is Wedderburn projective.
		
		Let $\Lambda = \End(P)^{\operatorname{op}}$ and let $M = \Hom(P, \Gamma)$. Then $M$ is a nonprojective generator. If we can show that $\Ext^{>0}(M,M)=0$, then we will have shown that $\Lambda$ does not satisfy ARC.
		
		We have functors $(M,-)\colon\mod\Lambda \to \mod\Gamma$ and $(P,-)\colon\mod\Gamma \to \mod\Lambda$. By \cref{thm:hom_generator_equivalence} $(M, -)$ is fully faithful and $(P,-)\circ (M,-) = \operatorname{id}_{\Lambda}$.
		
		Let $0\to M \to Q_0 \to Q_1$ be an injective copresentation of $M$. Applying $(M,-)$ we get an injective copresentation of $\Gamma$. We conclude that all the projective-inejctive modules are in the essential image of $(M,-)$.
		
		In other words if $I^\bullet$ is the minimal injective resolution of $\Gamma$ then $Q^\bullet := (P, I^\bullet)$ is the minimal injective resolution of $M$, and $(M, Q^\bullet)=I^\bullet$. This means that $(M, Q^\bullet)$ is exact away from 0, so $\Ext^{>0}(M,M)=0$. 
		
		But then $M$ is a nonprojectvie generator with $\Ext^{>0}(M,M)=0$, so $\Lambda$ does not satisfy ARC.
	\end{proof}
\end{theorem}

Combining the implications above we see that the strong Nakayama conjecture implies the Nakayama conjecture. There is however a much simpler proof of this fact which we include below.

\begin{prop}\cite{AR75} 
	The strong Nakayama conjecture implies the Nakayama conjecture
	\begin{proof}
		Assume $\Lambda^{\op}$ satisfies SNC and that the dominant dimension of $\Lambda$ is $\infty$. As noted in \cref{thm:SNC_implies_ARC} we have that $\Ext_\Lambda^\bullet(S, \Lambda) = \Ext_{\Lambda^{\op}}^\bullet(D\Lambda, DS) \neq 0 $. If $\Ext^\bullet(S, \Lambda)$ is nonzero that means the injective envelope $I(S)$ appears in the minimal injective resolution of $\Lambda$. If all injectives apear in the resolution and the dominant dimension is infinity then all injectives are projective. Thus $\Lambda$ is self injective, and hence $\Lambda$ satisfies NC. 
	\end{proof}
\end{prop}

\todo{Make a table of how the implications work}

\subsection{Wedderburn correspondence}\label{sec:wedderburn_correspondence}

\begin{prop}\label{prop:hom_generator_preserves_injectives}
	Let $M$ be a module and $I$ an injective module. If the projective cover of the socle of $I$ is a direct summand of $M$, then $(M,I)$ is an injective $\Gamma:=\End(M)^{\operatorname{op}}$-module.
	\begin{proof}
		Let $J \leq \Gamma$ be a left ideal and let $\psi\colon J \to (M,I)$ be any $\Gamma$-linear map. By \cref{lem:injectives_for_noetherian_ring} it is enough to show that $\psi$ factors through $\Gamma$. Assume $J$ is generated by $f_i$. If we can find $\gamma\colon M \to I$ such that $\gamma \circ f_i = \psi(f_i)$ then we would get our factorization by mapping $1\in \Gamma$ to $\gamma$.
		\begin{center}
			\begin{tikzcd}
			\bigoplus M \ar[dr, "\sum \psi(f_i)"] \ar[d, swap, "\sum f_i"]\\
			M \ar[r, swap, dashed, "\gamma"] & I
			\end{tikzcd}
		\end{center}
		Next we want to show that the kernel of $\sum \psi(f_i)$ contains the kernel of $\sum f_i$. To see this let $K$ be the kernel of $\sum f_i$ and let $K'$ be the kernel of $\sum \psi(f_i)$. If $K'$ does not contain $K$, then $Q:= K/K'\cap K$ is a nonzero module that is mapped injectively into $I$. So the socle of $Q$ is a summand of the socle of $I$. Then by assumption the projective cover of the socle of $Q$ is a direct summand of $M$. By the lifting property of projectives we get a map $M \to K$ such that the composition with $\sum \psi(f_i)$ is non-zero.
		
		Let $a_i$ be the composition 
		\begin{tikzcd}
		M \ar[r] & K \ar[r, hookrightarrow] & \bigoplus M \ar[r, "\pi_i"] & M
		\end{tikzcd}.
		Then we get $\sum f_i \circ a_i = 0$. Applying $\psi$ we get $\sum \psi(f_i)\circ a_i = 0$, which gives a contradiction. Thus $K'$ contains $K$.
		
		Using this we get the following commutative diagram:
		\begin{center}
			\begin{tikzcd}
			\bigoplus M \ar[d] \ar[dd, bend right=60, swap, "\sum f_i"] \ar[dr, "\sum \psi(f)"] \\
			(\bigoplus M)/ K \ar[r] \ar[d, hookrightarrow] & I\\
			M \ar[ur, dashed, swap, "\exists\gamma"]
			\end{tikzcd}
		\end{center}
		Since $I$ is injective it lifts monomorphisms so we know that $\gamma$ exists. Thus $(M, I)$ is an injective $\Gamma$-module.
	\end{proof}
\end{prop}

Now we will prove \cref{thm:hom_generator_equivalence}, restated here for the readers convenience.
\Wederburnequivalence*
\begin{proof}
	\begin{enumerate}[i)]
		\item[]
		\item By Yoneda's Lemma we have an equivalence $$(M,-)\colon\add M \to \add (M,M)=\proj\Gamma.$$ Since $M$ is a generator, $\Lambda$ is in $\add M$. So $$\End(P)=((M,\Lambda), (M,\Lambda)) = \End(\Lambda)=\Lambda^{\operatorname{op}}$$ {\centering and} $$(P,\Gamma)=((M,\Lambda),(M,M))=(\Lambda,M)=M.$$
		\item Let $X$ be a $\Lambda$-module. Since $\add M$ has only a finite number of indecomposables it is functorially finite. So we can take an $M$-resolution of $X$.
		$$\cdots \to M_1 \to M_0 \to X \to 0$$
		Since $\add M$ contains the projectives this is exact. Applying $(M,-)$ we get a projective resolution of $(M,X)$. Since $(M, X)$ is determined by its projective resolution and $X$ is determined by its $M$-resolution we need only show that $(P,-)\circ (M,-)$ is the identity on $\add M$. Then again by Yoneda's Lemma $(P, (M, M')) = (\Lambda, M')=M'$.
		\item Since $M$ is a generator it contains the projective cover of all simple modules. Then \cref{prop:hom_generator_preserves_injectives} gives us that $(M, -)$ maps injective modules to injective modules.
	\end{enumerate}
\end{proof}

\begin{prop}
	Let $P$ be a projective $\Gamma$-module, and let $\Lambda = \End(P)^{\operatorname{op}}$. Then $(P, -)\colon\mod \Gamma \to \mod \Lambda$ is fully faithful on $\add I(P/JP)$, where $P/JP$ denotes the top of $P$, and $I(P/JP)$ its injective envelope.
	\begin{proof}
		We want to show that the map $\Hom_\Gamma(I, I') \to \Hom_\Lambda((P,I), (P, I'))$ is an isomorphism. First we show injectivity. Let $f\colon I\to I'$ be a non-zero map. Then the socle of $\Image f$ is a semisimple submodule of $I'$, so it is in $\add P/JP$. Then there exists a nonzero map from $P$ to $\Image f$. Since $P$ is projective this lifts to a map $\hat{f}\colon P\to I$. Then $f \circ \hat{f}$ is non-zero, so $\Hom_\Gamma(I, I') \to \Hom_\Lambda((P,I), (P, I'))$ is injective.
		
		The argument for surjectivity is similar to that for \cref{prop:hom_generator_preserves_injectives}. Let $\psi\colon(P,I)\to (P, I')$ be a $\Lambda$-linear map. Let $f_i\colon P\to I$ generate $(P,I)$ as a $\Lambda$-module. Consider the diagram
		\begin{center}
		\begin{tikzcd}
			\bigoplus P \ar[r, "\sum f_i"] \ar[rd, swap, "\sum \psi(f_i)"] & I \ar[d, dashed, "?"]\\
			& I'
		\end{tikzcd}
		\end{center}
		We wish to show that there is a map at ? completing the diagram. We first show that $K':=\ker \sum \psi(f_i)$ contains $K:=\ker \sum f_i$. Assume for the sake of contradiction that it does not. Then $Q := K / K' \cap K$ is mapped injectively into $I'$ by $\sum \psi(f_i)$. So the socle of $Q$ is in $\add P/JP$, and we have a non-zero map $P \to Q$.
		
		Since $P$ is projective this extends to a map $P \to K$. Let $a_i$ be the compositions 
		\begin{tikzcd}
			P \ar[r] & K \ar[r] & \bigoplus P \ar[r, "\pi_i"] & P
		\end{tikzcd}
		Then clearly $\sum f_i \circ a_i = 0$, but $\sum \psi(f_i) \circ a_i$ is non-zero. Since $\psi$ is $\Lambda$-linear this is a contradiction, so $K'$ contains $K$.
		
		Then we get an induced diagram
		\begin{center}
			\begin{tikzcd}
			\bigoplus P \ar[d, two heads,]\\
			(\bigoplus P) / K \ar[r, hookrightarrow, "\sum f_i"] \ar[rd, swap, "\sum \psi(f_i)"] & I \ar[d, dashed, "\exists"]\\
			& I'
			\end{tikzcd}
		\end{center}
		Now because $I'$ is injective we know that there is a lift, and so $\Hom_\Gamma(I, I') \to \Hom_\Lambda((P,I), (P, I'))$ is surjective, and thus an isomorphism.
	\end{proof} 
\end{prop}

We conclude this section by giving a proof of \cref{thm:wedderburn_criterion}.

\Wederburncriterion*
\begin{proof}
	Let $\Gamma \to I_0 \to I_1$ be a minimal injective presentation of $\Gamma$. Then by \cref{prop:hom_generator_preserves_injectives}we have that $(P, I_0) \to (P,I_1)$ is an injective presentation of $(P,\Gamma)$. The proposition gives us that $(P,-)$ is fully faithful on $I_0$ and $I_1$. Since the endomorphisms of $\Gamma$ are exactly endomorphisms of $I_0 \to I_1$ up to homotopy this means that $$\Gamma=\End_\Gamma(\Gamma)^{\op} = \End_\Lambda((P, \Gamma))^{\op}$$
	So $P$ is Wedderburn projective.
\end{proof}


\section{Recollement}
In this section we will discuss a reduction technique known as recollement. The idea of reduction techniques is to reduce the work of proving an algebra has finite finitistic dimension to proving the same for ``simpler'' algebras. In \cref{sec:Triangular_matrix_rings} we will consider a reduction technique of triangular matrix algebras. The triangular matrix rings are closely related to recollements, and we discuss their relationship more closely in \todo{ref}

\begin{defn}[Recollement]\label{def:recollement}
	A \emph{recollement} between triangulated categories $\mathcal T'$, $\mathcal T$ and $\mathcal T''$ is a collection of six functors satisfying:
\begin{center}
\begin{tikzcd}[column sep=4cm]
\mathcal T' \ar[r, "i_*=i_!"{name=i}] & 
\ar[l, swap, "i^*"{name=il}, bend right=30] \ar[l, "i^!"{name=ir}, bend left=30]
\mathcal T \ar[r, "j^!=j^*"{name=j}] & 
\ar[l, swap, "j_!"{name=jl}, bend right=30] \ar[l, "j_*"{name=jr}, bend left=30]
\mathcal T''
\arrow[phantom, from=il, to=i, "\dashv" rotate=-90]
\arrow[phantom, from=i, to=ir, "\dashv" rotate=-90]
\arrow[phantom, from=jl, to=j, "\dashv" rotate=-90]
\arrow[phantom, from=j, to=jr, "\dashv" rotate=-90]
\end{tikzcd}	
\end{center}

\begin{enumerate}[(i)]
	\item All functors are exact, and we have adjoint pairs $(i^*, i_*)$, $(i_!, i^!)$, $(j_!, j^!)$, $(j^*, j_*)$. 
	\item \label{recollement:vanishing_composition}The composition $j^*i_*=0$ vanishes.
	\item \label{item:i_fully-faith} 
	We have natural isomorphisms $i^*i_* \cong i^!i_! \cong \operatorname{id}_{\mathcal T'}$ induced by the units and counits of the adjunctions. 
	\item \label{item:j_fully-faith}
	We have natural isomorphisms $j^!j_! \cong j^*j_* \cong \operatorname{id}_{\mathcal T''}$, also induced by the units and counits. 
	\item \label{recollement:triangles}
	For every $X \in \mathcal T$ we have the following distinguished triangles:
	\begin{center}
	\begin{tikzcd}
	j_!j^!X \ar[r, "\varepsilon"] & X \ar[r, "\eta"] & i_*i^* X \ar[r] & j_!j^!X[1]\\		
	i_!i^!X \ar[r, "\varepsilon"] & X \ar[r, "\eta"] & j_*j^* X \ar[r] & i_!i^!X[1].
	\end{tikzcd}
	\end{center}
\end{enumerate}
Note that (\ref{item:i_fully-faith}) and (\ref{item:j_fully-faith}) are equivalent to $i_*$, $j_!$, and $j_*$ being fully faithful.
\end{defn}

We are specifically interested in recollements where the triangulated categories in question are (bounded) derived categories of finite dimensional algebras.

\begin{lemma} \label{lem:adjoint_preserves_bounded_proj/inj}
	Let \begin{tikzcd}
	\D^b(\Lambda') \ar[r, "i_*"{name=i}, bend right=20] & 
	\ar[l, swap, "i^*"{name=il}, bend right=20]
	\D^b(\Lambda)
	\end{tikzcd} be exact functors with an adjoint pair $(i^*,i_*)$. Then $i^*$ preserves bounded projective complexes and $i_*$ preserves bounded injective complexes.
	\begin{proof}
		The bounded projective complexes can be characterized up to isomorphism as the complexes $P$ such that for any complex $Y$ there is an integer $t_Y$ with $\D^b(\Lambda) (P, Y[t])=0$ for $t\geq t_Y$. One can see this by using the equivalence $\D^b(\Lambda) \cong K^{-,b}(\proj\Lambda)$.
		
		Let $P$ be a bounded complex of projectives in $\D^b(\Lambda)$. Then we want to show that $i^*P$ is as well. Let $Y$ be any complex in $\D^b(\Lambda')$. Then $\D^b(\Lambda')(i^*P, Y[t]) = \D^b(\Lambda)(P, i_*Y[t])$, so since $P$ is a bounded complex of projectives there is $t_Y$ such that this vanishes for $t \geq t_Y$. 
		
		The statement for injectives is exactly dual, and so we do not write it out here, but leave it to the reader.
	\end{proof}
\end{lemma}

\begin{lemma} \label{lem:uniform_bound_on_homology}
	Let \begin{tikzcd}
	\D^b(\Lambda') \ar[r, "i_*"{name=i}] & 
	\ar[l, swap, "i^*"{name=il}, bend right=30] \ar[l, "i^!"{name=ir}, bend left=30]
	\D^b(\Lambda)
	\end{tikzcd} be exact functors with adjoint pairs $(i^*,i_*)$ and $(i_*, i^!)$. Then the homology of $i_*X$ is uniformly bounded for $X\in\mod\Lambda'$ considered as a complex concentrated in degree 0. I.e. there is an $r$, independent of $X$, such that $H^{j}(i_*X) = 0$ for $j\not\in(-r, r)$.
	\begin{proof}
		We first prove that there is an $r'$, independent of $X$, such that $H^{j}(i_*X)=0$ for $j \geq r'$.
		Let $P$ be $i^*\Lambda \in \D^b(\Lambda')$. Then by \cref{lem:adjoint_preserves_bounded_proj/inj} $P$ is a bounded complex of projectives.
		
		Thus there is an $r'$ such that $P^{-j}=0$ for $j \geq r'$. Then $$\D^b(\Lambda')(P, X[j]) = \D^b(\Lambda)(\Lambda, i_*X[j]) = H^{j}(i_*X)=0$$ for $j\geq r'$ and any $\Lambda'$-module $X$, when considered as a complex concentrated in degree 0.
		
		Next we prove that there is an $r''$ such that $H^{-j}(i_*X)=0$ for $j \geq r''$. The argument is completely dual. Let $I$ be $i^!D\Lambda \in \D^b(\Lambda') \cong K^{+,b}(\inj\Lambda')$. Then again by \cref{lem:adjoint_preserves_bounded_proj/inj} $I$ is a bounded complex of injectives.
		
		Thus there is an $r''$ such that $I^{j}=0$ for $j \geq r''$. Then $$\D^b(\Lambda')(X, I[j]) = \D^b(\Lambda)(i_*X, D\Lambda[j]) = H^{-j}(i_*X)=0$$ for $j\geq r''$ and any $\Lambda'$-module $X$, when considered as a complex concentrated in degree 0.
		
		Letting $r$ be the maximum of $r'$ and $r''$ we get that $H^{j}(X)$ is zero outside of $(-r, r)$.
	\end{proof}
\end{lemma}

Now that we have a good understanding of how the functors in a recollement interact with homology we can use this to say something about the projective dimension of modules, and thus about the finitistic dimension.

\begin{theorem}\cite[3.3]{Hap93}
	Given a recollement between bounded derived categories 
	\begin{center}
		\begin{tikzcd}[column sep=4cm]
		\D^b(\Lambda') \ar[r, "i_*=i_!"{name=i}] & 
		\ar[l, swap, "i^*"{name=il}, bend right=30] \ar[l, "i^!"{name=ir}, bend left=30]
		\D^b(\Lambda) \ar[r, "j^!=j^*"{name=j}] & 
		\ar[l, swap, "j_!"{name=jl}, bend right=30] \ar[l, "j_*"{name=jr}, bend left=30]
		\D^b(\Lambda''),
		\arrow[phantom, from=il, to=i, "\dashv" rotate=-90]
		\arrow[phantom, from=i, to=ir, "\dashv" rotate=-90]
		\arrow[phantom, from=jl, to=j, "\dashv" rotate=-90]
		\arrow[phantom, from=j, to=jr, "\dashv" rotate=-90]
		\end{tikzcd}	
	\end{center}
	 then we have that $\findim(\Lambda) < \infty$ if and only if we have that $\findim(\Lambda') < \infty$ and $\findim(\Lambda'') < \infty$.
	\begin{proof}
		Assume $\findim(\Lambda) < \infty$. We begin by showing that $\findim(\Lambda') < \infty$.
		
		Let $T = \Lambda' / \rad\Lambda'$ be the sum of all simple $\Lambda'$-modules. Then the projective dimension of $X$ is the largest $t$ for which $\Ext^t(X, T) \neq 0$. Let $X$ be a module in $\mod \Lambda'$ with finite projective dimension. We consider $X$ as a complex concentrated in degree 0. Then since $X$ is isomorphic to its projective resolution, by \cref{lem:adjoint_preserves_bounded_proj/inj} $i_*X$ is a bounded complex of projectives. Say:
		$$i_*X = 0 \to P^{-s} \to \cdots \to P^{s'} \to 0$$
		By \cref{lem:uniform_bound_on_homology} we know there is an $r$ independent of $X$ such that $H^{-j}(X)=0$ for $j \geq r$. Truncating $i_*X$ at $-r$ gives a projective resolution of $\ker d^{-r}_{i_*X}$. So $\ker d^{-r}_{i_*X}$ has projective dimension $-r-(-s) = s-r$. Since $\findim(\Lambda)<\infty$ this means that $s \leq r + \findim(\Lambda)$.
		
		Since $i_*T$ is in $\D^b(\Lambda)$ it is a bounded complex, in particular there is a $t_0$ such that $i_*T^{t}=0$ for $t \geq t_0$. Then by the bounds above $\D^b(\Lambda)(i_*X, i_*T[t]) = 0$ for $t \geq t_0 + s \geq t_0 + r + \findim(\Lambda)$. Since $i_*$ is fully faithful this equals $\D^b(\Lambda')(X, T[t])$, and so $\findim(\Lambda') \leq t_0 + r + \findim(\Lambda)$. In particular it is finite.
		
		The proof for $\findim(\Lambda'')$ is the same, just replacing $i_*$ with $j_!$. We leave writing out the details to the reader.
		
		For the converse assume $\Lambda'$ and $\Lambda''$ both have finite finitistic dimension. Let $T = \Lambda / \rad\Lambda$, and $X$ be a $\Lambda$-module with finite projective dimension, and consider both modules as a complex concentrated in degree 0. By \cref{def:recollement}(\ref{recollement:triangles}) we have distinguished triangles:
		\begin{center}
			\begin{tikzcd}
				j_!j^!X \ar[r] & X \ar[r] & i_*i^* X \ar[r] & j_!j^!X[1]\\		
				i_!i^!T \ar[r] & T \ar[r] & j_*j^* T \ar[r] & i_!i^!T[1].
			\end{tikzcd}
		\end{center}
		We write $(-,-)_m$ instead of $\D^b(\Lambda)(-,-[m])$, and make the following abbreviation:
		\begin{align*}
			X_j &:= j_!j^!X & X_i &:= i_*i^* X \\
			T_i &:= i_!i^!T & T_j &:= j_*j^* T.
		\end{align*} 
		Taking the long exact sequence in homfuntors we get the long exact sequences:
		\begin{center}
			\begin{tikzcd}[column sep=0.5cm]
			\cdots \ar[r] & (X, T_i)_m \ar[r] & (X, T)_m \ar[r] & (X, T_j)_m \ar[r] & (X, T_i)_{m+1} \ar[r] & \cdots\\
			\cdots \ar[r] & (X_i, T_i)_m \ar[r] & (X, T_i)_m \ar[r] & (X_j, T_i)_m \ar[r] & (X_i, T_i)_{m+1} \ar[r] & \cdots\\
			\cdots \ar[r] & (X_i, T_j)_m \ar[r] & (X, T_j)_m \ar[r] & (X_j, T_j)_m \ar[r] & (X_i, T_j)_{m+1} \ar[r] & \cdots\\
			\end{tikzcd}
		\end{center}
		Using the fact that $j^*i_* = j^!i_! = 0$ from \cref{def:recollement}(\ref{recollement:vanishing_composition} we deduce that
		\[\arraycolsep=2pt\def\arraystretch{1.33}
		\begin{array}{ccccccl}
			(X_i, T_j)_m &=& (i_*i^* X, j_*j^* T)_m &=& (j^*i_*i^*X, j^*T)_m &=& 0\\
			\span\span\span\text{and}&\\
			(X_j, T_i)_m &=& (j_!j^!X, i_!i^!T)_m &=& (j^!X, j^!i_!i^!T)_m &=& 0.
		\end{array}\]
		Combining this with the long exact sequences gives us that 
		$$(X_i, T_i)_m = (X, T_i)_m \text{ and } (X_j, T_j)_m = (X, T_j)_m.$$ 
		If we can show that $(X_i, T_i)_m$ and $(X_j, T_j)_m$ are bounded, then $(X, T_i)_m$ and $(X, T_j)_m$ would be bounded as well. Consequently we would have that $(X, T)_m$ is bounded. This would give us a bound on the projective dimension of $X$.
		
		We start by bounding $(X, T_i)_m = (X_i, T_i)_m$. First note that 
		\begin{align*}
			(X_i, T_i)_m = (i_*i^* X, i_!i^!T)_m = (i^*i_*i^* X, i^!T)_m = (i^* X, i^!T)_m
		\end{align*}
		Since $X$ has finite projective dimension we can think of it as a bounded complex of projectives. Then by \cref{lem:adjoint_preserves_bounded_proj/inj} $i^*X$ is as well. By the second half of \cref{lem:uniform_bound_on_homology} (using $(i^*, i_*)$ instead of $(i_*, i^!)$) we have that there is an $r$ such that $H^{-j}(i^*X)=0$ for all $j \geq r$. This means that thinking of $i^*X$ as a complex of projectives, it is 0 in degree $-t$ for all $t \geq r + \pd\ker d^{-r}_{i^*X}$, in particular it is 0 for all $t\geq r + \findim(\Lambda')$. Since $i^!T$ is a bounded complex, it has an upper bound, say $t_0$. Thus $(i^* X, i^!T)_m = 0$ for all $m \geq t_0 + r + \findim(\Lambda')$.
		
		The bound on $(X, T_j)_m$ is similar, using the finitistic dimension of $\Lambda''$. Taking the maximum of these two bounds we get a bound on $(X, T)_m$, which gives a bound on the projective dimension independent of $X$, hence a bound on $\findim(\Lambda)$. 
	\end{proof}
\end{theorem}

\subsection{Triangular matrix rings}\label{sec:Triangular_matrix_rings}
\subsection{Triangular matrix rings}\label{sec:Triangular_matrix_rings}

In this section we will relate the finitistic dimension of the triangular matrix ring $\Lambda = \begin{pmatrix}
R & 0\\
M& S
\end{pmatrix}$ to the finitistic dimension of $R$ and $S$. Specifically the finitistic dimension of $\Lambda$ will be finite if the finitistic dimensions of both $R$ and $S$ are finite. 

In \cref{sec:recollemt_of_triangular_rings} we give some further conditions on $M$ for which we get a recollement between the bounded derived categories of $S$, $R$ and $\Lambda$.

We will first define the concept of a comma category and describe some of its homological properties. In \cref{thm:findim_of_comma_cat} we give a bound on the finitistic dimension of the comma category. Then in \cref{prop:triangular_matrix_is_comma_cat} we show that for $\Lambda$ a triangular matrix ring as above, we have that $\mod \Lambda$ is isomorphic to the comma category of $M\otimes_R - \colon \mod R \to \mod S$, which means we get a bound on $\findim(\Lambda)$.

\begin{defn}[Comma category]
	Let $\mathcal A$ and $\mathcal B$ be categories and let $F\colon\mathcal A \to \mathcal B$ be a functor. Then the \emph{comma category} $(F, \mathcal  B)$ has as objects triplets $(A, B, f)$ with $A \in \mathcal  A$, $B \in \mathcal  B$, and $f\colon FA \to B$ a morphism in $\mathcal  B$. The morphisms are pairs $(\alpha, \beta)\colon(A, B, f) \to (A', B', f')$ with $\alpha\colon A \to A'$ and $\beta\colon B \to B'$ such that the following diagram commutes:
	\begin{center}
		\begin{tikzcd}
			FA \ar[r, "f"] \ar[d, swap, "F\alpha"] & B \ar[d, "\beta"]\\
			FA' \ar[r, "f'"] & B'.
		\end{tikzcd}
	\end{center}
	The composition is what one would expect. Namely, $(\alpha, \beta) \circ (\alpha', \beta') = (\alpha \circ \alpha ', \beta \circ \beta')$.
\end{defn}

\begin{prop}\label{prop:comma-cat_abelian}
	If $\mathcal A$ and $\mathcal B$ are abelian categories and $F$ is right exact, then the comma category $(F, \mathcal  B)$ is abelian. Further a sequence
	\begin{center}
	\begin{tikzcd}
		(A'', B'', f'') \ar[r]{}{(\alpha', \beta')} & (A, B, f)\ar[r]{}{(\alpha, \beta)}  & (A', B', f')
	\end{tikzcd}
	\end{center}
	is exact if and only if the two related sequences in $\mathcal A$ and $\mathcal B$ are exact.
	\begin{center}
		\begin{tikzcd}
			A'' \ar[r, "\alpha'"] & A\ar[r, "\alpha"]  & A'\\
			B'' \ar[r, "\beta'"] & B\ar[r, "\beta"]  & B'
		\end{tikzcd}
	\end{center}
	\begin{proof}
		We need to show that $(F, \mathcal B)$ has kernels and cokernels, and that for any map the image equals the coimage. First we show that it contains kernels. Let $(\alpha, \beta)\colon(A, B, f) \to (C, D, g)$ be a morphism in the comma category. Then we have a diagram:
		\begin{center}
		\begin{tikzcd}
			& F \ker \alpha \ar[r, "F\iota_\alpha"] \ar[d, dashed, "\theta"]& FA \ar[r, "F\alpha"] \ar[d, "f"] & FC \ar[d, "g"]\\
			0 \ar[r] & \ker \beta \ar[r, "\iota_\beta"] & B \ar[r, "\beta"] & D
		\end{tikzcd}
		\end{center}
		Since $\beta f F\iota_\alpha = f' F\alpha F \iota_\alpha = 0$ there is a unique $\theta$ making the diagram commute. I claim the kernel of $(\alpha, \beta)$ is $(\ker \alpha, \ker \beta, \theta)$. Indeed if $(\alpha', \beta')\colon (A', B', f') \to (A, B, f)$ is any map such that $(\alpha, \beta) \circ (\alpha', \beta') = 0$, then $\alpha\alpha'=0$ and $\beta\beta'=0$. This means both $\alpha'$ and $\beta'$ factor uniquely through $\iota_\alpha$ and $\iota_\beta$. Let $\alpha''$ and $\beta''$ be the morphisms such that $\alpha' = \iota_\alpha \circ \alpha''$ and $\beta' = \iota_\beta \circ \beta''$. Then we claim $(\alpha', \beta')$ factors through $(\iota_\alpha, \iota_\beta)$ as indicated in the diagram below.
		\begin{center}
			\begin{tikzcd}
			FA' \ar[d, "f'"] \ar[r, "F\alpha''"] & F \ker \alpha \ar[r, "F\iota_\alpha"] \ar[d, "\theta"]& FA  \ar[d, "f"] \\
			B' \ar[r, "\beta''"] & \ker \beta \ar[r, "\iota_\beta"] & B 
			\end{tikzcd}
		\end{center}
		The only thing left to verify is that the left square commutes. This follows from the outer rectangle commuting, and that $\iota_\beta$ is a monomorphism.
		
		Showing that cokernels exists is similar, but relies on $F$ being right exact. The construction is completely dual, but to verify commutativity at the end instead of using that $\iota_\beta$ is mono we must use that $F\pi_\alpha\colon FA' \to F\cok \alpha$ is an epimorphism. This follows from $F$ being right exact. We leave the details to the reader. 
		
		Since kernels and cokernels are directly induced by the kernels and cokernels in $\mathcal A$ and $\mathcal B$ it is clear that a sequence in $(F, \mathcal B)$ is exact if and only if the two related sequences are exact. Similarly that the image equals the coimage follows from this being true in $\mathcal A$ and $\mathcal B$.
	\end{proof}
\end{prop}

For the rest of this section we assume $F$ is a right exact functor between abelian catgeories so that the comma category is abelian. We also assume $\mathcal A$ and $\mathcal B$ has enough projectives. In particular we are interested in the case when $\mathcal A$ and $\mathcal B$ are module categories over finite dimensional algebras.

\begin{defn}
	For $\mathcal A$ and $\mathcal B$ abelian categories and $F$ right exact we define the following functors:
		\begin{center}
		\begin{tikzcd}[row sep=3pt]
		T\colon\mathcal A \times \mathcal{B} \ar[r]& (F, \mathcal B)\\
		(A, B)  \ar[r, mapsto]& (A, B \oplus FA, FA \hookrightarrow FA \oplus B)\\
		(\alpha, \beta)  \ar[r, mapsto]& (\alpha, F\alpha \oplus \beta)
		\end{tikzcd}
		\end{center}
		
		\begin{center}
			 \begin{tikzcd}[row sep=3pt]
			U\colon(F, \mathcal B) \ar[r]& \mathcal A \times \mathcal{B}\\
			(A, B, f) \ar[r, mapsto]& (A, B)\\
			(\alpha, \beta) \ar[r, mapsto]& (\alpha, \beta)
			\end{tikzcd}
			\hspace{1cm}
		\begin{tikzcd}[row sep=3pt]
		C\colon(F, \mathcal B)\ar[r]& \mathcal A \times \mathcal B\\
		(A, B, f)  \ar[r, mapsto]& (A, \cok f)\\
		(\alpha, \beta)  \ar[r, mapsto]& (\alpha, \hat{\beta})
		\end{tikzcd}
		\end{center}

		\begin{center}
		\begin{tikzcd}[row sep=3pt]
		Z\colon\mathcal A \times \mathcal{B} \ar[r]& (F, B)\\
		(A, B)  \ar[r, mapsto]& (A, B, 0)\\
		(\alpha, \beta)  \ar[r, mapsto]& (\alpha, \beta)
		\end{tikzcd}
		\end{center}
\end{defn}

\begin{prop}
	With the definitions above $U$ and $Z$ become exact functors.
	\begin{proof}
		Using the characterization of exact sequences shown in \cref{prop:comma-cat_abelian} a short exact sequence in $(F, \mathcal B)$ is a commutative diagram
		\begin{center}
			\begin{tikzcd}
				 & FA'' \ar[r, "F\alpha'"] \ar[d, "f''"] & FA \ar[r, "F\alpha"] \ar[d, "f"] & FA' \ar[r] \ar[d, "f'"] & 0\\
				0 \ar[r] & B'' \ar[r, "\beta'"] & B \ar[r, "\beta"] & B' \ar[r] & 0
			\end{tikzcd}
		\end{center}
		such that the sequences 
		\begin{center}
		\begin{tikzcd}
		0 \ar[r] & A'' \ar[r, "\alpha'"] & A \ar[r, "\alpha"] & A' \ar[r] & 0\\
		0 \ar[r] & B'' \ar[r, "\beta'"] & B \ar[r, "\beta"] & B' \ar[r] & 0
		\end{tikzcd} 
		\end{center}
		are short exact. Since when we apply $U$ we simply get the product of these two sequences, $U$ is exact.
		
		Similarly for $Z$ since the two sequences we start with are assumed to be exact the resulting sequence will be exact by the characterization in \cref{prop:comma-cat_abelian}.
	\end{proof}
\end{prop}

\begin{prop}\cite[Proposition~1.3]{FGR75}
	The pairs of functors $(T, U)$ and $(C, Z)$ form adjoint pairs.
	\begin{proof}
		We want to establish an isomorphism
		$$\Hom(T(A, B), (A', B', f)) \cong \Hom((A, B), (A', B')).$$ 
		A morphism $\left(\alpha, 
		\begin{bmatrix}
		\beta & \gamma
		\end{bmatrix}\right) \colon 
		T(A, B) \to (A', B', f)$  is given by a commutative diagram
		\begin{center}
		\begin{tikzcd}[ampersand replacement=\&, row sep = 25pt]
			T(A, B) \colon
			\ar[d, swap]{}{\left(\alpha, \begin{bmatrix}
					\beta & \gamma
			\end{bmatrix}\right)} 
		\& FA 
			\ar{r}{\begin{bmatrix}
			0 \\ 1
			\end{bmatrix}} 
			\ar[d, swap, "F\alpha"]
		\& B \oplus FA 
			\ar{d}{\begin{bmatrix}
			\beta & \gamma
			\end{bmatrix}} \\
			(A', B', f)\colon \& FA' \ar[r, "f"] \& B'.
		\end{tikzcd}
		\end{center}
		The isomorphism is then given by sending this to $(\alpha, \beta)$. This is clearly surjective. 
		
		For injectivity assume $(\alpha, \beta) = 0$, then $\gamma = \begin{bmatrix}
		\beta & \gamma
		\end{bmatrix}\begin{bmatrix}
		0 \\ 1
		\end{bmatrix} = fF\alpha= 0$. So the map is injective, and $(T, U)$ is an adjoint pair.
		
		Next we consider $(C, Z)$. We want an isomorphism 
		\begin{align*}
			\Hom(C(A, B, f), (A', B')) &= \Hom((A, \cok f), (A', B')) \\
			&\cong \Hom((A, B, f), (A', B', 0)).
		\end{align*} 
		A morphism in $\Hom((A, B, f), (A', B', 0))$ is a commutative diagram
		\begin{center}
			\begin{tikzcd}
			FA \ar[r, "f"] \ar[d, swap, "F\alpha"] & B \ar[d, "\beta"]\\
			FA' \ar[r, "0"] & B'
			\end{tikzcd}
		\end{center}
		Since $\beta f = 0$, we have that $\beta$ factors through the cokernel of $f$ uniquely. Let the factorization be given by the map $\beta'\colon \cok f \to B'$. Then we send this diagram to $(\alpha, \beta')$. Since the choice of $\beta'$ was unique this is an isomorphism, so $(C, Z)$ is an adjoint pair.
	\end{proof}
\end{prop}

\begin{cor}
	The functors $T$ and $C$ preserve projective objects.
	\begin{proof}
		What we need to check is that for projective objects $P$ and $Q$ in $(\mathcal A \times \mathcal B)$ and $(F, \mathcal B)$ respectively we have that $\Hom(TP, -)$ and $\Hom(CQ, -)$ are exact. By adjointness these are equal to $\Hom(P, U-)$ and $\Hom(Q, Z-)$ respectively. Since $U$ and $Z$ are exact this holds, and so $T$ and $C$ preserve projective objects.
	\end{proof}
\end{cor}

We will now use these four functors to understand the structure of projective objects in the comma category, and consequently projective resolutions.

\begin{prop}\cite[Corollary~1.6c]{FGR75}
	For a projective object $P$ in $(F, \mathcal B)$ we have that $T(C(P)) \cong P$, in particular all projectives are of the form $T(P')$ for a projective $P' \in \mathcal A \times \mathcal B$.
	\begin{proof}
		Let $P$ be given by $f\colon FA \to B$. Applying $C$ we get $(A, \cok f)$. We have morphisms $P \to ZC(P)$ and $TC(P) \to ZC(P)$ given by the following diagram
		\begin{center}
		\begin{tikzcd}
			FA \ar[d, equal] \ar[r, "f"] & B \ar[d, two heads]\\
			FA \ar[r, "0"] & \cok f\\
			FA \ar[u, equal] \ar[r, hookrightarrow]  & \cok f \oplus FA. \ar[u, swap,  two heads]
		\end{tikzcd}
		\end{center}
		By the projective property of $P$ there is some morphism $\beta$ factorizing the map $P \to ZC(P)$, which gives us the diagram:
		\begin{center}
			\begin{tikzcd}
			FA \ar[d, equal] \ar[r, "f"] & B \ar[d, "\beta"]\\
			FA \ar[d, equal] \ar[r, hookrightarrow]  & \cok f \oplus FA \ar[d,  two heads]\\
			FA \ar[r, "0"] & \cok f.
			\end{tikzcd}
		\end{center}
		Since $FA \hookrightarrow \cok f \oplus FA$ is split mono, $f$ is split mono. This means that $B$ splits as a direct sum of the image and cokernel of $f$, i.e. $B$ is isomorphic to $\cok f \oplus \Image f \cong \cok f \oplus FA$. From the diagram we see that $\beta$ induces an isomorphism on each component, and thus $\beta$ is an isomorphism. So we have $P \cong TC(P)$.
	\end{proof}
\end{prop}

\begin{prop}\cite[Lemma~4.16]{FGR75}\label{prop:pd_in_commacat}
	Let $X = (A, B, f)$ be an object in the comma category. Then $\pd X \geq \pd A$, and if $A=0$ then $\pd X = \pd B$.
	\begin{proof}
		We first show that $\pd X \geq \pd A$. Note that there is an equality $\pd C(X) = \max\{ \pd A, \pd \cok f \}$ so we always have $\pd C(X) \geq \pd A$. If $\pd X = \infty$ then the statement holds so let us assume $\pd X = n < \infty$. We proceed by induction on $n$. If $n=0$ then $C(X)$ is projective so $\pd X = \pd C(X) = \pd A = 0$. Next assume the statement holds whenever the projective dimension is less than $n$. Let $P \to A$ and $P' \to \cok f$ be epimorphisms from projectives. Then we have an epimorphism $T(P, P') \to X$. If we let $\Omega A$ be the kernel of $P \to A$ and $X' = (\Omega A, K, \theta)$ be the kernel of $T(P, P') \to X$ as shown in the following diagram
		\begin{center}
		\begin{tikzcd}
			& F\Omega A \ar[r] \ar[d, "\theta", swap] & FP \ar[r] \ar[d, hookrightarrow] & FA \ar[r] \ar[d, "f"] & 0\\
			0 \ar[r] & K \ar[r] & P' \oplus FP \ar[r] & B \ar[r] & 0,
		\end{tikzcd}
		\end{center}
		then we have $\pd A \leq \pd \Omega A + 1$ and $\pd X = \pd X' + 1$. By induction we have that $\pd X' \geq \pd \Omega A$ and so $\pd X \geq \pd \Omega A +1 \geq \pd A$. 
		
		If $A=0$ then we can associate $C(X)=(0, B)$ with $B$. Any projective resolution $P_B^\bullet$ of $B$ gives a resolution of $X$ by $T(0, P_B^\bullet)$, and any resolution $P_X^\bullet$ of $X$ gives a resolution of $(0, B)$ by $C(P_X^\bullet)$. Thus $\pd X = \pd B$.
	\end{proof}
\end{prop}

Now we are ready for the main theorem of this section, where we give an upper bound on the finitistic dimension of the comma category.

\begin{theorem}\cite[Theorem~4.20]{FGR75}\label{thm:findim_of_comma_cat}
	The finitistic dimension of the comma category $(F, \mathcal B)$ is bounded above by $\findim(\mathcal A) + \findim(\mathcal B) + 1$.
	\begin{proof}
		Let $X=(A, B, f)$ be an element of the comma category with finite projective dimension. Let $P_A^\bullet$ be a projective resolution of $A$ shorter than $\findim(\mathcal A)$. Similar to what we did in \cref{prop:pd_in_commacat} define $P_X^0$ to be $T(P_A^0, P(\cok f))$ where $P(\cok f)$ is a projective module with an epimorphism onto $\cok f$. Then we have that the kernel of $P_X^0 \to X$ is 
		\begin{tikzcd}
			F\Omega A \ar[r, "\theta^0"] & K^0.
		\end{tikzcd}
		We continue inductively defining $P_X^n$ to be $T(P_A^n, \cok \theta^{n-1})$. Then $\Omega^{\findim(\mathcal A)+1} X = (0, K^{\findim(\mathcal A)}, 0)$. Then by \cref{prop:pd_in_commacat} we know that $\pd\Omega^{\findim(\mathcal A)+1}X = \pd K^{\findim(\mathcal A)} \leq \findim(\mathcal B)$. So $$\pd X \leq \findim(\mathcal A) + \findim(\mathcal B) + 1.$$
	\end{proof}
\end{theorem}

Before applying this to triangular matrix rings, let us have a look at a simple example.

\begin{example}\label{ex:triangular_matrix_ring}
	If $k$ is a field, $\mathcal A = \mathcal B = \mod k$, and $F$ is the identity, then the comma category $(F, \mathcal B)$ is equivalent to the category of finite dimensional representations of $A_2$ over $k$. 
\end{example}

In this example $\mathcal A$ and $\mathcal B$ both have finitistic dimension 0 while $(F, \mathcal B)$ has finitistic dimension 1. So the bound shown above is tight. 

\begin{defn}[Triangular matrix ring]
	Let $R$ and $S$ be rings, and let $M$ be an $S$-$R$-bimodule. Then the \emph{triangular matrix ring} $\begin{pmatrix}
	R & 0\\
	M & S
	\end{pmatrix}$ is the ring of all matricies $\begin{bmatrix}
	r & 0\\
	m & s
	\end{bmatrix}$ with $r\in R$, $s\in S$, and $m\in M$. The multplication is given by
	$$\begin{bmatrix}
	r & 0\\
	m & s
	\end{bmatrix}\begin{bmatrix}
	r' & 0\\
	m' & s'
	\end{bmatrix}=\begin{bmatrix}
	rr' & 0\\
	mr' + sm' & ss'
	\end{bmatrix}.$$
\end{defn}

We have already hinted at an example of this in \cref{ex:triangular_matrix_ring}. The algebra $kA_2$ is isomorphic to the matrix ring $\begin{pmatrix}
k & 0\\
k & k
\end{pmatrix}$, and we saw how $\mod kA_2$ becomes the comma category for a functor between $\mod k$ and $\mod k$. In fact whenever $\Lambda$ is a triangular matrix ring, the module category $\mod \Lambda$ will be the comma category for some functor.

\begin{prop}\label{prop:triangular_matrix_is_comma_cat}
	If $\Lambda = \begin{pmatrix}
	R & 0\\
	M & S
	\end{pmatrix}$ is a triangular matrix ring, then $\mod \Lambda$ is isomorphic to the comma category $(M \otimes_R -, \mod S)$.
	\begin{proof}
		Notice, if $N$ is a $\Lambda$-module, then as an abelian group $N$ splits as a direct sum into
		$$N= N_R \oplus N_S :=
		\begin{bmatrix}
		1 & 0\\
		0 & 0
		\end{bmatrix}N \oplus
		\begin{bmatrix}
		0 & 0\\
		0 & 1
		\end{bmatrix}N.$$
		
		By restriction of scalars we can think of $N_R$ as an $R$-module and $N_S$ as an $S$-module. Further multiplication by $\begin{bmatrix}
		0 & 0\\
		m & 0
		\end{bmatrix}$ is 0 on $N_S$ and maps $N_R$ into $N_S$. So $N$ consists of an $R$-module $N_R$, an $S$-module $N_S$ and a $S$-$R$-linear map $M \to \Hom_\mathbb{Z}(N_R, N_S)$, or equivalently a $S$-linear map $M \otimes_R N_R \to N_S$.
		
		This gives us the equivalence between $\mod \Lambda$ and $(M \otimes_R -, \mod S)$.
	\end{proof}
\end{prop} 

\begin{cor}
	When $\Lambda$ is the triangular matrix algebra above, then 
	$$\findim(\Lambda) \leq \findim(R) + \findim(S)+1.$$
\end{cor}

\subsection{Recollements for triangular matrix rings}\label{sec:recollemt_of_triangular_rings}

There is an analogues definition of recollement between abelian categories. If $\Lambda$ is a triangulated matrix algebra as above then we do get a recollement of abelian categories

\begin{center}
	\begin{tikzcd}[column sep=4cm]
		\mod S \ar[r, ""{name=i}]{}{\operatorname{inc}} & 
		\ar[l, swap, ""{name=il}, bend right=30]{}{\Lambda / \Lambda e_R \Lambda \otimes_\Lambda -} \ar[l, ""{name=ir}, bend left=30]{}{\Hom(\Lambda e_S, -)}
		\mod \Lambda \ar[r, ""{name=j}]{}{\Hom(\Lambda e_R, -)=e_R\Lambda \otimes-} & 
		\ar[l, swap, ""{name=jl}, bend right=30]{}{\Lambda e_R \otimes -} \ar[l, ""{name=jr}, bend left=30]{}{\Hom(e_R\Lambda, -)}
		\mod R
		\arrow[phantom, from=il, to=i, "\dashv" rotate=-90]
		\arrow[phantom, from=i, to=ir, "\dashv" rotate=-90]
		\arrow[phantom, from=jl, to=j, "\dashv" rotate=-90]
		\arrow[phantom, from=j, to=jr, "\dashv" rotate=-90]
	\end{tikzcd}	
\end{center}

\todo{something about recoll of abcats}

By taking derived functors we get a recollement of unbounded derived categories, which also restricts to a recollement between $D^-(S)$, $D^-(\Lambda)$ and $D^-(R)$\cite[Corollary~15]{Ko91}.

This does not in general restrict to a recollement of bounded derived categories, but if $M$ has finite projective dimension both as an $R$-module and an $S$-module then it does. 


% Does the same idea of recollement works if we consider $\D^-$ instead of $\D^b$? Clearly not since Happell restricts to $\D^b$, where is does it break? ANSWER: In $\D^-$ we cannot characterize bounded complexes of injectives in the same way, so we cannot bound $H^i(X)$ from below. 

% Note $R = e_R\Lambda e_R$ and $S=\Lambda/\Lambda e_R \Lambda$.

% This gives a recollement of abelian categories. Under what conditions does this extend to one for derived categories?? When $M$ is projective as $S$ and $R$ module then the top functors are exact. The other functors are always exact. If $M$ has finite projective dimension then the derived functors should be well defined on the bounded derived category. Can we say something meaningful in the unbounded case \todo{?}


\section{Contravariant finiteness}
\section{Contravariantly finite subcategories}\label{sec:contravariantly_finite}

%Results are generalized in \cite{Trl01}

In this section we will study the structure of contravariantly finite resolving subcategories. One example of a resolving subcategory is the subcategory of modules with finite projective dimension, which we denote by $\mathcal P^\infty$. In \cref{thm:contravariantly_finite_resolving_is_Xi_filtered} we give a description of the structure of a contravariantly finite resolving subcategory from the approximations of the simple modules. As a corollary we get that an algebra has finite finitistic dimension when $\mathcal P^\infty$ is contravariantly finite. \cref{exam:not_contravariantly_finite}, discovered by Igusa--Smalø--Todorov, shows that $\mathcal P^\infty$ can fail to be contravariantly finite even for monomial algebras with radical cubed equal to 0.

It is known that $\mathcal P^\infty$ is contravariantly finite when the algebra is stably equivalent to a hereditary algebra. This was shown by Auslander--Reiten in their original paper \cite{AR91}. We consider a generalization of this class in \cref{sec:stable_hereditary_algebras} through the perspective of the Igusa--Todorov-function.

Throughout this section we, as usual, assume $\Lambda$ is a finite dimensional algebra, though it should be noted that all the results still hold if we instead let $\Lambda$ be an artin algebra.

\begin{defn}[Resolving]
	A full subcategory of an abelian category is called \emph{resolving} if 
	\begin{enumerate}[i)]
		\item It is closed under extensions.
		\item It contains the projectives.
		\item It contains the kernel of any epimorphism between two of its objects.
	\end{enumerate}
\end{defn}

Note that $\mathcal P^\infty$ is a resolving subcategory.

In the next few propositions we will consider a resolving subcategory $\mathcal X$, and its $\Ext$-orthogonal complement
\begin{align*}
	\mathcal Y := \ker \Ext^{\geq 1}(\mathcal X, -) = \{Y \in \mathcal C \mid \Ext^i(X, Y)=0, \forall X \in \mathcal X, \forall i \geq 1\},
\end{align*}
which we now show is equal to 
\begin{align*}
\ker \Ext^1(\mathcal X, -) = \{Y \in \mathcal C \mid \Ext^1(X, Y)=0, \forall X \in \mathcal X\}.
\end{align*}

\begin{lemma}\label{lem:resolving_ext_vanish}
	Let $\mathcal X$ be a resolving subcategory. Then $\Ext^1(\mathcal X, Y) = 0$ implies that $\Ext^i(\mathcal X, Y)=0$ for all $i \geq 1$.
	\begin{proof}
		Since $\mathcal X$ contains the projectives, $\Omega X$ is the kernel of an epimorphism between objects in $\mathcal X$. Thus $\mathcal X$ contains all syzygies, and we have $\Ext^i(X, Y) = \Ext^1(\Omega^{i-1}X, Y) = 0$.
	\end{proof}
\end{lemma}

\begin{prop}\label{prop:complement_closed_under_extension}
	Let $\mathcal X$ be a full subcategory. Then the $\Ext$-orthogonal complement $\mathcal Y := \ker\Ext^{i}(\mathcal X, -)$ is closed under extensions.
	\begin{proof}
		Let $0 \to Y \to E \to Y' \to 0$ be an extension of objects in $\mathcal Y$, and let $X$ be an object of $\mathcal X$. Then we get an exact sequence  
		\begin{center}
			\begin{tikzcd}
			0=\Ext^i(X, Y) \ar[r] & \Ext^i(X, E) \ar[r] & \Ext^i(X, Y')=0
			\end{tikzcd}
		\end{center}
		Thus $\Ext^i(X, E)=0$, and so $E$ is in $\mathcal Y$.
	\end{proof}
\end{prop}

\begin{lemma} \label{lem:exact_sequence_from_approximation}
	Let $\mathcal X$ be a contravariantly finite, resolving subcategory of $\mod \Lambda$. Then for every object $C \in \mod\Lambda$ there is a short exact sequence 
	$$0 \to Y \to X \to C \to 0$$ 
	with $X\to C$ minimal $\mathcal X$-approximation and $\Ext^i(\mathcal X, Y)=0$ for all $i \geq 1$.
	\begin{proof}
		Since $\mathcal X$ is contravariantly finite, $C$ has a minimal $\mathcal X$-approximation $X \to C$. Since $\mathcal X$ contains the projective cover of $C$ this approximation must be an epimorphism. So it is part of a short exact sequence $$0 \to Y \to X \to C \to 0.$$ Let $X'$ be an arbitrary object in $\mathcal X$. Taking the long exact sequence in $\Ext(X', -)$ gives us
		\begin{center}
		\begin{tikzcd}
			\Hom(X', Y) \ar[r]&\Hom(X', X) \ar[r] & \Hom(X', C) \ar[dll, overlay, out=-15, in=165] \\ \Ext^1(X', Y) \ar[r] & \Ext(X', X)^1 \ar[r] & \Ext^1(X', C)
		\end{tikzcd}
		\end{center}
		Since $X \to C$ is an approximation, we know that $\Hom(X', X) \to \Hom(X', C)$ is epi. Thus if we can prove that $\Ext^1(X', X) \to \Ext^1(X', C)$ is mono we would have that $\Ext^1(X', Y)=0$. 
		
		Assume we have an element of $\Ext^1(X', X)$ that is mapped to 0, i.e. we have a commutative diagram
		\begin{center}
			\begin{tikzcd}
			0 \ar[r] & X \ar[r] \ar[d] & E \ar[r] \ar[d] & X' \ar[d, equal] \ar[r] & 0\\
			0 \ar[r] & C \ar[r] & C \oplus X' \ar[r] & X' \ar[r] & 0.
			\end{tikzcd}
		\end{center}
		Since $\mathcal X$ is closed under extensions $E$ is in $\mathcal X$. By composing with projection $C\oplus X' \to C$ we get a commutative triangle
		\begin{center}
			\begin{tikzcd}
			 X \ar[r] \ar[d] & E \ar[dl]\\ 
			 C 
			\end{tikzcd}
		\end{center}
		Since $X \to C$ is an approximation we get that $E \to C$ factors through $X$. The endomorphism $X \to E \to X$ leaves the approximation unchanged, so by minimality it must be an isomorphism. Hence 
		\begin{center}
			\begin{tikzcd}
				0 \ar[r] & X \ar[r] & E \ar[r] & X' \ar[r] & 0
			\end{tikzcd}
		\end{center}
		is split and $\Ext^1(X', X) \to \Ext^1(X', C)$ is injective. Thus we have that $\Ext^1(X', Y)=0$, and by \cref{lem:resolving_ext_vanish} we get $\Ext^i(X', Y)=0$ for all $i\geq 1$.
	\end{proof}
\end{lemma}

We now prove the main theorem of this section, about the structure of approximations for a resolving subcategory.

\begin{theorem} \cite[3.8]{AR91} \label{thm:contravariantly_finite_resolving_is_Xi_filtered}
	Let $\mathcal X$ be a contravariantly finite, resolving subcategory of $\mod \Lambda$. Let $X_i$ be the minimal approximation of $S_i$. Then any $X \in \mathcal X$ is a direct summand of an $X_i$-filtered module.
	\begin{proof}
		The first part of the proof is to show by induction on length that any module $C$ is in an exact sequence $0 \to Y \to X \to C \to 0$ with $X$ $X_i$-filtered and $\Ext^1(\mathcal X, Y)=0$.
		
		For the base case if $C=S_i$ is simple, then by \cref{lem:exact_sequence_from_approximation} we have an exact sequence $0 \to Y \to X_i \to C \to 0$ with the desired properties stated above. 
		
		For the induction step, assume it holds for all modules of length less than $n$, and let $C$ be a module of length $n$. Then by Jordan-Hölder $C$ is the extension of two modules of length less than $n$. Say
		\begin{center}
			\begin{tikzcd}
			0 \ar[r] & C' \ar[r] & C \ar[r] & C'' \ar[r] & 0.
			\end{tikzcd}
		\end{center}
		Applying the induction hypothesis we get a diagram on the form
		\begin{center}
			\begin{tikzcd}
			& 0 \ar[d] &&0 \ar[d] &\\
			& Y' \ar[d] && Y'' \ar[d] & \\
			 & X' \ar[d] && X'' \ar[d] & \\
			0 \ar[r] & C' \ar[d] \ar[r] & C \ar[r] & C'' \ar[r] \ar[d] & 0\\
			& 0 &&0 &
			\end{tikzcd}
		\end{center}
		Taking the pullback of $X'' \to C''$ we get a diagram
		\begin{center}
			\begin{tikzcd}
			0 \ar[r] & C' \ar[r] \ar[d, equal] & E \ar[r] \ar[d] & X'' \ar[r] \ar[d] & 0\\
			0 \ar[r] & C' \ar[r] \ar[d] & C \ar[r] \ar[d] & C'' \ar[r] \ar[d] & 0\\
			&0&0&0&
			\end{tikzcd}
		\end{center}
		Since $Y'$ satisfies $\Ext^1(\mathcal X, Y') = 0$ by \cref{lem:resolving_ext_vanish} we have $\Ext^2(\mathcal X, Y')=0$. In particular from the long exact sequence
		\begin{center}
			\begin{tikzcd}[column sep=10pt]
				 0=\Ext^1(X'', Y) \ar[r] & \Ext^1(X'', X') \ar[r] & \Ext^1(X'', C) \ar[r] & \Ext^2(X'', Y)=0
			\end{tikzcd}
		\end{center}
		 we get that $X' \to C'$ induces an isomorphism $\Ext^1(X'', X') \to \Ext^1(X'', C)$. Thus the short exact sequence $0 \to C' \to E \to X'' \to 0$  must come from a sequence $0 \to X' \to X \to X'' \to 0$. This gives us a diagram
		\begin{center}
			\begin{tikzcd}
			& 0 \ar[d] &&0 \ar[d] &\\
			 & Y' \ar[d] && Y'' \ar[d] & \\
			0 \ar[r] & X' \ar[d] \ar[r] & X \ar[r] \ar[d] & X'' \ar[d] \ar[r] & 0\\
			0 \ar[r] & C' \ar[d] \ar[r] & C \ar[r] & C'' \ar[r] \ar[d] & 0\\
			& 0 &&0 &
			\end{tikzcd}
		\end{center}
		Applying the Snake Lemma we can fill out the diagram:
		\begin{center}
			\begin{tikzcd}
			& 0 \ar[d] & 0\ar[d] &0 \ar[d] &\\
			0 \ar[r] & Y' \ar[d] \ar[r] & Y \ar[r] \ar[d] & Y'' \ar[d] \ar[r] & 0\\
			0 \ar[r] & X' \ar[d] \ar[r] & X \ar[r] \ar[d] & X'' \ar[d] \ar[r] & 0\\
			0 \ar[r] & C' \ar[d] \ar[r] & C \ar[r] \ar[d] & C'' \ar[r] \ar[d] & 0\\
			& 0 &0&0 &
			\end{tikzcd}
		\end{center}
		Since $X$ is an extension of $X_i$-filtered modules, it is also $X_i$-filtered. Since $Y$ is the extension of $Y''$ and $Y'$ it follows from \cref{prop:complement_closed_under_extension} that $\Ext^1(\mathcal X, Y)=0$.
		
		Hence any $C$ fits into a sequence $0 \to Y \to X \to C \to 0$ with $X$ being $X_i$-filtered and $\Ext^{1}(\mathcal X, Y)=0$.
		
		Now suppose that $C$ is in $\mathcal X$, and let $0 \to Y \to X \to C \to 0$ be as before. Then we get that
		\begin{center}
		\begin{tikzcd}
		\Hom(C, X) \ar[r] & \Hom(C, C) \ar[r] & \Ext^1(C,Y) = 0
		\end{tikzcd}
		\end{center}
		is exact, and thus $C$ is a direct summand of $X$. So every object in $\mathcal X$ is a direct summand of an $X_i$-filtered module.
	\end{proof}
\end{theorem}

Applying this to $\mathcal P^\infty$ we get our wanted result about the finitistic dimension.

\begin{cor}\label{cor:contravariant_finite_implies_FDC}
	If $\mathcal P^\infty$ is contravariantly finite, then the finitistic dimension is the supremum of the projective dimension of the approximations of the simple modules. In particular it is finite.
\end{cor}

To finish this section of we give two examples. The first example is due to Igusa--Smalø--Todorov, which shows that $\mathcal P^\infty$ need not be contravariantly finite even for monomial algebras with $J^3 = 0$.

\begin{example}\cite[Proposition~2.3]{IST90}\label{exam:not_contravariantly_finite}
	Let $\Lambda$ be the path algebra of 
	\begin{center}
	\begin{tikzcd}[column sep = 50pt]
		1 \ar[r, "\alpha", bend left=45] \ar[r, "\beta"] & 2 \ar[l, "\gamma", bend left = 45]
	\end{tikzcd}
	\end{center}
	with relations $\alpha \gamma$, $\beta\gamma$, and $\gamma\alpha$ over an algebraically closed field $k$. Then $\findim(\Lambda) = 1$, but $\mathcal P^\infty$ is not contravariantly finite.
	
	\begin{proof}
		The indecomposable projective $\Lambda$-modules are given by the following quivers
		\begin{center}
			\begin{tikzcd}[column sep=7pt]
				&1 \ar[dl, swap, "\alpha"] \ar[dr, "\beta"]&\\
				2&&2 \ar[d, "\gamma"]\\
				&&1
			\end{tikzcd}
			\hspace{2cm}
			\begin{tikzcd}
				2\ar[d, "\gamma"]\\
				1
			\end{tikzcd}
		\end{center}
		Note that both the indecomposable projectives have even dimension, so any projective module has even dimension. Then if $X$ is a module with finite projective dimension, since $\dim X = \sum (-1)^i \dim P_X^i$ the dimension of $X$ is also even. In particular the two simple modules have infinite projective dimension.
		
		The radical of $P_1$ is $P_2\oplus S_2$ and the radical of $P_2$ is $S_1$, so the radical of an arbitrary projective looks like $P_2^n \oplus S_1^m \oplus S_2^n$. Let $P \to X$ be the projective cover of a module with finite projective dimension. Then $\Omega X$ is a submodule of $JP = P_2^n \oplus S_1^m \oplus S_2^n$. Let $M$ be an indecomposable summand of $\Omega X$, and consider the composition $M \to JP \to P_2$ for any possible projection to $P_2$. If this is epi then we must have $M = P_2$. If none of these are epi then $M$ is contained in $JP_2^n \oplus S_1^m \oplus S_2^n = S_1^{m+n} \oplus S_2^n$. This would mean $M=S_1$ or $M=S_2$, but $S_1$ and $S_2$ both have infinite projective dimension. Thus we must have $\Omega X$ projective, and so $\pd X \leq 1$.
		
		Next we want to show that $S_1$ has no minimal approximation by modules with finite projective dimension. Assume for the sake of contradiction that $X \to S_1$ is such a minimal approximation. Then we claim that $P_2$ is not a submodule of $X$. If $X$ had $P_2$ as a submodule, then since $\Hom(P_2, S_1) = 0$ the approximation would factor through $X'=X/P_2$. From the short exact sequence $0 \to P_2 \to X \to X' \to 0$ it follows that 
		$$\pd X' \leq \max\{\pd P_2 + 1, \pd X\} < \infty,$$ 
		and so $X'$ would give an approximation of shorter length, contradicting the minimality of $X$.
		
		% because $X'$ would also have finite projective dimension. Which can be seen in the diagram below.
		% \begin{center}
		% \begin{tikzcd}
		% 	&& 0 \ar[d] & 0 \ar[d] &\\
		% 	0\ar[r]& P^1_X \ar[d, equal]\ar[r] & P^1_X \oplus P_1 \arrow[dr, phantom, "\usebox\pullback" , very near start, color=black] \ar[d]\ar[r] & P_1\ar[d]\ar[r] & 0\\
		% 	0 \ar[r] & P^1_X \ar[r] & P^0_X \ar[d]\ar[r] & X\ar[d]\ar[r] & 0\\
		% 	&& X' \ar[d]\ar[r, equal] & X' \ar[d] \\
		% 	&&0&0
		% \end{tikzcd}
		% \end{center}

		This means that $\gamma X = 0$, because if there was an element $x \in X$ with $\gamma x \neq 0$, then $(e_2 x)$ would be a submodule of $X$ isomorphic to $P_2$. So $X$ is a $\Lambda/(\gamma)$ module. 
		
		The algebra $\Lambda/(\gamma)$ is the path algebra of the 2-Kronecker quiver, whose representation theory is well understood (c.f. \cite[Chapter~VIII.7]{ARS97} or \cite[Chapter~3.2]{Ring84}). Specifically $\Lambda/(\gamma)$ can be associated with the subquiver highlighted below. 
		\begin{center}
			\begin{tikzcd}[column sep = 50pt]
			1 \ar[r, "\alpha", bend left=45] \ar[r, "\beta"] & 2 \ar[l, opacity=0.3, "\gamma", bend left = 45]
			\end{tikzcd}
		\end{center}
		The indecomposable modules are as given in the table below.
		
		\begin{center}
		\begin{tabular}{ccc}
			\begin{tikzcd}[ampersand replacement=\&, column sep = 45pt]
			k^n \ar[bend left=35, r, "\begin{bmatrix}
			I_n\\ 0
			\end{bmatrix}"pos=0.55] \ar[swap, r, "\begin{bmatrix}
			0\\I_n
			\end{bmatrix}"]\& k^{n+1}
			\end{tikzcd}
			&
			\begin{tikzcd}[ampersand replacement=\&, column sep = 45pt, row sep=40pt]
			k^n \ar[bend left=35]{r}{J(n, \lambda)} \ar[swap, bend right=0]{r}{I_n} \& k^{n}\\
			k^n \ar[bend left=35]{r}{I_n} \ar[swap, bend right=0]{r}{J(n, 0)} \& k^{n}\\
			\end{tikzcd}
			&
			\begin{tikzcd}[ampersand replacement=\&, column sep = 45pt]
			k^{n+1} \ar[bend left=35, pos=0.45]{r}{\begin{bmatrix}
				I_n & 0
				\end{bmatrix}} \ar[swap, bend right=0]{r}{\begin{bmatrix}
				0 & I_n
				\end{bmatrix}} \& k^{n}
			\end{tikzcd}
			\\
			preprojective & regular & preinjective
		\end{tabular}
		\end{center}
		
	We see that the preprojective and preinjective modules both have odd dimension, so they will have infinite projective dimension as $\Lambda$-modules. We can easily verify that the $\Lambda/(\gamma)$-modules 
	\begin{tikzcd}
	k \ar[bend left=25]{r}{\lambda} \ar[swap, bend right=0]{r}{1} & k\\
	\end{tikzcd}
	all have finite projective dimension as $\Lambda$-modules and that they have a nonzero map onto $S_1$. So each of these modules would need to have a nonzero map to $X$. But it is easy to verify that there is a nonzero homomorphism between the regular modules only if they have the same value of $\lambda$. So for it to be possible for $X$ to factorize all these maps we would need $X$ to have infinitely many direct summands. Since we are working with finitely generated modules this is impossible, hence $S_1$ has no approximation, and the subcategory is not contravariantly finite.
	\end{proof}
\end{example}

In the next example we look at the opposite algebra of $\Lambda$, for which $\mathcal P^\infty$ is contravariantly finite for $\Gamma$. This shows that there is no immediate relationship between $\mathcal P^\infty$ being contravariantly finite for $\Lambda$ and for $\Lambda^{\op}$.

\begin{example}
	Let $\Gamma$ be the opposite algebra of the one in \cref{exam:not_contravariantly_finite}. That is, $\Gamma$ is the path algebra of 
	\begin{center}
		\begin{tikzcd}[column sep = 50pt]
		2 \ar[r, "\hat{\alpha}", bend left=45] \ar[r, "\hat{\beta}"] & 1 \ar[l, "\hat{\gamma}", bend left = 45]
		\end{tikzcd}
	\end{center}
	with relations $\hat{\gamma}\hat{\alpha}$, $\hat{\gamma}\hat{\beta}$, and $\hat{\alpha}\hat{\gamma}$. Then $\mathcal P^\infty$ is contravariantly finite. In other words the subcategory of $\Lambda$-modules with finite injective dimension is covariantly finite.
	\begin{proof}
		The indecomposable projective $\Gamma$-modules are given by the following quivers 
		\begin{center}
			\begin{tikzcd}
				1 \ar[d, "\hat{\gamma}"]\\2 \ar[d, "\hat{\beta}"]\\1
			\end{tikzcd}
			\hspace{2cm}
			\begin{tikzcd}[column sep=7pt]
				&2 \ar[dl, swap, "\hat{\alpha}"] \ar[dr, "\hat{\beta}"]&\\
				1&&1
			\end{tikzcd}
		\end{center}
		
		Similar to before, notice that the indecomposable projective modules are 3-dimensional and thus every module with finite projective dimension will have dimension a multiple of 3. So in particular the simple modules have infinite projective dimension. 
		
		Let $X$ be a module with finite projective dimension, and let $P$ be its projective cover. We have that $\Omega X$ is a submodule of $JP$. Notice that $\hat{\alpha} J = \hat{\gamma} J = 0$, so $\Omega X$ is a $\Gamma/(\hat{\alpha}, \hat{\gamma})$-module. But $\Gamma/(\hat{\alpha}, \hat{\gamma})$ is simply isomorphic to the path algebra of  
		\begin{tikzcd}
			2 \ar[r] & 1
		\end{tikzcd},
		over which there are just 3 indecomposable modules. We already know that the simple modules cannot be summands of $\Omega X$, because they have infinite projective dimension. The non-simple module
		\begin{tikzcd}
		k \ar[r, "1"] & k
		\end{tikzcd}
		is 2-dimensional and thus also has infinite projective dimension over $\Gamma$. So we conclude that $\Omega X = 0$, so $X$ is projective.
		
		So the only modules with finite projective dimension are the projectives themselves. In particular there are only a finite number of indecomposable modules with finite projective dimension. So the subcategory is contravariantly finite. 
	\end{proof}
\end{example}


%\section{Repdimension}
%
\begin{defn}[Representation dimension]
	Let $\Lambda$ be a finite dimensional algebra. The \emph{representation dimension} of $\Lambda$, denoted $\repdim(\Lambda)$, is the minimal global dimension of $\End(M)^{\operatorname{op}}$ for $M$ a generator-cogenerator in $\mod\Lambda$. We call a generator-cogenerator that achieves this minimum an \emph{Auslander-generator}.
\end{defn}

\begin{defn}[$\mathcal M$-resolutions]
	Let $X$ be an object in $\mod\Lambda$ and $\mathcal M$ a contravariantly finite subcategory.
	\begin{center}
	\begin{tikzcd}
		\cdots \ar[rd, two heads] \ar[r] & M_2 \ar[rd, two heads] \ar[r] & M_1 \ar[rd, two heads] \ar[r] & M_0 \ar[rd, two heads]\\
		&\Omega_M^3 X \ar[u, hook] & \Omega_M^2  \ar[u, hook] X & \Omega_M X  \ar[u, hook] & X
	\end{tikzcd}
	\end{center}
	If the maps $M_n \twoheadrightarrow \Omega_M^nX$ are minimal right $\mathcal M$-approximations for $n\geq 0$ (they need not be surjective), and $\Omega_M^{n+1} \hookrightarrow M_n$ are their kernels, then this is a minimal \emph{$\mathcal M$-resolution} of $X$. The \emph{$\mathcal M$-res-dimension} of $X$ is the length of this sequence of (nonzero) $M_i$'s, and the $\mathcal M$-res-dimension of $\Lambda$ is the supremum of the dimension on its objects.
\end{defn}

\begin{prop}
	If the representation dimension of $\Lambda$ is at least $2$, then $\repdim(\Lambda) - 2$ equals the minimum of $M$-$\operatorname{res-dim}(\mod \Lambda)$ for $M$ both generator and cogenerator. In fact, for any generator-cogenerator, $M$-$\operatorname{res-dim}(\mod \Lambda)$ is two less than the global dimension of $\End(M)^{\op}$.
	
	\begin{proof}
		Let $M$ be a generator-cogenerator. We first show that the global dimension of $\End(M)^{\op}$ is less than or equal to $M$-$\operatorname{res-dim}(\mod \Lambda) + 2$. 
		
		The functor $\Hom(M, -)$ is an equivalence from $\add M$ to $\proj\End(M)^{\op}$, which maps minimal $M$-approximations to projective covers. Let $X$ be any module in $\mod\End(M)^{\op}$ with projective dimension at least 2. Then it has a projective presentation $$\Omega^2X \to (M,M_1) \to (M,M_0) \to X.$$
		Because of the equivalence this is induced by a map $f\colon M_1\to M_0$. Since $\Hom(M,-)$ is left exact we have that $\Omega^2X \cong \Hom(M, \ker f)$, and so the projective dimension of $X$ is $2$ plus the $M$-res-dimension of $\ker f$. Hence the global dimension $\End(M)^{\op}$ is less than or equal to $M$-$\operatorname{res-dim}(\mod \Lambda) + 2$. 
		
		Next we prove the other inequality.
		
		Since $M$ is a cogenerator any module $Y$ in $\mod\Lambda$ has a copresentation 
		\begin{center}
		\begin{tikzcd}
			0 \ar[r] & Y \ar[r] & M_0 \ar[r, "f"] & M_1.
		\end{tikzcd}
		\end{center}
		Applying $(M,-) := \Hom(M,-)$ we get
		\begin{center}
		\begin{tikzcd}
		0 \ar[r] & (M,Y) \ar[r] & (M,M_0) \ar[r]{}{(M,f)} & (M,M_1) \ar[r] & \cok(M,f) \ar[r] & 0.
		\end{tikzcd}
		\end{center}
		If the projective dimension of $\cok(M,f)$ is less than 2, then $(M, Y)$ is a direct summand of $(M, M_0)$. This means that $(M,Y) \cong (M, M')$, so the minimal $M$-approximation of $Y$ is $M'$, and $(M, \Omega_M Y) = 0$. Since $M$ is a generator this means $\Omega_M Y = 0$ and thus the $M$-res-dimension of $Y$ is 0.
		
		So provided the projective dimension of $\cok(M,f)$ is larger than or equal to 2, it equals the $M$-res-dimension of $Y$ plus 2. In particular the global dimension of $\End(M)^{\op}$ is larger than or equal to $M$-$\operatorname{res-dim}(\mod \Lambda) + 2$. Hence they are equal.
	\end{proof}
\end{prop}

\begin{prop}
	The representation dimension of an artin algebra is always finite. \cite{Iya02}
\end{prop}

\begin{theorem}
	The representation dimension of $\Lambda$ is less than or equal to 2 if and only if $\Lambda$ is representation finite.
	\begin{proof}
		Assume $\Lambda$ is representation finite and let $M$ be the direct sum of all indecomposable modules (up to iso). Then $M$ is a generator-cogenerator. Let $X$ be an $\End(M)^{\operatorname{op}}$-module with projective presentation 
		$$(M,M_1) \to (M, M_0) \to X \to 0.$$ 
		Let $M_2$ be the kernel of $M_1 \to M_0$. Since $M$ is the sum of all indecomposables $M_2$ is in $\add M$, so 
		$$0 \to (M, M_2) \to (M,M_1) \to (M, M_0) \to X \to 0$$ 
		is a projective resolution of $X$. So $\Lambda$ has representation dimension at most 2.
		
		Assume $\Lambda$ has representation dimension at most 2, and let $M$ be an Auslander-generator. We want to show that $\add M = \mod\Lambda$. Let $X$ be any $\Lambda$-module, and let $$0 \to X \to I_0 \to I_1$$ be a minimal injective presentation. If $I_0 \to I_1$ is split then $X$ is injective and thus in $\add M$. Let $M_X$ be a minimal $M$-approximation of $X$, let $\Omega_M X$ be the kernel of the approximation, and let $Y$ be the cokernel of $(M, I_0) \to (M, I_1)$. Then $$(M,\Omega_M X) \to (M,M_X) \to (M, I_0) \to (M, I_1) \to Y \to 0$$ is a minimal exact sequence. Since the global dimension of $\End(M)^{\operatorname{op}}$ is at most 2 this means that $(M, \Omega_M X)=0$. Consequently we have that $\Omega_M X = 0$ and that $X=M_X$, so $X$ is in $\add M$. Thus $\Lambda$ is representation finite.
	\end{proof}
\end{theorem}

\begin{theorem}\cite[Corollary~8]{IgTo05}
	If $\Lambda = \End_\Gamma(P)^{\op}$ for an algebra $\Gamma$ with global dimension at most 3, and $P$ projective, then $\findim(\Lambda) < \infty$.
	\begin{proof}
		Let $X$ be any $\Lambda$-module with finite projective dimension. Then it has a projective presentation $(P, P_1) \to (P,P_0) \to X \to 0$ where $(P,P_i)=\Hom_\Gamma(P,P_i)$ with $P_i \in \add P$. Since $(P,-)$ is an equivalence from $\add P$ to $\proj\Lambda$ this corresponds to a map $P_1 \to P_0$ which we can extend to a projective resolution in $\Gamma$:
		\begin{center}
			\begin{tikzcd}
			0 \ar[r] & P_3 \ar[r] & P_2 \ar[r] & P_1 \ar[r] & P_0.
			\end{tikzcd}
		\end{center}
		Applying the exact functor $(P, -)$, we get an exact sequence
		\begin{center}
			\begin{tikzcd}
			0 \ar[r] & (P,P_3) \ar[r] & (P,P_2) \ar[r] & (P,P_1) \ar[r] & (P,P_0)\ar[r] & X \ar[r] & 0.
			\end{tikzcd}
		\end{center}
		Truncating this we get a short exact sequence
		\begin{center}
			\begin{tikzcd}
			0 \ar[r] & (P, P_3) \ar[r] & (P, P_2) \ar[r] & \Omega^2 X \ar[r] & 0.
			\end{tikzcd}
		\end{center}
		Then by \cref{thm:projdim_bounded_by_psi} the projective dimension of $\Omega^2 X$ is bounded by $\psi((P, P_3)\oplus (P, P_2))+1$. Which means
		$$\pd X \leq \psi((P, P_3)\oplus (P, P_2))+3 \leq \psi((P,\Gamma))+3$$
		Since this bound doesn't depend on $X$, $\Lambda$ has finite finitistic dimension.
	\end{proof} 
\end{theorem}

\begin{cor}
	If $\repdim(\Lambda) \leq 3$ then $\findim(\Lambda) < \infty$.
	\begin{proof}
		If $\Lambda$ has rep-dimension less than or equal to 3 then by \cref{prop:repdim_auslander_generator} there is a generator-cogenerator $M$ in $\mod\Lambda$ such that $\Gamma := \End_\Lambda(M)$ has global dimension 3 or less. Then since $M$ is a generator $\Lambda$ is in $\add M$ and so $\Hom_\Lambda(M, \Lambda)$ is a projective $\Gamma$-module with $\End_\Gamma(\Hom_\Lambda(M, \Lambda)) = \End_\Lambda(\Lambda) = \Lambda$.
	\end{proof}
\end{cor}


\section{The Igusa--Todorov functions} \label{sec:Igusa-Todorov}
\section{The Igusa--Todorov functions} \label{sec:Igusa-Todorov}

In this section we introduce the Igusa--Todorov functions, which are important tools for bounding the projective dimensions of modules in $\mod \Lambda$. The main theorem is \cref{thm:projdim_bounded_by_psi} in which we give a bound for the projective dimension of modules in a short exact sequence. In \cref{sec:repdimension} we use this to show that algebras with representation dimension at most 3, has finite finitistic dimension, and in \cref{sec:stable_hereditary_algebras} we give an example of a class of algebras which are known to have representation dimension 3.

From this point forward we let $K_0$ be the abelian group generated by isomorphism classes of modules in $\mod\Lambda$, with the relations that $[A\oplus B] - [A] - [B] = 0$ for any modules $A$ and $B$, and $[P]=0$ when $P$ is projective. We define the linear map $L\colon K_0\to K_0$ by $L[A] = [\Omega A]$. For any module $X$, we let $[\add X]$ be the finitely generated subgroup of $K_0$ generated by modules in $\add X$. 

Fitting's lemma [\cref{thm:Fittings_lemma}] tells us that there is an integer $\eta_X$ such that $L\colon L^m[\add X] \to L^{m+1}[\add X]$ is an isomorphism for every $m \geq \eta_X$. We use this to define two important functions from $\mod \Lambda$ to $\mathbb N$.

\begin{defn}[The Igusa--Todorov functions]
	We define two functions $\phi$ and $\psi$ from $\mod\Lambda$ to $\mathbb N$. For a module $M \in \mod\Lambda$ we define $\phi(M)$ to be the integer $\eta_M$ coming from Fitting's lemma, as explained above. In other words, $\phi(M)$ is the smallest integer such that $$L\colon L^m[\add M] \to L^{m+1}[\add M]$$ is an isomorphism for every $m \geq \phi(M)$. We define $\psi(M)$ in a similar way, but adding on an extra term to account for the structure of $\Omega^{\phi(M)}M$. 
	$$\psi(M) = \phi(M) + \sup\left\lbrace\pd Z \; \middle| \; \pd Z < \infty, Z \in \add \Omega^{\phi(M)}M\right\rbrace$$
\end{defn}

We now list the properties needed to prove our main theorem.

\begin{lemma} \cite[Lemma~3]{IgTo05} \label{lem:properties_of_psi}
	\begin{enumerate}[i)]
		\item $\psi(M) = \pd M$, when $\pd M < \infty$.
		\item $\psi(M^k) = \psi(M)$.
		\item $\psi(M) \leq \psi(M\oplus N)$.
		\item If $Z$ is a direct summand of $\Omega^n(M)$ where $n \leq \phi(M)$ and $\pd Z < \infty$, then $\pd Z + n \leq \psi(M)$.
	\end{enumerate}
	\begin{proof}
		\begin{enumerate}[i)]
			\item[] %empty line
			\item If $\pd M < \infty$, then $L^m \neq 0$ for $m < \pd M$, and $L^m =0$ for $m \geq \pd M$. So $\psi(M)=\phi(M)=\pd M$.
			\item The subcategory $\add M^k = \add M$, and $\psi$ is defined only in terms of the additive subcategory $\add M$.
			\item  The subcategory $\add M$ is contained in $\add M\oplus N$, so if $L$ is injective when restricted to $L^m(\add M\oplus N)$ then $L$ is injective when restricted to $L^m(\add M)$. Thus we have $\phi(M) \leq \phi({M\oplus N})$. Further $$\Omega^{\phi({M\oplus N})-\phi(M)}\left(\add\Omega^{\phi(M)}M \right) \subseteq \add\Omega^{\phi({M\oplus N})} M\oplus N,$$ 
			so $\psi(M) \leq \psi(M\oplus N)$.
			\item Let $p=\pd Z$ and $k = \phi(M) - n$. Then $\Omega^k Z$ is in $\add \Omega^{\phi(M)}M$, so $\pd\Omega^k Z + \phi(M) \leq \psi(M)$. Thus $$\pd Z + n = p + n = (p-k) + \phi(M) \leq \pd\Omega^k Z + \phi(M) \leq \psi(M).$$
		\end{enumerate}
	\end{proof}
\end{lemma}

We will now apply these properties to get a bound on the projective dimension of modules in a short exact sequence in terms of the $\psi$-function.

\begin{theorem}\cite[Theorem~4]{IgTo05} \label{thm:projdim_bounded_by_psi}
	Let $0 \to A \to B \to C \to 0$ be a short exact sequence of modules with $\pd C < \infty$. Then $\pd C \leq \psi(A\oplus B)+1$.
	\begin{proof}
		Let $P_A^\bullet$ and $P_C^\bullet$ be the minimal projective resolutions of $A$ and $C$. Then we get a map of short exact sequences
		\begin{center}
		\begin{tikzcd}
			0 \ar[r]  & P_A^0 \ar[r] \ar[d] & P_A^0 \oplus P_C^0 \ar[r] \ar[d] & P_C^0 \ar[r] \ar[d] & 0\\
			0 \ar[r] & A \ar[r] & B \ar[r] & C \ar[r] & 0 
		\end{tikzcd}
		\end{center}
		Applying the Snake Lemma we get $0 \to \Omega A \to \Omega B \oplus P \to \Omega C \to 0$ for some projective module $P$. Thus for some $n \leq \pd C$ we have $L^n[A] = L^n[B]$, and let $n$ be the minimal such number. Clearly $n \leq \phi(A\oplus B
			)$. Let $X = \Omega^n A = \Omega^n B$, then our sequence of $n$-syzygies looks like
		\begin{center}
			\begin{tikzcd}
			0 \ar[r] & X \ar[r] & X\oplus P \ar[r] & \Omega^nC \ar[r] & 0.
			\end{tikzcd}
		\end{center}
		Let $f$ be the composition
		\begin{tikzcd}
			X \ar[r] & X \oplus P \ar[r, "\pi_X"] & X.
		\end{tikzcd}
		Then by Fitting's lemma $X$ breaks as a direct sum into two components $X = Z \oplus Y$ such that $f = f_Z \oplus f_Y$ with $f_Y$ an isomorphism and $f_Z$ nilpotent. In other words the sequence above can be written as
		\begin{center}
			\begin{tikzcd}
			0 \ar[r] & Z\oplus Y \ar[r] & Z \oplus Y\oplus P \ar[r] & \Omega^nC \ar[r] & 0.
			\end{tikzcd}
		\end{center}
		with the left map being
		$$\begin{bmatrix}
			f_Z & 0\\
			0 & f_Y\\
			* & *
		\end{bmatrix} \sim
		\begin{bmatrix}
		f_Z & 0\\
		0 & 1_Y\\
		* & 0
		\end{bmatrix} $$
		So by changing basis this restricts to another short exact sequence
		\begin{center}
			\begin{tikzcd}
			0 \ar[r] & Z \ar[r] & Z \oplus P \ar[r] & \Omega^nC \ar[r] & 0.
			\end{tikzcd}
		\end{center}
		Let $T = \Lambda/J$ and apply the long exact sequence in $\Ext(-, T)$. Then we get an exact sequence
		\begin{center}
			\begin{tikzcd}
			\Ext^k(Z, T) \ar[r] & \Ext^k(Z \oplus P, T) \ar[r] & \Ext^{k+1}(\Omega^nC, T) 
			\end{tikzcd}
		\end{center}
		where the left map is induced by $f_Z$ since $\Ext^k(Z \oplus P, T) \cong \Ext^k(Z, T)$. Since $f_Z$ is nilpotent this map is surjective if and only if $\Ext^k(Z, T)=0$. We know that, since $\Omega^nC$ has finite projective dimension, $\Ext^{k+1}(\Omega^n C, T)$ is 0 for $k$ large enough. Then we must have that $\Ext^k(Z, T)=0$, and thus $Z$ has finite projective dimension. Specifically we have $\pd\Omega^n C -1 \leq \pd Z \leq \pd\Omega^n C$.
		
		Since $Z$ is a direct summand of $\Omega^n (A\oplus B)$, by \cref{lem:properties_of_psi} we have that $\pd Z + n \leq \psi(A \oplus B)$, and thus $\pd \Omega^n C - 1 + n = \pd C - 1 \leq \psi(A \oplus B)$.
	\end{proof}
\end{theorem}

With a bit of diagram chasing we can extend this theorem to get a bound for $\pd A$ and $\pd B$ as well.

\begin{cor}\label{cor:projdim_bounded_by_psi}
	Let $0 \to A \to B \to C \to 0$ be a short exact sequence of modules. 
	\begin{enumerate}[i)]
		\item \label{cor:projdim_bounded_by_psi_i}
		If $\pd A < \infty$, then $\pd A \leq \psi(\Omega B \oplus \Omega C)+1$.
		\item \label{cor:projdim_bounded_by_psi_ii}
		If $\pd B < \infty$ then $\pd B \leq \psi(\Omega A \oplus \Omega^2 C) + 2$.
	\end{enumerate}
	\begin{proof}
		Let $P_B \to B$ be a projective cover of $B$. Then we have a commutative diagram:
		\begin{center}
			\begin{tikzcd}
			0 \ar[r]  & 0 \ar[r] \ar[d] & P_B \ar[r] \ar[d] & P_B \ar[r] \ar[d] & 0\\
			0 \ar[r] & A \ar[r] & B \ar[r] & C \ar[r] & 0 
			\end{tikzcd}
		\end{center}
		Applying the Snake Lemma we get a short exact sequence $$0 \to \Omega B \to \Omega C \oplus P \to A \to 0$$ for some projective module $P$. Then using the theorem we have that if $\pd A \leq \infty$, then $\pd A \leq \psi(\Omega B \oplus \Omega C \oplus P) + 1 = \psi(\Omega B \oplus \Omega C) + 1$.
		
		Applying the same reasoning to $0 \to \Omega B \to \Omega C \oplus P \to A \to 0$ gives us that if $\pd B \leq \infty$, then $\pd\Omega B \leq \psi(\Omega A \oplus \Omega^2 C) + 1$. Hence $\pd B \leq  \psi(\Omega A \oplus \Omega^2 C) + 2$.
	\end{proof}
\end{cor}

These are all the results we need about the Igusa--Todorov functions. We will now use them to find families of algebras with $\findim(\Lambda) < \infty.$

%% MOVED
%\begin{theorem}\cite[Corollary~8]{IgTo05}
%	If $\Lambda = \End_\Gamma(P)^{\op}$ for an algebra $\Gamma$ with global dimension at most 3, and $P$ projective, then $\findim(\Lambda) < \infty$.
%	\begin{proof}
%		Let $X$ be any $\Lambda$-module with finite projective dimension. Then it has a projective presentation $(P, P_1) \to (P,P_0) \to X \to 0$ where $(P,P_i)=\Hom_\Gamma(P,P_i)$ with $P_i \in \add P$. Since $(P,-)$ is an equivalence from $\add P$ to $\proj\Lambda$ this corresponds to a map $P_1 \to P_0$ which we can extend to a projective resolution in $\Gamma$:
%		\begin{center}
%			\begin{tikzcd}
%			0 \ar[r] & P_3 \ar[r] & P_2 \ar[r] & P_1 \ar[r] & P_0.
%			\end{tikzcd}
%		\end{center}
%		Applying the exact functor $(P, -)$, we get an exact sequence
%		\begin{center}
%			\begin{tikzcd}
%			0 \ar[r] & (P,P_3) \ar[r] & (P,P_2) \ar[r] & (P,P_1) \ar[r] & (P,P_0)\ar[r] & X \ar[r] & 0.
%			\end{tikzcd}
%		\end{center}
%		Truncating this we get a short exact sequence
%		\begin{center}
%			\begin{tikzcd}
%			0 \ar[r] & (P, P_3) \ar[r] & (P, P_2) \ar[r] & \Omega^2 X \ar[r] & 0.
%			\end{tikzcd}
%		\end{center}
%		Then by \cref{thm:projdim_bounded_by_psi} the projective dimension of $\Omega^2 X$ is bounded by $\psi((P, P_3)\oplus (P, P_2))+1$. Which means
%		$$\pd X \leq \psi((P, P_3)\oplus (P, P_2))+3 \leq \psi((P,\Gamma))+3$$
%		Since this bound doesn't depend on $X$, $\Lambda$ has finite finitistic dimension.
%	\end{proof} 
%\end{theorem}
%
%\begin{cor}
%	If $\repdim(\Lambda) \leq 3$ then $\findim(\Lambda) < \infty$.
%	\begin{proof}
%		If $\Lambda$ has rep-dimension less than or equal to 3 then by \cref{prop:repdim_auslander_generator} there is a generator-cogenerator $M$ in $\mod\Lambda$ such that $\Gamma := \End_\Lambda(M)$ has global dimension 3 or less. Then since $M$ is a generator $\Lambda$ is in $\add M$ and so $\Hom_\Lambda(M, \Lambda)$ is a projective $\Gamma$-module with $\End_\Gamma(\Hom_\Lambda(M, \Lambda)) = \End_\Lambda(\Lambda) = \Lambda$.
%	\end{proof}
%\end{cor}

\subsection{Representation dimension}\label{sec:repdimension}

\begin{defn}[Representation dimension]
	Let $\Lambda$ be a finite dimensional algebra. The \emph{representation dimension} of $\Lambda$, denoted $\repdim(\Lambda)$, is the minimal global dimension of $\End(M)^{\operatorname{op}}$ for $M$ a generator-cogenerator in $\mod\Lambda$. We call a generator-cogenerator that achieves this minimum an \emph{Auslander-generator}.
\end{defn}

\begin{defn}[$\mathcal M$-resolutions]
	Let $X$ be an object in $\mod\Lambda$ and $\mathcal M$ a contravariantly finite subcategory.
	\begin{center}
	\begin{tikzcd}
		\cdots \ar[rd, two heads] \ar[r] & M_2 \ar[rd, two heads] \ar[r] & M_1 \ar[rd, two heads] \ar[r] & M_0 \ar[rd, two heads]\\
		&\Omega_M^3 X \ar[u, hook] & \Omega_M^2  \ar[u, hook] X & \Omega_M X  \ar[u, hook] & X
	\end{tikzcd}
	\end{center}
	If the maps $M_n \twoheadrightarrow \Omega_M^nX$ are minimal right $\mathcal M$-approximations for $n\geq 0$ (they need not be surjective), and $\Omega_M^{n+1} \hookrightarrow M_n$ are their kernels, then this is a minimal \emph{$\mathcal M$-resolution} of $X$. The \emph{$\mathcal M$-res-dimension} of $X$ is the length of this sequence of (nonzero) $M_i$'s, and the $\mathcal M$-res-dimension of $\Lambda$ is the supremum of the dimension on its objects.
\end{defn}

\begin{prop}
	If the representation dimension of $\Lambda$ is at least $2$, then $\repdim(\Lambda) - 2$ equals the minimum of $M$-$\operatorname{res-dim}(\mod \Lambda)$ for $M$ both generator and cogenerator. In fact, for any generator-cogenerator, $M$-$\operatorname{res-dim}(\mod \Lambda)$ is two less than the global dimension of $\End(M)^{\op}$.
	
	\begin{proof}
		Let $M$ be a generator-cogenerator. We first show that the global dimension of $\End(M)^{\op}$ is less than or equal to $M$-$\operatorname{res-dim}(\mod \Lambda) + 2$. 
		
		The functor $\Hom(M, -)$ is an equivalence from $\add M$ to $\proj\End(M)^{\op}$, which maps minimal $M$-approximations to projective covers. Let $X$ be any module in $\mod\End(M)^{\op}$ with projective dimension at least 2. Then it has a projective presentation $$\Omega^2X \to (M,M_1) \to (M,M_0) \to X.$$
		Because of the equivalence this is induced by a map $f\colon M_1\to M_0$. Since $\Hom(M,-)$ is left exact we have that $\Omega^2X \cong \Hom(M, \ker f)$, and so the projective dimension of $X$ is $2$ plus the $M$-res-dimension of $\ker f$. Hence the global dimension $\End(M)^{\op}$ is less than or equal to $M$-$\operatorname{res-dim}(\mod \Lambda) + 2$. 
		
		Next we prove the other inequality.
		
		Since $M$ is a cogenerator any module $Y$ in $\mod\Lambda$ has a copresentation 
		\begin{center}
		\begin{tikzcd}
			0 \ar[r] & Y \ar[r] & M_0 \ar[r, "f"] & M_1.
		\end{tikzcd}
		\end{center}
		Applying $(M,-) := \Hom(M,-)$ we get
		\begin{center}
		\begin{tikzcd}
		0 \ar[r] & (M,Y) \ar[r] & (M,M_0) \ar[r]{}{(M,f)} & (M,M_1) \ar[r] & \cok(M,f) \ar[r] & 0.
		\end{tikzcd}
		\end{center}
		If the projective dimension of $\cok(M,f)$ is less than 2, then $(M, Y)$ is a direct summand of $(M, M_0)$. This means that $(M,Y) \cong (M, M')$, so the minimal $M$-approximation of $Y$ is $M'$, and $(M, \Omega_M Y) = 0$. Since $M$ is a generator this means $\Omega_M Y = 0$ and thus the $M$-res-dimension of $Y$ is 0.
		
		So provided the projective dimension of $\cok(M,f)$ is larger than or equal to 2, it equals the $M$-res-dimension of $Y$ plus 2. In particular the global dimension of $\End(M)^{\op}$ is larger than or equal to $M$-$\operatorname{res-dim}(\mod \Lambda) + 2$. Hence they are equal.
	\end{proof}
\end{prop}

\begin{prop}
	The representation dimension of an artin algebra is always finite. \cite{Iya02}
\end{prop}

\begin{theorem}
	The representation dimension of $\Lambda$ is less than or equal to 2 if and only if $\Lambda$ is representation finite.
	\begin{proof}
		Assume $\Lambda$ is representation finite and let $M$ be the direct sum of all indecomposable modules (up to iso). Then $M$ is a generator-cogenerator. Let $X$ be an $\End(M)^{\operatorname{op}}$-module with projective presentation 
		$$(M,M_1) \to (M, M_0) \to X \to 0.$$ 
		Let $M_2$ be the kernel of $M_1 \to M_0$. Since $M$ is the sum of all indecomposables $M_2$ is in $\add M$, so 
		$$0 \to (M, M_2) \to (M,M_1) \to (M, M_0) \to X \to 0$$ 
		is a projective resolution of $X$. So $\Lambda$ has representation dimension at most 2.
		
		Assume $\Lambda$ has representation dimension at most 2, and let $M$ be an Auslander-generator. We want to show that $\add M = \mod\Lambda$. Let $X$ be any $\Lambda$-module, and let $$0 \to X \to I_0 \to I_1$$ be a minimal injective presentation. If $I_0 \to I_1$ is split then $X$ is injective and thus in $\add M$. Let $M_X$ be a minimal $M$-approximation of $X$, let $\Omega_M X$ be the kernel of the approximation, and let $Y$ be the cokernel of $(M, I_0) \to (M, I_1)$. Then $$(M,\Omega_M X) \to (M,M_X) \to (M, I_0) \to (M, I_1) \to Y \to 0$$ is a minimal exact sequence. Since the global dimension of $\End(M)^{\operatorname{op}}$ is at most 2 this means that $(M, \Omega_M X)=0$. Consequently we have that $\Omega_M X = 0$ and that $X=M_X$, so $X$ is in $\add M$. Thus $\Lambda$ is representation finite.
	\end{proof}
\end{theorem}

\begin{theorem}\cite[Corollary~8]{IgTo05}
	If $\Lambda = \End_\Gamma(P)^{\op}$ for an algebra $\Gamma$ with global dimension at most 3, and $P$ projective, then $\findim(\Lambda) < \infty$.
	\begin{proof}
		Let $X$ be any $\Lambda$-module with finite projective dimension. Then it has a projective presentation $(P, P_1) \to (P,P_0) \to X \to 0$ where $(P,P_i)=\Hom_\Gamma(P,P_i)$ with $P_i \in \add P$. Since $(P,-)$ is an equivalence from $\add P$ to $\proj\Lambda$ this corresponds to a map $P_1 \to P_0$ which we can extend to a projective resolution in $\Gamma$:
		\begin{center}
			\begin{tikzcd}
			0 \ar[r] & P_3 \ar[r] & P_2 \ar[r] & P_1 \ar[r] & P_0.
			\end{tikzcd}
		\end{center}
		Applying the exact functor $(P, -)$, we get an exact sequence
		\begin{center}
			\begin{tikzcd}
			0 \ar[r] & (P,P_3) \ar[r] & (P,P_2) \ar[r] & (P,P_1) \ar[r] & (P,P_0)\ar[r] & X \ar[r] & 0.
			\end{tikzcd}
		\end{center}
		Truncating this we get a short exact sequence
		\begin{center}
			\begin{tikzcd}
			0 \ar[r] & (P, P_3) \ar[r] & (P, P_2) \ar[r] & \Omega^2 X \ar[r] & 0.
			\end{tikzcd}
		\end{center}
		Then by \cref{thm:projdim_bounded_by_psi} the projective dimension of $\Omega^2 X$ is bounded by $\psi((P, P_3)\oplus (P, P_2))+1$. Which means
		$$\pd X \leq \psi((P, P_3)\oplus (P, P_2))+3 \leq \psi((P,\Gamma))+3$$
		Since this bound doesn't depend on $X$, $\Lambda$ has finite finitistic dimension.
	\end{proof} 
\end{theorem}

\begin{cor}
	If $\repdim(\Lambda) \leq 3$ then $\findim(\Lambda) < \infty$.
	\begin{proof}
		If $\Lambda$ has rep-dimension less than or equal to 3 then by \cref{prop:repdim_auslander_generator} there is a generator-cogenerator $M$ in $\mod\Lambda$ such that $\Gamma := \End_\Lambda(M)$ has global dimension 3 or less. Then since $M$ is a generator $\Lambda$ is in $\add M$ and so $\Hom_\Lambda(M, \Lambda)$ is a projective $\Gamma$-module with $\End_\Gamma(\Hom_\Lambda(M, \Lambda)) = \End_\Lambda(\Lambda) = \Lambda$.
	\end{proof}
\end{cor}


\subsection{Stably hereditary algebras}\label{sec:stable_hereditary_algebras}
In this section we will show that the class of stably hereditary algebras has repdimension at most 3, and thus that they have finite finitistic dimension.

\begin{defn}[(co)torsionfree]
	A module is called \textit{torsionfree} if it is a submodule of a projective module. Dually, a module is called \textit{cotorsionfree} if it is a factormodule of an injective.
\end{defn}

\begin{defn}[Stably hereditary algebra]
	An algebra is called \textit{stably hereditary} if any indecomposable torsionfree module is projective or simple, and any indecomposable cotorsionfree moule is injective or simple. 
\end{defn}

This generalizes the definition of hereditary algebra by also allowing simple modules to be (co)torsionfree.

\begin{defn}[The stable category]
	For an algebra $\Lambda$, \textit{the stable category} $\underline{\mod}\Lambda$ has the same objects as $\mod\Lambda$, but the homsets are given by $$\Hom_{\underline{\mod}\Lambda}(M, N) = \Hom_\Lambda(M,N)/\mathcal{P}(M,N)$$
	where $\mathcal{P}(M,N)$ is the ideal of all morphisms factoring through a projective.
\end{defn}

\begin{prop}
	If for an algebra $\Lambda$ there is a hereditary algebra $H$ such that $\underline{\mod}\Lambda \cong \underline{\mod}H$ then $\Lambda$ is stably hereditary.
	\begin{proof}
		\cite[Lemma~4.12]{AR91} \todo{+ a bit more...} \cite{AR73}
	\end{proof}
\end{prop}

The converse of the above proposition does not hold without more assumptions, but stably hereditary algebras generalize the idea of algebras stably equivalent to hereditary algebras.

\begin{theorem}\cite[Theorem~3.5]{Xi02}
	Stably hereditary algebras has repdimension at most 3.
	\begin{proof}
		Let $V$ be the direct sum of all the indecomposable projectives, all the indecomposable injectives, and all the simple modules. Then $V$ is a generator-cogenerator. So by \cref{prop:repdim_auslander_generator} if we can show that the global dimension of $\Gamma:=\End(V)^{op}$ is 3 or less, then we are done.
		
		We will show that for any $\Lambda$-module $M$ there is a short exact sequence $0 \to V_3 \to V_3 \to M \to 0$ with $V_i$ in $\add V$, and such that $0 \to (V, V_3) \to (V, V_2) \to (V, M) \to 0$ is exact. We will use this to construct short projective resolutions for $\mod\Gamma$. To construct $V_3$ and $V_2$ let $M'$ be the sum of the maximal injective summand of $M$ and all simple submodules of $M$. Then let $P$ be the projective cover of $M/M'$. Taking the pullback of $M \to M/M' \leftarrow P$ gives us the diagram:
		\begin{center}
		\begin{tikzcd}[column sep = 15pt, row sep = 25pt]
			   && 0 \ar[d] & 0 \ar[d]\\
			   && K \ar[d] \ar[r, equal] & K \ar[d]\\
			0 \ar[r] & M' \ar[r] \ar[d, equal] & M'\oplus P \ar[r]\ar[d] & P\ar[r]\ar[d] & 0\\
			0 \ar[r] & M' \ar[r] & M \ar[r]\ar[d] & M/M' \ar[r]\ar[d] & 0\\
			&&0&0 
		\end{tikzcd}
		\end{center}
		I claim that $0 \to K \to M'\oplus P \to M \to 0$ is the desired sequence. Firstly $M'\oplus P$ is clearly in $\add V$ since it is the sum of an injective, a semisimple, and a projective module. Further $K$ is a submodule of $P$, hence torsionfree. So since $\Lambda$ is stably hereditary $K$ is the sum of a projective and a semisimple module, so $K$ is also in $\add V$.
		
		Next we need to show that $0 \to (V, K) \to (V, M'\oplus P) \to (V, M) \to 0$ is exact. The only thing needed to show here is that $(V, M'\oplus P) \to (V, M)$ is surjective. We do this by showing that $(W, M'\oplus P) \to (W, M)$ is surjective for any indecomposable summand of $V$. If $W$ is projective this holds by definition. If $W$ is simple then any map from $W$ to $M$ factors through the socle and hence through $M'$, so it's surjective. Lastly if $W$ is injective then the image of $W$ in $M$ is a cotorsionfree module, so it is the sum of simple modules and an injective module. Hence the map from $W$ to $M$ factors through $M'$.
		
		Now we use this to show that the global dimension of $\Gamma$ is at most 3. Let $N$ be any $\Gamma$-module. Then it has a projective presentation
		\begin{center}
		\begin{tikzcd}
			(V,V_1) \ar[r, "f\circ-"] & (V,V_0) \ar[r] & N \ar[r] & 0
		\end{tikzcd}
		\end{center}
		If we let $M$ denote the kernel of $f$ and we choose $V_3$ and $V_2$ as above then we get a projective resolution of $N$ by
		\begin{center}
			\begin{tikzcd}[column sep=20pt]
			0\ar[r] & (V,V_3) \ar[r] & (V,V_2) \ar[r] & (V,V_1) \ar[r] & (V,V_0) \ar[r] & N \ar[r] & 0.
			\end{tikzcd}
		\end{center}
		This shows that the projective dimension of $N$ is at most 3, and since $N$ was arbitrary the global dimension of $\Gamma$ is at most 3. So the repdimension of $\Lambda$ is at most 3.
	\end{proof}
\end{theorem}


\subsection{Special biserial algebras}
\subsection{Special biserial algebras}\label{sec:special_biserial_algebras}
\cite{EHIS04}

In this section we shall consider two finite dimensional algebras, with a homomorphism between them. We denote these by $\Lambda$ and $\Gamma$, and we denote their radicals by $J_\Lambda$ and $J_\Gamma$ respectively.

\begin{defn}[Coinduced module]
	Given a homomorphism of algebras $\psi\colon\Lambda \to \Gamma$ we can consider every $\Gamma$-module as a $\Lambda$-module, where multiplication by $\lambda$ is given by multiplication with $\psi(\lambda)$. This defines a functor $\mod\Gamma \to \mod\Lambda$ known as \emph{restriction of scalars}. The right adjoint to this functor is called the \emph{coinduction functor}. For a $\Lambda$-module $M$ the coinduced module is defined to be
	$$M' := \Hom_\Lambda(\Gamma, M)$$
	where we consider $\Gamma$ as a $\Lambda$-$\Gamma$-bimodule through restriction of scalars. If we identify $M$ with $\Hom_\Lambda(\Lambda, M)$ then the counit of the adjunction is given by precomposing with $\psi$. Specifically we get the map
	\begin{center}
		\begin{tikzcd}
			M'\ar[r, "\varepsilon_M"] & M\\
			f \ar[r, mapsto] & f(\psi(1)) = f(1).
		\end{tikzcd}
	\end{center}
\end{defn}

\begin{prop}\cite[Lemma~2.2]{EHIS04}\label{prop:coinduction_right_adjoint}
	The coinduced functor as defined above is the right adjoint to restriction of scalars, and $\varepsilon$ is the counit.
	\begin{proof}
		Let $M$ be a $\Lambda$-module and let $N$ $\Gamma$-module. Then we have a Hom-Tensor adjunction
		$$\Hom_\Gamma(N, \Hom_\Lambda(\Gamma, M)) \cong \Hom_\Lambda(\Gamma\otimes_\Gamma N, M).$$
		Notice that $_\Lambda\Gamma\otimes_\Gamma N \cong _\Lambda N$ is exactly restriction of scalars. Further the counit $\Gamma\otimes_\Gamma M' = M' \to M$ is given by $f\mapsto f(1)$, which is exactly how we defined $\varepsilon$ above.
	\end{proof}
\end{prop}

Next, in preperation for \cref{thm:radical_embedding_repdim_3}, we restrict to the case where $\psi$ is the inclusion of a radical embbeding.

\begin{defn}[Radical embedding]
	A subalgebra $\Lambda \subseteq \Gamma$ is called a \emph{radical embedding} if the two radicals coincide, $J_\Lambda = J_\Gamma$.
\end{defn}

\begin{lemma}\cite[Lemma~2.3]{EHIS04}\label{lem:epsilon_semisimple_(co)kernel}
	If $\Lambda \subseteq \Gamma$ is a radical embedding, then $\ker \varepsilon_M$ and $\cok\varepsilon_M$ are both semisimple for any $\Lambda$-module $M$.
	\begin{proof}
		If we apply $\Hom_\Lambda(-, M)$ to the short exact sequence coming from $\psi$, $0 \to \Lambda \to \Gamma \to \Gamma/\Lambda \to 0$, we get 
		\begin{center}
			\begin{tikzcd}
				0 \ar[r] & \Hom(\Gamma/\Lambda, M) \ar[r] & M' \ar[r, "\varepsilon_M"] & M \ar[dr, two heads] \ar[r] & \Ext^1(\Gamma/\Lambda, M)\\
				  &&    && \cok\varepsilon_M \ar[u, hookrightarrow]
			\end{tikzcd}
		\end{center}
		Thus $\Hom(\Gamma/\Lambda, M)$ is the kernel of $\varepsilon_M$ and the cokernel is a submodule of $\Ext^1(\Gamma/\Lambda, M)$. Since $J_\Gamma = J_\Lambda \subseteq \Lambda$ we have that $(\Gamma/\Lambda)J_\Lambda = 0$. Thus $J_\Lambda\Hom(\Gamma/\Lambda, M)$ and $J_\Lambda\Ext^1(\Gamma/\Lambda, M)$ are both 0, which means they are both semisimple. Since $\cok\varepsilon_M$ is a submodule of $\Ext^1(\Gamma/\Lambda, M)$, it is also semisimple.
	\end{proof}
\end{lemma}

We now use the radical embedding to say somethign about the representation dimension of $\Lambda$.

INTERESTINGQUESTION: what happens if we replace the hypothesis $\Gamma$-repfinite with repdim of $\Gamma= n-1$. Same proof should give repdim $\Lambda=n$ except there might be problem with the exactness of the sequence since there are more $\Lambda$-linear maps. Find counterexample\todo{?} 

\begin{theorem}\label{thm:radical_embedding_repdim_3}
	If $\Gamma$ is representation finite and $\Lambda \subseteq \Gamma$ is a radical embedding, then the representation dimension of $\Lambda$ is at most 3.
	\begin{proof}
		Since $\Gamma$ is representation finite there is a finite set of indecomposable $\Gamma$-modules up to isomorphism. Let $X$ be the direct sum of all of these. Since $\Lambda$ is a subalgebra of $\Gamma$ we can consider $X$ as a $\Lambda$-module. Now define $V$ to be $\Lambda \oplus D\Lambda \oplus X$, i.e. $V$ is the sum of all projective $\Lambda$-modules, all injective $\Lambda$-modules, and all $\Gamma$-modules. We claim that $V$-$\operatorname{res-dim}(\Lambda) \leq 1$, which by \cref{prop:repdim_resdim+2} would imply that $\repdim(\Lambda) \leq 3$.
		
		As in \cref{thm:stably_hereditary_repdim_3} we do this by showing that for any  $\Lambda$-module $M$ there is a short exact sequence
		\begin{center}
			\begin{tikzcd}
			0 \ar[r] & V_1 \ar[r] & V_0 \ar[r] & M \ar[r] & 0
			\end{tikzcd}
		\end{center}
		with $V_i$ in $\add V$, such that 
		\begin{center}
			\begin{tikzcd}
			0 \ar[r] & (V,V_1) \ar[r] & (V,V_0) \ar[r] & (V,M) \ar[r] & 0
			\end{tikzcd}
		\end{center}
		is exact. 
		
		Now let $M$ be any $\Lambda$-module. If $M$ is injective, then $M$ is in $\add V$, and so we may simply choose $V_2 = M$ and $V_1=0$. From here on out assume that $M$ has no injective summands. 
		
		Let $M'$ be the coinduced module of $M$, and $\varepsilon_M\colon M' \to M$ be the map coming from the counit. Now if we let $P$ be the projective cover of $\cok \varepsilon_M$ then we get a surjective map $M' \oplus P \to M$. Since $M'$ is a $\Gamma$-module and $P$ is projective $M'\oplus P$ is in $\add V$. We let this be our $V_0$.
		
		Next, we let $V_1$ be the kernel of the map $V_0 \to M$. Then we wish to show that this is in $\add V$. Since $M \to \cok\varepsilon_M$ is an epimorphism and $P \to \cok\varepsilon_M$ is a projective cover, we can lift this to a morphism $P \to M$. Taking the pullback along $\Image \varepsilon_M \to M$ we get a commutative diagram:
		\begin{center}
		\begin{tikzcd}
			0 \ar[r] & K \ar[r]\ar[d]\arrow[dr, phantom, "\usebox\pullback" , very near start, color=black] & P \ar[r]\ar[d] & \cok\varepsilon_M \ar[r] \ar[d, equal] & 0\\
			0 \ar[r] & \Image \varepsilon_M \ar[r] & M \ar[r] & \cok\varepsilon_M \ar[r] & 0
		\end{tikzcd}
		\end{center} 
		By \cref{lem:epsilon_semisimple_(co)kernel} we have that $ \cok\varepsilon_M $ is semisimple, and thus $K = J_\Lambda P$. Since $J_\Lambda=J_\Gamma$ this means that $J_\Lambda P$ is a $\Gamma$-module, and thus is in $\add V$. Next we take the pullback again, this time along $M' \to \Image \varepsilon_M$. 
		\begin{center}
			\begin{tikzcd}
			0\ar[r]& \ker\varepsilon_M \ar[d, equal]\ar[r] & M'\prod\limits_M J_\Lambda P\arrow[dr, phantom, "\usebox\pullback" , very near start, color=black]\ar[r]\ar[d] & J_\Lambda P\ar[r]\ar[d] & 0\\
			0\ar[r]& \ker\varepsilon_M \ar[r]& M'\ar[r] & \Image \varepsilon_M\ar[r] &  0
			\end{tikzcd}
		\end{center} 
		Notice that $M'\prod\limits_M J_\Lambda P = M'\prod\limits_M P$, which is the kernel of $V_0 \to M$. In other words it is equal to $V_1$.
		
		Since $J_\Lambda P$ is a $\Gamma$-module we get a map of abelian groups by postcomposing with $\varepsilon_M$:
		\begin{center}
			\begin{tikzcd}
			\Hom_\Gamma(J_\Lambda P, M') \ar[r, "\varepsilon_M \circ -"] & \Hom_\Lambda(J_\Lambda P, M)\\
			f \ar[r, mapsto]& (p \mapsto f(p)(1))
			\end{tikzcd}
		\end{center} 
		This is excatly the isomoprhism of the adjuntion between restriction of scalars and the coinduction functor in \cref{prop:coinduction_right_adjoint}.

		In other words the map $P \to \Image \varepsilon_M$ factorizes through $M'$. Then using the pullback property, we get that the map $V_1 \to J_\Lambda P$ splits, and so $V_1 = \ker\varepsilon_M \oplus J_\Lambda P$. We have already established that $J_\Lambda P$ is a $\Gamma$-module. By \cref{lem:epsilon_semisimple_(co)kernel} we have that $\ker\varepsilon_M$ is semisimple. Thus $V_1$ is in $\add V$. \todo{kernel is not $Gamma$-module..?}
		
		Lastly we show that we get an exact sequence
		\begin{center}
			\begin{tikzcd}
			0\ar[r] & (V, V_1) \ar[r] & (V, V_2) \ar[r] & (V, M) \ar[r] & 0.
			\end{tikzcd}
		\end{center} 
		The only thing we need to show is that the last map is surjective. We do this by verifying the three cases for an indecomposable summand of $V$. Firstly let $W$ be a $\Gamma$-module. Then $\Hom_\Lambda(W, V_2)$ breaks up as a direct sum into $\Hom_\Lambda(W, M') \oplus \Hom_\Lambda(W, P)$. We saw in \cref{prop:coinduction_right_adjoint} that the composition
		\begin{tikzcd}
			\Hom_\Gamma(W, M') \ar[r, "\subseteq"] & \Hom_\Lambda(W, M') \ar[r] & \Hom_\Lambda(W, M)
		\end{tikzcd}
		is an isomorphism. Thus the map $\Hom_\Lambda(W, M') \to \Hom_\Lambda(W, M)$ is surjective.
		
		If $W$ is projective, then $\Hom_\Lambda(W, -)$ is exact, and there is nothing we need to show.
		
		If $W$ is an indecomposable injective, since we assumed $M$ had no injective summands, a map $W \to M$ cannot be injective. This means that it factors through $W/\socle(W)$. Since $D(W/\socle(W)) = (DW)J_\Lambda = (DW)J_\Gamma$ this means that $W/\socle(W)$ is a $\Gamma$-module. Then from the argument above it follows that the map is surjective.
		
		This shows that the global $V$-$\operatorname{res-dim}(\Lambda) \leq 1$, and thus the representation dimension of $\Lambda$ is at most 3.
	\end{proof}
\end{theorem}

Now we move away from the case where $\psi$ is a radical embedding, and instead look at a specific quotient map.

\begin{theorem}\label{thm:mod_out_socle}
	Let $\Lambda$ be a basic finite dimensional algebra and let $P$ be a basic projective-injective $\Lambda$-module. Then the socle of $P$ is a two-sided ideal, which allows us to define the ring $\Gamma := \Lambda / \socle P$. Then we have that $\repdim(\Lambda) \leq \max\{2, \repdim(\Gamma)\}$. 
	\begin{proof}
		First we show that the socle of $P$ is a two-sided ideal. Multiplication on the right defines a homomorphism $-\cdot \lambda\colon \Lambda \to \Lambda$. Any homomorphism maps the socle to the socle, so $(\socle P) \cdot \lambda \subseteq \socle \Lambda$. Now let $s \in \socle P$ be some element such that $s\lambda$ is non-zero. Then the injective envelope $I(s)$ is a direct summand of $P$ and thus projective-injective. Further since $-\cdot \lambda\colon (s) \to (s\lambda)$ is an injective map, $I(s)$ is mapped injectively into $\Lambda$ by $-\cdot \lambda$, which means $-\cdot\lambda\colon I(s) \to \Lambda$ splits. Since $\Lambda$ is basic this means that $I(s)\lambda \subseteq P$, and thus $s\lambda \in \socle P$, so the socle of $P$ is a two-sided ideal.
		
		Next we note that any indecomposable $\Lambda$-module is either a $\Gamma$-module, or a direct summand of $P$. To see this, let $M$ be any indecomposable $\Lambda$-module and consider $(\socle P)M$. If this is zero, then $M$ is a $\Gamma$-module. If on the other hand there is some $s \in \socle P$ and $m \in M$ such that $sm \neq 0$, then let $I(s)$ be the injective envelope of $s$ and let $e$ be the idempotent such that $I(s) = \Lambda e$. Then we get a map $I(s) \to M$ which maps $\lambda e$ to $\lambda e m$. Since $sm \neq 0$ this maps the socle of $I(s)$ injectively. Now, since $I(s)$ is injective this mean that $I(s)$ is a direct summand of $M$. Since $M$ is indecomposable we have that $M \cong I(s)$, and thus $M$ is a direct summand of $P$.
		
		Now we show that $\repdim(\Lambda) \leq \max\{2, \repdim(\Gamma)\}$. By \cref{prop:repdim_resdim+2} it suffices to find a generator-cogenerator $V$ such that $V$-$\operatorname{res-dim}(\mod \Lambda) \leq \max\{0, \repdim(\Gamma)-2\}$. Let $N$ be the generator-cogenerator in $\mod\Gamma$ that achieves the minimal resolution dimension. Then we claim $V = N \oplus P$ is our desired generator-cogenerator. This is a generator-cogenerator because any indecomposable projective or injective module that is not a summand of $P$ will be  a summand of $N$, since all $\Lambda$-modules that are not summands of $P$ are $\Gamma$-modules.
		
		To show that $V$-$\operatorname{res-dim}(\mod \Lambda) \leq \max\{0, \repdim(\Gamma)-2\}$ we explicitly construct the resolutions. Let $M$ be an indecomposable $\Lambda$-module. Then we wish to construct an exact sequence
		\begin{center}
		\begin{tikzcd}
			0 \ar[r] & V_n \ar[r] & \cdots \ar[r] & V_1 \ar[r] & V_0 \ar[r] & M \ar[r] & 0
		\end{tikzcd}
		\end{center} 
		such that $V_i$ is in $\add V$, $n\leq \max\{0, \repdim(\Gamma)-2\}$, and $\Hom(V, -)$ is exact on the sequence. If $M$ is a summand of $P$ we may choose $V_0=M$ and $V_i=0$ for $i>0$.
		
		If $M$ is not a summand of $P$ then $M$ is a $\Gamma$-module. Then we already have an exact sequence
		\begin{center}
			\begin{tikzcd}
				0 \ar[r] & N_n \ar[r] & \cdots \ar[r] & N_1 \ar[r] & N_0 \ar[r] & M \ar[r] & 0
			\end{tikzcd}
		\end{center} 
		with $N_i \in \add N$. Since $\Lambda \to \Gamma$ is surjective we get that $\Hom_\Lambda(N, -)=\Hom_\Gamma(N,-)$ on $\Gamma$-modules. So if we apply $\Hom_\Lambda(N, -)$ to the sequence it remains exact. Lastly since $\Hom(V,-) = \Hom(N, -) \oplus \Hom(P, -)$ and $\Hom(P, -)$ is an exact functor, if we apply $\Hom(V, -)$ to the sequences it still remains exact. Thus $V$-$\operatorname{res-dim}(\mod \Lambda) \leq \max\{0, \repdim(\Gamma)-2\}$ and $\repdim(\Lambda) \leq \max\{2, \repdim(\Gamma)\}$. 
	\end{proof}
\end{theorem}

\begin{defn}[Special biserial algebra]
	A finite dimensional algebra $\Lambda$ is called \emph{special biserial} if it is isomorphic to a path algebra $kQ/I$ such that
	\begin{itemize}
		\item Each vertex in $Q$ is the initial vertex for at most two arrows, and the terminal vertex for at most two arrows.
		\item For any arrow $\beta$ in $Q$ there is at most on arrow $\alpha$ such that $\alpha\beta \not\in I$ and at most one arrow $\gamma$ such that $\beta\gamma \not\in I$.
	\end{itemize}
	A special biserial algebra is called a \emph{string algebra} if it is also monomial. I.e. $I$ is generated by paths. 
\end{defn}

\begin{prop}\label{prop:special_biserail_algerbas_are_binomial}
	If $\Lambda = kQ/I$ is special biserial, then $I$ is generated by monomial and binomial relations. Further if $\gamma + t\gamma'$ is a binomial relation such that $\gamma \not\in I$, then $(\gamma)$ is the socle of a projective-injective module.
	\begin{proof}
		Let $\rho$ be a relation. Then we may assume $\rho$ is some linear combinations of paths which start in the same vertex and end in the same vertex. Assume by induction that $\rho$ is a combination of $n$ distinct paths for some $n\geq 3$, and let $\gamma^1$, $\gamma^2$, and $\gamma^3$ be three of those paths. Write each path as a composition of arrows $\gamma^1 = \alpha^1_{t_1} \cdots \alpha^1_1\alpha^1_0$,  $\gamma^2 = \alpha^2_{t_2} \cdots \alpha^2_1\alpha^2_0$, and  $\gamma^3 = \alpha^3_{t_3} \cdots \alpha^3_1\alpha^3_0$.
		
		Since there can be at most two arrows out of any vertex, it cannot be the case that $\alpha^1_0$, $\alpha^2_0$, and $\alpha^3_0$ are all distinct. Let us assume $\alpha^1_0 = \alpha^2_0$. Since we assume $\gamma^1$ and $\gamma^2$ are distinct there must be a smallest $k$ such that $\alpha^1_k \neq \alpha^2_k$. But then it must be the case that either $\alpha^1_k\alpha^1_{k-1}$ or $\alpha^2_k\alpha^1_{k-1}$ is a relation. That means that either $\gamma^1$ or $\gamma^2$ is a relation. Thus $\rho$ is the sum of a monomial relation and a relation that is the linear combination of $(n-1)$ paths. Then by induction each relation in $I$ is the sum of binomial relations.
		
		Now let $\gamma + t\gamma'$ be a binomial relation such that $\gamma \not\in I$. Let $i$ be the origin vertex of $\gamma$, let $j$ be the terminal vertex, and let $e_i$ and $e_j$ be the corresponding idempotents. Then we claim that $\Lambda e_i$ is projective-injective, and that $(\gamma)$ is its socle.
		
		As above decompose the two paths into a product of arrows $\gamma = \alpha_{t}\cdots \alpha_1\alpha_0$ and $\gamma' = \alpha'_{t'}\cdots \alpha_1\alpha_0$, and let $k$ be the smallest integer such that $\alpha_k \neq \alpha'_k$. If $k$ is bigger than 0, then as before we get that either $\alpha_k\alpha_{k-1}$ or $\alpha'_k\alpha_{k-1}$ is a relation. Consequently both $\gamma$ and $\gamma'$ would be relations contradicting our assumption. Similarly if we let $k$ be the smallest integer such that $\alpha_{t-k} \neq \alpha'_{t'-k}$ we get that $k$ cannot be bigger than 0, by exactly the same argument. This means that $\alpha_0 \neq \alpha'_0$ and that $\alpha_t \neq \alpha'_{t'}$, which will be important later.

		We show that $(\gamma)$ is simple, by showing that $\alpha\gamma$ is a relation for every arrow $\alpha$. We have that $\alpha(\gamma + t\gamma')$ is a relation. Since $\alpha_t \neq \alpha'_{t'}$ we have that either $\alpha\alpha_t = 0$ or $\alpha\alpha'_{t'} = 0$. If $\alpha\alpha_t=0$, then $\alpha\gamma=0$ and we are done. If $\alpha\alpha'_{t'} = 0$, then $\alpha\gamma'=0$ which means that $\alpha\gamma = \alpha(\gamma + t\gamma') - t\alpha\gamma'$ is as well. So $(\gamma)$ is simple and hence in the socle of $\Lambda e_i$.
		
		\todo{blabla}

		$\Lambda e_i \cong D e_j \Lambda$, $e_i \mapsto \gamma^*$
		 
	\end{proof} 
\end{prop}

This explains where the name \emph{special biserial} comes from; the radical of each indecomposable projective of a special biserial algebra is biserial. I.e. it is the sum of two uniserial modules. In fact for an indecomposable projective $P$, either $P$ is uniserial or $JP/\socle P$ is the direct sum of two uniserial modules.

\begin{center}
\setlength{\tabcolsep}{30pt}
\begin{tabular}{ccc}
	\begin{tikzcd}
	\bullet\ar[d]\\
	\bullet\ar[d]\\
	\vdots\ar[d]\\
	\bullet\ar[d]\\
	\bullet
	\end{tikzcd}
	&
	\begin{tikzcd}[ampersand replacement=\&, column sep = 10pt]
	\&\bullet\ar[dl]\ar[dr]\\
	\bullet\ar[d] \&\& \bullet\ar[d]\\
	\vdots\ar[d] \&\& \vdots\ar[d]\\
	\bullet \&\& \bullet\ar[d]\\
	\&\&\bullet
	\end{tikzcd}
	&
	\begin{tikzcd}[ampersand replacement=\&, column sep = 10pt]
	\&\bullet\ar[dl]\ar[dr]\\
	\bullet\ar[d] \&\& \bullet\ar[d]\\
	\vdots\ar[d] \&\& \vdots\ar[d]\\
	\bullet\ar[dr] \&\& \bullet\ar[dl]\\
	\&\bullet
	\end{tikzcd}
\end{tabular}

The possible shapes for an indecomposable projective module.
\end{center}

Combining \cref{thm:mod_out_socle} and \cref{prop:special_biserail_algerbas_are_binomial} we can reduce the problem of computing the representation dimension of a special biserial algebra to monomial algebras, by modding out all binomial relations. 

Special biserial algebras that are monomial are called string algebras. Well known examples of these are gentle algebras. We now combine everything we have proved so far.

\begin{theorem}\cite[Corollary~1.3]{EHIS04}
	If $\Lambda = kQ/I$ is a special biserial algebra, then $\repdim(\Lambda) \leq 3$, and thus $\findim(\Lambda) < \infty$.
	\begin{proof}
		By \cref{thm:mod_out_socle} we may assume $\Lambda$ is a string algebra. If we can construct a radical embedding of $\Lambda$ into a representation finite algebra, then by \cref{thm:radical_embedding_repdim_3} our result would follow.

		For any vertex $l \in Q$ define $E(l)$ to be the set of arrows ending in $l$ and $S(l)$ the set of arrows starting in $l$. Define $c(\Lambda)$ to be the sum of the number of vertices with $|E(l)| \geq 2$ and the number of vertices with $|S(l)| \geq 2$. The proof goes by induction on $c(\Lambda)$.

		If $c(\Lambda)=0$, then $Q$ is the disjoint union of linearly oriented quivers of type $\mathbb{A}$ and cyclically oriented quivers of type $\tilde{\mathbb{A}}$. Finite dimensional algebras arising from such quivers are well known to be representation finite (c.f. \cite[Chapter~VI.2]{ARS97} or \cite[Chapter~V.3]{ASS06}), and so the identity map on $\Lambda$ is a radical embedding into an algebra of finite representation type.
		\begin{center}
			\setlength{\tabcolsep}{30pt}
			\begin{tabular}{cc}
				\begin{tikzcd}[ampersand replacement=\&]
					1\ar[r] \& {} \ar[r, dashed, no head] \& {} \ar[r] \& n
				\end{tikzcd}
				&
				\begin{tikzcd}[ampersand replacement=\&, row sep= 15pt, column sep = 15pt]
					\&0 \ar[r] \& 1 \ar[dr]\\
					n \ar[ur] \&\&\& 2 \ar[dl]\\
					\& \phantom{3} \ar[ul] \&  \phantom{3} \ar[l, dashed, no head]
				\end{tikzcd}\\
				&\\
				Linearly oriented 
				&
				Cyclically oriented\\
				quiver of type $\mathbb{A}_n$
				&
				quiver of type $\tilde{\mathbb{A}}_n$
			\end{tabular}
		\end{center}
		If $c(\Lambda) = n \leq 1$, then there is a vertex $l$ with either $E(l)=2$ or $S(l)=2$. We now construct a new string algebra $\Gamma$ and a radical embedding $\Lambda \to \Gamma$ such that $c(\Gamma) \leq n-1$.

		The two cases are completely symmetric, so we only show the case $E(l)=2$ here. Let $\alpha_1$ and $\alpha_2$ be the two arrows ending in $l$. Define the quiver $Q'$ to have the same vertices as $Q$, except we replace $l$ by two vertices $l_1$ and $l_2$. The arrows of $Q'$ are exactly the same, except now $\alpha_1$ ends in $l_1$ and $\alpha_2$ ends in $l_2$. For any arrow $\beta \in Q$ that starts in $l$, the corresponding arrow in $Q'$ starts in $l_1$ if and only if $\beta\alpha_1$ is not a relation.

		We may consider $I$ as an ideal in $kQ'$ simply by setting paths to 0 if they are no longer defined in $Q'$. Then $\Gamma := kQ'/I$ is a string algebra, and the map $\Lambda \to \Gamma$ that sends $e_l$ to $e_{l_1}+e_{l_2}$ and all other paths to themselves is a radical embedding.

		For each vertex $k \neq l$, we have $E_\Lambda(k) = E_\Gamma(k)$ and $S_\Lambda(k) = S_\Gamma(k)$, $E_\Lambda(l) = 2$,  $E_\Gamma(l_1) = E_\Gamma(l_2) = 1$, and $S_\Lambda(l) = S_\Gamma(l_1)+S_\Gamma(l_2)$. Since $S_\Lambda(l) \leq 2$ it follows that $c(\Gamma) \leq n-1$.

		By induction there is a radical embedding of $\Lambda$ into an algebra $\Gamma$ with $c(\Gamma)=0$, which is representation finite. Then by \cref{thm:radical_embedding_repdim_3} we get that $\repdim(\Lambda) \leq 3$, and by \cref{cor:repdim_less_than_3_implies FDC} we have $\findim(\Lambda) < \infty$.
	\end{proof}
\end{theorem}

\section{Vanishing radical powers}
We remind the reader that throughout this section $\Lambda$ is a finite dimensional algebra, and $J$ is its radical. The Loewy length of an algebra is the smallest integer $n$ such that $J^n = 0$. In this section show that algebras with short Loewy length have finite finitistic dimension.

The may theorem of this section is \cref{thm:half_rep_finite} in which we prove that ``half representation finite'' algebras satisfies the finitistic dimension conjecture. The reader should note that \cref{thm:J2_equals_0_implies_FDC} and \cref{thm:J3_equals_0_implies_FDC} are special cases of \cref{thm:half_rep_finite}, but we include alternate proofs here.

\begin{theorem}\label{thm:J2_equals_0_implies_FDC}
	If $J^2=0$ then $\findim(\Lambda) < \infty$.
	\begin{proof}
		Let $d = \max\{\pd S_i \mid \pd S_i < \infty\}$ where $S_i$ ranges over the simple $\Lambda$-modules. Let $M$ be a module with $\pd M < \infty$. Let $P \to M$ be a projective cover. Then $\Omega M$ is contained in $JP$ and since $J^2P=0$, $\Omega M$ is annihilated by $J$ and is thus semisimple. This means $\pd \Omega M \leq d$, and thus $\pd M \leq d+1$. So $\findim(\Lambda) \leq d+1 < \infty$.
	\end{proof}
\end{theorem}

The proof for the case of $J^3=0$ uses the Igusa--Todorov function from \cref{sec:Igusa-Todorov}.

\begin{theorem}\cite[Corollary~6]{IgTo05}\label{thm:J3_equals_0_implies_FDC}
	If $J^3=0$ then $\findim(\Lambda) < \infty$.
	\begin{proof}
		Let $M$ be a module with $\pd M < \infty$, and let $P^0 \to M$ be its projective cover. Since $\Omega M \subseteq JP^0$ we have $J^2\Omega M = 0$. Let $P \to \Omega M$ be a projective cover. Since $J^2\Omega M = 0$ we can factorize this as $P \to P/J^2P \to \Omega M$, and we get a short exact sequence
		\begin{center}
		\begin{tikzcd}
			0 \ar[r] & (\Omega^2 M + J^2P) / J^2 P \ar[r] & P / J^2 P \ar[r] & \Omega M \ar[r] & 0
		\end{tikzcd}
		\end{center}
		Let $\psi$ be the Igusa--Todorov function as introduced in \cref{sec:Igusa-Todorov}. Since $\Omega^2 M \subseteq JP$ we have that $(\Omega^2 M + J^2P) / J^2 P$ is semisimple. Then by \cref{lem:properties_of_psi} $\psi((\Omega^2 M + J^2P) / J^2 P) \leq \psi(\Lambda / J)$, and $\psi(P / J^2 P) \leq \psi(\Lambda / J^2)$.
		
		Applying \cref{thm:projdim_bounded_by_psi} to the short exact sequence above we thus get $\pd \Omega M \leq \psi(\Lambda / J \oplus \Lambda / J^2) + 1$, and so $\pd M \leq \psi(\Lambda / J \oplus \Lambda / J^2) + 2$, and $\findim(\Lambda) < \infty$.
	\end{proof}
\end{theorem}

The main theorem of this section is just a very slight generalization of the proof of the $J^3=0$ case. 

\begin{theorem}\cite{Wang94}\label{thm:half_rep_finite}
	If $J^{2l+1} = 0$ and $\Lambda / J^l$ is representation-finite, then $\findim(\Lambda) < \infty$.
	\begin{proof}
		Let $M$ be a module with $\pd M < \infty$. We have a short exact sequence 
		\begin{center}
			\begin{tikzcd}
			0 \ar[r] & J^l\Omega M \ar[r] & \Omega M \ar[r] & \Omega M / J^l\Omega M \ar[r] & 0.
			\end{tikzcd}
		\end{center}
		Since $\Omega M \subseteq JP^0_M$ we have $J^{2l}\Omega M = 0$. This means that $J^l\Omega M$ and $\Omega M / J^l\Omega M$ are $\Lambda / J^l$-modules. We use this, the fact that $\Lambda / J^l$ is representation finite, and the Igusa--Todorov function to create a bound for $\pd M$.
		
		Applying \cref{cor:projdim_bounded_by_psi} (\ref{cor:projdim_bounded_by_psi_ii}) we have that:
		$$ \pd \Omega M \leq \psi(\Omega (J^l\Omega M)\oplus\Omega^2(\Omega M / J^l\Omega M)) + 2.$$ 
		Since $\Lambda / J^l$ is representation finite, there are only finitely many indecomposable $\Lambda / J^l$-modules, up to isomorphism. Let $V$ be the sum of all of them. Then since $J^l\Omega M$ and $\Omega M / J^l\Omega M$ are in $\add V$, using \cref{lem:properties_of_psi} we have that 
		$$\psi(\Omega (J^l\Omega M)\oplus\Omega^2(\Omega M / J^l\Omega M)) \leq \psi(\Omega V \oplus \Omega^2 V).$$
		So $\pd M \leq \psi(\Omega V \oplus \Omega^2 V) + 3$, and thus $\findim(\Lambda) < \infty$.
	\end{proof}
\end{theorem}


\section{Monomial algebras}\label{sec:monomial_algebras}
\cite{GKK91, IgZa90}

In this section we will show a particularly nice way to construct a minimal projective resolution of the right module $\Lambda / J$ for a monomial algebra $\Lambda$. We will use this to compute $\Tor_i(\Lambda /J, M)$ and/or $\Ext^i(M, D\Lambda/J)$ to get a bound on the projective dimension of all modules $M$.

\begin{defn}[Monomial algebra]
	A \emph{monomial algebra} is a path algebra with admissible relations that are generated by monomials. That is, we do not allow the generators for the relations to consist of nontrivial linear combinations of paths.
\end{defn}
\todo{We may assume rho contains J2}
\begin{defn}[$m$-chains]\cite{GKK91}
	Let $\Lambda = k\Gamma / (\rho)$ be a monomial algebra, with $\rho$ a minimal generating set of paths. As usual we define $\Gamma_0$ to be the vertices of $\Gamma$, and $\Gamma_1$ to be the arrows. Recursively define the set of $(m-1)$-chains, $\Gamma_m$, as the paths $\gamma$ with the following criteria:
	\begin{enumerate}[i)]
		\item $\gamma = \beta\delta\tau$ with $\beta \in \Gamma_{m-2}$, $\beta\delta \in \Gamma_{m-1}$, and $\tau$ a non-zero path of length at least 1.
		\item $\delta\tau$ is 0 in $\Lambda$, i.e. it is in the ideal of relations.
		\item $\gamma$ is left-minimal in the sense that if $\gamma = \gamma' \sigma$ such that $\gamma'$ satisfies the above conditions, then $\gamma = \gamma'$.
	\end{enumerate}
\end{defn}

The sets of $m$-chains will become the generating sets for the projectives in our projective resolution. But first we will prove some properties of them.

\begin{lemma}\label{lem:unique_factorization_of_chains}
	Any $\gamma\in \Gamma_m$ for $m \geq 1$ can be factored uniquely as $\gamma_1\gamma_0$ with $\gamma_1 \in \Gamma_{m-1}$, and $\gamma_0$ a non-zero path of length at least 1.
	\begin{proof}
		When $m=1$ this should be clear, since $\Gamma_1$ is the set of arrows, and $\Gamma_0$ is the set of vertices, so if $\gamma \in \Gamma_1$ is an arrow $i\to j$ then $\gamma = e_j\gamma$.
		
		When $m > 1$ we know from the definition of $\Gamma_m$ that $\gamma$ can be written as $\gamma_1\gamma_0$. Assume there is another decomposition $\gamma = \gamma'_1\gamma'_0$. Then without loss of generality we may assume that $\gamma'_1$ is shorter than $\gamma_1$. Then there is a $\sigma$ such that $\gamma'_1\sigma = \gamma_1$. By minimality this means that $\gamma'_1=\gamma_1$, and so the decomposition is unique.
	\end{proof} 
\end{lemma} 

From now on we will write $R$ for the ring $\Lambda/J$, which we identify with the subring of $\Lambda$ generated by the paths of length 0. Let $k\Gamma_m$ be the free vector space generated by $\Gamma_m$. Notice that $k\Gamma_m$ has a canonical structure as a $R-R$-bimodule. This means we can get projective right $\Lambda$-modules $P^m := k\Gamma_m\otimes_R\Lambda$.

\begin{prop}\label{prop:projective_res_of_top_monomial_alg}
Define the map $\delta_m \colon P^m \to P^{m-1}$ by $\delta_m(\gamma \otimes \alpha) = \gamma_1 \otimes \gamma_0\alpha$ where $\gamma_1\gamma_0$ is the unique decomposition of $\gamma$, and define $\delta_0 \colon P^0 \to \Lambda /J$ by $\delta_0(e_i\otimes \alpha) = e_i\alpha + J$. Then we have a minimal projective resolution of the right $\Lambda$-module $\Lambda/J$ by

\begin{center}
\begin{tikzcd}
	\cdots \ar[r] & P^3 \ar[r, "\delta_3"] & P^2 \ar[r, "\delta_2"] & P^1 \ar[r, "\delta_1"] & P^0 \ar[d, two heads, "\delta_0"]\ar[r] & 0\\
	&&&&\Lambda / J
\end{tikzcd}
\end{center}
\end{prop}

Before proving this proposition we require a lemma.

\begin{lemma}\cite[Lemma~2.1]{GKK91}\label{lem:decompose_kernel_delta_m}
	Let $M$ be a $\Lambda$-module, and $x$ an element in the kernel of of $\delta_m\otimes M\colon P^m\otimes_R M \to P^{m-1} \otimes_R M$. Write $x$ on the form
	$$x = \sum_j\sum_{k=0}^{n_j} \gamma_j  \gamma_j^k \otimes m_j^k$$
	with $\gamma_i \in \Gamma_{m-1}$ and $\gamma_i \neq \gamma_j$ and $\gamma_j^k \neq \gamma_j^l$ when $i \neq j$ and $k \neq l$. Then 
	$$\sum_{k=0}^{n_j} \gamma_j  \gamma_j^k \otimes m_j^k$$
	is also in the kernel for each $j$.
	\begin{proof}
		Let $x$ be as given above. Applying $\delta_m\otimes M$ we get that 
		$$\sum_j \gamma_j \otimes \sum_{k=0}^{n_j} \gamma_j^{k}m_j^{k}=0.$$ 
		Since the $\gamma_j$'s are distinct we can deduce that 
		$$ \sum_{k=0}^{n_j} \gamma_j^{k}m_j^{k}=0.$$ 
		Since $\Lambda$ only has monomial relations, and by the minimality of the $\gamma_j^k$'s none of them divide each other, we have that $\gamma_j^{k}\alpha_j^{k}=0$. Thus 
		$$\sum_{k=0}^{n_j} \gamma_j  \gamma_j^k \otimes m_j^k$$
		is also in the kernel of $\delta_m \otimes M$.
	\end{proof}
\end{lemma}

\begin{proof}[Proof of \cref{prop:projective_res_of_top_monomial_alg}]
	For all $i$ the module $P^i$ is projective as a right $\Lambda$-module and the image of $\delta_m$ is clearly contained in $P^{m-1}J$, so the only thing left to show is exactness. First we show that $\delta_m\delta_{m-1}=0$. Let $\gamma\otimes \alpha$ be in $P^m$ for $m \geq 2$. Then we can decompose $\gamma$ uniquely as $\gamma_2\gamma_1\gamma_0$ and $\delta_m\delta_{m-1}(\gamma\otimes \alpha) = \gamma_2\otimes\gamma_1\gamma_0\alpha$. By the way we defined $\Gamma_m$, $\gamma_1\gamma_0$ is 0 in $\Lambda$, and so $\gamma_2\otimes\gamma_1\gamma_0\alpha = 0$.
	
	Next we want to show that $\Ker\delta_{m-1} \subseteq \Image\delta_m$. Let $x$ be in the kernel of $\delta_{m-1}$. By \cref{lem:decompose_kernel_delta_m} it is sufficent to assume $x$ is of the form
	$$\sum_k \gamma  \gamma_k \otimes \alpha_k.$$
	Then $\sum_k \gamma_k\alpha_k = 0$. Because of this we have that $\gamma\gamma_k\alpha_k=\zeta_k\sigma_k$ for some $m$-chain $\zeta_k$ and some path $\sigma_k$ (possibly of length 0). This gives us that $x$ is the image of 
	$$\sum_k \zeta_k\otimes \sigma_k$$ 
	by $\delta_m$. Hence $\Ker\delta_{m-1} \subseteq \Image\delta_m$, and the sequence is exact. So this gives a minimal projective resolution of $\Lambda/J$ as a right $\Lambda$-module.
\end{proof}

\begin{defn}
	We call a path $\tau$ in $\Gamma$ a \emph{special segment} for $\Lambda = k\Gamma/(\rho)$ if there is a path $\gamma$ such that $\gamma\tau$ is a minimal relation.
\end{defn}

Note that when we decompose an $m$-chain $\gamma$ in \cref{lem:unique_factorization_of_chains} into $\gamma_1\gamma_0$ then $\gamma_0$ is a special segment, and that the set of special segments is finite.

\begin{lemma}\cite[Theorem~2.2]{GKK91}\label{lem:monomial_relation_repetition}
	Let $d$ be the number of special segments for $\Lambda$. If $s \geq d+3$ and $\gamma$ is in $\Gamma_s$, then for any integer $N$ there is an $n \geq N$ and a $\hat{\gamma} \in \Gamma_n$ such that for any path $\tau$ we have $\gamma\tau \in \Gamma_{s+r}$ if and only if $\hat{\gamma}\tau \in \Gamma_{n+r}$.
	\begin{proof}
		Applying \cref{lem:unique_factorization_of_chains} recursively we get that $\gamma$ can be written as $\gamma = \tau_0\tau_1\cdots \tau_{s-1}$ where $\tau_0\tau_1 \cdots \tau_{i-1} \in \Gamma_i$. In particular each $\tau_i$ is a special segment.
		
		Since $s \geq d+3$ we must have that there exists $i$ and $j$, $1\leq i < j \leq s-1$ such that $\tau_i=\tau_j$. Let $\beta = \tau_{i+1}\tau_{i+2}\cdots\tau_j$. Then $$\gamma_k := \tau_0\tau_1\cdots\tau_{j-1}\tau_j\beta^k\tau_{j+1}\cdots\tau_{s-1} \in \Gamma_{s + k(j-i)}$$
		where $\beta^k$ means $\beta$ repeated $k$ times. If we now choose $k$ large enough such that $s+k(j-i) \geq N$ we can choose $n=s+k(j-i)$ and $\hat{\gamma}=\gamma_k$. Then we see that for any path $\tau$, the composition $\gamma\tau$ is in $\Gamma_{s+r}$ if and only if $\hat{\gamma}\tau$ is in $\Gamma_{n+r}$.
	\end{proof}
\end{lemma}

\begin{theorem}\cite[Corollary~2.4]{GKK91}
	Let $\Lambda = k\Gamma/(\rho)$ be a monomial relation algebra. Then $\findim(\Lambda) \leq d+3$ where $d$ is the number of special segments for $\Lambda$.
	\begin{proof}
		Let $M$ be a module of finite projective dimension and let $N$ be $\pd M$. The projective dimension of $M$ can be characterized as the largest integer $c$ such that $\Tor_c(\Lambda/J, M) \neq 0$. We will show that this is at most $d+3$. Let $s \geq d+3$ be an integer. Then we want to show that $\Tor_{s+1}(\Lambda/J, M)=0$. We compute this by taking the projective resolution of $\Lambda/J$ found in \cref{prop:projective_res_of_top_monomial_alg} and tensoring with $M$.
		\begin{center}
			\begin{tikzcd}
				\cdots \ar[r] & k\Gamma_{s+2} \otimes M  \ar[r, "\delta_{s+2}\otimes M"] & k\Gamma_{s+1} \otimes M \ar[r, "\delta_{s+1}\otimes M"]  & k\Gamma_{s} \otimes M \ar[r] & \cdots
			\end{tikzcd}
		\end{center}
		Let $x$ be in the kernel of $\delta_{s+1}\otimes M$. Then by \cref{lem:decompose_kernel_delta_m} we may assume $x$ is on the form
		$$x = \sum_j \gamma \gamma_j \otimes m_j.$$
		Now since $\gamma$ is in $\Gamma_s$ \cref{lem:monomial_relation_repetition} gives us that there is an $n \geq N$ and a $\hat{\gamma} \in \Gamma_n$ such that $\gamma\tau$ is in $\Gamma_{s+r}$ if and only if $\hat{\gamma}\tau$ is in $\Gamma_{n+r}$.
		
		Then $\hat{x} = \sum \hat{\gamma}\gamma_j \otimes m_j$ is in the kernel of $\delta_{n+1}\otimes M$. Since $n+1>N=\pd M$ the complex is exact at $n+1$. This means that there are elements $\gamma_j^k$ and $m_j^k$ such that
		$$\hat{x} = \delta_{n+2} \left(\sum_j \sum_{k=0}^{n_j} \hat{\gamma}\gamma_j\gamma_j^k \otimes m_j^k\right) = 
		\sum_j \sum_{k=0}^{n_j} \hat{\gamma}\gamma_j \otimes \gamma_j^k m_j^k$$
		Since $\hat{\gamma}\gamma_j\gamma_j^k$ is in $\Gamma_{n+2}$ if and only if $\gamma\gamma_j\gamma_j^k$ is in $\Gamma_{s+2}$ we have that
		$$x = \delta_{s+2} \left(\sum_j \sum_{k=0}^{n_j} {\gamma}\gamma_j\gamma_j^k \otimes m_j^k\right)$$
		and thus $\Tor_{s+1}(\Lambda/J, M)=0$ so $\pd M \leq d+3$.
	\end{proof}
\end{theorem}




\section{Unbounded derived category}\label{sec:Unbounded_derived_category}

So far we have been focused on the finite dimensional version of the finitistic dimension, known as the little finitistic dimension. Namely 
$$\findim(\Lambda) = \sup\{\pd M \mid M \in \mod\Lambda, \pd M < \infty\}.$$

In this section we will consider infinite dimensional modules, and thus it is natural for us to look at the infinite dimensional version of the finitistic dimension, known as the big finitistic dimension. It is defined, as you would expect, by considering not just finite dimensional modules, but all $\Lambda$-modules:

$$\Findim(\Lambda) = \sup\{\pd M \mid M \in \Mod\Lambda, \pd M < \infty\}.$$

Note that $\findim(\Lambda) \leq \Findim(\Lambda)$ and so if we can show that $\Findim(\Lambda) < \infty$ we have also shown that $\findim(\Lambda) < \infty$.

In \cref{thm:FDC_implies_VC} we showed that if $\findim(\Lambda) < \infty$, then $D\Lambda$ becomes a generator in $\D^b(\Lambda)$. In this section we show that if we instead consider the unbounded derived category of all $\Lambda$-modules, then we get an analogous converse result.

\begin{defn}[Localizing subcategory]
	A full subcategory of a triangulated category $\mathcal T$ is called \emph{localizing} if
	\begin{enumerate}[i)]
		\item It is triangulated. I.e. it is closed under shifts and cones.
		\item It is closed under arbitrary coproducts.
	\end{enumerate}
	For a class of objects $\mathcal S \subset \mathcal T$ we call the smallest localizing subcategory that contains $\mathcal S$ the localizing category generated by $\mathcal S$, and we write $\langle \mathcal S \rangle$.
\end{defn}

It's a well known fact that $\Lambda$ generates the derived category as a localizing subcategory. We also have a dual notion, a colocalizing subcategory. Similarly it is true that $D\Lambda$ generates the derived category as a colocalizing subcategory. In the below theorem we do something a bit unexpected, we ask whether the derived category also is generated by $D\Lambda$ as a localizing subcategory.

\begin{theorem}\cite[Theorem~4.3]{Rick19}\label{thm:injectives_generate_implies_FDC}
	If the localizing subcategory generated by $D\Lambda$ is the entire unbounded derived category, then $\Findim(\Lambda) < \infty$.
	\begin{proof}
		Assume $\Findim(\Lambda) = \infty$. Then there are modules $M_i$ with projective dimension $i$ for every $i \geq 0$. Let $P_i$ be the minimal projective resolution of $M_i$, and consider $\bigoplus P_i[-i]$ and $\prod P_i[-i]$. Both of these have homology $M_i$ in degree $i$, and are concentrated in non-negative degrees.
		
		The inclusion from the sum to the product is clearly a quasi-isomorphism. We want to show that it is not a homotopy equivalence. Assume for the sake of contradiction that it was. Then tensoring with $\Lambda/J$ would give us another homotopy equivalence. Since $\Lambda/J$ is finitely presented tensoring preserves both products and coproducts. Because all the resolutions were minimal, tensoring with $\Lambda/J$ gives us a complex with differentials equal to 0. In degree 0 we get $$\bigoplus \Tor_i(\Lambda/J, M_i) \to \prod \Tor_i(\Lambda/J, M_i) .$$
		Since $\Tor_i(\Lambda/J, M_i)$ is nonzero for every $M_i$ this map is not an isomorphism, and so we don't have a homotopy equivalence.
		
		Let $C$ be the cone of $\bigoplus P_i[-i] \to \prod P_i[-i]$. Then $C$ is 0 in the derived category, but non-zero in the homotopy category. Since $\Lambda$ is artinian the product of projectives is projective\cite[Theorem~3.3]{Chase60}, so $\prod P_i[-i]$ is a complex of projectives, which means that $C$ is a complex of projectives. 
		
		In other words $C$ is an acyclic lower bounded complex of projectives that is not contractible. Tensoring with $D\Lambda$ is an equivalence from projectives to injectives with inverse $\Hom(D\Lambda, -)$ (c.f. \cref{thm:Proj_Inj_equivalence} in the appendix), so $D\Lambda \otimes C$ is a lower bounded complex of injectives that is not contractible. Such a complex cannot be acyclic so $D\Lambda \otimes C$ has homology, and is thus non-zero in $\D(\Lambda)$.
		
		The homology of $C$ is 0, so $K(\Lambda)(\Lambda, C[i]) = 0$. Applying the equivalence $D\Lambda \otimes -$ we get 
		$$0=K(\Lambda)(D\Lambda, D\Lambda \otimes C [i])=\D(\Lambda)(D\Lambda, D\Lambda \otimes C [i]).$$ 
		
		The full subcategory of objects $X$ with $\D(\Lambda)(X, D\Lambda \otimes C [i]) = 0$ is localizing and contains $D\Lambda$, so it contains $\langle D\Lambda \rangle$.
		
		This means that $D\Lambda \otimes C$ is not in $\langle D\Lambda \rangle$, and so that can not be the entire derived category.
	\end{proof}
\end{theorem}

\begin{theorem}\cite[Theorem~4.4]{Rick19}
	For a finite dimensional algebra $\Lambda$ we have $\Findim(\Lambda) < \infty$ if and only if $D\Lambda^\perp \cap \D^+(\Lambda) = 0$.
	\begin{proof}
		In the theorem above we proved that when the finitistic dimension is infinite then there is a non-zero complex in $\D^+(\Lambda)$ perpendicular to $D\Lambda$. 
		
		The proof of the converse is the same as for \cref{thm:FDC_implies_VC}. If we have a non-zero object $X \in D\Lambda^\perp \cap \D^+(\Lambda)$, then by replaccing $X$ by its minimal injective resolution we see that $\D(\Lambda)(D\Lambda, X)$ is an acyclic minimal complex of projectives that continue arbitrarily to the right. So the cokernels have arbitrarily large projective dimension. 
	\end{proof}
\end{theorem}

\section{Summary}

FDC holds for the following classes of algebras

\begin{itemize}
	\item \textbf{Big FDC:}
	\item Representation finite algebras
	\begin{proof}
		The supremum over a finite set is finite so $\findim(\Lambda) < \infty$ for a representation finite algebra.
	\end{proof}
	\item Monomial algebras
	\begin{proof}
		This was shown in \cref{sec:monomial_algebras}.
	\end{proof}
	\item Gorenstein algebras
	\begin{proof}
		An algebra is said to be Gorenstein if all injectives have finite projective dimension and all projectives have finite injective dimension. In particular the $\Lambda$-module $\Lambda$ is isomorphic to a finite injective resolution in the derived category. So $\Lambda$ is in the localizing category generated by injectives. Then \cref{thm:injectives_generate_implies_FDC} gives us that $\Findim(\Lambda) < \infty$, and therefor also $\findim(\Lambda) < \infty$.
	\end{proof}
	\item Finite global dimension
	\item Self injective
	\item $J^2 = 0$
	\item Derived equivalent to the above
	\item Local algebras
	\begin{proof}
		Local algebras are local artinian rings. So if $\Lambda$ is local then $\findim(\Lambda)=0$.
	\end{proof}
	\item \textbf{only small FDC is known?:}
	\item Stably hereditary algebras
	\item Special biserial algebras
	\item "half rep-finite" algebras, i.e. $\Lambda/J^l$ rep-finite $J^{2l+1}=0$.
\end{itemize}

Not sure where to put this, ill put it here for now
\begin{theorem}
	Local artinian rings have finitistic dimension zero.
	\begin{proof}
		Assume there is a non-projective module with finite projective dimension. Then in particular we have one with projective dimension equal to 1. Since all finitely generated projectives are free this means we have a short exact sequence
		\begin{center}
			\begin{tikzcd}
				0 & R^n & R^m & M & 0
			\end{tikzcd}
		\end{center}
	with $R^n$ contained in $JR^m$. Let $k$ be the minimal integer such that $J^k=0$. Let $a$ be a generator in $R^n$ and let $r$ be a non-zero element of $J^{k-1}$. Then $ra$ is non-zero, but is mapped to something in $J^{k-1}JR^m=0$, thus the map is not injective which gives a contradiction. 
	\end{proof}
\end{theorem}

\section{Dual conjectures}

Many of the cases are equivalent to their dual statements. Some are not.
\begin{itemize}
	\item Given a recollement of the bounded derived category you get one for $\Lambda^{\operatorname{op}}$. \todo{Look at examples of reccolement to see how it translates.}
	\item Just because the subcategory of modules with finite projective dimension is contravariantly finite does not mean the subcategory of modules with finite injective dimension has to be covariantly finite. See \cref{exam:not_contravariantly_finite}.
	\item repdim of $\Lambda$ equals the repdim of $\Lambda^{\operatorname{op}}$.
	\begin{proof}
		If $M$ is an auslander generator for $\Lambda$ then $DM$ is an auslander generator for $\Lambda^{\operatorname{op}}$.
	\end{proof}
	\item If $J^{2l+1} = 0$ and $\Lambda/J^l$ is repfinite then the same is true for $\Lambda^{\operatorname{op}}$.
	\item If $\Lambda$ is monomial then so is $\Lambda^{\operatorname{op}}$.
	\item Injective generates implies the weaker property that projective cogenerate for the opposite algebra. This is also sufficient to prove the algebra satisfies FDC.\cite[Section~5]{Rick19} 
\end{itemize}

Similarly for the weaker conjectures
\begin{itemize}
	\item GSC says the injective dimension of $\Lambda$ is finite if and only if the injective dimension of $\Lambda^{\operatorname{op}}$ is finite. This statement is symmetric with respect to $\Lambda$ and $\Lambda^{\operatorname{op}}$. So the dual is equivalent.
	\item NC: Certainly $\Lambda$ is self injective if and only $\Lambda^{\operatorname{op}}$ is. \todo{Can the dominant dimension of the opposite algebra be different? Arbitrary different?}
	\item For all the others it seems just as difficult as solving the conjecture to connect it to it's dual.
\end{itemize}

\begin{appendices}
\section{Appendix: Homological algebra}\label{sec:appendix}
\section{Homological algebra}\label{sec:appendix}

In this section we collect relevant theorems from homological algebra that would be distracting within the text itself.

\begin{lemma}\cite[Chapter I, theorem 3.2]{CE56} \label{lem:injectives_for_noetherian_ring}
	Let $R$ be a noetherian ring. Then an $R$-module $Q$ is injective if and only if it has the injective lifting property for inclusions of ideals into $R$.
	\begin{proof}
		If $Q$ is injective then $Q$ has the lifting property for all monomorphisms, so one direction is clear. Assume we have a diagram
		\begin{center}
			\begin{tikzcd}
			Q\\
			M \ar[u, "f"] \ar[r, hook] & N \ar[ul, dashed]
			\end{tikzcd}
		\end{center}
		We want to show that the dashed arrow exists. Let $S$ be the partially ordered set $\{(M', f'): M \leq M', f'|_M = f\}$. By Zorn's lemma this has a maximal element $(M', f')$. Assume $M' \neq N$, then there is an element $x \in N - M'$. The set of $r$ such that $rx \in M'$ forms an ideal $I$. Define the map $g: I \to Q$ by $I(r) = f'(rx)$. By hypothesis $g$ lifts to a map $\tilde{g}:R \to Q$. Let $q$ be $\tilde{g}(1)$. Then $\tilde{f}: M' + Rx \to Q$ defined by $\tilde{f}(m + rx) = f'(m) + rq$ gives us a bigger element of $S$, contradicting maximality. Thus $M'=N$ and $Q$ is injective.
	\end{proof}
\end{lemma}

\begin{theorem}
	Let $R$ be a noetherian ring. Then an arbitrary coproduct of injectives is injective.
	\begin{proof}
		By the lemma above it is enough to show the lifting property on ideals of $R$. Let $I$ be an ideal and $f:I \to \bigoplus_i Q_i$ be a map to a coproduct of injectives. Since $R$ is notherian $I$ is finitely generated so $f$ factors through a finite sum $I \to \bigoplus_{i=0}^n Q_i \to \bigoplus Q_i$. Since finite coproducts of injectives are injective we are done.
		\begin{center}
			\begin{tikzcd}
			\bigoplus Q_i\\
			\bigoplus\limits_{i=0}^n Q_i \ar[u]\\
			I \ar[u] \ar[r, hook] & R \ar[ul, dashed]
			\end{tikzcd}
		\end{center}
	\end{proof}
\end{theorem}

\begin{theorem}\cite[Chapter I, Exercise 8]{CE56}
	Let $R$ be a noetherian ring. Then direct limits of injectives is injective.
	\begin{proof}
		By the lemma above it is enough to show the lifting property on ideals of $R$. Let $I$ be an ideal and let $Q = \lim\limits_{\rightarrow} Q_i$ be a direct limit of injectives.
		
		Since $R$ is noetherian $I$ is finitely presented, say $R^n \to R^m \to I \to 0$. Applying $\Hom(-,Q)$ we get an exact sequence 
		\begin{center}
			\begin{tikzcd}
			0 \ar[r] & \Hom(I, Q) \ar[r] & \Hom(R^m, Q) \ar[r] & \Hom(R^n, Q)
			\end{tikzcd}
		\end{center}
		Since direct limits are exact we also have an exact sequence
		\begin{center}
			\begin{tikzcd}
			0 \ar[r] & \lim\limits_{\rightarrow}\Hom(I, Q_i) \ar[r] & \lim\limits_{\rightarrow}\Hom(R^m, Q_i) \ar[r] & \lim\limits_{\rightarrow}\Hom(R^n, Q_i)
			\end{tikzcd}
		\end{center}
		We also have a natural map $\lim\limits_{\rightarrow}\Hom(-, Q_i) \to \Hom(-, Q)$. $\Hom(R^n, Q_i)$ just equals $Q_i^n$, so this map is an isomorphism at $R^n$. Then by the five lemma applied to the two sequences above we get that $\Hom(I, Q) \cong \lim\limits_{\rightarrow}\Hom(I, Q_i)$ for all ideals $I$. So since 
		\begin{center}
			\begin{tikzcd}
			\lim\limits_{\rightarrow}\Hom(R, Q_i) \ar[r] & \lim\limits_{\rightarrow}\Hom(I, Q_i) \ar[r] & 0
			\end{tikzcd}
		\end{center}
		is exact, we get that
		\begin{center}
			\begin{tikzcd}
			\Hom(R, Q) \ar[r] & \Hom(I, Q) \ar[r] & 0
			\end{tikzcd}
		\end{center}
		is exact. Hence $Q$ is injective.
	\end{proof}
\end{theorem}

\begin{theorem}\label{thm:local_artin_ring_Findim_0}
	If $R$ is a local artinian ring, then all modules with finite projective dimensions are projective. In other words we have that $\Findim(R) = 0$.
	\begin{proof}
		Assume there is a non-projective module with finite projective dimension. Then in particular we have one with projective dimension equal to 1. Since all projective modules are free this means we have a short exact sequence
		\begin{center}
			\begin{tikzcd}
			0 & R^{(I')} & R^{(I)} & M & 0
			\end{tikzcd}
		\end{center}
		where $R^{(I')}$ maps into $JR^{(I)}$. Let $k$ be the minimal integer such that $J^k=0$. Let $a$ be a generator in $R^{(I')} $ and let $r$ be a non-zero element of $J^{k-1}$. Then $ra$ is non-zero, but is mapped to something in $J^{k-1}JR^m=0$, thus the map is not injective which gives a contradiction. 
	\end{proof}
\end{theorem}

\begin{theorem}\label{thm:Proj_Inj_equivalence}
	Let $\Lambda$ be an artin algebra. Then we have an equivalence of categories 
	\begin{center}
		\begin{tikzcd}[column sep = 50pt]
		\operatorname{Proj}\Lambda \ar[r, bend left=10, "D\Lambda \otimes -"] & \operatorname{Inj}\Lambda \ar[l, bend left=10]{}{\Hom(D\Lambda, -)}
		\end{tikzcd}
	\end{center}
	where the tensor product is over $\Lambda$, and $\Hom(D\Lambda, X)$ is considered as a $\Lambda$-module by considering $D\Lambda$ as a bimodule.
	\begin{proof}
		First we note the following isomorphisms of $\Lambda$-modules when evaluating the functors at $\Lambda$ and $D\Lambda$
		\begin{align*}
		\Hom(D\Lambda, D\Lambda \otimes \Lambda) &\cong \End(D\Lambda)\\
		&\cong \End(\Lambda_\Lambda) \\
		&\cong \Lambda
		\end{align*} 
		\begin{center}
			and
		\end{center}
		\begin{align*}
		D\Lambda \otimes \Hom(D\Lambda, D\Lambda) &\cong D\Lambda \otimes \Lambda\\
		&\cong D\Lambda.
		\end{align*}
		Since $D\Lambda$ is finitely presented $D\Lambda \otimes -$ and $\Hom(D\Lambda, -)$ preserve both products and coproducts. Then since $\operatorname{Proj}\Lambda = \operatorname{Add}\Lambda$ and $\operatorname{Inj}\Lambda = \operatorname{Prod}D\Lambda$ it follows from the equations above that $\Hom(D\Lambda, D\Lambda \otimes -)$ and $D\Lambda \otimes \Hom(D\Lambda, -)$ are isomorphic to the identity on $\operatorname{Proj}\Lambda$ and $\operatorname{Inj}\Lambda$ respectively. 
		
		Lastly we verify that the maps are well defined. Since $\Lambda$ is an artin algebra each injective module is the injective envelope of its socle. Since the socle is semisimple it is the direct sum of simple modules. Thus each injective is the sum of indecomposable injective modules, and hence we have that $\operatorname{Add}D\Lambda = \operatorname{Inj}\Lambda$. It is true for any ring that $\operatorname{Add}\Lambda = \operatorname{Proj}\Lambda$, and so we have the following:
		
		$$D\Lambda\otimes (\operatorname{Proj}\Lambda) = D\Lambda\otimes (\operatorname{Add}\Lambda) = \operatorname{Add} D\Lambda = \operatorname{Inj}\Lambda,$$
		\begin{center}
			and
		\end{center}
		$$\Hom(D\Lambda, \operatorname{Inj}\Lambda) =\Hom(D\Lambda, \operatorname{Add}D\Lambda) = \operatorname{Add} \Lambda = \operatorname{Proj}\Lambda.$$
		So the maps induce an equivalence of categories.
	\end{proof}
\end{theorem}

\begin{theorem}[Fitting's Lemma]\label{thm:Fittings_lemma}
	Let $R$ be a ring, $M$ an $R$-module, and $L\colon M \to M$ an endomorphism. If $X$ is a noetherian submodule of $M$, then there exists a positive integer $\eta_X$ such that $L|_{L^n(X)}\colon L^n(X) \to M$ is injective for all $n \geq \eta_X$.
	\begin{proof}
		We have an increasing sequence of submodules of $X$ given by:
		$$\ker L \cap X \subseteq \ker L^2 \cap X \subseteq \ker L^3 \cap X \subseteq \cdots$$
		Since $X$ is noetherian this sequence stabilizes, i.e. there is an integer $\eta_X$ such that $\ker L^n \cap X = \ker L^{n+1} \cap X$ for all $n \geq \eta_X$. We know that $L^n(X) \cong X / \ker L^n \cap X$, and that through this isomorphism the map $L \colon L^n(X) \to M$ is induced by $L^{n+1} \colon X / \ker L^n \cap X \to L^{n+1}(X) \subseteq M$. Since for $n \geq \eta_X$ we have that $\ker L^n \cap X = \ker L^{n+1}\cap X$ this map is injective, and so the theorem holds.
	\end{proof}
\end{theorem}

Interesting examples of Fitting's Lemma comes from $R$ being a noetherian ring and $X$ being a finitely generated modules. In particular the case when $R = \mathbb Z$ appears in \cref{sec:Igusa-Todorov}. An important special case of Fitting's Lemma that comes up when working with artinian rings is when $X=M$ and $X$ has finite length. Remember that over an artin ring all finitely generated modules have finite length.

\begin{cor}\label{cor:fittings_lemma_artin}
	Let $X$ be a module of finite length, and let $L\colon X\to X$ be an endomorphism. Then $L$ splits as a direct sum $L_1 \oplus L_2 \colon X_1 \oplus X_2 \to X_1 \oplus X_2$ such that $L_1$ is nilpotent and $L_2$ is an isomorphism.
	\begin{proof}
		Since $X$ has finite length it is noetherian, thus we can apply Fitting's Lemma. Let $n$ be the positive integer we get from Fitting's Lemma, and let $K$ be $\ker L^{n}$. We wish to show that $X$ is the direct sum of $K$ and $L^n(X)$. Note that since $L$ is inejctive when restricted to $L^n(X)$ we have that $K \cap L^n(X)=0$, so all we have to show is that $X = K + L^n(X)$.
	
		We have a short exact sequence
		\begin{center}
			\begin{tikzcd}
				0 \ar[r] & K\ar[r] & X\ar[r] & L^{n}(X)\ar[r] & 0.
			\end{tikzcd}
		\end{center}
		From this we conclude that the length of $L^{n}(X)$ is equal to the length of $X$ minus the length of $K$. Since $\ker L^n = \ker L^{2n}$ we also have that the length of $L^n(X)$ and $L^{2n}(X)$ are equal. Since $L^{2n}(X)$ is a submodule of $L^n(X)$ this means that $L^n(X)=L^{2n}(X)$. Thus $L$ restricts to an automorphism on $L^n(X)$. Let $\psi$ be its inverse. Then for any $x \in X$ we have $x = \psi L^n(x) + x - \psi L^n(x)$. Clearly $\psi L^n(x)$ is in $L^n(X)$. Applying $L^n$ to $x-\psi L^n(x)$ we get
		\begin{align*}
			L^n(x-\psi L^n(x)) &= L^n(x) - L^n \psi L^n (x)\\
			&= L^n(x) - L^n(x)\\
			&= 0
		\end{align*}
		Thus $ x - \psi L^n(x)$ is in the kernel and so $X = K \oplus L^n(X)$. Then we see that $L$ breaks down as a direct sum $L = L_1 \oplus L_2$ with $L_1\colon K \to K$ nilpotent and $L_2 \colon L^n(X) \to L^n(X)$ an isomorphism.
	\end{proof}
\end{cor}
\end{appendices}

\section{Personal appendix}
\begin{theorem}
	The global dimension of an artin algebra is the supremum of $k$ with $\Ext^k(T,T)\neq 0$ ($T$ sum of simples). This is also the supremum of projective dimension and supremum of injective dimension.
	\begin{proof}
		For a minimal projective resolution $\Hom(-,T)$ makes the differentials 0, and similarly with $\Hom(T,-)$ and injective resolutions. So $\Ext^k(M, T)$ is only 0 exactly when $k>\pd M$, similarly $\Ext^k(T,M)$ is only 0 when $k$ is bigger than the injective dimension. Since any module is built by extensions of simples you can prove by induction, and the long exact sequence in $\Ext(-,T)$ you get that any module has projective dimension less than or equal to that of $T$. Similarly for injective dimension.
	\end{proof}
\end{theorem}

$\findim(\Lambda)$ need not equal $\findim(\Lambda^{\operatorname{op}}) = \sup\{ $injective dimension of $M | M$ has finite injective dimension$ \}$.

\begin{example} \cite{Gre20}
	Let $\Lambda=k \left.
	\left[\begin{tikzcd}
	\ar[out=150,in=210, loop, swap, looseness=3, "a"] 1 \ar[r, bend left=15, "b"] & 2 \ar[l, bend left=15, "c"]
	\end{tikzcd}\right] \middle/ (a^2, ac, ba, cbc) \right.$. Then $\findim(\Lambda) \geq 1$, but $\findim(\Lambda^{\operatorname{op}})=0$.
	\begin{proof}
		The module $\mymatrix{1\\1} = P_1/P_2 $ ($k^2$ where $a$ acts by $\begin{bsmallmatrix}
			0 & 1\\0&0
		\end{bsmallmatrix}$, and $b$ and $c$ act trivially)
		has projective dimension 1, so $\findim(\Lambda) \geq 1$. The projective/injective modules of $\Lambda$ are:
		$$ P_1 = \mymatrix{
			&1&\\
			1 && 2\\
			&&1\\
			&&2
		},\quad P_2 = \mymatrix{
			2\\1\\2
		},\quad I_1 = \mymatrix{
			&&1\\
			1&&2\\
			&1&
		},\quad I_2 = \mymatrix{
			1\\2\\1\\2
		} $$
		If $\findim(\Lambda^{\operatorname{op}})>0$ there would be a module with finite non-zero injective resolution. In particular it would end with a non-split epimorphism between injectives. I claim this would mean there is a non-split epimorphism $I \to I_i$ from an injective to an indecomposable injective. Obviously we get epimorphisms by composing with the projections onto summands, so we want to show that they are not split. Assume that they are, that is the map looks like
		
		\begin{center}
		\begin{tikzcd}[ampersand replacement=\&]
			I_i \oplus I \ar{r}{
				\begin{bmatrix}
				1 & 0\\ f & g
				\end{bmatrix}
			} \ar[swap]{rd}{
				\begin{bmatrix}
				1 & 0
				\end{bmatrix}
			} \& I_i \oplus I' \ar[]{d}{
				\begin{bmatrix}
				1 & 0
				\end{bmatrix}
			}\\
			\& I_i
		\end{tikzcd}.
		\end{center}
		We see that by changing basis in the domain we get the matrix $\begin{bmatrix}
		1&0\\0&g
		\end{bmatrix}$. Thus $I_i$ is mapped isomorphically to itself, which doesn't happen in a minimal resolution.
		
		The only thing left to show is that there are no non-split epimorphisms from injective modules to $I_1$ and $I_2$.
	\end{proof}
\end{example}

\clearpage

\bibliography{mybib}
%\bibliography{intro_bib}
\bibliographystyle{alpha}
%\printbibliography
\end{document}