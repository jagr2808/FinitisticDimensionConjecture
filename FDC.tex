\documentclass[11pt, a4paper, english]{article}
\usepackage[utf8]{inputenc}
\usepackage{babel, amsmath, amsthm, amssymb, amsfonts, mathtools, centernot, enumerate}
\usepackage{thmtools, thm-restate}
\usepackage{tikz-cd}
\usepackage{tikz-3dplot}
\usepackage{caption}
\usepackage{intcalc}
\usepackage{stmaryrd}
\usepackage{multicol}
\usepackage{cite}
\usepackage{hyperref}
\usepackage[capitalise]{cleveref}

\usepackage[toc,page]{appendix}
\usepackage{fancyhdr}
\usepackage{todonotes}
\newcommand\tab[1][1cm]{\hspace*{1}}
\DeclarePairedDelimiter{\ceil}{\lceil}{\rceil}

\newtheorem{theorem}{Theorem}[section]
\declaretheorem[name=Theorem,sibling=theorem]{restate-thm}
\newtheorem{conj}{Conjecrture}
\newtheorem{cor}{Corollary}[theorem]
\newtheorem{prop}[theorem]{Proposition}
\newtheorem{lemma}[theorem]{Lemma}
\theoremstyle{definition}
\newtheorem{defn}[theorem]{Definition}
\newtheorem{example}[theorem]{Example}

\newcommand{\C}{\mathbb{C}}
\newcommand{\Z}{\mathbb{Z}}
\DeclareMathOperator{\Hom}{Hom}
\DeclareMathOperator{\Ext}{Ext}
\DeclareMathOperator{\Tor}{Tor}
\DeclareMathOperator{\End}{End}
\DeclareMathOperator{\Aut}{Aut}
\DeclareMathOperator{\Image}{Im}
\DeclareMathOperator{\Ker}{Ker}
\DeclareMathOperator{\cok}{Cok}
\DeclareMathOperator{\depth}{depth}
\DeclareMathOperator{\findim}{findim}
\DeclareMathOperator{\Findim}{Findim}
\DeclareMathOperator{\domdim}{domdim}
\DeclareMathOperator{\repdim}{repdim}
\DeclareMathOperator{\inj}{inj}
\DeclareMathOperator{\proj}{proj}
\DeclareMathOperator{\op}{op}
\DeclareMathOperator{\pd}{pd}
\DeclareMathOperator{\add}{add}
\DeclareMathOperator{\Mod}{Mod}
\def\mod{\operatorname{mod}}
\DeclareMathOperator{\rad}{rad}

\newsavebox{\pullback}
\sbox\pullback{%
	\begin{tikzpicture}%
	\draw (0,0) -- (1ex,0ex);%
	\draw (1ex,0ex) -- (1ex,1ex);%
	\end{tikzpicture}
}

\newcommand{\mymatrix}[1]{\begin{matrix}#1\end{matrix}}

\DeclareMathOperator{\D}{\mathcal{D}}

\setlength{\parindent}{0em}
\setlength{\parskip}{1em}

\pagestyle{fancy}
\fancyhead{}
%\fancyhead[LO, LE]{\small\emph{McKay correspondence}}
\fancyhead[LO, LE]{\small\nouppercase\rightmark}
\fancyfoot[CO, CE]{\thepage}
%\fancyfoot[RO, RE]{\thepage}
\renewcommand{\sectionmark}[1]{\markboth{}{\emph{\thesection~#1}}}
%\renewcommand{\subsectionmark}[1]{}% Remove \subsection from header

\begin{document}
\title{Finitistic dimension conjecture}
\author{Jacob Fjeld Grevstad}
\date{2020}
\maketitle
\pagenumbering{roman}

\begin{abstract}
FDC yo! This is abstract!
\end{abstract}
\clearpage

\tableofcontents
\clearpage

\section*{Notation}
\addcontentsline{toc}{section}{\protect\numberline{}Notation}%
\markboth{section}{Notation}
$k$ a field $\Lambda$ findim alg, $J$ radical

$\mod\Lambda$ finite dimensional (left-)modules, $\Mod\Lambda$ all (left-)modules.

$_\Gamma M_\Lambda$ is a $\Gamma-\Lambda$-bimodule. Left $\Gamma$-module, right $\Lambda$-module

Quiver/path algebra. Multiplication is written right to left

$\Hom_{\mathcal C}(M, N)$ can be written as $\mathcal C(M,N)$ or sometimes simply $(M,N)$

$D: \mod \Lambda \to \mod\Lambda^{\operatorname{op}}$ is the duality $DM = \Hom(M, k)$

$\D^b(\Lambda)$ bounded derived category, $K^b$, $K^{+,b}$, $K^{-, b}$, $\D$, etc. $X^{\geq n}$ hard truncation, $[i]$ shift

All functors are additive

$I(M)$ injective envelope, $P_M^0$ projective cover, $\cdots \to P_M^1 \to P_M^0$ projective resolution. $\Omega^n M$ syzygy.

$\pd$ projective dimension

\newpage

\pagenumbering{arabic}
\section*{Introduction}
\addcontentsline{toc}{section}{\protect\numberline{}Introduction}%
\markboth{section}{Introduction}

This is an introduction

\section{The homological conjectures}

\begin{itemize}
	\item FDC - finitistic dimesnion conjecture
	
	Finitistic dimension is always finite
	
	\item WTC - Watamatsu tilting conjecture
	
	A module is called watamatsu tilting if
	\begin{itemize}
		\item $\Ext^n(T,T)=0$ for all $n >0$.
		\item There is an exact sequence $$\eta: 0 \to \Lambda \to T_0 \to T_1 \to \cdots$$ where $T_i$ is in $\add T$.
		\item $\Hom(\eta, T)$ is exact. I.e. $\Ext^1(\Ker f, T)=0$ for every $f$ in $\eta$.
	\end{itemize}
	WTC says that any watamatsu tilting module with finite projective dimension is a tilting module. I.e $\eta$ can be chosen to be bounded.
	
	\item GSC - Gorenstein symmetry conjecture
	
	The injective dimension of $_\Lambda\Lambda$ is finite if and only if the projective dimension of $D(\Lambda_\Lambda)$ is finite.
	
	\item NuC - Nunke condition
	
	If $X \neq 0$ then there is an $n \geq 0$ such that $\Ext^n(D\Lambda, X) \neq 0$. 
	
	\item SNC - strong Nakayama conjecture
	
	For every simple module $S$ there is an $n \geq 0$ such that $\Ext^n(D\Lambda, S) \neq 0$. 
	
	\item ARC - Auslander Reiten conjecture
	
	If $\Ext^n(M, M \oplus \Lambda) = 0$ for all $n > 0$ then $M$ is projective. 
	
	\item NC - Nakayama conjecture
	
	If $\Lambda$ has infinite dominant dimension then $\Lambda$ is self-injective.
	
\end{itemize}

\subsection{Implications}
\begin{tikzcd}
FDC \ar[r]\ar[d] & WTC \ar[r] & GSC\\
NuC\ar[r] & SNC\ar[r] & ARC\ar[r] & NC
\end{tikzcd}

\begin{theorem} \cite[1.2]{Hap93} \label{thm:findim_implies_inj_generate}
	\begin{enumerate}[i)]
		\item If $\findim(\Lambda) < \infty$ (FDC) then $K^b(\inj\Lambda)^\perp = 0$.
		\item If $K^b(\inj\Lambda)^\perp = 0$ then for any $X\neq 0$ there exists $i$ such that, $\Ext^i(D(\Lambda), X) \neq 0$ (NuC).
	\end{enumerate}
	\begin{proof}
		\begin{enumerate}[i)]
			\item[]
			\item Let $I^\bullet \in K^b(\inj\Lambda)^\perp$ be non-zero. Since $\D^b(\Lambda) \cong K^{+,b}(\inj\Lambda)$ we may assume $I^\bullet$ is a complex of injectives, and WLOG we may assume it concentrated in degrees $i \geq 0$, and that $d^0:I^0 \to I^1$ is not split mono. Since if its concentrated in degrees $i \geq k$ we can just shift it, and if $d^0$ is split mono then replacing $I^0$ by $0$, and $I^1$ be $I^1/I^0$ gives a homotopic complex.
			
			$\Hom(D\Lambda, I^i)$ is in $\add\Hom(D\Lambda, D\Lambda) = \add\Lambda$ so $\Hom(D\Lambda, I^\bullet)$ is a complex of projectives.
			
			\begin{center}
			\begin{tikzcd}
			0 \ar[r] \ar[d] & D\Lambda \ar[r] \ar[d, "f"] \ar[dl, dashed]& 0 \ar[d]\\
			I^{i-1} \ar[r, "d^{i-1}"] & I^i \ar[r, "d^i"] & I^{i+1}
			\end{tikzcd}
			\end{center}
			
			Since $I^\bullet$ is in $K^b(\inj\Lambda)^\perp$ and $D\Lambda$ is in $K^b(\inj\Lambda)$, whenever $d^if=0$, $f^\bullet$ is homotopic to 0. Meaning $f$ factors through $d^{i-1}$. This means that $\Hom(D\Lambda, I^\bullet)$ is an exact complex. Further since $\Hom(D\Lambda, -)$ is an equivalence between $\inj\Lambda$ and $\proj\Lambda$ we have that $\Hom(D\Lambda, d^0)$ is not split mono.
			
			$\cok\Hom(D\Lambda, d^i)$ has a projective resolution of length $i$. This resolution is the direct sum of its minimal resolution and an acyclic bounded complex of projectives. Since bounded acyclic complexes of projectives are split and $\Hom(D\Lambda, d^0)$ is not, we must have that the minimal resolution has length $i$, and so $\findim(\Lambda) = \infty$.
			
			\item Assume there is an $X \neq 0$ with $\Ext^i(D\Lambda, X) = 0$ for all $i \geq 0$. Then $X$ considered as a stalk complex is in $K^b(\inj\Lambda)^\perp$. Proceed by induction: If $I[-i] \in K^b(\inj\Lambda)$ is a stalk complex then $\D^b(I[-i], X) = \Ext^i(I, X)$. This is 0 because $D\Lambda$ is the sum of the indecomposable injectives.
			
			Let $I \in K^b(\inj\Lambda)$ be a complex of width $n$. WLOG assume $I$ concentrated in degrees $0 \leq i \leq n-1$. Then $$I^{>0} \to I \to I^{0} \to I^{>0}[1]$$ is a triangle, and $I^{>0}$ has width $n-1$. Taking the long exact sequence in $\D^b(-,X)$ it follows that $\D^b(I, X)=0$. 
		\end{enumerate}
	\end{proof}
\end{theorem}

\begin{prop}WTC $\Rightarrow$ GSC
	\begin{proof}
	$D(\Lambda_\Lambda)$ is watamatsu tilting. WTC then gives us that if $D\Lambda$ has finite projective dimension then $\Lambda$ has a finite injective dimension.
	
	For the other direction assume $_\Lambda\Lambda$ has finite injective dimension. Then $D(_\Lambda\Lambda)$ has finite projective dimension, so WTC gives us that $\Lambda_\Lambda$ has finite injective dimension. Which means $D(\Lambda_\Lambda)$ has finite projective dimension.
	\end{proof}
\end{prop}

\begin{prop}
	ARC is equivalent to $M$ a generator with $\Ext^n(M, M) = 0$ for $n > 0$ implies $M$ projective.
	\begin{proof}
		Assume ARC and that $M$ satisfies the hypothesis. Then since $M$ is a generator $\Lambda$ is in $\add M$ and thus $\Ext^n(M, \Lambda)=0$. So $\Ext^n(M, M\oplus \Lambda)=0$ and $M$ is projective.
		
		For the other direction Assume $M$ satisfies $\Ext^n(M, M \oplus \Lambda)=0$. Then $\Ext^n(M \oplus \Lambda, M\oplus \Lambda) = 0$, so $M \oplus \Lambda$ is projective, which means that $M$ is projective. 
	\end{proof}
\end{prop}

\begin{prop}
	SNC $\Rightarrow$ ARC
	\begin{proof}
		$\Ext^i(D\Lambda, S) = \Ext^i(DS, \Lambda)$, so SNC means that for every simple there is an $i$ such that $\Ext^i(S, \Lambda) \neq 0$.
		
		Assume $M$ is a nonprojective generator such that $\Ext^n(M, M)=0$ for all $n>0$. Let $\Gamma$ be $\End(M)^{\operatorname{op}}$, and let
		\begin{center}
		\begin{tikzcd}
			M \ar[r] & I_0 \ar[r] & I_1 \ar[r] & \cdots
		\end{tikzcd}
		\end{center}
		be an injective resolution of $M$. Since $\Ext^n(M,M)=0$ when we apply $(M,-):=\Hom(M,-)$ we get an exact sequence.
		\begin{center}
		\begin{tikzcd}
			\Gamma \ar[r] & (M,I_0) \ar[r] & (M,I_1) \ar[r] & \cdots
		\end{tikzcd}
		\end{center}
		By \cref{prop:hom_generator_equivalence} this is an injective resolution of $\Gamma$.\todo{prop cited before stated}
		
		Since $M$ is a non-projective generator it has every indecomposable projective as a summand and a nonprojective summand. So $M$ has more indecomposable summands than $\Lambda$ which means that $\Gamma$ has more indecomposable projectives than $\Lambda$. It follows that $\Gamma$ also has more injectives and thus has an injective not on the form $(M, I)$. Let $Q$ be such an injective and let $S$ be its socle. Then $\Hom_\Gamma(S, (M, I_i)) = 0$ for all $i$, so $\Ext^i(S, \Gamma) = 0$ for all $i$. Thus $\Gamma$ does not satisfy SNC.
	\end{proof}
\end{prop}

The next proposition requires part of the theory of Wedderburn projectives. The relevant theory is proven in \cref{sec:wedderburn_correspondence} below.

\begin{prop}
	ARC $\Rightarrow$ NC
	\begin{proof}
		Assume $\Gamma$ has dominant dimension $\infty$, but is not self injective, and let
		\begin{center}
		\begin{tikzcd}
			0 \ar[r] & \Gamma \ar[r] & I_0 \ar[r] & I_1
		\end{tikzcd}
		\end{center}
		be an injective copresentation of $\Gamma$. Let $P$ be the sum of the projective covers of all nonisomorphic simple modules in the socle of $I_0$. Then by \cref{prop:wedderburn_criterion} we have that $P$ is Wedderburn projective.
		
		Let $\Lambda = \End(P)^{\operatorname{op}}$ and let $M = \Hom(P, \Gamma)$. Then $M$ is a nonprojective generator, we want to show that $\Ext^{>0}(M,M)=0$.
		
		We have functors $(M,-):\mod\Lambda \to \mod\Gamma$ and $(P,-):\mod\Gamma \to \mod\Lambda$. By \cref{prop:hom_generator_equivalence} $(M, -)$ is fully faithful and $(P,-)\circ (M,-) = id_{\Lambda}$.
		
		Let $0\to M \to Q_0 \to Q_1$ be an injective copresentation of $M$. Applying $(M,-)$ we get an injective copresentation of $\Gamma$. We conclude that all the projective-inejctive modules are in the essential image of $(M,-)$.
		
		In other words if $I^\bullet$ is the minimal injective resolution of $\Gamma$ then $Q^\bullet := (P, I^\bullet)$ is the minimal injective resolution of $M$, and $(M, Q^\bullet)=I^\bullet$. This means that $(M, Q^\bullet)$ is exact away from 0, so $\Ext^{>0}(M,M)=0$. 
		
		But then $M$ is a nonprojectvie generator with $\Ext^{>0}(M,M)=0$, so $\Lambda$ does not satisfy ARC.
	\end{proof}
\end{prop}

\begin{prop}\cite{AR75}
	SNC $\Rightarrow$ NC
	\begin{proof}
		$\Ext(D\Lambda, S) = \Ext(DS, \Lambda)$. $\Ext(DS, \Lambda)$ being nonzero means $I(DS)$ appears in the injective resolution of $\Lambda$. If all injectives apear in the resolution and the dominant dimension is infinity then all injectives are projective. Thus $\Lambda$ is self injective.
	\end{proof}
\end{prop}

\subsection{Wedderburn correspondence}\label{sec:wedderburn_correspondence}
\begin{prop} \label{prop:hom_generator_equivalence}
Let $\Lambda$ be an artin algebra and $M$ a generator. Let $\Gamma = \End(M)^{\operatorname{op}}$ and $P=(M, \Lambda)$. Then we have the following:
\begin{itemize}
	\item $\End(P)^{\operatorname{op}} = \Lambda$ and $(P, \Gamma) = M$.

\begin{proof}
	By Yonedas lemma we have an equivalence $(M,-):\add M \to \add (M,M)=\proj\Gamma$. Since $M$ is a generator $\Lambda$ is in $\add M$. So $$\End(P)=((M,\Lambda), (M,\Lambda)) = \End(\Lambda)=\Lambda^{\operatorname{op}}$$ \centering{and} $$(P,\Gamma)=((M,\Lambda),(M,M))=(\Lambda,M)=M.$$
\end{proof}

\item $(P,-)\circ (M,-)$ is the identity on $\mod\Lambda$.

\begin{proof}
	Let $X$ be a $\Lambda$-module. Since $\add M$ has only a finite number of indecomposables it is functorially finite. So we can take an $M$-resolution of $X$.
	$$\cdots \to M_1 \to M_0 \to X \to 0$$
	Since $\add M$ contains the projectives this is exact. Applying $(M,-)$ we get a projective resolution of $(M,X)$. Since $(M, X)$ is determined by its projective resolution and $X$ is determined by its $M$-resolution we need only show that $(P,-)\circ (M,-)$ is the identity on $\add M$. Then again by Yonedas lemma $(P, (M, M')) = (\Lambda, M')=M'$.
\end{proof}
\end{itemize}
\end{prop}

\begin{prop}\label{prop:hom_generator_preserves_injectives}
	Let $M$ be a module and $I$ an injective module. If the projective cover of the socle of $I$ is a direct summand of $M$, then $(M,I)$ is an injective $\Gamma:=\End(M)^{\operatorname{op}}$-module.
	\begin{proof}
		Let $J \leq \Gamma$ be a left ideal and let $\psi: J \to (M,I)$ be any $\Gamma$-linear map. By \cref{lem:injectives_for_noetherian_ring} it is enough to show that $\psi$ factors through $\Gamma$. Assume $J$ is generated by $f_i$. If we can find $\gamma: M \to I$ such that $\gamma \circ f_i = \psi(f_i)$ then we would get our factorization by mapping $1\in \Gamma$ to $\gamma$.
		\begin{center}
			\begin{tikzcd}
			\bigoplus M \ar[dr, "\sum \psi(f_i)"] \ar[d, swap, "\sum f_i"]\\
			M \ar[r, swap, dashed, "\gamma"] & I
			\end{tikzcd}
		\end{center}
		Next we want to show that the kernel of $\sum \psi(f_i)$ contains the kernel of $\sum f_i$. To see this let $K$ be the kernel of $\sum f_i$ and let $K'$ be the kernel of $\sum \psi(f_i)$. If $K'$ does not contain $K$, then $Q:= K/K'\cap K$ is a nonzero module that is mapped injectively into $I$. So the socle of $Q$ is a summand of the socle of $I$. Then by assumption the projective cover of the socle of $Q$ is a direct summand of $M$. By the lifting property of projectives we get a map $M \to K$ such that the composition with $\sum \psi(f_i)$ is non-zero.
		
		Let $a_i$ be the composition 
		\begin{tikzcd}
			M \ar[r] & K \ar[r, hookrightarrow] & \bigoplus M \ar[r, "\pi_i"] & M
		\end{tikzcd}.
		Then we get $\sum f_i \circ a_i = 0$. Applying $\psi$ we get $\sum \psi(f_i)\circ a_i = 0$, which gives a contradiction. Thus $K'$ contains $K$.
		
		Using this we get the following commutative diagram:
		\begin{center}
			\begin{tikzcd}
			\bigoplus M \ar[d] \ar[dd, bend right=60, swap, "\sum f_i"] \ar[dr, "\sum \psi(f)"] \\
			(\bigoplus M)/ K \ar[r] \ar[d, hookrightarrow] & I\\
			M \ar[ur, dashed, swap, "\exists\gamma"]
			\end{tikzcd}
		\end{center}
		Since $I$ is injective it lifts monomorphisms so we know that $\gamma$ exists. Thus $(M, I)$ is an injective $\Gamma$-module.
	\end{proof}
\end{prop}

\begin{defn}[Wedderburn projective]
	Let $\Gamma$ be an artin algebra and $P$ a finitely generated projective. Let $\Lambda = \End(P)^{\operatorname{op}}$ and $M=(P, \Gamma)$. $P$ is said to be \emph{Wedderburn projective} if $\End(M)^{\operatorname{op}}=\Gamma$.
\end{defn}

\begin{prop}\label{prop:wedderburn_criterion}
	If $P$ contains the projective cover of all simple modules that appear in the socle of an injective copresentation of $\Gamma$, then $P$ is Wedderburn projective.
\end{prop}

To prove this we first need the next proposition as a lemma.

\begin{prop}
	Let $P$ be a projective $\Gamma$-module, and let $\Lambda = \End(P)^{\operatorname{op}}$. Then $(P, -):\mod \Gamma \to \mod \Lambda$ is fully faithful on $\add I(P/JP)$.
	\begin{proof}
		We want to show that the map $\Hom_\Gamma(I, I') \to \Hom_\Lambda((P,I), (P, I'))$ is an isomorphism. Let's first show injectivity. Let $f:I\to I'$ be a non-zero map. Then the socle of $\Image f$ is a semisimple submodule of $I'$, so it is in $\add P/JP$. Then there exists a nonzero map from $P$ to $\Image f$. Since $P$ is projective this lifts to a map $\hat{f}:P\to I$. Then $f \circ \hat{f}$ is non-zero, so $\Hom_\Gamma(I, I') \to \Hom_\Lambda((P,I), (P, I'))$ is injective.
		
		The argument for surjectivity is similar to that for \cref{prop:hom_generator_preserves_injectives}. Let $\psi:(P,I)\to (P, I')$ be a $\Lambda$-linear map. Let $f_i:P\to I$ generate $(P,I)$ as a $\Lambda$-module. Consider the diagram
		\begin{center}
		\begin{tikzcd}
			\bigoplus P \ar[r, "\sum f_i"] \ar[rd, swap, "\sum \psi(f_i)"] & I \ar[d, dashed, "?"]\\
			& I'
		\end{tikzcd}
		\end{center}
		We wish to show that there is a map at ? completing the diagram. We first show that $K':=\ker \sum \psi(f_i)$ contains $K:=\ker \sum f_i$. Assume for the sake of contradiction that it does not. Then $Q := K / K' \cap K$ is mapped injectively into $I'$ by $\sum \psi(f_i)$. So the socle of $Q$ is in $\add P/JP$, and we have a non-zero map $P \to Q$.
		
		Since $P$ is projective this extends to a map $P \to K$. Let $a_i$ be the compositions 
		\begin{tikzcd}
			P \ar[r] & K \ar[r] & \bigoplus P \ar[r, "\pi_i"] & P
		\end{tikzcd}
		Then clearly $\sum f_i \circ a_i = 0$, but $\sum \psi(f_i) \circ a_i$ is non-zero. Since $\psi$ is $\Lambda$-linear this is a contradiction, so $K'$ contains $K$.
		
		Then we get an induced diagram
		\begin{center}
			\begin{tikzcd}
			\bigoplus P \ar[d, two heads,]\\
			(\bigoplus P) / K \ar[r, hookrightarrow, "\sum f_i"] \ar[rd, swap, "\sum \psi(f_i)"] & I \ar[d, dashed, "\exists"]\\
			& I'
			\end{tikzcd}
		\end{center}
		Now because $I'$ is injective we know that there is a lift, and so $\Hom_\Gamma(I, I') \to \Hom_\Lambda((P,I), (P, I'))$ is surjective, and thus an isomorphism.
	\end{proof} 
\end{prop}

\begin{cor}
	\cref{prop:wedderburn_criterion}
	\begin{proof}
		Let $\Gamma \to I_0 \to I_1$ be a minimal injective presentation of $\Gamma$. Then by \cref{prop:hom_generator_preserves_injectives}we have that $(P, I_0) \to (P,I_1)$ is an injective presentation of $(P,\Gamma)$. The proposition gives us that $(P,-)$ is fully faithful on $I_0$ and $I_1$. Since the endomorphisms of $\Gamma$ are exactly endomorphisms of $I_0 \to I_1$ up to homotopy this means that $$\Gamma^{\operatorname{op}}=\End_\Gamma(\Gamma) = \End_\Lambda((P, \Gamma))$$
		So $P$ is Wedderburn projective.
	\end{proof}
\end{cor}


\section{Recollement}
In this section we will discuss a reduction technique known as recollement. The idea of reduction techniques is to reduce the work of proving an algebra has finite finitistic dimension to proving the same for ``simpler'' algebras. In \cref{sec:Triangular_matrix_rings} we will consider a reduction technique of triangular matrix algebras. The triangular matrix rings are closely related to recollements, and we discuss their relationship more closely in \todo{ref}

OVERGANG?

\begin{defn}[Recollement]\label{def:recollement}
	A \emph{recollement} between triangulated categories $\mathcal T'$, $\mathcal T$ and $\mathcal T''$ is a collection of six functors satisfying:
\begin{center}
\begin{tikzcd}[column sep=4cm]
\mathcal T' \ar[r, "i_*=i_!"{name=i}] & 
\ar[l, swap, "i^*"{name=il}, bend right=30] \ar[l, "i^!"{name=ir}, bend left=30]
\mathcal T \ar[r, "j^!=j^*"{name=j}] & 
\ar[l, swap, "j_!"{name=jl}, bend right=30] \ar[l, "j_*"{name=jr}, bend left=30]
\mathcal T''
\arrow[phantom, from=il, to=i, "\dashv" rotate=-90]
\arrow[phantom, from=i, to=ir, "\dashv" rotate=-90]
\arrow[phantom, from=jl, to=j, "\dashv" rotate=-90]
\arrow[phantom, from=j, to=jr, "\dashv" rotate=-90]
\end{tikzcd}	
\end{center}

\begin{enumerate}[(i)]
	\item All functors are exact, and we have adjoint pairs $(i^*, i_*)$, $(i_!, i^!)$, $(j_!, j^!)$, $(j^*, j_*)$. 
	\item \label{recollement:vanishing_composition}The composition $j^*i_*=0$ vanishes.
	\item \label{item:i_fully-faith} 
	We have natural isomorphisms $i^*i_* \cong i^!i_! \cong \operatorname{id}_{\mathcal T'}$ induced by the units and counits of the adjunctions. 
	\item \label{item:j_fully-faith}
	We have natural isomorphisms $j^!j_! \cong j^*j_* \cong \operatorname{id}_{\mathcal T''}$, also induced by the units and counits. 
	\item \label{recollement:triangles}
	For every $X \in \mathcal T$ we have the following distinguished triangles:
	\begin{center}
	\begin{tikzcd}
	j_!j^!X \ar[r, "\varepsilon"] & X \ar[r, "\eta"] & i_*i^* X \ar[r] & j_!j^!X[1]\\		
	i_!i^!X \ar[r, "\varepsilon"] & X \ar[r, "\eta"] & j_*j^* X \ar[r] & i_!i^!X[1].
	\end{tikzcd}
	\end{center}
\end{enumerate}
Note that (\ref{item:i_fully-faith}) and (\ref{item:j_fully-faith}) are equivalent to $i_*$, $j_!$, and $j_*$ being fully faithful.
\end{defn}

We are specifically interested in recollements where the triangulated categories in question are (bounded) derived categories of finite dimensional algebras.

OVERGANG?

\begin{lemma} \label{lem:adjoint_preserves_bounded_proj/inj}
	Let \begin{tikzcd}
	\D^b(\Lambda') \ar[r, "i_*"{name=i}, bend right=20] & 
	\ar[l, swap, "i^*"{name=il}, bend right=20]
	\D^b(\Lambda)
	\end{tikzcd} be exact functors with an adjoint pair $(i^*,i_*)$. Then $i^*$ preserves bounded projective complexes and $i_*$ preserves bounded injective complexes.
	\begin{proof}
		The bounded projective complexes can be characterized up to isomorphism as the complexes $P$ such that for any complex $Y$ there is an integer $t_Y$ with $\D^b(\Lambda) (P, Y[t])=0$ for $t\geq t_Y$. One can see this by using the equivalence $\D^b(\Lambda) \cong K^{-,b}(\proj\Lambda)$.
		
		Let $P$ be a bounded complex of projectives in $\D^b(\Lambda)$. Then we want to show that $i^*P$ is as well. Let $Y$ be any complex in $\D^b(\Lambda')$. Then $\D^b(\Lambda')(i^*P, Y[t]) = \D^b(\Lambda)(P, i_*Y[t])$, so since $P$ is a bounded complex of projectives there is $t_Y$ such that this vanishes for $t \geq t_Y$. 
		
		The statement for injectives is exactly dual, and so we do not write it out here, but leave it to the reader.
	\end{proof}
\end{lemma}

FILLER

\begin{lemma} \label{lem:uniform_bound_on_homology}
	Let \begin{tikzcd}
	\D^b(\Lambda') \ar[r, "i_*"{name=i}] & 
	\ar[l, swap, "i^*"{name=il}, bend right=30] \ar[l, "i^!"{name=ir}, bend left=30]
	\D^b(\Lambda)
	\end{tikzcd} be exact functors with adjoint pairs $(i^*,i_*)$ and $(i_*, i^!)$. Then the homology of $i_*X$ is uniformly bounded for $X\in\mod\Lambda'$ considered as a complex concentrated in degree 0. I.e. there is an $r$, independent of $X$, such that $H^{j}(i_*X) = 0$ for $j\not\in(-r, r)$.
	\begin{proof}
		We first prove that there is an $r'$, independent of $X$, such that $H^{j}(i_*X)=0$ for $j \geq r'$.
		Let $P$ be $i^*\Lambda \in \D^b(\Lambda')$. Then by \cref{lem:adjoint_preserves_bounded_proj/inj} $P$ is a bounded complex of projectives.
		
		Thus there is an $r'$ such that $P^{-j}=0$ for $j \geq r'$. Then $$\D^b(\Lambda')(P, X[j]) = \D^b(\Lambda)(\Lambda, i_*X[j]) = H^{j}(i_*X)=0$$ for $j\geq r'$ and any $\Lambda'$-module $X$, when considered as a complex concentrated in degree 0.
		
		Next we prove that there is an $r''$ such that $H^{-j}(i_*X)=0$ for $j \geq r''$. The argument is completely dual. Let $I$ be $i^!D\Lambda \in \D^b(\Lambda') \cong K^{+,b}(\inj\Lambda')$. Then again by \cref{lem:adjoint_preserves_bounded_proj/inj} $I$ is a bounded complex of injectives.
		
		Thus there is an $r''$ such that $I^{j}=0$ for $j \geq r''$. Then $$\D^b(\Lambda')(X, I[j]) = \D^b(\Lambda)(i_*X, D\Lambda[j]) = H^{-j}(i_*X)=0$$ for $j\geq r''$ and any $\Lambda'$-module $X$, when considered as a complex concentrated in degree 0.
		
		Letting $r$ be the maximum of $r'$ and $r''$ we get that $H^{j}(X)$ is zero outside of $(-r, r)$.
	\end{proof}
\end{lemma}

Now that we have a good understanding of how the functors in a recollement interact with homology, we can use this to say something about the projective dimension of modules, and thus about the finitistic dimension.

\begin{theorem}\cite[3.3]{Hap93}
	Given a recollement between bounded derived categories 
	\begin{center}
		\begin{tikzcd}[column sep=4cm]
		\D^b(\Lambda') \ar[r, "i_*=i_!"{name=i}] & 
		\ar[l, swap, "i^*"{name=il}, bend right=30] \ar[l, "i^!"{name=ir}, bend left=30]
		\D^b(\Lambda) \ar[r, "j^!=j^*"{name=j}] & 
		\ar[l, swap, "j_!"{name=jl}, bend right=30] \ar[l, "j_*"{name=jr}, bend left=30]
		\D^b(\Lambda''),
		\arrow[phantom, from=il, to=i, "\dashv" rotate=-90]
		\arrow[phantom, from=i, to=ir, "\dashv" rotate=-90]
		\arrow[phantom, from=jl, to=j, "\dashv" rotate=-90]
		\arrow[phantom, from=j, to=jr, "\dashv" rotate=-90]
		\end{tikzcd}	
	\end{center}
	 then we have that $\findim(\Lambda) < \infty$ if and only if we have that $\findim(\Lambda') < \infty$ and $\findim(\Lambda'') < \infty$.
	\begin{proof}
		Assume $\findim(\Lambda) < \infty$. We begin by showing that $\findim(\Lambda') < \infty$.
		
		Let $T = \Lambda' / \rad\Lambda'$ be the sum of all simple $\Lambda'$-modules. Then the projective dimension of $X$ is the largest $t$ for which $\Ext^t(X, T) \neq 0$. Let $X$ be a module in $\mod \Lambda'$ with finite projective dimension. We consider $X$ as a complex concentrated in degree 0. Then since $X$ is isomorphic to its projective resolution, by \cref{lem:adjoint_preserves_bounded_proj/inj} $i_*X$ is a bounded complex of projectives. Say:
		$$i_*X = 0 \to P^{-s} \to \cdots \to P^{s'} \to 0$$
		By \cref{lem:uniform_bound_on_homology} we know there is an $r$ independent of $X$ such that $H^{-j}(i_*X)=0$ for $j \geq r$. Truncating $i_*X$ at $-r$ gives a projective resolution of $\ker d^{-r}_{i_*X}$. So $\ker d^{-r}_{i_*X}$ has projective dimension $-r-(-s) = s-r$. Since $\findim(\Lambda)<\infty$ this means that $s \leq r + \findim(\Lambda)$.
		
		Since $i_*T$ is in $\D^b(\Lambda)$ it is a bounded complex, in particular there is a $t_0$ such that $i_*T^{t}=0$ for $t \geq t_0$. Then by the bounds above $\D^b(\Lambda)(i_*X, i_*T[t]) = 0$ for $t \geq t_0 + s \geq t_0 + r + \findim(\Lambda)$. Since $i_*$ is fully faithful this equals $\D^b(\Lambda')(X, T[t])$, and so $\findim(\Lambda') \leq t_0 + r + \findim(\Lambda)$. In particular it is finite.
		
		The proof for $\findim(\Lambda'')$ is the same, just replacing $i_*$ with $j_!$. We leave writing out the details to the reader.
		
		For the converse assume $\Lambda'$ and $\Lambda''$ both have finite finitistic dimension. Let $T = \Lambda / \rad\Lambda$, and $X$ be a $\Lambda$-module with finite projective dimension, and consider both modules as a complex concentrated in degree 0. By \cref{def:recollement}(\ref{recollement:triangles}) we have distinguished triangles:
		\begin{center}
			\begin{tikzcd}
				j_!j^!X \ar[r] & X \ar[r] & i_*i^* X \ar[r] & j_!j^!X[1]\\		
				i_!i^!T \ar[r] & T \ar[r] & j_*j^* T \ar[r] & i_!i^!T[1].
			\end{tikzcd}
		\end{center}
		We write $(-,-)_m$ instead of $\D^b(\Lambda)(-,-[m])$, and make the following abbreviation:
		\begin{align*}
			X_j &:= j_!j^!X & X_i &:= i_*i^* X \\
			T_i &:= i_!i^!T & T_j &:= j_*j^* T.
		\end{align*} 
		Taking the long exact sequence in homfuntors we get the long exact sequences:
		\begin{center}
			\begin{tikzcd}[column sep=0.5cm]
			\cdots \ar[r] & (X, T_i)_m \ar[r] & (X, T)_m \ar[r] & (X, T_j)_m \ar[r] & (X, T_i)_{m+1} \ar[r] & \cdots\\
			\cdots \ar[r] & (X_i, T_i)_m \ar[r] & (X, T_i)_m \ar[r] & (X_j, T_i)_m \ar[r] & (X_i, T_i)_{m+1} \ar[r] & \cdots\\
			\cdots \ar[r] & (X_i, T_j)_m \ar[r] & (X, T_j)_m \ar[r] & (X_j, T_j)_m \ar[r] & (X_i, T_j)_{m+1} \ar[r] & \cdots\\
			\end{tikzcd}
		\end{center}
		Using the fact that $j^*i_* = j^!i_! = 0$ from \cref{def:recollement}(\ref{recollement:vanishing_composition}) we deduce that
		\[\arraycolsep=2pt\def\arraystretch{1.33}
		\begin{array}{ccccccl}
			(X_i, T_j)_m &=& (i_*i^* X, j_*j^* T)_m &=& (j^*i_*i^*X, j^*T)_m &=& 0\\
			\span\span\span\text{and}&\\
			(X_j, T_i)_m &=& (j_!j^!X, i_!i^!T)_m &=& (j^!X, j^!i_!i^!T)_m &=& 0.
		\end{array}\]
		Combining this with the long exact sequences gives us that 
		$$(X_i, T_i)_m = (X, T_i)_m \text{ and } (X_j, T_j)_m = (X, T_j)_m.$$ 
		If we can show that $(X_i, T_i)_m$ and $(X_j, T_j)_m$ are bounded, then $(X, T_i)_m$ and $(X, T_j)_m$ would be bounded as well. Consequently we would have that $(X, T)_m$ is bounded. This would give us a bound on the projective dimension of $X$.
		
		We start by bounding $(X, T_i)_m = (X_i, T_i)_m$. First note that since  $i^*i_* \cong \operatorname{id}$ we have that
		\begin{align*}
			(X_i, T_i)_m = (i_*i^* X, i_!i^!T)_m = (i^*i_*i^* X, i^!T)_m = (i^* X, i^!T)_m
		\end{align*}
		Since $X$ has finite projective dimension we can think of it as a bounded complex of projectives. Then by \cref{lem:adjoint_preserves_bounded_proj/inj} $i^*X$ is as well. By the second half of \cref{lem:uniform_bound_on_homology} (using $(i^*, i_*)$ instead of $(i_*, i^!)$) we have that there is an $r$ such that $H^{-j}(i^*X)=0$ for all $j \geq r$. This means that thinking of $i^*X$ as a complex of projectives, it is 0 in degree $-t$ for all $t \geq r + \pd\ker d^{-r}_{i^*X}$, in particular it is 0 for all $t\geq r + \findim(\Lambda')$. Since $i^!T$ is a bounded complex, it has an upper bound, say $t_0$. Thus $(i^* X, i^!T)_m = 0$ for all $m \geq t_0 + r + \findim(\Lambda')$.
		
		The bound on $(X, T_j)_m$ is similar, using the finitistic dimension of $\Lambda''$. Taking the maximum of these two bounds we get a bound on $(X, T)_m$, which gives a bound on the projective dimension independent of $X$, hence a bound on $\findim(\Lambda)$. 
	\end{proof}
\end{theorem}

\subsection{Triangular matrix rings and vertex removal}
Something somethign triangular.

\begin{defn}[Comma category]
	Let $\mathcal A$ and $\mathcal B$ be categories and $F:\mathcal A \to \mathcal B$ a functor. Then the comma category $(F, \mathcal  B)$ has as objects triplets $(A, B, f)$ with $A \in \mathcal  A$, $B \in \mathcal  B$, and $f: FA \to B$ a morphism in $\mathcal  B$. The morphisms are pairs $(\alpha, \beta):(A, B, f) \to (A', B', f')$ with $\alpha: A \to A'$ and $\beta: B \to B'$ such that the following diagram commutes:
	\begin{center}
		\begin{tikzcd}
			FA \ar[r, "f"] \ar[d, swap, "F\alpha"] & B \ar[d, "\beta"]\\
			FA' \ar[r, "f'"] & B'.
		\end{tikzcd}
	\end{center}
\end{defn}

\begin{prop}\label{prop:comma-cat_abelian}
	If $\mathcal A$ and $\mathcal B$ are abelian categories and $F$ is right exact, then the comma category $(F, \mathcal  B)$ is abelian.
	\begin{proof}
		We need to show that $(F, \mathcal B)$ has kernels, has cokernels, and that image equals coimage. First we show kernels. Let $(\alpha, \beta):(A, B, f) \to (A', B', f')$ be a morphism in the comma category. Then we have a diagram:
		\begin{center}
		\begin{tikzcd}
			& F \ker \alpha \ar[r, "F\iota_\alpha"] \ar[d, dashed, "\theta"]& FA \ar[r, "F\alpha"] \ar[d, "f"] & FA' \ar[d, "f'"]\\
			0 \ar[r] & \ker \beta \ar[r, "\iota_\beta"] & B \ar[r, "\beta"] & B'
		\end{tikzcd}
		\end{center}
		Since $\beta f F\iota_\alpha = f' F\alpha F \iota_\alpha = 0$ there is a unique $\theta$ making the diagram commute. I claim the kernel of $(\alpha, \beta)$ is $(\ker \alpha, \ker \beta, \theta)$. Indeed if any $(\alpha', \beta')$ is any map such that $(\alpha, \beta) \circ (\alpha', \beta') = 0$ then $\alpha\alpha'=0$ and $\beta\beta'=0$ so both $\alpha'$ and $\beta'$ factor uniquely through $\iota_\alpha$ and $\iota_\beta$.
		\begin{center}
			\begin{tikzcd}
			FA'' \ar[d, "f''"] \ar[r, "\alpha''"] & F \ker \alpha \ar[r, "F\iota_\alpha"] \ar[d, dashed, "\theta"]& FA  \ar[d, "f"] \\
			B'' \ar[r, "\beta''"] & \ker \beta \ar[r, "\iota_\beta"] & B 
			\end{tikzcd}
		\end{center}
		The only thing left to verify is that the left square commutes. This follows from the outer rectangle commuting, and that $\iota_\beta$ is a monomorphism.
		
		Showing that cokernels exists is similar, but relies on $F$ being right exact. The construction is completely dual, but to verify commutativity at the end instead of using that $\iota_\beta$ is mono we must use that $F\pi_\alpha: FA' \to F\cok \alpha$ is an epimorphism. This follows from $F$ being right exact. I leave the details to the reader. \todo{or do I?}
		
		Now the image equaling the coimage follows from $\mathcal A$ and $\mathcal B$ being abelian, and the way we constructed the kernels and cokernels.
	\end{proof}
\end{prop}

For the rest of this section we will assume $F$ is a right exact functor between abelian catgeories so that the comma category is abelian. We will also assume $\mathcal A$ and $\mathcal B$ has enough projectives, whenever we mention projective objects. In particular we will be interested in the case when $\mathcal A$ and $\mathcal B$ are module categories over finite dimensional algebras.

\begin{defn}
	For $\mathcal A$ and $\mathcal B$ abelian catgeories and $F$ right exact we define the following functors:
		\begin{center}
		 \begin{tikzcd}[row sep=3pt]
			U:(F, \mathcal B) \ar[r]& \mathcal A \times \mathcal{B}\\
			(A, B, f) \ar[r, mapsto]& (A, B)\\
			(\alpha, \beta) \ar[r, mapsto]& (\alpha, \beta)
		\end{tikzcd}
		\begin{tikzcd}[row sep=3pt]
		T:\mathcal A \times \mathcal{B} \ar[r]& (F, \mathcal B)\\
		(A, B)  \ar[r, mapsto]& (A, B \oplus FA, FA \hookrightarrow FA \oplus B)\\
		(\alpha, \beta)  \ar[r, mapsto]& (\alpha, F\alpha \oplus \beta)
		\end{tikzcd}
		\end{center}
		
		\begin{center}
		\begin{tikzcd}[row sep=3pt]
		C:(F, \mathcal B)\ar[r]& \mathcal A \times \mathcal B\\
		(A, B, f)  \ar[r, mapsto]& (A, \cok f)\\
		(\alpha, \beta)  \ar[r, mapsto]& (\alpha, \hat{\beta})
		\end{tikzcd}
		\begin{tikzcd}[row sep=3pt]
		Z:\mathcal A \times \mathcal{B} \ar[r]& (F, B)\\
		(A, B)  \ar[r, mapsto]& (A, B, 0)\\
		(\alpha, \beta)  \ar[r, mapsto]& (\alpha, \beta)
		\end{tikzcd}
		\end{center}
\end{defn}

\begin{prop}
	With the definitions above $U$ and $Z$ become exact functors.
	\begin{proof}
		Using the characterization of exact sequences shown in \cref{prop:comma-cat_abelian} a short exact sequence in $(F, \mathcal B)$ is a commutative diagram
		\begin{center}
			\begin{tikzcd}
				0 \ar[r] & FA'' \ar[r, "F\alpha'"] \ar[d, "f''"] & FA \ar[r, "F\alpha"] \ar[d, "f"] & FA' \ar[r] \ar[d, "f'"] & 0\\
				0 \ar[r] & B'' \ar[r, "\beta'"] & B \ar[r, "\beta"] & B' \ar[r] & 0
			\end{tikzcd}
		\end{center}
		such that the short exact sequences 
		\begin{center}
		\begin{tikzcd}
		0 \ar[r] & A'' \ar[r, "\alpha'"] & A \ar[r, "\alpha"] & A' \ar[r] & 0\\
		0 \ar[r] & B'' \ar[r, "\beta'"] & B \ar[r, "\beta"] & B' \ar[r] & 0
		\end{tikzcd} 
		\end{center}
		are both exact. Since when we apply $U$ we simply get the product of these two sequences, $U$ is exact.
		
		Similarly for $Z$ since the two sequences we start with are assumed to be exact the resulting sequence will be exact by the characterization in \cref{prop:comma-cat_abelian}.
	\end{proof}
\end{prop}

\begin{prop}\cite[Proposition~1.3]{FGR75}
	The functors $(T, U)$ and $(C, Z)$ form adjoint pairs.
	\begin{proof}
		We want to establish an isomorphism between $\Hom(T(A, B), (A', B', FA' \to B'))$ and $\Hom((A, B), (A', B'))$. A morphism in $\Hom(T(A, B), (A', B', FA' \to B'))$  is given by a commutative diagram
		\begin{center}
		\begin{tikzcd}[ampersand replacement=\&]
			FA \ar{r}{\begin{bmatrix}
			0 \\ 1
			\end{bmatrix}} 
			\ar[d, swap, "F\alpha"]
		\& B \oplus FA \ar{d}{\begin{bmatrix}
			\beta & \gamma
			\end{bmatrix}} \\
			FA' \ar[r, "f"] \& B'
		\end{tikzcd}
		\end{center}
		The isomorphism is then given by sending this to $(\alpha, \beta)$. This is clearly surjective. For injectivity assume $(\alpha, \beta) = 0$, then $\gamma = \begin{bmatrix}
		\beta & \gamma
		\end{bmatrix}\begin{bmatrix}
		0 \\ 1
		\end{bmatrix} = fF\alpha= 0$, so the map is injective. So $(T, U)$ is an adjoint pair.
		
		Next we consider $(C, Z)$. We want an isomorphism between $\Hom(C(A, B, f), (A', B')) = \Hom((A, \cok f), (A', B'))$ and $\Hom((A, B, f), (A', B', 0))$. A morphism in $\Hom((A, B, f), (A', B', 0))$ is a commutative diagram
		\begin{center}
			\begin{tikzcd}
			FA \ar[r, "f"] \ar[d, swap, "F\alpha"] & B \ar[d, "\beta"]\\
			FA' \ar[r, "0"] & B'
			\end{tikzcd}
		\end{center}
		Since $\beta f = 0$, we have that $\beta$ factors through the cokernel of $f$ uniquely. Let the factorization be given by the map $\beta': \cok f \to B'$. Then we send this diagram to $(\alpha, \beta')$. Since the choice of $\beta'$ was unique this is an isomorphism, so $(C, Z)$ is an adjoint pair.
	\end{proof}
\end{prop}

\begin{cor}
	The functors $T$ and $C$ preserve projective objects.
	\begin{proof}
		What we need to check is that for projective objects $P$ and $Q$ in $(\mathcal A \times \mathcal B)$ and $(F, \mathcal B)$ respectively we have that $\Hom(TP, -)$ and $\Hom(CQ, -)$ are exact. By adjointness these are equal to $\Hom(P, U-)$ and $\Hom(Q, Z-)$ respectively. Since $U$ and $Z$ this holds, and so $T$ and $C$ preserve projective objects.
	\end{proof}
\end{cor}

\begin{prop}\cite[Corollary~1.6c]{FGR75}
	For a projective object $P$ in $(F, \mathcal B)$ we have that $T(C(P)) \cong P$, in particular all projectives are of the form $T(P')$ for a projective $P' \in \mathcal A \times \mathcal B$.
	\begin{proof}
		Let $P$ be given by $f:FA \to B$. Applying $C$ we get $(A, \cok f)$. We have morphisms $P \to ZC(P)$ and $TC(P) \to ZC(P)$ given by the following diagram
		\begin{center}
		\begin{tikzcd}
			FA \ar[d, equal] \ar[r, "f"] & B \ar[d, two heads]\\
			FA \ar[r, "0"] & \cok f\\
			FA \ar[u, equal] \ar[r, hookrightarrow]  & \cok f \oplus FA \ar[u, swap,  two heads]
		\end{tikzcd}
		\end{center}
		By the projective property of $P$ there is some morphism $\beta$ factorizing the map $P \to ZC(P)$ giving us the diagram:
		\begin{center}
			\begin{tikzcd}
			FA \ar[d, equal] \ar[r, "f"] & B \ar[d, "\beta"]\\
			FA \ar[d, equal] \ar[r, hookrightarrow]  & \cok f \oplus FA \ar[d,  two heads]\\
			FA \ar[r, "0"] & \cok f
			\end{tikzcd}
		\end{center}
		Since $FA \hookrightarrow \cok f \oplus FA$ is split mono $f$ is split mono, and consequently $\beta$ is an isomorphism. So we have $P \cong TC(P)$.
	\end{proof}
\end{prop}

\begin{prop}\cite[Lemma~4.16]{FGR75}\label{prop:pd_in_commacat}
	Let $X = (A, B, f)$ be an object in the commacategory. Then $\pd X \geq \pd A$, and if $A=0$ then $\pd X = \pd B$.
	\begin{proof}
		We first show that $\pd X \geq \pd A$. Note that $\pd C(X) = \max\{ \pd A, \pd \cok f \}$ so we always have $\pd C(X) \geq \pd A$. If $\pd X = \infty$ then the statement holds so let us assume $\pd X = n < \infty$. We proceed by induction. If $n=0$ then $C(X)$ is projective so $\pd X = \pd C(X) = \pd A = 0$. Next assume the statement holds for whenever the projective dimension is less than $n$. Let $P \to A$ and $P' \to \cok f$ be epimorphisms from projectives. Then we have an epimorphism $T(P, P') \to X$. If we let $\Omega A$ be the kernel of $P \to A$ and $X' = (\Omega A, K, \theta)$ be the kernel of $T(P, P') \to X$ as shown in the following diagram
		\begin{center}
		\begin{tikzcd}
			& F\Omega A \ar[r] \ar[d, "\theta", swap] & FP \ar[r] \ar[d, hookrightarrow] & FA \ar[r] \ar[d, "f"] & 0\\
			0 \ar[r] & K \ar[r] & P' \oplus FP \ar[r] & B \ar[r] & 0
		\end{tikzcd}
		\end{center}
		Then we have $\pd A \leq \pd \Omega A + 1$ and $\pd X = \pd X' + 1$. By induction we have that $\pd X' \geq \pd \Omega A$ and so $\pd X \geq \pd \Omega A +1 \geq \pd A$. 
		
		If $A=0$ then we can associate $C(X)=(0, B)$ with $B$. Any projective resolution $P_B^\bullet$ of $B$ gives a resolution of $X$ by $T(0, P_B^\bullet)$, and any resolution $P_X^\bullet$ of $X$ gives a resolution of $(0, B)$ by $C(P_X^\bullet)$. 
	\end{proof}
\end{prop}

\begin{theorem}\cite[Theorem~4.20]{FGR75}
	The finitistic dimension of the comma category $(F, \mathcal B)$ is bounded above by $\findim(\mathcal A) + \findim(\mathcal B) + 1$.
	\begin{proof}
		Let $X=(A, B, f)$ be an element of the commacategory with finite projective dimension. Let $P_A^\bullet$ be a projective resolution of $A$ shorter than $\findim(\mathcal A)$. Similar to what we did in \cref{prop:pd_in_commacat} define $P_X^0$ to be $T(P_A^0, P(\cok f))$ where $P(\cok f)$ is a projective module with an epimorphism onto $\cok f$. Then we have that the kernel of $P_X^0 \to X$ is 
		\begin{tikzcd}
			F\Omega A \ar[r, "\theta^0"] & K^0.
		\end{tikzcd}
		We continue inductively defining $P_X^n$ to be $T(P_A^n, \cok \theta^{n-1})$. Then $\Omega^{\findim(\mathcal A)+1} X = (0, K^{\findim(\mathcal A)}, 0)$. Then by \cref{prop:pd_in_commacat} we know that $\pd\Omega^{\findim(\mathcal A)+1}X = \pd K^{\findim(\mathcal A)} \leq \findim(\mathcal B)$. So $\pd X \leq \findim(\mathcal A) + \findim(\mathcal B) + 1$.
	\end{proof}
\end{theorem}

\begin{example}
	If $k$ is a field, $\mathcal A = \mathcal B = \mod k$ and $F$ is the identity, then the comma category $(F, \mathcal B)$ is equivalent to the category of finite dimensional representations of $A_2$ over $k$. Then $\mathcal A$ and $\mathcal B$ both have finitistic dimension 0 while $(F, \mathcal B)$ has finitistic dimension 1. So the bound shown above is tight. 
\end{example}

\begin{defn}[Triangular matrix ring]
	Let $R$ and $S$ be rings, and let $M$ be an $S-R$-bimodule. Then the triangular matrix ring $\begin{pmatrix}
	R & 0\\
	M & S
	\end{pmatrix}$ is the ring of all matricies $\begin{bmatrix}
	r & 0\\
	m & s
	\end{bmatrix}$ with $r\in R$, $s\in S$, and $m\in M$. The multplication is given by
	$$\begin{bmatrix}
	r & 0\\
	m & s
	\end{bmatrix}\begin{bmatrix}
	r' & 0\\
	m' & s'
	\end{bmatrix}=\begin{bmatrix}
	rr' & 0\\
	mr' + sm' & ss'
	\end{bmatrix}.$$
\end{defn}

Notice if $N$ is a module over the matrix ring $\begin{pmatrix}
R & 0\\
M & S
\end{pmatrix}$ then as an abelian group $N$ splits as a direct sum into
$$N=
\begin{bmatrix}
1 & 0\\
0 & 0
\end{bmatrix}N \oplus
\begin{bmatrix}
0 & 0\\
0 & 1
\end{bmatrix}N.$$

By restriction of scalars we can think of $N_R:=\begin{bmatrix}
	1 & 0\\
	0 & 0
\end{bmatrix}N$ as an $R$-module and $N_S:=\begin{bmatrix}
0 & 0\\
0 & 1
\end{bmatrix}N$ as an $S$-module. Further multiplication by $\begin{bmatrix}
0 & 0\\
m & 0
\end{bmatrix}$ is 0 on $N_S$ and maps $N_R$ into $N_S$. So $N$ consists of an $R$-module $N_R$, an $S$-module $N_S$ and a $S-R$-linear map $M \to \Hom_\mathbb{Z}(N_R, N_S)$, or equivalently a $S$-linear map $M \otimes_R N_R \to N_S$. This means that $\mod \begin{pmatrix}
R & 0\\
M & S
\end{pmatrix}$ is equivalent to the comma category $(\mod R, \mod S, M \otimes_R -)$. So we have that $$\findim\begin{pmatrix}
R & 0\\
M & S
\end{pmatrix} \leq \findim(R) + \findim(S)+1$$.

apply theorem to them

vertecies


\section{Contravariant finiteness}
\section{Contravariantly finite subcategories}\label{sec:contravariantly_finite}

%Results are generalized in \cite{Trl01}

In this section we will study the structure of contravariantly finite resolving subcategories. One example of a resolving subcategory is the subcategory of modules with finite projective dimension, which we denote by $\mathcal P^\infty$. In \cref{thm:contravariantly_finite_resolving_is_Xi_filtered} we give a description of the structure of a contravariantly finite resolving subcategory from the approximations of the simple modules. As a corollary we get that an algebra has finite finitistic dimension when $\mathcal P^\infty$ is contravariantly finite. \cref{exam:not_contravariantly_finite}, discovered by Igusa--Smalø--Todorov, shows that $\mathcal P^\infty$ can fail to be contravariantly finite even for monomial algebras with radical cubed equal to 0.

It is known that $\mathcal P^\infty$ is contravariantly finite when the algebra is stably equivalent to a hereditary algebra. This was shown by Auslander--Reiten in their original paper \cite{AR91}. We consider a generalization of this class in \cref{sec:stable_hereditary_algebras} through the perspective of the Igusa--Todorov-function.

Throughout this section we, as usual, assume $\Lambda$ is a finite dimensional algebra, though it should be noted that all the results still hold if we instead let $\Lambda$ be an artin algebra.

\begin{defn}[Resolving]
	A full subcategory of an abelian category is called \emph{resolving} if 
	\begin{enumerate}[i)]
		\item It is closed under extensions.
		\item It contains the projectives.
		\item It contains the kernel of any epimorphism between two of its objects.
	\end{enumerate}
\end{defn}

Note that $\mathcal P^\infty$ is a resolving subcategory.

In the next few propositions we will consider a resolving subcategory $\mathcal X$, and its $\Ext$-orthogonal complement
\begin{align*}
	\mathcal Y := \ker \Ext^{\geq 1}(\mathcal X, -) = \{Y \in \mathcal C \mid \Ext^i(X, Y)=0, \forall X \in \mathcal X, \forall i \geq 1\},
\end{align*}
which we now show is equal to 
\begin{align*}
\ker \Ext^1(\mathcal X, -) = \{Y \in \mathcal C \mid \Ext^1(X, Y)=0, \forall X \in \mathcal X\}.
\end{align*}

\begin{lemma}\label{lem:resolving_ext_vanish}
	Let $\mathcal X$ be a resolving subcategory. Then $\Ext^1(\mathcal X, Y) = 0$ implies that $\Ext^i(\mathcal X, Y)=0$ for all $i \geq 1$.
	\begin{proof}
		Since $\mathcal X$ contains the projectives, $\Omega X$ is the kernel of an epimorphism between objects in $\mathcal X$. Thus $\mathcal X$ contains all syzygies, and we have $\Ext^i(X, Y) = \Ext^1(\Omega^{i-1}X, Y) = 0$.
	\end{proof}
\end{lemma}

\begin{prop}\label{prop:complement_closed_under_extension}
	Let $\mathcal X$ be a full subcategory. Then the $\Ext$-orthogonal complement $\mathcal Y := \ker\Ext^{i}(\mathcal X, -)$ is closed under extensions.
	\begin{proof}
		Let $0 \to Y \to E \to Y' \to 0$ be an extension of objects in $\mathcal Y$, and let $X$ be an object of $\mathcal X$. Then we get an exact sequence  
		\begin{center}
			\begin{tikzcd}
			0=\Ext^i(X, Y) \ar[r] & \Ext^i(X, E) \ar[r] & \Ext^i(X, Y')=0
			\end{tikzcd}
		\end{center}
		Thus $\Ext^i(X, E)=0$, and so $E$ is in $\mathcal Y$.
	\end{proof}
\end{prop}

\begin{lemma} \label{lem:exact_sequence_from_approximation}
	Let $\mathcal X$ be a contravariantly finite, resolving subcategory of $\mod \Lambda$. Then for every object $C \in \mod\Lambda$ there is a short exact sequence 
	$$0 \to Y \to X \to C \to 0$$ 
	with $X\to C$ minimal $\mathcal X$-approximation and $\Ext^i(\mathcal X, Y)=0$ for all $i \geq 1$.
	\begin{proof}
		Since $\mathcal X$ is contravariantly finite, $C$ has a minimal $\mathcal X$-approximation $X \to C$. Since $\mathcal X$ contains the projective cover of $C$ this approximation must be an epimorphism. So it is part of a short exact sequence $$0 \to Y \to X \to C \to 0.$$ Let $X'$ be an arbitrary object in $\mathcal X$. Taking the long exact sequence in $\Ext(X', -)$ gives us
		\begin{center}
		\begin{tikzcd}
			\Hom(X', Y) \ar[r]&\Hom(X', X) \ar[r] & \Hom(X', C) \ar[dll, overlay, out=-15, in=165] \\ \Ext^1(X', Y) \ar[r] & \Ext(X', X)^1 \ar[r] & \Ext^1(X', C)
		\end{tikzcd}
		\end{center}
		Since $X \to C$ is an approximation, we know that $\Hom(X', X) \to \Hom(X', C)$ is epi. Thus if we can prove that $\Ext^1(X', X) \to \Ext^1(X', C)$ is mono we would have that $\Ext^1(X', Y)=0$. 
		
		Assume we have an element of $\Ext^1(X', X)$ that is mapped to 0, i.e. we have a commutative diagram
		\begin{center}
			\begin{tikzcd}
			0 \ar[r] & X \ar[r] \ar[d] & E \ar[r] \ar[d] & X' \ar[d, equal] \ar[r] & 0\\
			0 \ar[r] & C \ar[r] & C \oplus X' \ar[r] & X' \ar[r] & 0.
			\end{tikzcd}
		\end{center}
		Since $\mathcal X$ is closed under extensions $E$ is in $\mathcal X$. By composing with projection $C\oplus X' \to C$ we get a commutative triangle
		\begin{center}
			\begin{tikzcd}
			 X \ar[r] \ar[d] & E \ar[dl]\\ 
			 C 
			\end{tikzcd}
		\end{center}
		Since $X \to C$ is an approximation we get that $E \to C$ factors through $X$. The endomorphism $X \to E \to X$ leaves the approximation unchanged, so by minimality it must be an isomorphism. Hence 
		\begin{center}
			\begin{tikzcd}
				0 \ar[r] & X \ar[r] & E \ar[r] & X' \ar[r] & 0
			\end{tikzcd}
		\end{center}
		is split and $\Ext^1(X', X) \to \Ext^1(X', C)$ is injective. Thus we have that $\Ext^1(X', Y)=0$, and by \cref{lem:resolving_ext_vanish} we get $\Ext^i(X', Y)=0$ for all $i\geq 1$.
	\end{proof}
\end{lemma}

We now prove the main theorem of this section, about the structure of approximations for a resolving subcategory.

\begin{theorem} \cite[3.8]{AR91} \label{thm:contravariantly_finite_resolving_is_Xi_filtered}
	Let $\mathcal X$ be a contravariantly finite, resolving subcategory of $\mod \Lambda$. Let $X_i$ be the minimal approximation of $S_i$. Then any $X \in \mathcal X$ is a direct summand of an $X_i$-filtered module.
	\begin{proof}
		The first part of the proof is to show by induction on length that any module $C$ is in an exact sequence $0 \to Y \to X \to C \to 0$ with $X$ $X_i$-filtered and $\Ext^1(\mathcal X, Y)=0$.
		
		For the base case if $C=S_i$ is simple, then by \cref{lem:exact_sequence_from_approximation} we have an exact sequence $0 \to Y \to X_i \to C \to 0$ with the desired properties stated above. 
		
		For the induction step, assume it holds for all modules of length less than $n$, and let $C$ be a module of length $n$. Then by Jordan-Hölder $C$ is the extension of two modules of length less than $n$. Say
		\begin{center}
			\begin{tikzcd}
			0 \ar[r] & C' \ar[r] & C \ar[r] & C'' \ar[r] & 0.
			\end{tikzcd}
		\end{center}
		Applying the induction hypothesis we get a diagram on the form
		\begin{center}
			\begin{tikzcd}
			& 0 \ar[d] &&0 \ar[d] &\\
			& Y' \ar[d] && Y'' \ar[d] & \\
			 & X' \ar[d] && X'' \ar[d] & \\
			0 \ar[r] & C' \ar[d] \ar[r] & C \ar[r] & C'' \ar[r] \ar[d] & 0\\
			& 0 &&0 &
			\end{tikzcd}
		\end{center}
		Taking the pullback of $X'' \to C''$ we get a diagram
		\begin{center}
			\begin{tikzcd}
			0 \ar[r] & C' \ar[r] \ar[d, equal] & E \ar[r] \ar[d] & X'' \ar[r] \ar[d] & 0\\
			0 \ar[r] & C' \ar[r] \ar[d] & C \ar[r] \ar[d] & C'' \ar[r] \ar[d] & 0\\
			&0&0&0&
			\end{tikzcd}
		\end{center}
		Since $Y'$ satisfies $\Ext^1(\mathcal X, Y') = 0$ by \cref{lem:resolving_ext_vanish} we have $\Ext^2(\mathcal X, Y')=0$. In particular from the long exact sequence
		\begin{center}
			\begin{tikzcd}[column sep=10pt]
				 0=\Ext^1(X'', Y) \ar[r] & \Ext^1(X'', X') \ar[r] & \Ext^1(X'', C) \ar[r] & \Ext^2(X'', Y)=0
			\end{tikzcd}
		\end{center}
		 we get that $X' \to C'$ induces an isomorphism $\Ext^1(X'', X') \to \Ext^1(X'', C)$. Thus the short exact sequence $0 \to C' \to E \to X'' \to 0$  must come from a sequence $0 \to X' \to X \to X'' \to 0$. This gives us a diagram
		\begin{center}
			\begin{tikzcd}
			& 0 \ar[d] &&0 \ar[d] &\\
			 & Y' \ar[d] && Y'' \ar[d] & \\
			0 \ar[r] & X' \ar[d] \ar[r] & X \ar[r] \ar[d] & X'' \ar[d] \ar[r] & 0\\
			0 \ar[r] & C' \ar[d] \ar[r] & C \ar[r] & C'' \ar[r] \ar[d] & 0\\
			& 0 &&0 &
			\end{tikzcd}
		\end{center}
		Applying the Snake Lemma we can fill out the diagram:
		\begin{center}
			\begin{tikzcd}
			& 0 \ar[d] & 0\ar[d] &0 \ar[d] &\\
			0 \ar[r] & Y' \ar[d] \ar[r] & Y \ar[r] \ar[d] & Y'' \ar[d] \ar[r] & 0\\
			0 \ar[r] & X' \ar[d] \ar[r] & X \ar[r] \ar[d] & X'' \ar[d] \ar[r] & 0\\
			0 \ar[r] & C' \ar[d] \ar[r] & C \ar[r] \ar[d] & C'' \ar[r] \ar[d] & 0\\
			& 0 &0&0 &
			\end{tikzcd}
		\end{center}
		Since $X$ is an extension of $X_i$-filtered modules, it is also $X_i$-filtered. Since $Y$ is the extension of $Y''$ and $Y'$ it follows from \cref{prop:complement_closed_under_extension} that $\Ext^1(\mathcal X, Y)=0$.
		
		Hence any $C$ fits into a sequence $0 \to Y \to X \to C \to 0$ with $X$ being $X_i$-filtered and $\Ext^{1}(\mathcal X, Y)=0$.
		
		Now suppose that $C$ is in $\mathcal X$, and let $0 \to Y \to X \to C \to 0$ be as before. Then we get that
		\begin{center}
		\begin{tikzcd}
		\Hom(C, X) \ar[r] & \Hom(C, C) \ar[r] & \Ext^1(C,Y) = 0
		\end{tikzcd}
		\end{center}
		is exact, and thus $C$ is a direct summand of $X$. So every object in $\mathcal X$ is a direct summand of an $X_i$-filtered module.
	\end{proof}
\end{theorem}

Applying this to $\mathcal P^\infty$ we get our wanted result about the finitistic dimension.

\begin{cor}\label{cor:contravariant_finite_implies_FDC}
	If $\mathcal P^\infty$ is contravariantly finite, then the finitistic dimension is the supremum of the projective dimension of the approximations of the simple modules. In particular it is finite.
\end{cor}

To finish this section of we give two examples. The first example is due to Igusa--Smalø--Todorov, which shows that $\mathcal P^\infty$ need not be contravariantly finite even for monomial algebras with $J^3 = 0$.

\begin{example}\cite[Proposition~2.3]{IST90}\label{exam:not_contravariantly_finite}
	Let $\Lambda$ be the path algebra of 
	\begin{center}
	\begin{tikzcd}[column sep = 50pt]
		1 \ar[r, "\alpha", bend left=45] \ar[r, "\beta"] & 2 \ar[l, "\gamma", bend left = 45]
	\end{tikzcd}
	\end{center}
	with relations $\alpha \gamma$, $\beta\gamma$, and $\gamma\alpha$ over an algebraically closed field $k$. Then $\findim(\Lambda) = 1$, but $\mathcal P^\infty$ is not contravariantly finite.
	
	\begin{proof}
		The indecomposable projective $\Lambda$-modules are given by the following quivers
		\begin{center}
			\begin{tikzcd}[column sep=7pt]
				&1 \ar[dl, swap, "\alpha"] \ar[dr, "\beta"]&\\
				2&&2 \ar[d, "\gamma"]\\
				&&1
			\end{tikzcd}
			\hspace{2cm}
			\begin{tikzcd}
				2\ar[d, "\gamma"]\\
				1
			\end{tikzcd}
		\end{center}
		Note that both the indecomposable projectives have even dimension, so any projective module has even dimension. Then if $X$ is a module with finite projective dimension, since $\dim X = \sum (-1)^i \dim P_X^i$ the dimension of $X$ is also even. In particular the two simple modules have infinite projective dimension.
		
		The radical of $P_1$ is $P_2\oplus S_2$ and the radical of $P_2$ is $S_1$, so the radical of an arbitrary projective looks like $P_2^n \oplus S_1^m \oplus S_2^n$. Let $P \to X$ be the projective cover of a module with finite projective dimension. Then $\Omega X$ is a submodule of $JP = P_2^n \oplus S_1^m \oplus S_2^n$. Let $M$ be an indecomposable summand of $\Omega X$, and consider the composition $M \to JP \to P_2$ for any possible projection to $P_2$. If this is epi then we must have $M = P_2$. If none of these are epi then $M$ is contained in $JP_2^n \oplus S_1^m \oplus S_2^n = S_1^{m+n} \oplus S_2^n$. This would mean $M=S_1$ or $M=S_2$, but $S_1$ and $S_2$ both have infinite projective dimension. Thus we must have $\Omega X$ projective, and so $\pd X \leq 1$.
		
		Next we want to show that $S_1$ has no minimal approximation by modules with finite projective dimension. Assume for the sake of contradiction that $X \to S_1$ is such a minimal approximation. Then we claim that $P_2$ is not a submodule of $X$. If $X$ had $P_2$ as a submodule, then since $\Hom(P_2, S_1) = 0$ the approximation would factor through $X'=X/P_2$. From the short exact sequence $0 \to P_2 \to X \to X' \to 0$ it follows that 
		$$\pd X' \leq \max\{\pd P_2 + 1, \pd X\} < \infty,$$ 
		and so $X'$ would give an approximation of shorter length, contradicting the minimality of $X$.
		
		% because $X'$ would also have finite projective dimension. Which can be seen in the diagram below.
		% \begin{center}
		% \begin{tikzcd}
		% 	&& 0 \ar[d] & 0 \ar[d] &\\
		% 	0\ar[r]& P^1_X \ar[d, equal]\ar[r] & P^1_X \oplus P_1 \arrow[dr, phantom, "\usebox\pullback" , very near start, color=black] \ar[d]\ar[r] & P_1\ar[d]\ar[r] & 0\\
		% 	0 \ar[r] & P^1_X \ar[r] & P^0_X \ar[d]\ar[r] & X\ar[d]\ar[r] & 0\\
		% 	&& X' \ar[d]\ar[r, equal] & X' \ar[d] \\
		% 	&&0&0
		% \end{tikzcd}
		% \end{center}

		This means that $\gamma X = 0$, because if there was an element $x \in X$ with $\gamma x \neq 0$, then $(e_2 x)$ would be a submodule of $X$ isomorphic to $P_2$. So $X$ is a $\Lambda/(\gamma)$ module. 
		
		The algebra $\Lambda/(\gamma)$ is the path algebra of the 2-Kronecker quiver, whose representation theory is well understood (c.f. \cite[Chapter~VIII.7]{ARS97} or \cite[Chapter~3.2]{Ring84}). Specifically $\Lambda/(\gamma)$ can be associated with the subquiver highlighted below. 
		\begin{center}
			\begin{tikzcd}[column sep = 50pt]
			1 \ar[r, "\alpha", bend left=45] \ar[r, "\beta"] & 2 \ar[l, opacity=0.3, "\gamma", bend left = 45]
			\end{tikzcd}
		\end{center}
		The indecomposable modules are as given in the table below.
		
		\begin{center}
		\begin{tabular}{ccc}
			\begin{tikzcd}[ampersand replacement=\&, column sep = 45pt]
			k^n \ar[bend left=35, r, "\begin{bmatrix}
			I_n\\ 0
			\end{bmatrix}"pos=0.55] \ar[swap, r, "\begin{bmatrix}
			0\\I_n
			\end{bmatrix}"]\& k^{n+1}
			\end{tikzcd}
			&
			\begin{tikzcd}[ampersand replacement=\&, column sep = 45pt, row sep=40pt]
			k^n \ar[bend left=35]{r}{J(n, \lambda)} \ar[swap, bend right=0]{r}{I_n} \& k^{n}\\
			k^n \ar[bend left=35]{r}{I_n} \ar[swap, bend right=0]{r}{J(n, 0)} \& k^{n}\\
			\end{tikzcd}
			&
			\begin{tikzcd}[ampersand replacement=\&, column sep = 45pt]
			k^{n+1} \ar[bend left=35, pos=0.45]{r}{\begin{bmatrix}
				I_n & 0
				\end{bmatrix}} \ar[swap, bend right=0]{r}{\begin{bmatrix}
				0 & I_n
				\end{bmatrix}} \& k^{n}
			\end{tikzcd}
			\\
			preprojective & regular & preinjective
		\end{tabular}
		\end{center}
		
	We see that the preprojective and preinjective modules both have odd dimension, so they will have infinite projective dimension as $\Lambda$-modules. We can easily verify that the $\Lambda/(\gamma)$-modules 
	\begin{tikzcd}
	k \ar[bend left=25]{r}{\lambda} \ar[swap, bend right=0]{r}{1} & k\\
	\end{tikzcd}
	all have finite projective dimension as $\Lambda$-modules and that they have a nonzero map onto $S_1$. So each of these modules would need to have a nonzero map to $X$. But it is easy to verify that there is a nonzero homomorphism between the regular modules only if they have the same value of $\lambda$. So for it to be possible for $X$ to factorize all these maps we would need $X$ to have infinitely many direct summands. Since we are working with finitely generated modules this is impossible, hence $S_1$ has no approximation, and the subcategory is not contravariantly finite.
	\end{proof}
\end{example}

In the next example we look at the opposite algebra of $\Lambda$, for which $\mathcal P^\infty$ is contravariantly finite for $\Gamma$. This shows that there is no immediate relationship between $\mathcal P^\infty$ being contravariantly finite for $\Lambda$ and for $\Lambda^{\op}$.

\begin{example}\label{exam:contravariantly_finite_dual}
	Let $\Gamma$ be the opposite algebra of the one in \cref{exam:not_contravariantly_finite}. That is, $\Gamma$ is the path algebra of 
	\begin{center}
		\begin{tikzcd}[column sep = 50pt]
		2 \ar[r, "\hat{\alpha}", bend left=45] \ar[r, "\hat{\beta}"] & 1 \ar[l, "\hat{\gamma}", bend left = 45]
		\end{tikzcd}
	\end{center}
	with relations $\hat{\gamma}\hat{\alpha}$, $\hat{\gamma}\hat{\beta}$, and $\hat{\alpha}\hat{\gamma}$. Then $\mathcal P^\infty$ is contravariantly finite. In other words the subcategory of $\Lambda$-modules with finite injective dimension is covariantly finite.
	\begin{proof}
		The indecomposable projective $\Gamma$-modules are given by the following quivers 
		\begin{center}
			\begin{tikzcd}
				1 \ar[d, "\hat{\gamma}"]\\2 \ar[d, "\hat{\beta}"]\\1
			\end{tikzcd}
			\hspace{2cm}
			\begin{tikzcd}[column sep=7pt]
				&2 \ar[dl, swap, "\hat{\alpha}"] \ar[dr, "\hat{\beta}"]&\\
				1&&1
			\end{tikzcd}
		\end{center}
		
		Similar to before, notice that the indecomposable projective modules are 3-dimensional and thus every module with finite projective dimension will have dimension a multiple of 3. So in particular the simple modules have infinite projective dimension. 
		
		Let $X$ be a module with finite projective dimension, and let $P$ be its projective cover. We have that $\Omega X$ is a submodule of $JP$. Notice that $\hat{\alpha} J = \hat{\gamma} J = 0$, so $\Omega X$ is a $\Gamma/(\hat{\alpha}, \hat{\gamma})$-module. But $\Gamma/(\hat{\alpha}, \hat{\gamma})$ is simply isomorphic to the path algebra of  
		\begin{tikzcd}
			2 \ar[r] & 1
		\end{tikzcd},
		over which there are just 3 indecomposable modules. We already know that the simple modules cannot be summands of $\Omega X$, because they have infinite projective dimension. The non-simple module
		\begin{tikzcd}
		k \ar[r, "1"] & k
		\end{tikzcd}
		is 2-dimensional and thus also has infinite projective dimension over $\Gamma$. So we conclude that $\Omega X = 0$, so $X$ is projective.
		
		So the only modules with finite projective dimension are the projectives themselves. In particular there are only a finite number of indecomposable modules with finite projective dimension. So the subcategory is contravariantly finite. 
	\end{proof}
\end{example}


\section{Repdimension}

Many results based on the survey \cite{Opp09}.

\begin{defn}[Dominant dimension]
	Let 
	\begin{center}
		\begin{tikzcd}
			\Lambda \ar[r] & I^0 \ar[r] & I^1 \ar[r] & \cdots
		\end{tikzcd}
	\end{center}
	be the minimal injective resolution of $\Lambda$. Then the \emph{dominant dimension} of $\Lambda$, denoted $\domdim(\Lambda)$, is the infimum over $n$ such that $I^n$ is not projective. If all $I^n$ are projective we say the dominant dimension is $\infty$.
\end{defn}

\begin{defn}[Representation dimension]
	Let $A$ be defined by $$A = \{\Gamma | \domdim(\Gamma) \geq 2, \Lambda \text{ Morita equivalent to } \End_\Gamma(I_0(\Gamma))^{\operatorname{op}}\}$$ where $I_0(\Gamma)$ is the injective envelope of $\Gamma$. Then the \emph{representation dimension} of $\Lambda$, denoted $\repdim(\Lambda)$, is the infimum of the global dimensions of $\Gamma \in A$.
\end{defn}

\begin{prop}\label{prop:repdim_auslander_generator} \todo{This is the only thing I use, maybe I should ditch the rest}
	For a finite dimensional algebra $\Lambda$, the representation dimension of $\Lambda$ is the same as minimal global dimension of $\End(M)^{\operatorname{op}}$ for $M \in \mod\Lambda$ being both a generator and cogenerator. 
	\begin{proof}
		Consider $\Gamma \in A$. Since $\domdim(\Gamma) \geq 1$, $I_0(\Gamma)$ is the sum of all projective-injective modules (some probably several times). 
		
		Let $\mathcal S$ be the set of all $\Gamma$-modules with a copresentation
		\begin{center}
		\begin{tikzcd}
			0 \ar[r] & X \ar[r] & I_0 \ar[r] & I_1
		\end{tikzcd}
		\end{center}
		with $I_i$ in $\add I_0(\Gamma)$. In particular $\Gamma$ is in $\mathcal S$, because $domdim\Gamma \geq 2$.
		
		The Yoneda embedding gives a duality 
		$$\Hom_\Gamma(-,I_0(\Gamma))\colon\add I_0(\Gamma) \to \proj\End_\Gamma(I_0(\Gamma)),$$ 
		and thus we get an equivalence 
		$$D\Hom_\Gamma(-,I_0(\Gamma))\colon\add I_0(\Gamma) \to \inj\End_\Gamma(I_0(\Gamma))^{\operatorname{op}}$$
		Since $I_0(\Gamma)$ is injective $D\Hom(-,I_0(\Gamma))$ is exact and preserves kernels, so extends to an equivalence
		$$D\Hom_\Gamma(-,I_0(\Gamma))\colon\mathcal S \to \mod \End_\Gamma(I_0(\Gamma))^{\operatorname{op}}$$
		
		Since $\End_\Gamma(I_0(\Gamma))^{\operatorname{op}}$ is Morita equivalent to $\Lambda$, $\mathcal S$ is equivalent to $\mod \Lambda$. The module $\Gamma \in \mathcal S$ is clearly a generator. To see that it is a cogenerator note that $\Gamma$ contains all the projective-injective indecomposable objects as direct summands, so there is an injection $I_0(\Gamma) \to \Gamma^n$, and since $I_0(\Gamma)$ is a cogenerator in $\mathcal S$, $\Gamma$ is as well.
		
		Thus by the equivalence $\mathcal S \to \mod\Lambda$ there is a generator-cogenerator module $M$ such that $\End_\Lambda(M)^{\operatorname{op}} = \End_\Gamma(\Gamma)^{\operatorname{op}}=\Gamma$.
		
		The last step of the proof is showing that $\End(M)^{\operatorname{op}}$ is in $A$ whenever $M$ is a generator-cogenerator.
		
		Let $0 \to M \to I_0(M) \to I_1(M)$ be a minimal injective copresentation of $M$. Since $M$ is a cogenerator $I_i(M)$ is in $\add M$, thus we get an exact sequence of projective $\End(M)^{\operatorname{op}}$-modules
		\begin{align} \label{eq:injective_copres_of_endM}
		0 \to \End(M) \to \Hom(M, I_0(M)) \to \Hom(M, I_1(M)).
		\end{align}
		Now we have the following isomorphisms of $\End(M)^{\op}-\Lambda$-bimodules
		\begin{align*}
		\Hom_\Lambda(M, D\Lambda) &=
		\Hom_k(\Lambda\otimes M, k) \\&=
		\Hom_k(M, k) \\&=
		DM \\&=
		D\Hom_\Lambda(\Lambda, M)
		\end{align*}
		Since $\Lambda$ is in $\add M$, $\Hom(\Lambda, M)$ is projective as a $\End(M)$-module, and thus $D\Hom(\Lambda, M) = \Hom(M, D\Lambda)$ is injective as a $\End(M)^{\op}$-module. This means that (\ref{eq:injective_copres_of_endM}) is an injective copresentation, and thus $\domdim(\End(M)^{\op}) \geq 2$.
		
		Since $\Hom(M, I_0(M))$ is the beginning of an injective resolution of $\End(M)$, $I_0(\End(M))$, must be a direct summand. Then $\Hom(M, I_0(M)) / I_0(\End(M))$ would map injectively into $\Hom(M, I_1(M))$, but that would mean there is a direct summand of $I_0(M)$ mapping injectively into $I_1(M)$, contradicting minimality. Thus $\Hom(M, I_0(M)) = I_0(\End(M)^{\op})$.
		
		Let $I=I_0(M)$ and $\Gamma = \End_\Lambda(I)^{\op}$, then $D\Hom(-,I)$ is an exact equivalence from $\add I$ to $\inj\Gamma$. Since $I$ is an injective cogenerator $\add I = \inj\Lambda$. Then because $D\Hom(-,I)$ is exact it extends to an equivalence $\mod\Lambda \to \mod\Gamma$. So $\Lambda$ is Morita equivalent to $\Gamma = \End_{\Lambda}(I_0(M))^{\op} = \End_{R}(I_0(R))^{\op}$, where $R=\End(M)^{\op}$. Thus $\End(M)^{\op}$ is in $A$.
	\end{proof}
\end{prop}

\begin{defn}
	Let $X$ be an object of $\mod\Lambda$ and $\mathcal M$ a contravariantaly finite subcategory.
	\begin{center}
	\begin{tikzcd}
		\cdots \ar[rd, two heads] \ar[r] & M_2 \ar[rd, two heads] \ar[r] & M_1 \ar[rd, two heads] \ar[r] & M_0 \ar[rd, two heads]\\
		&\Omega_M^3 X \ar[u, hook] & \Omega_M^2  \ar[u, hook] X & \Omega_M X  \ar[u, hook] & X
	\end{tikzcd}
	\end{center}
	If the maps $M_n \twoheadrightarrow \Omega_M^nX$ are minimal right $\mathcal M$-approximations for $n\geq 0$ (they need not be surjective), and $\Omega_M^{n+1} \hookrightarrow M_n$ are their kernels, then this is a minimal \emph{$M$-resolution} of $X$. The \emph{$\mathcal M$-res-dimension} of $X$ is the length of this sequence of (nonzero) $M_i$'s, and the $\mathcal M$-res-dimension of $\Lambda$ is the supremum of the dimension on its objects.

\end{defn}

\begin{prop}
	Repdim-2 is the minimum of $M$-res-dim$(\mod \Lambda)$ for $M$ both generator and cogenrator (assuming repdim is at least 2).
	
	\begin{proof}
		The functor $\Hom(M, -)$ is an equivalence from $\add M$ to $\proj\End(M)$, which maps minimal $M$-approximations to projective covers. Let $X$ be any module in $\mod\End(M)$ with projective dimension at least 2. Then it has a projective presentation $$\Omega^2X \to (M,M_1) \to (M,M_0) \to X.$$
		Because of the equivalence this is induced by a map $f\colon M_1\to M_0$. Since $\Hom$ is left exact we have that $\Omega^2X \cong \Hom(M, \ker f)$, and so the projective dimension of $X$ is $2$ plus the $M$-res-dimension of $\ker f$.
		
		Since $M$ is a cogenerator any module $Y$ in $\mod\Lambda$ has a copresentation 
		\begin{center}
		\begin{tikzcd}
			0 \ar[r] & Y \ar[r] & M_0 \ar[r, "f"] & M_1.
		\end{tikzcd}
		\end{center}
		Applying $\Hom(M,-) =: (M,-)$ we get
		\begin{center}
		\begin{tikzcd}
		0 \ar[r] & (M,Y) \ar[r] & (M,M_0) \ar[r]{}{(M,f)} & (M,M_1) \ar[r] & \cok(M,f) \ar[r] & 0.
		\end{tikzcd}
		\end{center}
		If the projective dimension of $\cok(M,f)$ is less than 2, then $(M, Y)$ is a direct summand of $(M, M_0)$. This means that $(M,Y) \cong (M, M')$, so the minimal $M$-approximation of $Y$ is $M'$, and $(M, \Omega_M Y) = 0$. Since $M$ is a generator this means $\Omega_M Y = 0$ and thus the $M$-res-dimension of $Y$ is 0.
		
		So provided the projective dimension of $\cok(M,f)$ is larger than or equal to 2, it equals the $M$-res-dimension of $Y$ plus 2. In particular the global dimension of $\End(M)$ is 2 plus the $M$-res-dimension of $\mod\Lambda$, provided it is at least 2.
	\end{proof}
\end{prop}

\begin{prop}
	The repdimension of an artin algebra is always finite. \cite{Iya02}
\end{prop}

\begin{theorem}
	The repdimension of $\Lambda$ is less than or equal to 2 if and only if $\Lambda$ is representation finite.
	\begin{proof}
		Assume $\Lambda$ is representation finite and let $M$ be the direct sum of all indecomposable modules (up to iso). Then $M$ is a generator-cogenerator. Let $X$ be an $\End(M)^{\operatorname{op}}$-module with projective presentation $$(M,M_1) \to (M, M_0) \to X \to 0.$$ Let $M_2$ be the kernel of $M_1 \to M_0$. Since $M$ is the sum of all indecomposables $M_2$ is in $\add M$, so $$0 \to (M, M_2) \to (M,M_1) \to (M, M_0) \to X \to 0$$ is a projective resolution of $X$. So $\Lambda$ has repdimension at most 2.
		
		Assume $\Lambda$ has repdimension at most 2, and let $M$ be an auslander generator. We want to show that $\add M = \mod\Lambda$. Let $X$ be any $\Lambda$-module, and let $$0 \to X \to I_0 \to I_1$$ be a minimal injective presentation. If $I_0 \to I_1$ is split then $X$ is injective and thus in $\add M$. Let $M_X$ be a minimal $M$-approximation of $X$, let $\Omega_M X$ be the kernel of the approximation, and let $Y$ be the cokernel of $(M, I_0) \to (M, I_1)$. Then $$(M,\Omega_M X) \to (M,M_X) \to (M, I_0) \to (M, I_1) \to Y \to 0$$ is a minimal exact sequence. Since the global dimension of $\End(M)^{\operatorname{op}}$ is at most 2 this means that $(M, \Omega_M X)=0$. Consequently we have that $\Omega_M X = 0$ and that $X=M_X$, so $X$ is in $\add M$. Thus $\Lambda$ is representation finite.
	\end{proof}
\end{theorem}


\subsection{The Igusa-Todorov function} \label{sec:Igusa-Todorov}
In this section we let $K_0$ be the abelian group generated by isomorphism classes of modules in $\mod\Lambda$, with the relations that $[A\oplus B] - [A] - [B] = 0$ for any modules $A$ and $B$, and $[P]=0$ when $P$ is projective. We define the linear map $L\colon K\to K$ by $L[A] = [\Omega A]$. For any module $X$, we let $[\add X]$ be the finitely generated subgroup of $K_0$ generated by modules in $\add X$. Fitting's lemma\todo{maybe make appendix} tells us that there is an integer $\eta_X$ such that $L\colon L^m[\add X] \to L^{m+1}[\add X]$ is an isomorphism for every $m \geq \eta_X$. We use this to define two important functions from $\mod \Lambda$ to $\mathbb N$.

\begin{defn}[The Igusa--Todorov functions]
	We define two functions $\phi$ and $\psi$ from $\mod\Lambda$ to $\mathbb N$. For a module $M \in \mod\Lambda$ we define $\phi(M)$ to be the integer $\eta_M$ coming from Fitting's lemma, as explained above. In other words, $\phi(M)$ is the smallest integer such that $$L\colon L^m[\add M] \to L^{m+1}[\add M]$$ is an isomorphism for every $m \geq \phi(M)$. We define $\psi(M)$ in a similar way, but adding on an extra term to account for the structure of $\Omega^{\phi(M)}M$. 
	$$\psi(M) = \phi(M) + \sup\left\lbrace\pd Z \; \middle| \; \pd Z < \infty, Z \in \add \Omega^{\phi(M)}M\right\rbrace$$
\end{defn}

\begin{lemma} \cite[Lemma~3]{IgTo05} \label{lem:properties_of_psi}
	\begin{enumerate}[i)]
		\item $\psi(M) = \pd M$, when $\pd M < \infty$.
		\item $\psi(M^k) = \psi(M)$.
		\item $\psi(M) \leq \psi(M\oplus N)$.
		\item If $Z$ is a direct summand of $\Omega^n(M)$ where $n \leq \phi(M)$ and $\pd Z < \infty$, then $\pd Z + n \leq \psi(M)$.
	\end{enumerate}
	\begin{proof}
		\begin{enumerate}[i)]
			\item[] %empty line
			\item If $\pd M < \infty$, then $L^m \neq 0$ for $m < \pd M$, and $L^m =0$ for $m \geq \pd M$. So $\psi(M)=\phi(M)=\pd M$.
			\item The subcategory $\add M^k = \add M$, and $\psi$ is defined only in terms of the additive subcategory $\add M$.
			\item  The subcategory $\add M$ is contained in $\add M\oplus N$, so if $L$ is injective when restricted to $L^m(\add M\oplus N)$ then $L$ is injective when restricted to $L^m(\add M)$. Thus we have $\phi(M) \leq \phi({M\oplus N})$. Further $$\Omega^{\phi({M\oplus N})-\phi(M)}\left(\add\Omega^{\phi(M)}M \right) \subseteq \add\Omega^{\phi({M\oplus N})} M\oplus N,$$ 
			so $\psi(M) \leq \psi(M\oplus N)$.
			\item Let $p=\pd Z$ and $k = \phi(M) - n$. Then $\Omega^k Z$ is in $\add \Omega^{\phi(M)}M$, so $\pd\Omega^k Z + \phi(M) \leq \psi(M)$. Thus $$\pd Z + n = p + n = (p-k) + \phi(M) \leq \pd\Omega^k Z + \phi(M) \leq \psi(M).$$
		\end{enumerate}
	\end{proof}
\end{lemma}

\begin{theorem}\cite[Theorem~4]{IgTo05} \label{thm:projdim_bounded_by_psi}
	Let $0 \to A \to B \to C \to 0$ be a short exact sequence of modules with $\pd C < \infty$. Then $\pd C \leq \psi(A\oplus B)+1$.
	\begin{proof}
		Let $P_A^\bullet$ and $P_C^\bullet$ be the minimal projective resolutions of $A$ and $C$. Then we get a map of short exact sequences
		\begin{center}
		\begin{tikzcd}
			0 \ar[r]  & P_A^0 \ar[r] \ar[d] & P_A^0 \oplus P_C^0 \ar[r] \ar[d] & P_C^0 \ar[r] \ar[d] & 0\\
			0 \ar[r] & A \ar[r] & B \ar[r] & C \ar[r] & 0 
		\end{tikzcd}
		\end{center}
		Applying the Snake Lemma we get $0 \to \Omega A \to \Omega B \oplus P \to \Omega C \to 0$ for some projective module $P$. Thus for some $n \leq \pd C$ we have $L^n[A] = L^n[B]$, and let $n$ be the minimal such number. Clearly $n \leq \phi(A\oplus B
			)$. Let $X = \Omega^n A = \Omega^n B$, then our sequence of $n$-syzygies looks like
		\begin{center}
			\begin{tikzcd}
			0 \ar[r] & X \ar[r] & X\oplus P \ar[r] & \Omega^nC \ar[r] & 0.
			\end{tikzcd}
		\end{center}
		Let $f$ be the composition
		\begin{tikzcd}
			X \ar[r] & X \oplus P \ar[r, "\pi_X"] & X.
		\end{tikzcd}
		Then by Fitting's lemma $X$ breaks as a direct sum into two components $X = Z \oplus Y$ such that $f = f_Z \oplus f_Y$ with $f_Y$ an isomorphism and $f_Z$ nilpotent. In other words the sequence above can be written as
		\begin{center}
			\begin{tikzcd}
			0 \ar[r] & Z\oplus Y \ar[r] & Z \oplus Y\oplus P \ar[r] & \Omega^nC \ar[r] & 0.
			\end{tikzcd}
		\end{center}
		with the left map being
		$$\begin{bmatrix}
			f_Z & 0\\
			0 & f_Y\\
			* & *
		\end{bmatrix} \sim
		\begin{bmatrix}
		f_Z & 0\\
		0 & 1_Y\\
		* & 0
		\end{bmatrix} $$
		So by changing basis this restricts to another short exact sequence
		\begin{center}
			\begin{tikzcd}
			0 \ar[r] & Z \ar[r] & Z \oplus P \ar[r] & \Omega^nC \ar[r] & 0.
			\end{tikzcd}
		\end{center}
		Let $T = \Lambda/J$ and apply the long exact sequence in $\Ext(-, T)$. Then we get an exact sequence
		\begin{center}
			\begin{tikzcd}
			\Ext^k(Z, T) \ar[r] & \Ext^k(Z \oplus P, T) \ar[r] & \Ext^{k+1}(\Omega^nC, T) 
			\end{tikzcd}
		\end{center}
		where the left map is induced by $f_Z$ since $\Ext^k(Z \oplus P, T) \cong \Ext^k(Z, T)$. Since $f_Z$ is nilpotent this map is surjective if and only if $\Ext^k(Z, T)=0$. We know that, since $\Omega^nC$ has finite projective dimension, $\Ext^{k+1}(\Omega^n C, T)$ is 0 for $k$ large enough. Then we must have that $\Ext^k(Z, T)=0$, and thus $Z$ has finite projective dimension. Specifically we have $\pd\Omega^n C -1 \leq \pd Z \leq \pd\Omega^n C$.
		
		Since $Z$ is a direct summand of $\Omega^n (A\oplus B)$, by \cref{lem:properties_of_psi} we have that $\pd Z + n \leq \psi(A \oplus B)$, and thus $\pd \Omega^n C - 1 + n = \pd C - 1 \leq \psi(A \oplus B)$.
	\end{proof}
\end{theorem}

\begin{cor}\label{cor:projdim_bounded_by_psi}
	Let $0 \to A \to B \to C \to 0$ be a short exact sequence of modules. 
	\begin{enumerate}[i)]
		\item \label{cor:projdim_bounded_by_psi_i}
		If $\pd A < \infty$, then $\pd A \leq \psi(\Omega B \oplus \Omega C)+1$.
		\item \label{cor:projdim_bounded_by_psi_ii}
		If $\pd B < \infty$ then $\pd B \leq \psi(\Omega A \oplus \Omega^2 C) + 2$.
	\end{enumerate}
	\begin{proof}
		Let $P_B \to B$ be a projective cover of $B$. Then we have a commutative diagram:
		\begin{center}
			\begin{tikzcd}
			0 \ar[r]  & 0 \ar[r] \ar[d] & P_B \ar[r] \ar[d] & P_B \ar[r] \ar[d] & 0\\
			0 \ar[r] & A \ar[r] & B \ar[r] & C \ar[r] & 0 
			\end{tikzcd}
		\end{center}
		Applying the Snake Lemma we get a short exact sequence $$0 \to \Omega B \to \Omega C \oplus P \to A \to 0$$ for some projective module $P$. Then using the theorem we have that if $\pd A \leq \infty$, then $\pd A \leq \psi(\Omega B \oplus \Omega C \oplus P) + 1 = \psi(\Omega B \oplus \Omega C) + 1$.
		
		Applying the same reasoning to $0 \to \Omega B \to \Omega C \oplus P \to A \to 0$ gives us that if $\pd B \leq \infty$, then $\pd\Omega B \leq \psi(\Omega A \oplus \Omega^2 C) + 1$. Hence $\pd B \leq  \psi(\Omega A \oplus \Omega^2 C) + 2$.
	\end{proof}
\end{cor}

\begin{theorem}\cite[Corollary~8]{IgTo05}
	If $\Lambda = \End_\Gamma(P)^{\op}$ for an algebra $\Gamma$ with global dimension at most 3, and $P$ projective, then $\findim(\Lambda) < \infty$.
	\begin{proof}
		Let $X$ be any $\Lambda$-module with finite projective dimension. Then it has a projective presentation $(P, P_1) \to (P,P_0) \to X \to 0$ where $(P,P_i)=\Hom_\Gamma(P,P_i)$ with $P_i \in \add P$. Since $(P,-)$ is an equivalence from $\add P$ to $\proj\Lambda$ this corresponds to a map $P_1 \to P_0$ which we can extend to a projective resolution in $\Gamma$:
		\begin{center}
			\begin{tikzcd}
			0 \ar[r] & P_3 \ar[r] & P_2 \ar[r] & P_1 \ar[r] & P_0.
			\end{tikzcd}
		\end{center}
		Applying the exact functor $(P, -)$, we get an exact sequence
		\begin{center}
			\begin{tikzcd}
			0 \ar[r] & (P,P_3) \ar[r] & (P,P_2) \ar[r] & (P,P_1) \ar[r] & (P,P_0)\ar[r] & X \ar[r] & 0.
			\end{tikzcd}
		\end{center}
		Truncating this we get a short exact sequence
		\begin{center}
			\begin{tikzcd}
			0 \ar[r] & (P, P_3) \ar[r] & (P, P_2) \ar[r] & \Omega^2 X \ar[r] & 0.
			\end{tikzcd}
		\end{center}
		Then by \cref{thm:projdim_bounded_by_psi} the projective dimension of $\Omega^2 X$ is bounded by $\psi((P, P_3)\oplus (P, P_2))+1$. Which means
		$$\pd X \leq \psi((P, P_3)\oplus (P, P_2))+3 \leq \psi((P,\Gamma))+3$$
		Since this bound doesn't depend on $X$, $\Lambda$ has finite finitistic dimension.
	\end{proof} 
\end{theorem}

\begin{cor}
	If $\repdim(\Lambda) \leq 3$ then $\findim(\Lambda) < \infty$.
	\begin{proof}
		If $\Lambda$ has rep-dimension less than or equal to 3 then by \cref{prop:repdim_auslander_generator} there is a generator-cogenerator $M$ in $\mod\Lambda$ such that $\Gamma := \End_\Lambda(M)$ has global dimension 3 or less. Then since $M$ is a generator $\Lambda$ is in $\add M$ and so $\Hom_\Lambda(M, \Lambda)$ is a projective $\Gamma$-module with $\End_\Gamma(\Hom_\Lambda(M, \Lambda)) = \End_\Lambda(\Lambda) = \Lambda$.
	\end{proof}
\end{cor}

\subsection{Stably hereditary algebras}
In this section we will show that the class of stably hereditary algebras has repdimension at most 3, and thus that they have finite finitistic dimension.

\begin{defn}[(co)torsionfree]
	A module is called \textit{torsionfree} if it is a submodule of a projective module. Dually, a module is called \textit{cotorsionfree} if it is a factormodule of an injective.
\end{defn}

\begin{defn}[Stably hereditary algebra]
	An algebra is called \textit{stably hereditary} if any indecomposable torsionfree module is projective or simple, and any indecomposable cotorsionfree moule is injective or simple. 
\end{defn}

This generalizes the definition of hereditary algebra by also allowing simple modules to be (co)torsionfree.

\begin{defn}[The stable category]
	For an algebra $\Lambda$, \textit{the stable category} $\underline{\mod}\Lambda$ has the same objects as $\mod\Lambda$, but the homsets are given by $$\Hom_{\underline{\mod}\Lambda}(M, N) = \Hom_\Lambda(M,N)/\mathcal{P}(M,N)$$
	where $\mathcal{P}(M,N)$ is the ideal of all morphisms factoring through a projective.
\end{defn}

\begin{prop}
	If for an algebra $\Lambda$ there is a hereditary algebra $H$ such that $\underline{\mod}\Lambda \cong \underline{\mod}H$ then $\Lambda$ is stably hereditary.
	\begin{proof}
		\cite[Lemma~4.12]{AR91} \todo{+ a bit more...} \cite{AR73}
	\end{proof}
\end{prop}

The converse of the above proposition does not hold without more assumptions, but stably hereditary algebras generalize the idea of algebras stably equivalent to hereditary algebras.

\begin{theorem}\cite[Theorem~3.5]{Xi02}
	Stably hereditary algebras has repdimension at most 3.
	\begin{proof}
		Let $V$ be the direct sum of all the indecomposable projectives, all the indecomposable injectives, and all the simple modules. Then $V$ is a generator-cogenerator. So by \cref{prop:repdim_auslander_generator} if we can show that the global dimension of $\Gamma:=\End(V)^{op}$ is 3 or less, then we are done.
		
		We will show that for any $\Lambda$-module $M$ there is a short exact sequence $0 \to V_3 \to V_3 \to M \to 0$ with $V_i$ in $\add V$, and such that $0 \to (V, V_3) \to (V, V_2) \to (V, M) \to 0$ is exact. We will use this to construct short projective resolutions for $\mod\Gamma$. To construct $V_3$ and $V_2$ let $M'$ be the sum of the maximal injective summand of $M$ and all simple submodules of $M$. Then let $P$ be the projective cover of $M/M'$. Taking the pullback of $M \to M/M' \leftarrow P$ gives us the diagram:
		\begin{center}
		\begin{tikzcd}[column sep = 15pt, row sep = 25pt]
			   && 0 \ar[d] & 0 \ar[d]\\
			   && K \ar[d] \ar[r, equal] & K \ar[d]\\
			0 \ar[r] & M' \ar[r] \ar[d, equal] & M'\oplus P \ar[r]\ar[d] & P\ar[r]\ar[d] & 0\\
			0 \ar[r] & M' \ar[r] & M \ar[r]\ar[d] & M/M' \ar[r]\ar[d] & 0\\
			&&0&0 
		\end{tikzcd}
		\end{center}
		I claim that $0 \to K \to M'\oplus P \to M \to 0$ is the desired sequence. Firstly $M'\oplus P$ is clearly in $\add V$ since it is the sum of an injective, a semisimple, and a projective module. Further $K$ is a submodule of $P$, hence torsionfree. So since $\Lambda$ is stably hereditary $K$ is the sum of a projective and a semisimple module, so $K$ is also in $\add V$.
		
		Next we need to show that $0 \to (V, K) \to (V, M'\oplus P) \to (V, M) \to 0$ is exact. The only thing needed to show here is that $(V, M'\oplus P) \to (V, M)$ is surjective. We do this by showing that $(W, M'\oplus P) \to (W, M)$ is surjective for any indecomposable summand of $V$. If $W$ is projective this holds by definition. If $W$ is simple then any map from $W$ to $M$ factors through the socle and hence through $M'$, so it's surjective. Lastly if $W$ is injective then the image of $W$ in $M$ is a cotorsionfree module, so it is the sum of simple modules and an injective module. Hence the map from $W$ to $M$ factors through $M'$.
		
		Now we use this to show that the global dimension of $\Gamma$ is at most 3. Let $N$ be any $\Gamma$-module. Then it has a projective presentation
		\begin{center}
		\begin{tikzcd}
			(V,V_1) \ar[r, "f\circ-"] & (V,V_0) \ar[r] & N \ar[r] & 0
		\end{tikzcd}
		\end{center}
		If we let $M$ denote the kernel of $f$ and we choose $V_3$ and $V_2$ as above then we get a projective resolution of $N$ by
		\begin{center}
			\begin{tikzcd}[column sep=20pt]
			0\ar[r] & (V,V_3) \ar[r] & (V,V_2) \ar[r] & (V,V_1) \ar[r] & (V,V_0) \ar[r] & N \ar[r] & 0.
			\end{tikzcd}
		\end{center}
		This shows that the projective dimension of $N$ is at most 3, and since $N$ was arbitrary the global dimension of $\Gamma$ is at most 3. So the repdimension of $\Lambda$ is at most 3.
	\end{proof}
\end{theorem}


\subsection{Special biserial algebras}
\cite{EHIS04}

\section{Vanishing radical powers}
Throughout this section $\Lambda$ is a finite dimensional algebra, and $J$ is its radical.

\begin{theorem}
	If $J^2=0$ then $\findim(\Lambda) < \infty$.
	\begin{proof}
		Let $d = \max\{\pd S_i | \pd S_i < \infty\}$ where $S_i$ ranges over the simple $\Lambda$-modules. Let $M$ be a module with $\pd M < \infty$. Let $P \to M$ be a projective cover. Then $\Omega M$ is contained in $JP$ and since $J^2P=0$, $\Omega M$ is annihilated by $J$ and is thus semisimple. This means $\pd \Omega M \leq d$, and thus $\pd M \leq d+1$. So $\findim(\Lambda) \leq d+1 < \infty$.
	\end{proof}
\end{theorem}

\begin{theorem}\cite[Corollary~6]{IgTo05}
	If $J^3=0$ then $\findim(\Lambda) < \infty$.
	\begin{proof}
		Let $M$ be a module with $\pd M < \infty$, and let $P^0 \to M$ be its projective cover. Since $\Omega M \subseteq JP^0$ we have $J^2\Omega M = 0$. Let $P \to \Omega M$ be a projective cover. Since $J^2\Omega M = 0$ we can factorize this as $P \to P/J^2P \to \Omega M$, and we get a short exact sequence
		\begin{center}
		\begin{tikzcd}
			0 \ar[r] & (\Omega^2 M + J^2P) / J^2 P \ar[r] & P / J^2 P \ar[r] & \Omega M \ar[r] & 0
		\end{tikzcd}
		\end{center}
		Let $\psi$ be the Igusa-Todorov function as introduced in \cref{sec:Igusa-Todorov}. Since $\Omega^2 M \subseteq JP$ we have that $(\Omega^2 M + J^2P) / J^2 P$ is semisimple. Then by \cref{lem:properties_of_psi} $\psi((\Omega^2 M + J^2P) / J^2 P) = \psi(\Lambda / J)$, and $\psi(P / J^2 P) = \psi(\Lambda / J^2)$.
		
		Applying \cref{thm:projdim_bounded_by_psi} to the short exact sequence above we thus get $\pd \Omega M \leq \psi(\Lambda / J \oplus \Lambda / J^2) + 1$, and so $\pd M \leq \psi(\Lambda / J \oplus \Lambda / J^2) + 2$, and $\findim(\Lambda) < \infty$.
	\end{proof}
\end{theorem}

\begin{theorem}\cite{Wang94}
	If $J^{2l+1} = 0$ and $\Lambda / J^l$ is representation finite then $\findim(\Lambda) < \infty$.
	\begin{proof}
		Let $M$ be a module with $\pd M < \infty$. We have a short exact sequence 
		\begin{center}
			\begin{tikzcd}
			0 \ar[r] & J^l\Omega M \ar[r] & \Omega M \ar[r] & \Omega M / J^l\Omega M \ar[r] & 0.
			\end{tikzcd}
		\end{center}
		Since $\Omega M \subseteq JP^0_M$ we have $J^{2l}\Omega M = 0$. This means that $J^l\Omega M$ and $\Omega M / J^l\Omega M$ are $\Lambda / J^l$-modules. We will use this, the fact that $\Lambda / J^l$ is representation finite, and the Igusa-Todorov function to create a bound for $\pd M$.
		
		Applying \cref{cor:projdim_bounded_by_psi} we have that:
		$$ \pd \Omega M \leq \psi(\Omega (J^l\Omega M)\oplus\Omega^2(\Omega M / J^l\Omega M)) + 2.$$ 
		Since $\Lambda / J^l$ is representation finite there are only finitely many indecomposable $\Lambda / J^l$-modules, up to isomorphism. Let $S$ be the sum of all of them. Then since $J^l\Omega M$ and $\Omega M / J^l\Omega M$ are in $\add S$, using \cref{lem:properties_of_psi} we have that 
		$$\psi(\Omega (J^l\Omega M)\oplus\Omega^2(\Omega M / J^l\Omega M)) \leq \psi(\Omega S \oplus \Omega^2 S).$$
		So $\pd M \leq \psi(\Omega S \oplus \Omega^2 S) + 3$, and thus $\findim(\Lambda) < \infty$.
	\end{proof}
\end{theorem}


\section{Monomial algebras}\label{sec:monomial_algebras}
\cite{GKK91, IgZa90}

In this section we will show a particularly nice way to construct a minimal projective resolution of the right module $\Lambda / J$ for a monomial algebra $\Lambda$. We will use this to compute $\Tor_i(\Lambda /J, M)$ and/or $\Ext^i(M, D\Lambda/J)$ to get a bound on the projective dimension of all modules $M$.

\begin{defn}[Monomial algebra]
	A \emph{monomial algebra} is a path algebra with admissible relations that are generated by monomials. That is, we do not allow the generators for the relations to consist of nontrivial linear combinations of paths.
\end{defn}

\begin{defn}[$m$-chains]\cite{GKK91}
	Let $\Lambda = k\Gamma / (\rho)$ be a monomial algebra, with $\rho$ a minimal generating set of paths. As usual we define $\Gamma_0$ to be the vertices of $\Gamma$, and $\Gamma_1$ to be the arrows. Recursively define the set of $(m-1)$-chains, $\Gamma_m$, as the paths $\gamma$ with the following criteria:
	\begin{itemize}
		\item $\gamma = \beta\delta\tau$ with $\beta \in \Gamma_{m-2}$, $\beta\delta \in \Gamma_{m-1}$, and $\tau$ a non-zero path of length at least 1.
		\item $\delta\tau$ is 0 in $\Lambda$, i.e. it is ine the ideal of relations.
		\item $\gamma$ is left-minimal in the sense that if $\gamma = \gamma' \sigma$ such that $\gamma'$ satisfies the above conditions, then $\gamma = \gamma'$.
	\end{itemize}
\end{defn}

The $\Gamma_m$'s will become the generating sets for the projectives in our projective resolution. But first we will prove some properties of them.

\begin{lemma}
	Any $\gamma\in \Gamma_m$ for $m \geq 1$ can be factored uniquely as $\gamma_1\gamma_0$ with $\gamma_1 \in \Gamma_{m-1}$, and $\gamma_0$ a non-zero path of length at least 1.
	\begin{proof}
		When $m=1$ this should be clear, since $\Gamma_1$ is the set of arrows, and $\Gamma_0$ is the set of vertices, so if $\gamma \in \Gamma_1$ is an arrow $i\to j$ then $\gamma = e_j\gamma$.
		
		When $m > 1$ we know from the definition of $\Gamma_m$ that $\gamma$ can be written as $\gamma_1\gamma_0$. Assume there is another decomposition $\gamma = \gamma'_1\gamma'_0$. Then without loss of generality we may assume that $\gamma'_1$ is shorter than $\gamma_1$. Then there is a $\sigma$ such that $\gamma'_1\sigma = \gamma_1$. By minimality this means that $\gamma'_1=\gamma_1$, and so the decomposition is unique.
	\end{proof} 
\end{lemma} 

From now on we will write $R$ for $\Lambda/J$. Let $k\Gamma_m$ be the free vectorspace generated by $\Gamma_m$. Notice that $k\Gamma_m$ has a canonical structure as a $R$-$R$-bimodule. This means we can get projective right $\Lambda$-modules $P^m := k\Gamma_m\otimes_R\Lambda$.

Define the map $\delta_m : P^m \to P^{m-1}$ by $\delta_m(\gamma \otimes \alpha) = \gamma_1 \otimes \gamma_0\alpha$ where $\gamma_1\gamma_0$ is the unique decomposition of $\gamma$, and define $\delta_0:k\Gamma_0 \to \Lambda /J$ by $\delta_0(e_i\otimes \alpha) = e_i\alpha + J$. Then I claim we have a minimal projective resolution of the right $\Lambda$-module $\Lambda/J$ by

\begin{center}
\begin{tikzcd}
	\cdots \ar[r] & P^3 \ar[r, "\delta_3"] & P^2 \ar[r, "\delta_2"] & P^1 \ar[r, "\delta_1"] & P^0 \ar[d, two heads, "\delta_0"]\ar[r] & 0\\
	&&&&\Lambda / J
\end{tikzcd}
\end{center}

\begin{proof}
	For all $i$ $P^i$ is projective and the image of $\delta_m$ is clearly contained in $P^{m-1}J$, so the only thing left to show is exactness. First we show that $\delta_m\delta_{m-1}=0$. Let $\gamma\otimes \alpha$ be in $P^m$ for $m \geq 2$. Then we can decompose $\gamma$ uniquely as $\gamma_2\gamma_1\gamma_0$ and $\delta_m\delta_{m-1}(\gamma\otimes \alpha) = \gamma_2\otimes\gamma_1\gamma_0\alpha$. By the way we defined $\Gamma_m$, $\gamma_1\gamma_0$ is 0 in $\Lambda$, and so $\gamma_2\otimes\gamma_1\gamma_0\alpha = 0$.
	
	Next we want to show that $\Ker\delta_{m-1} \subseteq \Image\delta_m$. Let $\sum \gamma^i\otimes \alpha^i$ be in $\Ker\delta_{m-1}$. \todo{finish}
\end{proof}


\section{Unbounded derived category}

If we go to the unbounded derived category we can get a sort of converse to \cref{thm:findim_implies_inj_generate}.

\begin{theorem}\cite[Theorem~4.3]{Rick19}\label{thm:injectives_generate_implies_FDC}
	If the localizing category of $D\Lambda$ is the entire unbounded derived category then $\Findim(\Lambda) < \infty$. (Note the capital F meaning the finitistic dimesnion of $\Mod\Lambda$, which is bigger than or equal to that of $\mod\Lambda$).
	
	\begin{proof}
		Assume $\Findim(\Lambda) = \infty$. Then there are modules $M_i$ with projective dimension $i$ for every $i \geq 0$. Let $P_i$ be the minimal projective resolution of $M_i$, and consider $\bigoplus P_i[-i]$ and $\prod P_i[-i]$. Both of these have homology $M_i$ in degree $i$, and are concentrated in non-negative degrees.
		
		The inclusion from the sum to the product is clearly a quasi-isomorphism. We want to show that it is not a homotopy equivalence. Assume for the sake of contradiction that it was. Then tensoring with $\Lambda/J$ would give us another homotopy equivalence. Since $\Lambda/J$ is finitely presented tensoring preserves both products and coproducts. Because all the resolutions were minimal tensoring with $\Lambda/J$ gives us 0 differentials. In degree 0 we get $$\bigoplus \Tor_i(\Lambda/J, M_i) \to \prod \Tor_i(\Lambda/J, M_i) .$$
		Since $\Tor_i(\Lambda/J, M_i)$ is nonzero for every $M_i$ this map is not an isomorphism, and so we don't have a homotopy equivalence.
		
		So the cone of the inclusion $\bigoplus P_i[-i] \to \prod P_i[-i]$, $C$, is 0 in the derived category, but non-zero in the homotopy category. Since $\Lambda$ is artinian the product of projectives is projective\cite[Theorem~3.3]{Chase60}, so $\prod P_i[-i]$ is a complex of projectives, which means that $C$ is a complex of projectives. 
		
		In other words $C$ is an acyclic lower bounded complex of projectives that is not contractible. Tensoring with $D\Lambda$ is an equivalence from projectives to injectives with inverse $\D(\Lambda)(D\Lambda, -)$ \todo{maybe I should prove this or find refference}, so $D\Lambda \otimes C$ is an lower bounded complex of injectives that is not contractible. Such a complex cannot be acyclic so $D\Lambda \otimes C$ has homology.
		
		The homology of $C$ is 0, so $K(\Lambda)(\Lambda, C[i]) = 0$. Applying the equivalence $D\Lambda \otimes -$ we get $$\D(\Lambda)(D\Lambda, D\Lambda \otimes C [i])=K(\Lambda)(D\Lambda, D\Lambda \otimes C [i])=0.$$ This means that $D\Lambda \otimes C$ is not in the localizing category generated by $D\Lambda$, and so that can not be the entire derived category.
	\end{proof}
\end{theorem}

\begin{theorem}\cite[Theorem~4.4]{Rick19}
	$\Findim(\Lambda) < \infty$ if and only if $D\Lambda^\perp \cap \D^+(\Lambda) = 0$.
	\begin{proof}
		In the theorem above we proved that when the finitistic dimension is infinite then there is a non-zero complex in $\D^+(\Lambda)$ perpendicular to $D\Lambda$. 
		
		The proof of the converse is the same as for \cref{thm:findim_implies_inj_generate}. If we have a non-zero object $X \in D\Lambda^\perp \cap \D^+(\Lambda)$, then $\D(\Lambda)(D\Lambda, X)$ is an acyclic minimal complex of projectives that continue arbitrarily to the right. So the cokernels have arbitrarily big projective dimension. \todo{We see this by taking injective resolution of X}
	\end{proof}
\end{theorem}

\section{Summary}

FDC holds for the following classes of algebras

\begin{itemize}
	\item \textbf{Big FDC:}
	\item Representation finite algebras
	\begin{proof}
		The supremum over a finite set is finite so $\findim(\Lambda) < \infty$ for a representation finite algebra.
	\end{proof}
	\item Monomial algebras
	\begin{proof}
		This was shown in \cref{sec:monomial_algebras}.
	\end{proof}
	\item Gorenstein algebras
	\begin{proof}
		An algebra is said to be Gorenstein if all injectives have finite projective dimension and all projectives have finite injective dimension. In particular the $\Lambda$-module $\Lambda$ is isomorphic to a finite injective resolution in the derived category. So $\Lambda$ is in the localizing category generated by injectives. Then \cref{thm:injectives_generate_implies_FDC} gives us that $\Findim(\Lambda) < \infty$, and therefor also $\findim(\Lambda) < \infty$.
	\end{proof}
	\item Finite global dimension
	\item Self injective
	\item $J^2 = 0$
	\item Derived equivalent to the above
	\item Local algebras
	\begin{proof}
		Local algebras are local artinian rings. So if $\Lambda$ is local then $\findim(\Lambda)=0$.
	\end{proof}
	\item \textbf{only small FDC is known?:}
	\item Stably hereditary algebras
	\item Special biserial algebras
	\item "half rep-finite" algebras, i.e. $\Lambda/J^l$ rep-finite $J^{2l+1}=0$.
\end{itemize}

Not sure where to put this, ill put it here for now
\begin{theorem}
	Local artinian rings have finitistic dimension zero.
	\begin{proof}
		Assume there is a non-projective module with finite projective dimension. Then in particular we have one with projective dimension equal to 1. Since all finitely generated projectives are free this means we have a short exact sequence
		\begin{center}
			\begin{tikzcd}
				0 & R^n & R^m & M & 0
			\end{tikzcd}
		\end{center}
	with $R^n$ contained in $JR^m$. Let $k$ be the minimal integer such that $J^k=0$. Let $a$ be a generator in $R^n$ and let $r$ be a non-zero element of $J^{k-1}$. Then $ra$ is non-zero, but is mapped to something in $J^{k-1}JR^m=0$, thus the map is not injective which gives a contradiction. 
	\end{proof}
\end{theorem}

\section{Dual conjectures}

Many of the cases are equivalent to their dual statements. Some are not.
\begin{itemize}
	\item Given a recollement of the bounded derived category you get one for $\Lambda^{\operatorname{op}}$. \todo{Look at examples of reccolement to see how it translates.}
	\item Just because the subcategory of modules with finite projective dimension is contravariantly finite does not mean the subcategory of modules with finite injective dimension has to be covariantly finite. See \cref{exam:not_contravariantly_finite}.
	\item repdim of $\Lambda$ equals the repdim of $\Lambda^{\operatorname{op}}$.
	\begin{proof}
		If $M$ is an auslander generator for $\Lambda$ then $DM$ is an auslander generator for $\Lambda^{\operatorname{op}}$.
	\end{proof}
	\item If $J^{2l+1} = 0$ and $\Lambda/J^l$ is repfinite then the same is true for $\Lambda^{\operatorname{op}}$.
	\item If $\Lambda$ is monomial then so is $\Lambda^{\operatorname{op}}$.
	\item Injective generates implies the weaker property that projective cogenerate for the opposite algebra. This is also sufficient to prove the algebra satisfies FDC.\cite[Section~5]{Rick19} 
\end{itemize}

Similarly for the weaker conjectures
\begin{itemize}
	\item GSC says the injective dimension of $\Lambda$ is finite if and only if the injective dimension of $\Lambda^{\operatorname{op}}$ is finite. This statement is symmetric with respect to $\Lambda$ and $\Lambda^{\operatorname{op}}$. So the dual is equivalent.
	\item NC: Certainly $\Lambda$ is self injective if and only $\Lambda^{\operatorname{op}}$ is. \todo{Can the dominant dimension of the opposite algebra be different? Arbitrary different?}
	\item For all the others it seems just as difficult as solving the conjecture to connect it to it's dual.
\end{itemize}


\section{Personal appendix}
\begin{theorem}
	The global dimension of an artin algebra is the supremum of $k$ with $\Ext^k(T,T)\neq 0$ ($T$ sum of simples). This is also the supremum of projective dimension and supremum of injective dimension.
	\begin{proof}
		For a minimal projective resolution $\Hom(-,T)$ makes the differentials 0, and similarly with $\Hom(T,-)$ and injective resolutions. So $\Ext^k(M, T)$ is only 0 exactly when $k>\pd M$, similarly $\Ext^k(T,M)$ is only 0 when $k$ is bigger than the injective dimension. Since any module is built by extensions of simples you can prove by induction, and the long exact sequence in $\Ext(-,T)$ you get that any module has projective dimension less than or equal to that of $T$. Similarly for injective dimension.
	\end{proof}
\end{theorem}

$\findim(\Lambda)$ need not equal $\findim(\Lambda^{\operatorname{op}}) = \sup\{ $injective dimension of $M | M$ has finite injective dimension$ \}$.

\begin{example} \cite{Gre20}
	Let $\Lambda=k \left.
	\left[\begin{tikzcd}
	\ar[out=150,in=210, loop, swap, looseness=3, "a"] 1 \ar[r, bend left=15, "b"] & 2 \ar[l, bend left=15, "c"]
	\end{tikzcd}\right] \middle/ (a^2, ac, ba, cbc) \right.$. Then $\findim(\Lambda) \geq 1$, but $\findim(\Lambda^{\operatorname{op}})=0$.
	\begin{proof}
		The module $\mymatrix{1\\1} = P_1/P_2 $ ($k^2$ where $a$ acts by $\begin{bsmallmatrix}
			0 & 1\\0&0
		\end{bsmallmatrix}$, and $b$ and $c$ act trivially)
		has projective dimension 1, so $\findim(\Lambda) \geq 1$. The projective/injective modules of $\Lambda$ are:
		$$ P_1 = \mymatrix{
			&1&\\
			1 && 2\\
			&&1\\
			&&2
		},\quad P_2 = \mymatrix{
			2\\1\\2
		},\quad I_1 = \mymatrix{
			&&1\\
			1&&2\\
			&1&
		},\quad I_2 = \mymatrix{
			1\\2\\1\\2
		} $$
		If $\findim(\Lambda^{\operatorname{op}})>0$ there would be a module with finite non-zero injective resolution. In particular it would end with a non-split epimorphism between injectives. I claim this would mean there is a non-split epimorphism $I \to I_i$ from an injective to an indecomposable injective. Obviously we get epimorphisms by composing with the projections onto summands, so we want to show that they are not split. Assume that they are, that is the map looks like
		
		\begin{center}
		\begin{tikzcd}[ampersand replacement=\&]
			I_i \oplus I \ar{r}{
				\begin{bmatrix}
				1 & 0\\ f & g
				\end{bmatrix}
			} \ar[swap]{rd}{
				\begin{bmatrix}
				1 & 0
				\end{bmatrix}
			} \& I_i \oplus I' \ar[]{d}{
				\begin{bmatrix}
				1 & 0
				\end{bmatrix}
			}\\
			\& I_i
		\end{tikzcd}.
		\end{center}
		We see that by changing basis in the domain we get the matrix $\begin{bmatrix}
		1&0\\0&g
		\end{bmatrix}$. Thus $I_i$ is mapped isomorphically to itself, which doesnt happen in a minimal resolution.
		
		The only thing left to show is that there are no non-split epimorphisms from injective modules to $I_1$ and $I_2$.
	\end{proof}
\end{example}

\begin{lemma}\cite[Chapter I, theorem 3.2]{CE56} \label{lem:injectives_for_noetherian_ring}
	Let $R$ be a noetherian ring. Then an $R$-module $Q$ is injective if and only if it has the injective lifting property for inclusions of ideals into $R$.
	\begin{proof}
		If $Q$ is injective then $Q$ has the lifting property for all monomorphisms, so one direction is clear. Assume we have a diagram
		\begin{center}
		\begin{tikzcd}
			Q\\
			M \ar[u, "f"] \ar[r, hook] & N \ar[ul, dashed]
		\end{tikzcd}
		\end{center}
		We want to show that the dashed arrow exists. Let $S$ be the partially ordered set $\{(M', f'): M \leq M', f'|_M = f\}$. By Zorn's lemma this has a maximal element $(M', f')$. Assume $M' \neq N$, then there is an element $x \in N - M'$. The set of $r$ such that $rx \in M'$ forms an ideal $I$. Define the map $g: I \to Q$ by $I(r) = f'(rx)$. By hypothesis $g$ lifts to a map $\tilde{g}:R \to Q$. Let $q$ be $\tilde{g}(1)$. Then $\tilde{f}: M' + Rx \to Q$ defined by $\tilde{f}(m + rx) = f'(m) + rq$ gives us a bigger element of $S$, contradicting maximality. Thus $M'=N$ and $Q$ is injective.
	\end{proof}
\end{lemma}

\begin{theorem}
	Let $R$ be a noetherian ring. Then an arbitrary coproduct of injectives is injective.
	\begin{proof}
		By the lemma above it is enough to show the lifting property on ideals of $R$. Let $I$ be an ideal and $f:I \to \bigoplus_i Q_i$ be a map to a coproduct of injectives. Since $R$ is notherian $I$ is finitely generated so $f$ factors through a finite sum $I \to \bigoplus_{i=0}^n Q_i \to \bigoplus Q_i$. Since finite coproducts of injectives are injective we are done.
		\begin{center}
			\begin{tikzcd}
			\bigoplus Q_i\\
			\bigoplus\limits_{i=0}^n Q_i \ar[u]\\
			I \ar[u] \ar[r, hook] & R \ar[ul, dashed]
			\end{tikzcd}
		\end{center}
	\end{proof}
\end{theorem}

\begin{theorem}\cite[Chapter I, Exercise 8]{CE56}
	Let $R$ be a noetherian ring. Then direct limits of injectives is injective.
	\begin{proof}
		By the lemma above it is enough to show the lifting property on ideals of $R$. Let $I$ be an ideal and let $Q = \lim\limits_{\rightarrow} Q_i$ be a direct limit of injectives.
		
		Since $R$ is noetherian $I$ is finitely presented, say $R^n \to R^m \to I \to 0$. Applying $\Hom(-,Q)$ we get an exact sequence 
		\begin{center}
			\begin{tikzcd}
			0 \ar[r] & \Hom(I, Q) \ar[r] & \Hom(R^m, Q) \ar[r] & \Hom(R^n, Q)
			\end{tikzcd}
		\end{center}
		Since direct limits are exact we also have an exact sequence
		\begin{center}
			\begin{tikzcd}
			0 \ar[r] & \lim\limits_{\rightarrow}\Hom(I, Q_i) \ar[r] & \lim\limits_{\rightarrow}\Hom(R^m, Q_i) \ar[r] & \lim\limits_{\rightarrow}\Hom(R^n, Q_i)
			\end{tikzcd}
		\end{center}
		We also have a natural map $\lim\limits_{\rightarrow}\Hom(-, Q_i) \to \Hom(-, Q)$. $\Hom(R^n, Q_i)$ just equals $Q_i^n$, so this map is an isomorphism at $R^n$. Then by the five lemma applied to the two sequences above we get that $\Hom(I, Q) \cong \lim\limits_{\rightarrow}\Hom(I, Q_i)$ for all ideals $I$. So since 
		\begin{center}
			\begin{tikzcd}
			\lim\limits_{\rightarrow}\Hom(R, Q_i) \ar[r] & \lim\limits_{\rightarrow}\Hom(I, Q_i) \ar[r] & 0
			\end{tikzcd}
		\end{center}
		is exact, we get that
		\begin{center}
			\begin{tikzcd}
			\Hom(R, Q) \ar[r] & \Hom(I, Q) \ar[r] & 0
			\end{tikzcd}
		\end{center}
	is exact. Hence $Q$ is injective.
	\end{proof}
\end{theorem}

\clearpage

\bibliography{mybib}
%\bibliography{intro_bib}
\bibliographystyle{alpha}
%\printbibliography
\end{document}