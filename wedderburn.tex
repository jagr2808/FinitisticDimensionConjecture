To prove the last two implications we need some results from the theory of Wedderburn projectives. The results we need are stated below, and are proved in \cref{sec:wedderburn_correspondence}. We remind the reader that we write $(-,-)$ to mean $\Hom(-,-)$.

\begin{restatable}{restate-thm}{Wederburnequivalence} \label{thm:hom_generator_equivalence}
	Let $\Lambda$ be an artin algebra and $M$ a generator. Let $\Gamma = \End(M)^{\operatorname{op}}$ and $P=(M, \Lambda)$. Then we have the following:
	\begin{enumerate}[i)]
		\item We have an isomorphism of rings $\End_\Gamma(P)^{\operatorname{op}} \cong \Lambda$ and an isomorphism of $\Lambda$-modules $(P_\Lambda, \Gamma) \cong M$.
		\item The composition $(P,-)\circ (M,-)$ is the identity on $\mod\Lambda$.
		\item The functor $(M,-)$ maps injective $\Lambda$-modules to injective $\Gamma$-modules. 
	\end{enumerate}
\end{restatable}

\begin{defn}[Wedderburn projective]
	Let $\Gamma$ be an artin algebra and let $P$ be a finitely generated projective $\Gamma$-module. Let $\Lambda = \End(P)^{\operatorname{op}}$ and $M=(P, \Gamma)$. The module $P$ is said to be \emph{Wedderburn projective} if $\End(M)^{\operatorname{op}}=\Gamma$.
\end{defn}

\begin{restatable}{restate-thm}{Wederburncriterion}\label{thm:wedderburn_criterion}
	Let $\Gamma$ be an artin algebra and $P$ a projective $\Gamma$-module. If $P$ contains the projective cover of all simple modules that appear in the socle of an injective copresentation of $\Gamma$, then $P$ is Wedderburn projective.
\end{restatable}

\subsection{Wedderburn correspondence}\label{sec:wedderburn_correspondence}

In this section we give the relevant theory needed to prove the implications between homological conjectures involving the Auslander--Reiten conjecture.

The theory is about understanding the relationship between the functors $(P,-)\colon \mod \Gamma \to \mod\Lambda$ and $(M,-)\colon \mod\Lambda \to \mod\Gamma$, where $P$ is a projective $\Gamma$-module, $M=(P, \Gamma)$ and $\Lambda = \End(P)^{\op}$. First we show a general result about how the homfunctor interacts with injective modules.

\begin{prop}\label{prop:hom_generator_preserves_injectives}
	Let $M$ be a module and $I$ an injective module. If the projective cover of the socle of $I$ is in $\add M$, then $(M,I)$ is an injective $\Gamma:=\End(M)^{\operatorname{op}}$-module. In particular if $M$ is a generator then $(M,-)$ preserves injectives.
	\begin{proof}
		Let $J \leq \Gamma$ be a left ideal and let $\psi\colon J \to (M,I)$ be any $\Gamma$-linear map. By \cref{lem:injectives_for_noetherian_ring} \todo{[...] In the appendix} it is enough to show that $\psi$ factors through $\Gamma$ to conclude that $(M, I)$ is injective. Assume $J$ is generated by $\{f_i\}$. If we can find $\gamma\colon M \to I$ such that $\gamma \circ f_i = \psi(f_i)$ then we would get our factorization of $\psi$ by 
		\begin{tikzcd}
			J \ar[r, hookrightarrow] & \Gamma \ar[r, "\gamma \circ -"] & (M, I).
		\end{tikzcd} To construct such a $\gamma$ we consider the following diagram.
		\begin{center}
			\begin{tikzcd}
			\bigoplus M \ar[dr, "\sum \psi(f_i)"] \ar[d, swap, "\sum f_i"]\\
			M \ar[r, swap, dashed, "\gamma"] & I
			\end{tikzcd}
		\end{center}
		We want to show that the kernel of $\sum \psi(f_i)$ contains the kernel of $\sum f_i$, so that we can use the injective property of $I$. To see this let $K$ be the kernel of $\sum f_i$ and let $K'$ be the kernel of $\sum \psi(f_i)$. If $K'$ does not contain $K$, then $Q:= K/K'\cap K$ is a nonzero module that is mapped injectively into $I$. So the socle of $Q$ is a summand of the socle of $I$. Then by assumption the projective cover of the socle of $Q$ is in $\add M$, so there is a non-zero projective map $M \to Q$. By the lifting property of projectives we get a map $M \to K$ such that the composition with $\sum \psi(f_i)$ is non-zero.
		
		Let $a_i$ be the composition 
		\begin{tikzcd}[column sep=15pt]
		M \ar[r] & K \ar[r, hookrightarrow] & \bigoplus M \ar[r, "\pi_i"] & M.
		\end{tikzcd}
		Then we get $\sum f_i \circ a_i = 0$. Applying $\psi$ we get $\sum \psi(f_i)\circ a_i = 0$, which gives a contradiction since $a_i$ was explicitly constructed such that $\sum \psi(f_i)\circ a_i$ is non-zero. Thus $K'$ contains $K$.
		
		Using this we get the following commutative diagram:
		\begin{center}
			\begin{tikzcd}
			\bigoplus M \ar[d] \ar[dd, bend right=60, swap, "\sum f_i"] \ar[dr, "\sum \psi(f_i)"] \\
			\left(\bigoplus M\right)/ K \ar[r] \ar[d, hookrightarrow] & I\\
			M \ar[ur, dashed, swap, "\exists\gamma"]
			\end{tikzcd}
		\end{center}
		Since $I$ is injective it lifts monomorphisms so we know that $\gamma$ exists. Thus $(M, I)$ is an injective $\Gamma$-module.
	\end{proof}
\end{prop}

Combining the proposition we just proved with Yoneda's Lemma we have the necessary tools to prove \cref{thm:hom_generator_equivalence}. We restate it her for the convenience of the reader.
\Wederburnequivalence*
\begin{proof}
	\begin{enumerate}[i)]
		\item[]
		\item By Yoneda's Lemma we have an equivalence $$(M,-)\colon\add M \to \add (M,M)=\proj\Gamma.$$ Since $M$ is a generator, $\Lambda$ is in $\add M$. So 
		$$\End(P)=\End((M,\Lambda)) = \End(\Lambda)=\Lambda^{\operatorname{op}}$$ 
		\begin{center}
			and
		\end{center}
		$$(P,\Gamma)=((M,\Lambda),(M,M))=(\Lambda,M)=M.$$
		\item Let $X$ be a $\Lambda$-module. Since $\add M$ has only a finite number of indecomposable modules it is functorially finite. \todo{Have not defined this yet...} So we can take an $M$-resolution of $X$.
		$$\cdots \to M_1 \to M_0 \to X \to 0$$
		Since $\add M$ contains the projectives this is exact. Applying $(M,-)$ we get a projective resolution of $(M,X)$. Since $(M, X)$ is determined by its projective resolution and $X$ is determined by its $M$-resolution we need only show that $(P,-)\circ (M,-)$ is the identity on $\add M$. Then again by Yoneda's Lemma $(P, (M, M')) = (\Lambda, M')=M'$.
		\item Since $M$ is a generator it contains the projective cover of all simple modules. Then \cref{prop:hom_generator_preserves_injectives} gives us that $(M, -)$ maps injective modules to injective modules.
	\end{enumerate}
\end{proof}

\begin{prop}
	Let $P$ be a projective $\Gamma$-module, and let $\Lambda = \End(P)^{\operatorname{op}}$. Then $(P, -)\colon\mod \Gamma \to \mod \Lambda$ is fully faithful on $\add I(P/JP)$, where $P/JP$ denotes the top of $P$, and $I(P/JP)$ its injective envelope.
	\begin{proof}
		Let $I$ and $I'$ be in $\add I(P/JP)$. We want to show that the map $\Hom_\Gamma(I, I') \to \Hom_\Lambda((P,I), (P, I'))$ is an isomorphism. First we show injectivity. Let $f\colon I\to I'$ be a non-zero map. Then the socle of $\Image f$ is a semisimple submodule of $I'$, so it is in $\add P/JP$. Then there exists a nonzero map from $P$ to $\Image f$. Since $P$ is projective this lifts to a map $\hat{f}\colon P\to I$. Then $f \circ \hat{f}$ is non-zero, so $\Hom_\Gamma(I, I') \to \Hom_\Lambda((P,I), (P, I'))$ is injective.
		
		The argument for surjectivity is similar to that for \cref{prop:hom_generator_preserves_injectives}. Let $\psi\colon(P,I)\to (P, I')$ be a $\Lambda$-linear map. Let $f_i\colon P\to I$ generate $(P,I)$ as a $\Lambda$-module. Consider the following diagram:
		\begin{center}
			\begin{tikzcd}
			\bigoplus P \ar[dr, "\sum \psi(f_i)"] \ar[d, swap, "\sum f_i"]\\
			I \ar[r, swap, dashed, "?"] & I'
			\end{tikzcd}
		\end{center}
		We wish to show that there is a map at ? completing the diagram. We first show that $K':=\ker \sum \psi(f_i)$ contains $K:=\ker \sum f_i$. Assume for the sake of contradiction that it does not. Then $Q := K / K' \cap K$ is mapped injectively into $I'$ by $\sum \psi(f_i)$. So the socle of $Q$ is in $\add P/JP$, and we have a non-zero map $P \to Q$.
		
		Since $P$ is projective this extends to a map $P \to K$. Let $a_i$ be the compositions 
		\begin{tikzcd}[column sep = 15pt]
			P \ar[r] & K \ar[r] & \bigoplus P \ar[r, "\pi_i"] & P.
		\end{tikzcd}
		Then clearly $\sum f_i \circ a_i = 0$, but $\sum \psi(f_i) \circ a_i$ is non-zero. Since $\psi$ is $\Lambda$-linear this is a contradiction, so $K'$ contains $K$.
		
		Then we get an induced commutative diagram:
		\begin{center}
			\begin{tikzcd}
			\bigoplus P \ar[d] \ar[dd, bend right=60, swap, "\sum f_i"] \ar[dr, "\sum \psi(f)"] \\
			\left(\bigoplus P\right)/ K \ar[r] \ar[d, hookrightarrow] & I'\\
			I \ar[ur, dashed, swap, "\exists"]
			\end{tikzcd}
		\end{center}
		Now because $I'$ is injective we know that there is a lift, and so the map $\Hom_\Gamma(I, I') \to \Hom_\Lambda((P,I), (P, I'))$ is surjective, and thus an isomorphism.
	\end{proof} 
\end{prop}

We conclude this section by giving a proof of \cref{thm:wedderburn_criterion}.

\Wederburncriterion*
\begin{proof}
	Let $\Gamma \to I_0 \to I_1$ be a minimal injective presentation of $\Gamma$. Then by \cref{prop:hom_generator_preserves_injectives} we have that $(P, I_0) \to (P,I_1)$ is an injective presentation of $(P,\Gamma)$. The proposition gives us that $(P,-)$ is fully faithful on $I_0$ and $I_1$. Since the endomorphisms of $\Gamma$ are exactly endomorphisms of $I_0 \to I_1$ up to homotopy this means that $$\Gamma=\End_\Gamma(\Gamma)^{\op} = \End_\Lambda((P, \Gamma))^{\op}$$
	So $P$ is Wedderburn projective.
\end{proof}