
So far we have been focused on the finite dimensional version of the finitistic dimension, known as the little finitistic dimension. Namely 
$$\findim(\Lambda) = \sup\{\pd M \mid M \in \mod\Lambda, \pd M < \infty\}.$$

In this section we will consider infinite dimensional modules, and thus it is natural for us to look at the infinite dimensional version of the finitistic dimension, known as the big finitistic dimension. It is defined, as you would expect, by considering not just finite dimensional modules, but all $\Lambda$-modules:

$$\Findim(\Lambda) = \sup\{\pd M \mid M \in \Mod\Lambda, \pd M < \infty\}.$$

Note that $\findim(\Lambda) \leq \Findim(\Lambda)$ and so if we can show that $\Findim(\Lambda) < \infty$ we have also shown that $\findim(\Lambda) < \infty$.

In \cref{thm:FDC_implies_VC} we showed that if $\findim(\Lambda) < \infty$, then $D\Lambda$ becomes a generator in $\D^b(\Lambda)$. In this section we show that if we instead consider the unbounded derived category of all $\Lambda$-modules, then we get an analogous converse result.

\begin{theorem}\cite[Theorem~4.3]{Rick19}\label{thm:injectives_generate_implies_FDC}
	If the localizing subcategory generated by $D\Lambda$ is the entire unbounded derived category, then $\Findim(\Lambda) < \infty$.
	
	\begin{proof}
		Assume $\Findim(\Lambda) = \infty$. Then there are modules $M_i$ with projective dimension $i$ for every $i \geq 0$. Let $P_i$ be the minimal projective resolution of $M_i$, and consider $\bigoplus P_i[-i]$ and $\prod P_i[-i]$. Both of these have homology $M_i$ in degree $i$, and are concentrated in non-negative degrees.
		
		The inclusion from the sum to the product is clearly a quasi-isomorphism. We want to show that it is not a homotopy equivalence. Assume for the sake of contradiction that it was. Then tensoring with $\Lambda/J$ would give us another homotopy equivalence. Since $\Lambda/J$ is finitely presented tensoring preserves both products and coproducts. Because all the resolutions were minimal tensoring with $\Lambda/J$ gives us 0 differentials. In degree 0 we get $$\bigoplus \Tor_i(\Lambda/J, M_i) \to \prod \Tor_i(\Lambda/J, M_i) .$$
		Since $\Tor_i(\Lambda/J, M_i)$ is nonzero for every $M_i$ this map is not an isomorphism, and so we don't have a homotopy equivalence.
		
		So the cone of the inclusion $\bigoplus P_i[-i] \to \prod P_i[-i]$, $C$, is 0 in the derived category, but non-zero in the homotopy category. Since $\Lambda$ is artinian the product of projectives is projective\cite[Theorem~3.3]{Chase60}, so $\prod P_i[-i]$ is a complex of projectives, which means that $C$ is a complex of projectives. 
		
		In other words $C$ is an acyclic lower bounded complex of projectives that is not contractible. Tensoring with $D\Lambda$ is an equivalence from projectives to injectives with inverse $\Hom(D\Lambda, -)$ \todo{\cref{thm:Proj_Inj_equivalence} in appendix}, so $D\Lambda \otimes C$ is a lower bounded complex of injectives that is not contractible. Such a complex cannot be acyclic so $D\Lambda \otimes C$ has homology, and is thus non-zero in $\D(\Lambda)$.
		
		The homology of $C$ is 0, so $K(\Lambda)(\Lambda, C[i]) = 0$. Applying the equivalence $D\Lambda \otimes -$ we get 
		$$0=K(\Lambda)(D\Lambda, D\Lambda \otimes C [i])=\D(\Lambda)(D\Lambda, D\Lambda \otimes C [i]).$$ 
		This means that $D\Lambda \otimes C$ is not in the localizing category generated by $D\Lambda$, and so that can not be the entire derived category.
	\end{proof}
\end{theorem}

\begin{theorem}\cite[Theorem~4.4]{Rick19}
	$\Findim(\Lambda) < \infty$ if and only if $D\Lambda^\perp \cap \D^+(\Lambda) = 0$.
	\begin{proof}
		In the theorem above we proved that when the finitistic dimension is infinite then there is a non-zero complex in $\D^+(\Lambda)$ perpendicular to $D\Lambda$. 
		
		The proof of the converse is the same as for \cref{thm:FDC_implies_VC}. If we have a non-zero object $X \in D\Lambda^\perp \cap \D^+(\Lambda)$, then $\D(\Lambda)(D\Lambda, X)$ is an acyclic minimal complex of projectives that continue arbitrarily to the right. So the cokernels have arbitrarily big projective dimension. \todo{We see this by taking injective resolution of X}
	\end{proof}
\end{theorem}